\documentclass{article}
\title{Une amélioration des templates pour les nombres de Ramsey mutlticolores et les nombres de Schur}
\author{Romain Ageron}

\usepackage{amsmath}
\usepackage{amssymb}
\usepackage{amsthm}
\usepackage{dsfont}
\usepackage{mathtools}

\usepackage{enumerate}

\usepackage[french]{babel}
\usepackage[hyphens]{url}
\usepackage{hyperref}
\usepackage[utf8]{inputenc}
\usepackage{siunitx}
\usepackage{float}

\usepackage[table]{xcolor}
\usepackage{multirow}
\usepackage{nicematrix}
\usepackage[a4paper, total={6in, 8in}]{geometry}

\newtheorem{definition}{Definition}[section]
\newtheorem{theorem}[definition]{Théorème}
\newtheorem{corollary}[definition]{Corollaire}
\newtheorem{proposition}[definition]{Proposition}

\DeclareMathOperator{\cons}{Cons}
\DeclareMathOperator{\findef}{Default}

\newcommand{\defeq}{\vcentcolon=}


\begin{document}

\maketitle

\begin{abstract}
Récement, l'introduction de constructions basées sur des objets particuliers appelés "template" a permis d'améliorer plusieurs bornes inférieures pour les nombres Ramsey multicolores, les nombres de Schur et les nombres de Schur faible. Ce document de travail présente des améliorations apportées au concept de template, notamment une généralisation : les b-templates. Cela conduit à une amélioration des inégalités et bornes inférieures connues certains nombre de Ramsey multicolores et certains nombres de Schur.
\end{abstract}

\section{Organisation de ce document de travail}
La section \ref{sec:def} présente des notations, des définitions, ainsi que des propriétés utiles. La section \ref{sec:temp} présente mon approche et mes résultats sur les templates. elle commence par une explication des différences avec l'approche de Rowley. Elle fait aussi office de revue de l'état de l'art en précisant les résultats de Rowley en fin de section. La section \ref{sec:b-temp} introduit les b-templates, une généralisation des templates qui est utilisée dans la plupart des améliorations présentées. La section \ref{sec:results} compare les nouvelles inégalités et bornes inférieures à celles connues auparavant.

\section{Définitions et propriétés}
\label{sec:def}

Pour \(n \in \mathbb{N}\), on note \(K_n\) le graphe complet d'ordre \(n\).

\begin{definition}[Coloriage de Ramsey]
Soit \(n \in \mathbb{N}\), la taille du graphe, et soit \(r \in \mathbb{N}\), le nombre de couleurs. Soit \((k_c)_{1 \leqslant c \leqslant r} \in {\mathbb{N}^*}^r\). On appelle coloriage de Ramsey tout coloriage des arêtes de \(K_n\) avec \(r\) couleurs tel que pour toute couleur \(c \in [\![1, r]\!]\), le coloriage ne contient pas de sous-graphe complet \(K_{k_c}\) monochromatique en la couleur \(c\). \(R(k_1, ..., k_r ; n)\) désigne l'un (quelconque) de ces coloriages. Si tous les \(k_c\) sont égaux à un \(k\), on utilise également la notation \(R_r(k; n)\).
\end{definition}

\begin{theorem}[Théorème de Ramsey fini \cite{Ramsey}]
\label{thm:ram}
Soit \(r \in \mathbb{N}\) et soit \((k_c)_{1 \leqslant c \leqslant r} \in {\mathbb{N}^*}^r\). Alors il existe un entier \(n \in \mathbb{N}\) tel qu'il n'existe pas de \(R(k_1, ..., k_r ; n)\).
\end{theorem}

Le théorème \ref{thm:ram} justifie la définition suivante.

\begin{definition}[Nombre de Ramsey]
Soit \(r \in \mathbb{N}\) et soit \((k_c)_{1 \leqslant c \leqslant r} \in {\mathbb{N}^*}^r\). On définit le nombre de Ramsey \(R(k_1, ..., k_r)\) comme le plus petit entier \(n \in \mathbb{N}\) tel qu'il n'existe pas de \(R(k_1, ..., k_r ; n)\). Si tous les \(k_c\) sont égaux à un \(k\), ce nombre est également noté \(R_r(k)\).
\end{definition}

On remarque que, contrairement aux nombres de Schur, les nombres de Ramsey désigne le plus petit entier tel qu'il n'existe pas de coloriage, et non pas la taille maximale d'un coloriage.

\begin{definition}[Coloriage de Ramsey linéaire]
Soit \(n \in \mathbb{N}\), et soit \(r \in \mathbb{N}\). Soit \((k_c)_{1 \leqslant c \leqslant r} \in {\mathbb{N}^*}^r\). On indexe les sommets de \(K_n\) avec les entiers de \([\![0, n - 1]\!]\). On appelle coloriage de Ramsey linéaire tout \(R(k_1, ..., k_r ; n)\) pour lequel la couleur de l'arête \((u, v)\) dépend uniquement de \(|u - v|\). Un \(L(k_1, ..., k_r ; n)\) désigne l'un (quelconque) de ces coloriages. Si tous les \(k_c\) sont égaux à un \(k\), on utilise également la notation \(L_r(k; n)\).
\end{definition}

Le théorème \ref{thm:ram} justifie la définition suivante.

\begin{definition}[Nombre de Ramsey linéaire]
Soit \(r \in \mathbb{N}\) et soit \((k_c)_{1 \leqslant c \leqslant r} \in {\mathbb{N}^*}^r\). On définit le nombre de Ramsey linéaire \(L(k_1, ..., k_r)\) comme le plus petit entier \(n \in \mathbb{N}\) tel qu'il n'existe pas de \(L(k_1, ..., k_r ; n)\). Si tous les \(k_c\) sont égaux à un \(k\), ce nombre est également noté \(L_r(k)\).
\end{definition}

\begin{proposition}
\(\forall r \in \mathbb{N}, (k_c)_{1 \leqslant c \leqslant r} \in {\mathbb{N}^*}^r, L(k_1, ..., k_r) \leqslant R(k_1, ..., k_r)\)
\end{proposition}

On remarque qu'un \(L(k_1, ..., k_r ; n)\) induit un coloriage de \([\![1, n - 1]\!]\). Réciproquement, tout coloriage de \([\![1, n - 1]\!]\) correspond à un coloriage linéraire de \(K_n\). Le théorème suivant porte sur le lien entre un coloriage sur \([\![1, n]\!]\) et les nombres de cliques du coloriage linéaire de \(K_{n+1}\) associé.

\begin{theorem}[Lien avec les coloriages sans-sommes \cite{AbbottHanson}]
\label{thm:lien-partitions}
Soit \(r \in \mathbb{N}\) et soit \((k_c)_{1 \leqslant c \leqslant r} \in \mathbb{N}^r\). Soit \(n \in \mathbb{N}\) et soit \(\gamma\) un coloriage de \([\![1, n]\!]\) avec \(r\) couleurs. Alors les deux conditions suivantes sont équivalentes :
\begin{itemize}
\item dans le coloriage linéaire de \(K_{n+1}\) associé à \(\gamma\), la couleur \(c\) contient un sous-graphe complet de taille \(k_c\)
\item il existe un sous-ensemble \(S\) de \([\![1, n]\!]\) monochromatique de couleur \(c\) et de cardinal \(k_c - 1\) vérifiant : pour tous \(x, y \in S\) distincts, \(\gamma(|x - y|) = c\) 
\end{itemize}
\end{theorem}

Dans la suite, on utilise également \(L(k_1, ..., k_r ; n)\) pour désigner un coloriage de \([\![1, n - 1]\!]\)  associé à un \(L(k_1, ..., k_r ; n)\).

Le théorème \ref{thm:lien-partitions} donne le corollaire suivant.

\begin{corollary}[Lien avec les nombres de Schur]
On a \(S(n) = L_n(3) - 2 \leqslant R_n(3) - 2\). 
\end{corollary}

\section{Templates}
\label{sec:temp}

Dans cette section, j'utilise des notations et définitions différentes de celle de Rowley \cite{RowleyRamsey}. Notamment, les templates et constructions sont introduits différement et sont manipulés en termes de partitions et non de graphes. De plus la couleur spéciale est désignée de manière abstraite par un "t" au lieu d'un "3" qui pourrait être confondu avec l'un des \(k_c\). Par ailleurs, je définis un template directement comme une partition pour laquelle la construction fonctionne. Enfin, J'utilise une formulation différente de la condition sur les templates.

\begin{definition}[Permutation partielle]
Pour \(S\) un ensemble quelconque et \(a \in \mathbb{N}\) un entier naturel, on appelle permutation partielle à \(a\) éléments de \(S\) tout a-uplet d'éléments de \(S\) deux à deux distincts. L'ensemble des permutations partielles à \(a\) éléments de \(S\) est noté \(\mathfrak{S}_a(S)\), c'est-à-dire : \(\mathfrak{S}_a(S) = \left\{ (x_i)_{1 \leqslant i \leqslant a} ~|~ \forall (i, j) \in [\![1, a]\!]^2, i \neq j \implies x_i \neq x_j \right\}\). En particulier, \(\mathfrak{S}_2(S)\) désigne l'ensemble des couples d'éléments distincts. Si \(S = [\![1, n]\!]\) pour un \(n \in \mathbb{N}\), \(\mathfrak{S}_a(S)\) est plus simplement noté \(\mathfrak{S}_a^n\).
\end{definition}

Dans cette section, pour un entier \(n\) (en pratique sous-entendu), on note \(\pi\) l'application qui a un entier \(x \in \mathbb{Z}\) associe son unique représentant modulo \(n\) dans \([\![1, n]\!]\).

\begin{definition}[Reformulation des templates]
Soit \(n \in \mathbb{N}\). Soit \(r \in \mathbb{N}\) et soit \((k_c)_{1 \leqslant c \leqslant r} \in \mathbb{N}^r\) avec \(\forall c \in [\![1,r]\!], k_c \geqslant 3\). Un template de taille \(n\) à \(r + 1\) couleurs, noté \(T(k_1, ..., k_r, t; n)\), est défini comme une partition de \([\![1, n]\!]\) en \(r+1\) sous-ensembles \(A_1, ..., A_{r+1}\) vérifiant : 
\begin{itemize}
\item \(n \in A_{r+1}\)
\item pour la couleur spéciale \(r + 1\) : \(\forall (x,y) \in A_{r+1}^2, x + y \notin A_{r+1}\),
\item pour les autres couleurs \(c \in [\![1, r]\!]\) : \(\forall x \in \mathfrak{S}_{k_c - 1}(A_c), \exists (i, j) \in \mathfrak{S}_2^{k_c - 1}, \pi(x_i - x_j) \notin A_c\).
\end{itemize}
\end{definition}

La couleur \(r + 1\)  n'est pas nécessairement la dernière par ordre d'apparition, la désignation de \(r+1\) comme couleur spéciale signalée par \(t\) est une convention qui allège les notations en évitant de devoir écrire \(T(k_1, ..., k_{i - 1}, t,  k_{i + 1}, ..., k_r; n)\).

Dans le cas où tous les \(k_c\) sont égaux à 3, on retouve la définition des S-templates.

\begin{definition}[valeur par défaut]
TODO
\end{definition}

\begin{definition}[Méthode de construction]
Soit \((n, p) \in {\mathbb{N}^*}^2\) les tailles des coloriages. Soit \((r, s) \in {\mathbb{N}^*}^2\). On se donne un coloriage de \([\![1, n]\!]\) à \(r + 1\) couleurs (appelée \(\mathcal{H}\), pour horizontal), et un coloriage de \([\![1, p]\!]\) à \(s\) couleurs (appelée \(\mathcal{V}\), pour vertical).

Pour un \(b \in \mathbb{N}\), on se donne aussi un coloriage dit final \(\mathcal{F}\) de \([\![1, b]\!]\) à \(r + 1\) couleurs. Il est parfois utile de définir \(\mathcal{F}\) au cas par cas, mais on peut aussi lui attribuer une valeur par défaut : \(b = \min \mathcal{H}^{-1}(\{r + 1\}) - 1\) et \(\mathcal{F} = \findef(\mathcal{H}) \defeq \mathcal{H}_{|\,[\![1, b]\!]}\). 

On construit alors un coloriage \(\gamma\) de \([\![1, n \times p + b]\!]\) à \(r + s\) couleurs comme suit. De manière informelle, on répéte \(p\) fois le motif formé l'enchaînement des couleurs dans \(\mathcal{H}\) en remplaçant pour la \(i\)-ème itération la couleur spéciale par \(r + \mathcal{V}(i)\) puis on ajoute à la fin le motif formé l'enchaînement des couleurs dans \(\mathcal{F}\). Formellement :
\[\begin{array}{c c c c}
	\gamma: & [\![1, n \times p + b]\!]  & \longrightarrow &  [\![1, r + s]\!] \\
 	& x & \longmapsto & 
		\left\{\begin{array}{l l}
			\mathcal{H}(\pi(x)) & ~\text{si}~ x \leqslant n \times p ~\text{et}~ \mathcal{H}(\pi(x)) \neq r + 1 \\
			r + \mathcal{V}\left(\left\lceil \dfrac{x}{n} \right\rceil\right) & ~\text{si}~ x \leqslant n \times p ~\text{et}~ \mathcal{H}(\pi(x)) = r + 1 \\
			\mathcal{F}(x - n \times p) & ~\text{si}~ x > n \times p
		\end{array}\right.
\end{array}\]
\end{definition}

\begin{figure}[H]\begin{center}
\renewcommand{\arraystretch}{1.8}
\begin{NiceTabular}{|*{9}{c|}}[corners=SE,standard-cline,hlines]
\CodeBefore
	\cellcolor{red}{1-1}
	\cellcolor{green}{1-2}
	\cellcolor{green}{1-3}
	\cellcolor{red}{1-4}
	\cellcolor{cyan}{1-5}
	\cellcolor{cyan}{1-6}
	\cellcolor{red}{1-7}
	\cellcolor{cyan}{1-8}
	\cellcolor{cyan}{1-9}
	\cellcolor{red}{2-1}
	\cellcolor{green}{2-2}
	\cellcolor{green}{2-3}
	\cellcolor{red}{2-4}
	\cellcolor{yellow}{2-5}
	\cellcolor{yellow}{2-6}
	\cellcolor{red}{2-7}
	\cellcolor{yellow}{2-8}
	\cellcolor{yellow}{2-9}
	\cellcolor{red}{3-1}
	\cellcolor{green}{3-2}
	\cellcolor{green}{3-3}
	\cellcolor{red}{3-4}
	\cellcolor{yellow}{3-5}
	\cellcolor{yellow}{3-6}
	\cellcolor{red}{3-7}
	\cellcolor{yellow}{3-8}
	\cellcolor{yellow}{3-9}
	\cellcolor{red}{4-1}
	\cellcolor{green}{4-2}
	\cellcolor{green}{4-3}
	\cellcolor{red}{4-4}
	\cellcolor{cyan}{4-5}
	\cellcolor{cyan}{4-6}
	\cellcolor{red}{4-7}
	\cellcolor{cyan}{4-8}
	\cellcolor{cyan}{4-9}
	\cellcolor{red}{5-1}
	\cellcolor{green}{5-2}
	\cellcolor{green}{5-3}
	\cellcolor{red}{5-4}
\Body
	1 & 2 & 3 & 4 & 5 & 6 & 7 & 8 & 9 \\
	10 & 11 & 12 & 13 & 14 & 15 & 16 & 17 & 18 \\
	19 & 20 & 21 & 22 & 23 & 24 & 25 & 26 & 27 \\
	28 & 29 & 30 & 31 & 32 & 33 & 34 & 35 & 36 \\
	37 & 38 & 39 & 40 \\
\end{NiceTabular}
\caption{Construction d'un \(L_4(3;40)\) avec \(\mathcal{H} = T(3,3,t;9)\), \(\mathcal{V} = L_2(3;4)\) et \(\mathcal{F} = \findef(\mathcal{H})\)}
\end{center}\end{figure}

\begin{definition}[compatibilité avec un template]
TODO
\end{definition}

\begin{proposition}[compatibilité de la valeur par défaut]
TODO
\end{proposition}

\begin{theorem}[Construction en utilisant les templates]
Soit \(r \in \mathbb{N}\) et soit \((k_c)_{1 \leqslant c \leqslant r} \in \mathbb{N}^r\) avec \(\forall c \in [\![1,r]\!], k_c \geqslant 3\). Soit \(n \in \mathbb{N}^*\). Soit \(\mathcal{H}\) un coloriage de \([\![1,n]\!]\) à \(r + 1\) couleurs. Les deux conditions suivantes sont équivalentes.

\begin{enumerate}[(i)]
\item \(\mathcal{H} \in \mathcal{T}(k_1, ..., k_r, t; n)\)
\item \(\forall s \in \mathbb{N}^*, \forall l \in {[\![3, +\infty[\![}^s, \forall \mathcal{V} \in \mathcal{L}(l_1, ..., l_s ; p), \cons(\mathcal{H}, \mathcal{V}, \varnothing) \in  \mathcal{L}(k_1, ..., k_r, l_1, ..., l_s ; n \times p)\)
\end{enumerate}

On suppose maintenant que \(\mathcal{H} \in \mathcal{T}(k_1, ..., k_r, t; n)\). Soient \(s \in \mathbb{N}^*, l \in [\![3, +\infty[\![^s\) et \(\mathcal{V} \in \mathcal{L}(l_1, ..., l_s ; p)\). Soit \(b \in \mathbb{N}\) et soit \(\mathcal{F}\) un coloriage de \([\![1, b]\!]\) à \(r + 1\) couleurs compatible avec \(\mathcal{H}\). Alors \(\cons(\mathcal{H}, \mathcal{V}, \mathcal{F}) \in  \mathcal{L}(k_1, ..., k_r, l_1, ..., l_s ; n \times p + b)\).
\end{theorem}

Rowley évoque seulement \((i) \implies (ii)\) mais pas \((ii) \implies (i)\) qui montre l'optimalité des conditions dans la définition des templates.

La construction de Rowley comprend uniquement le coloriage finale par défaut, l'introduction du coloriage final variable et la notion de compatibilité sont nouvelles. Cela correspond à une extension aux templates pour les nombres de Ramsey du raffinement sur la dernière ligne pour les S-templates.
\section{b-Templates}
\label{sec:b-temp}

\section{Nouvelles inégalités et bornes inférieures}
\label{sec:results}

\bibliographystyle{IEEEtran}
\bibliography{biblio}

\end{document}