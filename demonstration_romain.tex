\documentclass{article}
\title{New theoretical lower bounds of weak Schur numbers using regular Schur numbers}
\author{Romain Ageron, Paul Casteras, Thibaut Pellerin, Yann Portella}
\usepackage{amsmath}
\usepackage{amssymb}
\usepackage{setspace}
\usepackage[top=1cm]{geometry}
\usepackage{graphicx}
\usepackage[hyphens]{url}
\usepackage{hyperref}
\usepackage[utf8]{inputenc}
\usepackage[english]{babel}

\newtheorem{definition}{Definition}[section]
\newtheorem{notation}[definition]{Notation}
\newtheorem{theorem}{Theorem}[section]
\newtheorem{corollary}{Corollary}[theorem]


\begin{document}
\maketitle
\section{Introduction, context and notations}
We start by defining sum-free and weakly sum-free subsets to introduce regular and weak Schur numbers.
\begin{definition}
A subset $A$ of $\mathbb{N}$ is said to be sum-free when :
\[ \forall (a,b) \in A^2 \text{, } a+b \notin A
\]
\end{definition}
\begin{definition}
A subset $B$ of $\mathbb{N}$ is said to be weakly sum-free when :
\[ \forall (a,b) \in B^2 \text{, } a \neq b \Longrightarrow a+b \notin B
\]
\end{definition}
Let us notice that a sum-free subset is also weakly sum-free, hence justifying the name of \textit{weakly} sum-free subsets. Given $m$ and $n$ two integers, we are interested in partitioning the set of integers from 1 to $m$ in $n$ weakly sum-free subsets.
\begin{notation}
We denote by $[\![1,n]\!]$ the set of integers $\{1, 2, ..., n\}$
\end{notation}
Schur proved in \hyperlink{label1}{\textbf{[1]}} that given a number of subsets $n$, there exists a value of $m$ such that there exists no partition of $[\![1,p]\!]$ in $n$ sum-free subsets for any $p \geqslant m$. This observation leads to the following definition.
\begin{definition}
Let $n \in \mathbb{N}^*$, there exists an integer that we note S(n) (\textit{respectively WS(n)}) such that $[\![1,S(n) + 1]\!]$ (respectively $[\![1,WS(n) + 1]\!]$) can not be partitioned in $n$ sum-free subsets (respectively in $n$ weakly sum-free subsets) while $[\![1,S(n)]\!]$ (respectively $[\![1,WS(n)]\!]$) can be. S(n) is called the n\textsuperscript{th} Schur number and WS(n) the n\textsuperscript{th} weak Schur number.
\end{definition}
The values of both Schur and weak Schur numbers are known only for small values of $n$, and our goal is to improve the lower bounds on weak Schur numbers for $n \geqslant 5$. At the time being, the best lower bounds for weak Schur numbers are established by Rowley in \hyperlink{label2}{\textbf{[2]}}. He greatly improved the former best values with a purely theoretical and constructive approach. More precisely, he managed to build interesting weakly sum-free partitions with sum-free partitions. With this, he found the following lower bounds :
\[ WS(n+1) \geqslant 4S(n) + 2
\]
\[
WS(n+2) \geqslant 13S(n) + 8
\]
Actually, Rowley's construction can be generalised to find a lower bound for $WS(n+m)$ for all $(n,m) \in (\mathbb{N}^*)^2$. This result and its demonstration are detailled in the next part.
\section{A new theoretical lower bound for WS(n+m)}
The following theorem, inspired by Rowley's work, was found and proved by Romain Ageron.
\begin{theorem}
Let $(n,p), (r,s) \in (\mathbb{N}^*)^2$. If there exists a partition of $r$ weakly sum-free subsets of $[\![1,n]\!]$ and a partition of $s$ sum-free subsets of $[\![1,p]\!]$ then there exists a partition of $r+s$ weakly sum-free subsets of $[\![1,p(n+\left \lfloor \frac{n}{2} \right \rfloor + 1)+n]\!]$
\end{theorem}
In particular, if we choose $n = WS(r)$ and $p = WS(s)$ in the last theorem, the next corollary follows.
\begin{corollary}
$ \forall (n,m) \in (\mathbb{N}^*)^2 \text{, } WS(n+m) \geqslant S(m) \left (WS(n) + \left \lfloor \frac{WS(n)}{2} \right \rfloor +1 \right) + WS(n)$
\end{corollary}
\textsc{Proof :} Let $(n,p), (r,s) \in (\mathbb{N}^*)^2$,  $N = p(n+\left \lfloor \frac{n}{2} \right \rfloor + 1)+n$, $\alpha = \left \lfloor \frac{n}{2} \right \rfloor > 0$ and $\beta = n + \alpha + 1$. We denote by $f$ the projection of the equivalence relation induced by the partition of $[\![1,n]\!]$ and $g$ the one induced by the partition of $[\![1,p]\!]$. Each equivalence class is represented by a single integer, therefore :
\[ f : [\![1,n]\!] \longrightarrow [\![1,r]\!] \text{ and } \forall (x,y) \in [\![1,n]\!]^2, \left\{
    \begin{array}{ll}
        x \neq y \\
        f(x) = f(y)
    \end{array}
\right.
\Longrightarrow f(x+y) \neq f(x)
\]
\[g : [\![1,p]\!] \longrightarrow [\![1,s]\!] \text{ and } \forall (x,y) \in [\![1,n]\!]^2 \text{, } f(x) = f(y) \Longrightarrow f(x+y) \neq f(x)
\]
Let us start by parting the integers of $[\![1,N]\!]$ in two subsets $\mathcal{A}$ and $\mathcal{B}$ where $\mathcal{A} = [\![1,\alpha]\!] \cup \{a\beta + u \mid (a,u) \in [\![0,p]\!] \times [\![\alpha + 1,n]\!]\}$ and $\mathcal{B} = \{a\beta + u \mid (a,u) \in [\![1,p]\!] \times [\![-\alpha,\alpha]\!]\}$.\\
\\
First, \underline{$\mathcal{A} \cap \mathcal{B} = \varnothing$} : \\
Suppose there exists $x \in \mathcal{A} \cap \mathcal{B} = \varnothing$. Then there are $(a,u) \in [\![0,p]\!] \times [\![\alpha + 1,n]\!]$ and $(b,v) \in [\![1,p]\!] \times [\![-\alpha,\alpha]\!]$ such that $x = a\beta + u = b\beta +v$. By definition of $\alpha$ and  $\beta$ we have $u \in [\![\alpha + 1,n]\!] \subset [\![0,\beta - 1]\!]$. From there, we distinguish two cases :
\begin{itemize}
\item If $v \in [\![0,\alpha]\!]$ then $v \in [\![0,\beta - 1]\!]$ and $v \neq u$ because $v < \alpha + 1 \leqslant u$
\item If $v \in [\![-\alpha,-1]\!]$, we note $\tilde{v} = \beta + v$ and thus have $x = (b-1)\beta + \tilde{v}$ with $\tilde{v} \in [\![\beta - \alpha,\beta - 1]\!] \subset [\![0,\beta - 1]\!]$ and $\tilde{v} \neq u$ because $u < n+1 = \beta - \alpha \leqslant \tilde{v}$.
\end{itemize}
In either cases, we run into a contradiction because of the remainder's uniqueness in the euclidean division of $x$ by $\beta$.\\
\\
Then, we have \underline{$\mathcal{A} \cup \mathcal{B} = [\![1,N]\!]$}:
\begin{itemize}
\item On one hand : $1 = \text{min($\mathcal{A}$)} \leqslant \text{max($\mathcal{A}$)} = p\beta + n = N$ and $1 \leqslant \beta - \alpha = \text{min($\mathcal{B}$)} \leqslant \text{max($\mathcal{B}$)} = p\beta + \alpha \leqslant N$, which gives $\mathcal{A} \cup \mathcal{B} \subset [\![1,N]\!]$.
\item On the other hand, let $x \in  [\![1,N]\!]$. If $x \leqslant \alpha$, we directly have $x \in \mathcal{A}$, let us then suppose that $x > \alpha$ and write $x = a\beta + u$ the euclidean division of $x$ by $\beta$. We have $x \leqslant N$, thus $a \leqslant p$. We distinguish three cases : \\
- If $u \in [\![0,\alpha]\!]$ then we necessarily have $a \geqslant 1$ because $x > \alpha$, and so $x \in \mathcal{B}$.\\
- If $u \in [\![\alpha + 1,n]\!]$, then $x \in \mathcal{A}$. \\
- If $u \in [\![n + 1,\beta - 1]\!]$ then $x = (a+1)\beta - (\beta - u)$ with $-\alpha \leqslant \beta - u \leqslant 0$. Furthermore, $a \leqslant p - 1$, else we would have $x > N$, and so $x \in \mathcal{B}$ \\
In any case, $x \in \mathcal{A} \cup \mathcal{B}$ and we can thus conclude that $[\![1,N]\!] \subset \mathcal{A} \cup \mathcal{B}$.
\end{itemize}
This first partition of $[\![1,N]\!]$ will help us to define our final partition by the projection of its equivalence relation. We thereby define $h : [\![1,N]\!] \longrightarrow [\![1,r+s]\!]$ as such :\\
- If $x \in \mathcal{A}$ then $h(x) = f(x \text{ mod } \beta)$ (well defined because $x \text{ mod } \beta \in [\![1,N]\!]$)\\
- If $x \in \mathcal{B}$ then $x = a\beta + u$ with a unique $(a,u) \in [\![1,p]\!] \times [\![-\alpha,\alpha]\!]$ and we define $h(x) = r + g(a)$\\
The fact that $(\mathcal{A}, \mathcal{B})$ is a partition of $[\![1,N]\!]$ ensures that this definition of $h$ is valid. We then have to verify that $h$ induces weakly sum-free subsets.\\
\\
\underline{The classes of equivalence $h(x)$ for $x \in \mathcal{A}$ are weakly sum-free :}
\\
\\
Let $(x,y) \in \mathcal{A}^2$ such that $h(x) = h(y)$, $x \neq y$ and $x + y \leqslant N$
\begin{itemize}
\item If $(x,y) \in [\![1,\alpha]\!]^2$ :\\
We have $x + y \leqslant 2\alpha \leqslant n$ and $x + y = 0\beta + x + y$, therefore $x + y \in \mathcal{A}$. Then, by definition : $h(x) = f(x)$, $h(y) = f(y)$ and $h(x+y) = f(x+y)$, which gives us, thanks to the property verified by $f$, that $h(x+y) \neq h(x)$.
\item If $(x,y) \in [\![1,\alpha]\!] \times ( \mathcal{A} \text{ \textbackslash} \text{ } [\![1,\alpha]\!] )$ :\\
We write $y = a\beta + u$ with $(a,u) \in [\![0,p]\!] \times [\![\alpha + 1,n]\!]$. Then $x+y = a\beta + x + u = (a+1)\beta + x + u - \beta$, and if $x + u > n$ it follows that $a \leqslant p-1$ since $x+y \leqslant N$, and $-\alpha \leqslant x + u - \beta \leqslant -1$. Therefore $x+y \in \mathcal{B}$ and $h(x+y) \neq h(x) = f(x)$ by definition of h. On the contrary, if $x - u \leqslant n$, then $x+y \in \mathcal{A}$ and $h(x+y) = f(x+u)$ because $x+u$ is actually the remainder of the euclidean division of $x+y$ by $\beta$. Moreover, $h(x) = f(x)$, $x < u$ and, with our initial hypothesis, $h(x) = h(y) = f(u)$. The property verified by $f$ gives us $f(x+u) \neq f(x)$ which can be rewritten as $h(x+y) \neq h(x)$.
\item If $(x,y) \in ( \mathcal{A} \text{ \textbackslash} \text{ } [\![1,\alpha]\!] ) \times [\![1,\alpha]\!]$ : \\
This case is handled exactly like the previous one by swaping the roles of $x$ and $y$.
\item If $(x,y) \in ( \mathcal{A} \text{ \textbackslash} \text{ } [\![1,\alpha]\!] )^2$ : \\
We write $x = a\beta + u$ and $y = b\beta + v$ with $(a,u)$ and $(b,v)$ in $[\![0,p]\!] \times [\![\alpha + 1,n]\!]$. Then $x+y = (a+b)\beta + u+v = (a+b+1)\beta + u + v - \beta$ with $a+b \leqslant p-1$ (else we would have $x+y > N$ because $u+v > n$) and $-\alpha \leqslant u + v - \beta \leqslant \alpha$, therefore $x+y \in \mathcal{B}$ and by definition $h(x+y) \neq h(x)$.
\end{itemize}
In any case, $h(x+y) \neq h(x)$ and the classes of equivalence $h(x)$ for $x \in \mathcal{A}$ are weakly sum-free.\\
\\
\underline{The classes of equivalence $h(x)$ for $x \in \mathcal{B}$ are weakly sum-free :}
\\
\\
Let $(x,y) \in \mathcal{B}^2$ such that $h(x) = h(y)$, $x \neq y$ and $x + y \leqslant N$.\\
We write $x = a\beta + u$ and $y = b\beta + v$ with $(a,u)$ and $(b,v)$ in $[\![1,p]\!] \times [\![-\alpha,\alpha]\!]$. We have $h(x) = r + g(a)$ and $h(y) = r + g(b)$, therefore $g(a) = g(b)$. We also have $x+y = (a+b)\beta + u +v$.\\
If $u + v \in [\![-\alpha,\alpha]\!]$, then $x+y \in \mathcal{B}$ and $h(x+y) = g(a+b)$, hence we can deduce that $h(x+y) \neq h(x)$ because of the property verified by $g$. On the contrary, if $u+v \notin [\![-\alpha,\alpha]\!]$, then necessarily $x+y \in \mathcal{A}$. Suppose $x+y \in \mathcal{B}$, then $x+y = c\beta + w$ with $(c,w) \in [\![1,p]\!] \times [\![-\alpha,\alpha]\!]$. Thus, $c\beta + w = (a+b)\beta + u + v$ and $(a+b-c)\beta = w-u-v$. Furthermore $a+b-c \neq 0$, else we would have $u+v = w \in [\![-\alpha,\alpha]\!]$. This finally leads to the following inequality :
\[\beta \leqslant |a+b-c|\beta = |w-u-v| \leqslant |w| + |u| + |v| \leqslant 3\alpha \leqslant n + \alpha < \beta
\]
which is absurd. We can therefore conclude that $x+y \in \mathcal{A}$ and by definition of $h$, $h(x+y) \neq h(x)$, proving that the classes of equivalence $h(x)$ for $x \in \mathcal{B}$ are weakly sum-free.\\
\\
Finally, we have showed that every classe of equivalence induced by $h$ is weakly sum-free, which ends the proof.
\section{References}
\hypertarget{label1}{\textbf{[1]}} Schur, I.. "Über Kongruenz x ... (mod. p.).." Jahresbericht der Deutschen Mathematiker-Vereinigung 25 (1917): 114-116. \url{http://eudml.org/doc/145475} \\
\hypertarget{label2}{\textbf{[2]}} F. Rowley, New Lower Bounds for Weak Schur Partitions, arXiv 2011.11292, \url{https://arxiv.org/pdf/2011.11292.pdf}
\end{document}
