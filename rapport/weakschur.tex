\section{Templates for weak Schur numbers}
\label{WeakSchur}

\qquad In this section, we generalize Rowley's constructions for weak Schur numbers \cite{RowleyWS} and give an analogous
for weak Schur numbers of Abbott and Hanson's construction for Schur numbers. By analogy with the previous section,
we then introduce WSF-templates as well as an associated sequence \(\WS^+(n)\). We find templates and use them to
provide new lower bounds for weak Schur numbers.

\subsection{Inequality for weak Schur numbers using Schur and weak Schur numbers}

\qquad Up to now, no equivalent for weak Schur numbers of Abbott and Hanson's construction for Schur numbers
\cite{AbbottHanson} was known. Here we give a general lower bound for weak Schur numbers as a function of both 
regular and weak Schur numbers. The following theorem, inspired by Rowley's inequalities for \(\WS (n+1)\) and 
\(\WS (n+2)\), was found and proved by Romain Ageron.

\begin{theorem}
\label{theorem}
\begin{sloppypar}
Let \((p,k), (q,n) \in (\mathbb{N}^*)^2\). If there exists a partition of \([\![1,q]\!]\) into \(n\) weakly sum-free
subsets and a partition of \([\![1,p]\!]\) into \(k\) sum-free
subsets then there exists a partition of \({[\![1,p(q+\left \lceil \frac{q}{2} \right \rceil + 1)+q]\!]}\) into \(n+k\)
weakly sum-free subsets.
\end{sloppypar}
\end{theorem}

In particular, by setting \(q = \WS (n)\) and \(p = S(k)\) in the above theorem, one obtains the following corollary.

\begin{corollary}
\( \forall (n,k) \in (\mathbb{N}^*)^2 \text{, } \WS (n+k) \geqslant S(k) \left (\WS (n) + \left \lceil \displaystyle \frac{\WS (n)}{2}
\right \rceil +1 \right) + \WS (n)\)
\end{corollary}

This can be seen as an equivalent for weak Schur numbers of Abott and Hanson's construction for Schur numbers. This formula includes
the results of Rowley \cite{RowleyWS} as a special case. For \(n>2\), this formula does not give new lower bounds.

\begin{remark}
The above inequality can be improved by adding 1 to the lower bound if \(\WS (n)\) is odd (more generally if \(q\) is
odd in the theorem). However, it makes the proof less clear and it is never useful in practice.
\end{remark}

Given that this theorem will appear as a particular case of a more general theorem after the introduction of
templates for weak Schur numbers, we only give here an intuitive explanation of the above theorem; a formal
\hyperref[proof_theorem]{proof} can be found in the appendix.

Let \((p, k), (q, n) \in (\mathbb{N}^*)^2\) such that there exists a partition of \([\![1,q]\!]\) into \(n\) weakly sum-free
subsets and a partition of \([\![1,p]\!]\) into \(k\) sum-free subsets. Let \(a \in \mathbb{N}\) with \(a > q\)
and let's try to build a coloring of \([\![1, ap + q]\!]\) into \(n + k\) weakly sum-free subsets. Let
\(l = a - b - 1\), \(r \in [\![1,q]\!]\) and \(w = a - l - r - 1 = b - r\).

First, put the integers of \([\![1, ap + q]\!]\) in the following table (with \(a\) columns and \(p + 1\) lines) 
and divide it into 3 blocks (the columns are numbered from \(-l\) to \(+q\)).

\begin{itemize}
	\item \(\mathcal{T}\) (the "tail"): the integers from 1 to \(q\). NB: this is line number 0.
	\item \(\mathcal{R}\) (the "rows"): the integers in columns \(-l\) to \(+r\) (excluding  \(\mathcal{T}\)).
	\item \(\mathcal{C}\) (the "columns"): the integers in the last \(w\) columns (excluding  \(\mathcal{T}\)).
\end{itemize}

Like SF-templates, \(\mathcal{R}\) and \(\mathcal{C}\) play the role of security zones for each other. Note that with
this numbering of columns, the column of the sum of two numbers is the only integer in \([\![-l,q]\!]\) equal to two the
sum of the columns modulo \(a\).

\renewcommand{\arraystretch}{1.7}
\setlength{\arraycolsep}{3pt}

\[
\begin{NiceArray}{*{13}{c}}[corners={NW,SW},hvlines,first-row,last-row,first-col]
\CodeBefore
	\cellcolor{green}{1-6}
	\cellcolor{green}{1-7}
	\cellcolor{cyan}{1-8}
	\cellcolor{green}{1-9}
	\cellcolor{cyan}{1-10}
	\cellcolor{cyan}{1-11}
	\cellcolor{cyan}{1-12}
	\cellcolor{green}{1-13}
	\cellcolor{red}{2-1}
	\cellcolor{red}{2-2}
	\cellcolor{red}{2-3}
	\cellcolor{red}{2-4}
	\cellcolor{red}{2-5}
	\cellcolor{red}{2-6}
	\cellcolor{red}{2-7}
	\cellcolor{red}{2-8}
	\cellcolor{red}{2-9}
	\cellcolor{cyan}{2-10}
	\cellcolor{cyan}{2-11}
	\cellcolor{cyan}{2-12}
	\cellcolor{green}{2-13}
	\cellcolor{yellow}{3-1}
	\cellcolor{yellow}{3-2}
	\cellcolor{yellow}{3-3}
	\cellcolor{yellow}{3-4}
	\cellcolor{yellow}{3-5}
	\cellcolor{yellow}{3-6}
	\cellcolor{yellow}{3-7}
	\cellcolor{yellow}{3-8}
	\cellcolor{yellow}{3-9}
	\cellcolor{cyan}{3-10}
	\cellcolor{cyan}{3-11}
	\cellcolor{cyan}{3-12}
	\cellcolor{green}{3-13}
	\cellcolor{yellow}{4-1}
	\cellcolor{yellow}{4-2}
	\cellcolor{yellow}{4-3}
	\cellcolor{yellow}{4-4}
	\cellcolor{yellow}{4-5}
	\cellcolor{yellow}{4-6}
	\cellcolor{yellow}{4-7}
	\cellcolor{yellow}{4-8}
	\cellcolor{yellow}{4-9}
	\cellcolor{cyan}{4-10}
	\cellcolor{cyan}{4-11}
	\cellcolor{cyan}{4-12}
	\cellcolor{green}{4-13}
	\cellcolor{red}{5-1}
	\cellcolor{red}{5-2}
	\cellcolor{red}{5-3}
	\cellcolor{red}{5-4}
	\cellcolor{red}{5-5}
	\cellcolor{red}{5-6}
	\cellcolor{red}{5-7}
	\cellcolor{red}{5-8}
	\cellcolor{red}{5-9}
	\cellcolor{cyan}{5-10}
	\cellcolor{cyan}{5-11}
	\cellcolor{cyan}{5-12}
	\cellcolor{green}{5-13}
	\cellcolor{cyan}{6-10}
	\cellcolor{cyan}{6-11}
	\cellcolor{cyan}{6-12}
	\cellcolor{green}{6-13}
\Body
	& & & & & & \Block{1-8}{\overbrace{\hphantom{---------------------------}}^{\mathcal{T}}} \\
	& & & & & & 1 & 2 & ... & r & r + 1 & ... & b - 1 & b \\
	\Block{5-1}{\mathcal{R} \left\{ \vphantom{\begin{array}{l} . \\ . \\ . \\ . \\ . \\ . \\ . \\ . \\ . \end{array}} \right.} 
	& a - l & a - l + 1 & ... & a - 1 & a & a + 1 & ... & a + r - 1 & a + r & a + r + 1 & ... & a + b - 1 & a + b \\
	& 2 a - l & ... & ... & ... & 2 a & ... & ... & ... & 2 a + r & ... & ... & ... & 2 a + b \\
	& ... & ... & ... & ... & ... & ... & ... & ... & ... & ... & ... & ... & ... \\
	& ... & ... & ... & ... & ... & ... & ... & ... & ... & ... & ... & ... & ... \\
	& p a - l & ... & ... & ... & p a & ... & ... & ... & p a + r & ... & ... & ... & p a + b \\
	& & & & & & & & & & \Block{1-4}{\underbrace{\hphantom{--------------}}_{\mathcal{C}}} \\
\end{NiceArray}
\]

\resetarraystretch
\setlength{\arraycolsep}{6pt}

\noindent \underline{\textbf{\(\mathcal{T}\) block}}
\par
We color this block using the weakly sum-free coloring of \([\![1,q]\!]\) with colors \(1, ..., n\).

\noindent \underline{\textbf{\(\mathcal{R}\) block}}
\par
In this block,  we use the colors \(n + 1, ..., n + k\). We color an integer \(x\) according to its line number (written \(\lambda(x)\)).
For every \(x \in \mathcal{R}\), we color \(x\) with \(n + c\) where \(c\) is the color of \(\lambda(x)\) in the sum-free coloring of  \([\![1,p]\!]\).
Let \((x, y) \in \mathcal{R}^2\). The cases are twofold.

\begin{itemize}
	\item \underline{\(\lambda(x+y) = \lambda(x) + \lambda(y)\)} \\
	In this case, we use the sum-free property of the coloring of \([\![1,p]\!]\) (in block \(\mathcal{C}\), we only 
	use colors \(1, ..., n\)).
	\item \underline{\(\lambda(x+y) \neq \lambda(x) + \lambda(y)\)} \\
	In this case, we do not have information about the color of \(\lambda(x+y)\). Thereby, we want to have 
	\(x+y \in \mathcal{C}\). A simple solution is to limit the horizontal movement, that is if the sum changes line 
	(that is its line number is different from \(\lambda(x) + \lambda(y)\)), not to move too far from \((\lambda(x) 
	+ \lambda(y)) a\) so that the sum stays in \(\mathcal{C}\). There, the maximal displacement to the left (resp. 
	to the right) is \(2l\) (resp. \(2r\)). Not crossing entirely \(\mathcal{C}\) by going to the left is then expressed 
	as \(-2l > -a + r\). Likewise, not going to far to the right is expressed as \(2r < a - l\). It can then be written 
	as \(\max(l, r) \leqslant w\).
\end{itemize}

\noindent \underline{\textbf{\(\mathcal{C}\) block}}
\par
In this block,  we use colors \(1, ..., n\). We color an integer \(x\) according to its column number, denoted by \(\tilde{\pi}(x)\). It is linked to the
projection on the first line, denoted by \(\pi\), by the relation \(\tilde{\pi}(x) = \pi(x) - a\). A simple solution is to color \(x\) with the same color
as \(\tilde{\pi}(x)\) in the weakly sum-free coloring of \([\![1,q]\!]\). As long as \(2b \leqslant a + r\) (not going two far to the right) and there
is no \(x \in \tilde{\pi}(\mathcal{C})\) such that \(2x \in \tilde{\pi}(\mathcal{C})\) (so that we do not have a sum in \(\mathcal{C}\) when
taking two numbers in the same column), the colors \(1, ..., n\) are sum-free.

In particular, taking \(w = l = \left\lceil \displaystyle \frac{q}{2} \right\rceil\) and \(r = \left\lfloor \displaystyle \frac{q}{2} \right\rfloor\) works, thus obtaining the above theorem.\\
\par
As in the previous section, we now introduce WSF-templates and the sequence \(\WS^+\) in order to generalize the above construction.

\subsection{Definition of \(\WS^+\)}

\qquad In this subsection, we introduce WSF-templates and prove calculative results for the general construction 
theorem on templates for weak Schur numbers.

\begin{definition}
Let \((a,b) \in (\mathbb{N}^*)^2\) such that \(a>b\). Define
\[ \pi_{a,b}:x \longmapsto (\operatorname{Id}+a\mathds{1}_{ [\![0,b]\!]})(x \mod a)\]
\end{definition}

If there is no confusion on the \(a\) and \(b\) to use, \(\pi_{a, b}\) is denoted by \(\pi\). Notice that 
\({\forall x \in \mathbb{Z}, \pi(x) = x \mod a}\) and \({\forall x \in [\![b + 1, a + b]\!], b + 1 \leqslant \pi(x) \leqslant a + b}\).

\begin{sloppypar}
\(\pi\) is the projection on the first line mentioned in the intuitive explanation. 
\end{sloppypar}

\begin{proposition}
\label{prop1}
\[
\forall x \in [\![b + 1, a + b]\!], \pi(x) = x
\]
\end{proposition}

\begin{proof}[\textsc{Proof.}]
\begin{sloppypar}
Let \(x \in [\![b + 1, a + b]\!]\). If \(x < a\) then \(x \mod a = x \notin [\![1, b]\!]\). Hence \(\pi(x) = x\). 
Otherwise, \({x \mod a = x - a \in [\![1, b]\!]}\). Hence \({\pi(x) = x - a + a = x}\).
\end{sloppypar}
\end{proof}

\begin{proposition}
\label{prop2}
Let x \(\in [\![1,b]\!]\) and \(y \in \mathbb{N}^*\). Then 
\[
\pi(x+\pi(y)) = \pi(x+y)
\]
\end{proposition}

\begin{proof}[\textsc{Proof.}]
It is a direct consequence of \(\pi(x) = x \mod a\). \\
\end{proof}

\begin{proposition}
\label{prop3}
Let x \(\in [\![1,b]\!]\) and \(y \in \mathbb{N}^*\) such that \(x+\pi(y) \leqslant a+b\). Then 
\[
\pi(x+y)=x+\pi(y)
\]
\end{proposition}

\begin{proof}[\textsc{Proof.}]
\(\pi(y) \geqslant b + 1\) and thus \(b + 1 \leqslant x + \pi(y) \leqslant a + b\).
\[
\begin{array}{l l l l}
	\pi(x + y) & = & \pi(x + \pi(y)) & proposition \ref{prop2} \\
	 & = & x + \pi(y) & proposition \ref{prop1}
\end{array}
\]
\end{proof}

\begin{proposition}
\label{prop4}
Let \((x,y)\in (\mathbb{N}^*)^2\). Then 
\[
\pi(\pi(x)+\pi(y))=\pi(x+y)
\]
\end{proposition}

\begin{proof}[\textsc{Proof.}]
It is a direct consequence of \(\pi(x) = x \mod a\). \\
\end{proof}

\begin{definition}
Let \((a,b) \in (\mathbb{N}^*)^2\) such that \(a>b\). Define
\[ \lambda_{a,b}:x \longmapsto 1+ \left\lfloor\dfrac{x-b-1}{a}\right\rfloor\]
\end{definition}

If there is no confusion on the \(a\) and \(b\) to use, \(\lambda_{a, b}\) is denoted by \(\lambda\).

\(\lambda\) is the function which maps an element \(x\) to its line number as mentioned in the intuitive explanation.

\begin{proposition}
\label{prop5}
Let \(x\in \mathbb{N}^*\). Then 
\[
x=a\lambda(x)+\pi(x)-a
\]
\end{proposition}

\begin{proof}[\textsc{Proof.}]
Let \((a,b)\in (\mathbb{N}^*)^2\) such that \(a>b\) and let \(x\in \mathbb{N}^*\). \\
\(a\lambda(x)+\pi(x)-a=a\left\lfloor\dfrac{x-b-1}{a}\right\rfloor+(x \mod a)+ a \mathds{1}_{ [\![0,b]\!]}(x \mod a)\)

\noindent  if \(x \mod a>b\) then \(a\lambda(x)+\pi(x)-a=a\left\lfloor\dfrac{x}{a}\right\rfloor+x \mod a=x\) \\
\vspace{1mm}
\noindent if \(x \mod a \leqslant b\) then
\(a\lambda(x)+\pi(x)-a=a \left( \left \lfloor \dfrac{x}{a} \right \rfloor - 1 \right)+x \mod a +a=x\) \\
\end{proof}

\begin{proposition}
\label{prop6}
Let \(x, y \in \mathbb{Z}\) such that \(\lambda(x + y) = \lambda(y)\). Then 
\[
\pi(x + y) = x + \pi(y)
\]
\end{proposition}

\begin{proof}[\textsc{Proof.}]
By applying twice proposition \ref{prop5}, \(a \lambda(x + y) + \pi(x + y) - a = x + y = x + a \lambda(y) + \pi(y) - a\). 
The result is then obtained by symplying the equality. \\
\end{proof}

\begin{proposition}
\label{prop7}
Let \(x, y \in \mathbb{Z}\) such that \(\pi(x) + \pi(y) \in [\![a + b + 1, 2 a + b]\!]\). Then 
\[
\lambda(x + y) = \lambda(x) + \lambda(y)
\]
\end{proposition}

\begin{proof}[\textsc{Proof.}]
By proposition \ref{prop5}, \(x + y = a (\lambda(x) + \lambda(y)) + \pi(x) + \pi(y) - 2 a\). Then
\[
\begin{array}{l l l}
	\lambda(x + y) & = & \left \lfloor \dfrac{x + y - b - 1}{a} \right \rfloor + 1 \vspace{1mm} \\
	 & = & \left \lfloor \dfrac{a (\lambda(x) + \lambda(y)) + \pi(x) + \pi(y) - 2 a - b - 1}{a} \right \rfloor + 1 \vspace{1mm} \\
	 & = & \lambda(x) + \lambda(y) - 1 \left \lfloor \dfrac{\pi(x) + \pi(y) - b - 1}{a}\right\rfloor \\
	 & = & \lambda(x)+\lambda(y) -1 +1 ~~~~\text{since } \pi(x) + \pi(y) \in [\![a + b + 1, 2 a + b]\!] \\
	 & =& \lambda(x)+\lambda(y)
\end{array}
\]
\end{proof}

\begin{definition}
Let \( (a,n,b) \in (\mathbb{N}^*)^3\) with \(a > b\). Let \((A_1,...,A_n)\) a partition of  \([\![1, a + b]\!]\).
This partition is said to be a \(b\)-weakly-sum-free template (\(b\)-WSF-template) with width \(a\) and \(n\) colors if

\begin{itemize}
\item \(\forall i \in [\![1, n]\!], A_i\) is weakly-sum-free
\item \(\forall i \in [\![1, n]\!], A_i\backslash [\![1, b]\!]\) is sum-free
\item For \(A_n\) (the special subset)
	\[
	\forall (x,y) \in A_n^2, \,x+y>b+2a \implies x+y-2a\notin A_n
	\]
\item For the others subsets
	\[
	\forall i \in [\![0,n-1]\!], \, \forall(x,y) \in A_i^2, \, x+y>a+b \implies \pi(x+y) \notin A_i
	\]
\end{itemize}
\end{definition}

Please note that the special color \(n\) is not necessarily the last color by order of appearance.

\begin{definition}
Let \( (n,b) \in (\mathbb{N}^*)^2\). If there exists \(a\) such that there exists a \(b\)-WSF-template with width \(a\) 
and \(n\) colors, define
\[
\WS_b^+(n)= \max \{a \in \mathbb{N}^*/ \text{there exists a } b \text{-WSF-template with width } a \text{ and } n \text{ colors} \}
\]
If no such \(a\) exists, set \(\WS_b^+(n) = 0\).
\end{definition}

\begin{definition}
Let \( n \in \mathbb{N}^*\). Define 
\[
\WS^+(n)=\max_{b\in \mathbb{N}^*} \WS_b^+(n)
\]
\end{definition}

\begin{proposition}
Let \(n \in [\![2, +\infty]\!]\). Then
\[
\frac{3}{2} \WS (n-1)+1 \leqslant \WS^+(n) \leqslant \WS (n)
\]
\end{proposition}

\begin{proof}[\textsc{Proof.}]
The lower bound comes from the analogous of Abott and Hanson's construction for weak Schur numbers.
The upper bound comes from the fact that a WSF-template with width \(a\) and \(n\) colors contains a partition of
\([\![1, a]\!]\) into \(n\) sum-free subsets. \\
\end{proof}

\begin{remark}
\(\WS^+\) and \(\WS\) have the same asymptotic growth rate.
\end{remark}

We now proceed to state and prove the main result of this article.


\subsection{Construction of weak Schur partitions using WSF-templates}

\begin{theorem}
Let \((a,n,b) \in (\mathbb{N}^*)^3\) with \(a > b\) and \( (p,k) \in (\mathbb{N}^*)^2\). If there exists a partition of \([\![1,p]\!]\) 
into \(k\) sum-free subsets and a \(b\)-WSF-template \((A_1,...,A_{n+1})\) with width \(a\) and \(n+1\) colors, 
then there exists a partition of \([\![1, p a + b]\!]\) into \(k+n\) weakly sum-free subsets.
\end{theorem}

In particular, by setting \(p = S(k)\) and \(a = \WS^+(n + 1)\) in the last theorem, the next corollary follows.

\begin{corollary}
Let \(n,k \in \mathbb{N}^*\) and set \( b_{max} = \max \{b\in \mathbb{N}^*/ \WS_b^+(n+1) = \WS^+(n+1)\}\).
Then
\[
\WS(n+k) \geqslant S(k) \WS^+(n+1)+b_{max}
\]
\end{corollary}

\begin{remark}
In the SF-template construction for Schur numbers, the additive constant comes from the fact that the special color does
not necessarily appear right at the begining of the repeating pattern. Likewise, \(b_{max}\) can actually be replaced by
\[
\max_{b \in \mathbb{N}^*} \left\{\min (A_{n+1} \backslash [\![1, b ]\!]) - 1~|~ \WS_b^+(n+1) = \WS^+(n+1) \right\}
\]
\end{remark}

\begin{proof}[\textsc{Proof.}]
\begin{sloppypar}
Let \((a,n,b) \in (\mathbb{N}^*)^3\) and \((p,k) \in (\mathbb{N}^*)^2\). Denote by \(f\) the coloring 
associated to the \(b\)-WSF-template  and \(g\) the one associated to the sum-free partition of \([\![1,p]\!]\); where 
\({f : [\![1, a + b]\!] \longrightarrow [\![1,n+1]\!]}\) and \({g : [\![1, p]\!]  \longrightarrow [\![1, k]\!]}\). Moreover, assume 
that the sum-free coloring of \([\![1, p]\!]\) is ordered.
\end{sloppypar}

\par
NB: In the following 5 predicates, the conditions \(x + y \leqslant p\)  and \(x + y \leqslant a + b\) are omitted for readability.
\par
The (weakly) sum-free conditions are expressed as 
\[
\forall (x,y) \in [\![1,a + b]\!]^2, \left\{
\begin{array}{l}
	f(x) = f(y) \\
	x \neq y
\end{array}
\right. \Longrightarrow f(x+y) \neq f(x)
\]
\[
\forall (x,y) \in [\![b+1,a + b]\!]^2, f(x) = f(y) \Longrightarrow f(x+y) \neq f(x)
\]
\[
\forall (x,y) \in [\![1,p]\!]^2, g(x) = g(y) \Longrightarrow g(x+y) \neq g(x)
\]

The additionnal constraints for the WSF-template are
\[
\forall (x,y) \in [\![1,a + b]\!]^2, \left\{
\begin{array}{l}
	f(x) = f(y) \leqslant n \\
	x + y > a + b
\end{array}
\right. \Longrightarrow f(\pi(x+y)) \neq f(x)
\]
\[
\forall (x,y) \in [\![1,a + b]\!]^2, \left\{
\begin{array}{l}
	f(x) = f(y) = n + 1 \\
	x + y > 2 a + b
\end{array}
\right. \Longrightarrow f(x+y - 2 a) \neq f(x)
\]

Here, we consider \(\pi\) defined in the previous subsection as an application defined on \([\![b + 1, p a + b]\!]\). 
Split \([\![1, p a + b]\!]\) into 3 subsets.

\begin{itemize}
	\item \(\mathcal{T} = [\![1, b]\!]\)
	\item \(\mathcal{C} = \pi^{-1}(f^{-1}([\![1, n]\!]))\)
	\item \(\mathcal{R} = \pi^{-1}(f^{-1}(\{n + 1\}))\)
\end{itemize}

Define a new coloring \(h\) as follows
\[
\begin{array}{c c c l}
	h : & [\![1, p a + b]\!] & \longrightarrow & [\![1,n+k]\!] \\
	& x & \longmapsto & 
	\left\{ \begin{array}{l l}
		f(x) & \text{if}~x \in \mathcal{T} \\
		f(\pi(x)) & \text{if}~x \in \mathcal{C} \\
		n + g(\lambda(x)) & \text{if}~x \in \mathcal{R}
	\end{array} \right.
\end{array}
\]

\(h\) is well defined since \((\mathcal{T}, \mathcal{C}, \mathcal{R})\) is a partition of  \([\![1, p a + b]\!]\). 
We now prove that \(h\) is a weakly sum-free coloring. Let \(x,y \in [\![1, p a + b]\!]\) such that \(x \neq y\), 
\(h(x) = h(y)\) and \(x+y \leqslant p a+ b\). We claim that \(h(x+y) \neq h(x)\). Distinguish between six cases 
according to the subsets \((\mathcal{T}, \mathcal{C}, \mathcal{R})\) to which \(x\) and \(y\) belong. It is 
sufficient to check only six cases out of nine since \(x\) and \(y\) play symmetric roles. \\

\noindent \underline{\textbf{Case 1:} \((x,y) \in \mathcal{T}^2\)}
\par
If \(x + y \leqslant b\) then \(h(x+y)=f(x+y)\). Otherwise, \(b < x+y < a+b\) since \(b < a\) and 
therefore \(\pi(x + y) = x +y\) (proposition \ref{prop1}). Hence in both cases \(h(x+y)=f(x+y)\). Given that 
\(f\) is a weakly sum-free coloring, \(f(x + y) \neq f(x)\) since \(f(x)=h(x)=h(y)=f(y)\) and \(x \neq y \). That 
is \(h(x + y) \neq h(x)\). \\

\noindent \underline{\textbf{Case 2:} \((x,y) \in \mathcal{T} \times \mathcal{C}\)}
\par
Given that \(h(x) = h(y)\) and by definition of \(h\), \(f(x) = f(\pi(y))\). Besides, \(f(\pi(y)) \leqslant n\) since 
\(y \in \mathcal{C}\). Distinguish between two cases according to the value of \(x + \pi(y)\).
\begin{itemize}
\item \begin{sloppypar}
	If \(x + \pi(y) \leqslant a + b\) then \(f(x + \pi(y)) = f(\pi(x + y))\) (proposition \ref{prop3}). Given that \(f\) is 
	a weakly sum-free coloring, \(f(x+\pi(y)) \neq f(x)\) since \(f(x)=f(\pi(y))\) and \(x \neq \pi(y)\) since
	\({x \leqslant b < \pi(y)}\).
	\end{sloppypar}
\item \begin{sloppypar}
	If \(x+\pi(y)> a+b\) then given that \(f\) is a WSF-template and since \({f(x) = f(\pi(y)) \leqslant n}\), 
	\({f(\pi(x+\pi(y))) \neq f(x)}\).~Furthermore \({f(\pi(x+\pi(y))) = f(\pi(x+y))}\) (proposition \ref{prop2}), such that 
	\({f(\pi(x+ y)) \neq f(x)}\).
	\end{sloppypar}
\end{itemize}
\par
Hence in both cases \(f(\pi(x+y)) \neq f(x)\). If  \(f(\pi(x+y)) \leqslant n\) then \(h(x+y) = f(\pi(x+y))\). Therefore 
\(h(x+y) \neq h(x)\) since \(f(x) = h(x)\). Otherwise, \(f(\pi(x+y)) = n + 1\) and thus \(h(x+y) > n\). In particular, 
\(h(x + y) \neq h(x)\) since \(h(x) = h(y) \leqslant n\). \\

\noindent \underline{\textbf{Case 3:} \((x,y) \in \mathcal{T} \times \mathcal{R}\)}
\par
Necessarily \(h(x) = h(y) = n + 1\). Distinguish between two cases according to the value of \(\lambda(x+y)\).
\begin{itemize}
\item If \(\lambda(y)=\lambda(x+y)\) then \(\pi(x + y) = x + \pi(y)\) (proposition \ref{prop6}). By definition of 
	\(h\), \(f(x) = f(\pi(y))\). Given that \(f\) is a weakly sum-free coloring, \(f(x + \pi(y)) \neq f(x)\) since 
	\(f(x) = f(\pi(y))\) and \(x \neq \pi(y)\) since \({x \leqslant b < \pi(y)}\). Hence \(h(x + y) \neq h(x)\).
\item If \(\lambda(y) \neq \lambda(x + y)\) then \(\lambda(x + y) = \lambda(y) + 1\) since \(x \leqslant b < a\). 
	Besides, \(n + 1 = h(y) = n +  g(\lambda(y))\). Hence \(g(\lambda(y)) = 1\). Moreover \(g(1) = 1\) since \(g\) 
	is an ordered coloring. Therefore, given that \(g\) is sum-free, \(g(\lambda(y) + 1) \neq 1\). If \(\pi(x + y) \in 
	A_{n + 1}\) then \(h(x + y) = n + g(\lambda(x + y)) \neq n + 1\). Otherwise, \(h(x + y) \leqslant n\). Hence 
	in both cases \(h(x + y) \neq h(x)\).
\end{itemize}

\noindent \underline{\textbf{Case 4:} \((x,y) \in \mathcal{C}^2\)}
\par
By definition of \(h\) and since \(h(x)=h(y)\), \(f(\pi(x)) = f(\pi(y))\). Distinguish between two cases according 
to the value of \(\pi(x)+\pi(y)\).
\begin{itemize}
\item If \(\pi(x) + \pi(y) \leqslant a+b\) then \(\pi(x)+\pi(y) = \pi(x + y)\). Hence \(f(\pi(x + y)) \neq f(\pi(x))\) since 
	\(f\) is sum-free for \(x>b\)
\item \begin{sloppypar} 
	If \(\pi(x)+\pi(y)>a+b\) then given that \(f\) is a WSF-template, \({f(\pi(\pi(x)+\pi(y))) \neq f(\pi(x))}\) since 
	\({f(\pi(x)) = f(\pi(y))}\). Besides,  \({f(\pi(\pi(x)+\pi(y))) = f(\pi(x + y))}\) (proposition \ref{prop4}). Hence \({f(\pi(x + y)) 
	\neq  f(\pi(x))}\).
	\end{sloppypar}
\end{itemize}
\par
Hence in both cases \(f(\pi(x+y)) \neq f(x)\). If  \(f(\pi(x+y)) \leqslant n\) then \(h(x+y) = f(\pi(x+y))\). Therefore 
\(h(x+y) \neq h(x)\) since \(f(x) = h(x)\). Otherwise, \(f(\pi(x+y)) = n + 1\) and thus \(h(x+y) > n\). In particular, 
\(h(x + y) \neq h(x)\) since \(h(x) = h(y) \leqslant n\). \\

\noindent \underline{\textbf{Case 5:} \((x,y) \in \mathcal{C} \times \mathcal{R}\)}
\par
By definition of \(h\), \(h(x) \neq h(y)\).\\

\noindent \underline{\textbf{Case 6:} \((x,y) \in \mathcal{R}^2\)}
\par
In particular \(f(\pi(x)) = f(\pi(y))=n + 1\). Distinguish between three cases according to the value of \(\pi(x) + \pi(y)\).
\begin{itemize}
\item If \(\pi(x) + \pi(y) \in [\![a + b + 1, 2 a + b]\!]\) then \(\lambda(x + y) = \lambda(x) + \lambda(y)\) 
	(proposition \ref{prop7}). By definition of \(h\) and since \(h(x) = h(y)\), \(g(\lambda(x)) = g(\lambda(y))\).
	Hence, \(h(\lambda(x + y)) \neq h(\lambda(x))\) since \(h\) is a sum-free coloring. If \(f(x+y) \geqslant n + 1\) 
	then \(h(x + y) = n + g(\lambda(x + y))\). And \(h(x) = n + g(\lambda(x))\). Therefore, \(h(x + y)  \neq h(x)\). 
	Otherwise \(h(x+y) \leqslant n < h(x)\). In particular \(h(x + y) \neq h(x)\).
\item If \(\pi(x)+\pi(y)>2a+b\) then \(f(\pi(\pi(x)+\pi(y))) \neq f(\pi(x)) = n + 1\) since \(f\) is a \(b\)-WSF template and 
	\(f(\pi(x)) = f(\pi(y))\). Given that \(\pi(\pi(x)+\pi(y)) = \pi(x+y)\) (proposition \ref{prop4}), \(f(\pi(x+y)) \neq n + 1\).
\item \begin{sloppypar}
	If \(\pi(x)+\pi(y)\leqslant b+a\) then, given that \(\pi(x)+\pi(y) \geqslant b\) and \(f_{| [\![b, a + b ]\!]}\) is 
	sum-free, \({f(\pi(x) + \pi(y)) \neq f(\pi(x)) = n + 1}\). That is \({f(\pi(x + y)) \neq n + 1}\) (proposition \ref{prop1}).
	\end{sloppypar}
\end{itemize}
\par
In both of the last two cases, \({f(\pi(x + y)) \neq n + 1}\) that is \(x+y \in \mathcal{C}\). Therefore \(h(x+y) < n \leqslant h(x)\). 
In particular, \(h(x + y) \neq h(x)\). \\
\end{proof}

The general lower bound for weak Schur numbers in function of both regular and weak Schur numbers can be seen as a particular
case of WSF-template in the same way Abott and Hanson's construction can be seen as a particular case of SF-template. Acutally,
like for SF-templates, the additive constant of a WSF-template can be improved by weakening the hypotheses made on
the last row. The principle behind it is the same as in the analogous proposition for SF-templates.

\begin{proposition}
Let \((b, k, a) \in (\mathbb{N}^*)^3\) and let \(f\) be a coloring associated to a \(b\)-WSF-template with width \(p\) and \(k\) colors. Let
\(c \in \mathbb{N}\) (\(c = \min (A_{k+1} \backslash [\![1, b ]\!]) - 1\) works) and assume there there exists a coloring \(g\) of
\([\![b + 1, b + c]\!]\) with \(k\) colors such that for all \(c \in [\![1, k]\!]\),

\begin{itemize}
\item \(\forall (x, y) \in  [\![1, a + b]\!] \times  [\![b + 1, a + b]\!],  \left\{
	\begin{array}{l}
		f(x) = f(y) \\
		\pi(x + y) \leqslant b + c
	\end{array}
	\right. \implies g(\pi(x + y)) \neq f(x)\)
\item \(\forall (x, y) \in  [\![1, a + b]\!] \times  [\![b + 1, b + c]\!], \left\{
	\begin{array}{l}
		f(x) = g(y) \\
		\pi(x + y) \leqslant b + c
	\end{array}
	\right. \implies g(\pi(x + y)) \neq f(x)\)
\end{itemize}

Then, for every \(n \in \mathbb{N}^*\), by using on the last row the coloring \(x \longmapsto g(x - p S(n))\), we have\\
\[ \WS(n+k) \geqslant \WS^+(k+1) S(n) + b + c\]
\end{proposition}

The WSF-templates can actually be fine-tuned further. However, it gives only minor improvements (most likely only an additive
constant) at the cost of dramatically increasing the size of the search space. Therefore, it does not seem relevant to
use this sophistications given that we could not even find good WSF-templates with 5 colors using a computer (here good
means better than those obtain by combining smaller templates).

\par These modifications work as follows. One may notice that the first row (excluding the "tail") has constraints that other rows
do not have because of the tail, especially if the special color appears in the tail as well. Thus allowing to have a coloring on the
first row different from the coloring of the other rows would weaken the constraints. Acutally, one may even go further by
noticing that on the one hand the first (ordered) color of the sum-free partition used for the extension procedure has more
more constraints than the other colors of the sum-free partition since the first row is of this color and is more constrained than
the other rows, but that on the other hand it has more degrees of freedom than the other colors of the sum-free partition since in
the sum-free partition there cannot be two consecutive numbers of this color. As a result, it removes  some constraints imposed
 by the first row on the other rows.

\par
To sum up, one can look for a generalised WSF-template that uses a special coloring for the tail and the first row, a
coloring dedicated to the rows whose number is not 1 but is in the first color in the sum-free partition, a coloring for all
the other rows and a special coloring for the last numbers (as previously explained for the improvement of the additive
constant of WSF-templates).

We also have a similar theorem where only \(\WS^+\) is involved.

\begin{theorem}
Let \((k,p) \in (\mathbb{N}^*)^2\) and \((a, n, b) \in (\mathbb{N}^*)^3\) . If there exists a SF-template with width 
\(p\) and \(k+1\) colors and a \(b\)-WSF-template with width \(a\) and \(n\) colors, then there exists \(pb\)-WSF-template 
with width \(pq\) and \((n+k)\) colors.
\end{theorem}

And the associated inequality

\begin{corollary}
Let \(n, k \in \mathbb{N}^*\). Then
\[
\WS^+(n+k) \geqslant S^+(k+1) \WS^+(n)
\]
\end{corollary}

\begin{proof}[\textsc{Proof.}]
The idea is the same as in the previous theorem. The only difference is the WSF property inherited
from the WSF-template. \\
\end{proof}



\subsection{Inequalities and new lower bounds for Weak Schur numbers}

\qquad We exhibited WSF-templates using a SAT solver, hence providing lower bound on \(WS^+\) and inequalities 
of the type \(WS(n+k) \geqslant a S(n) + b\). We have sought templates providing the greatest value of 
\((a, b)\) (for the lexicographic order).

\[
\begin{array}{r r r r r}
	WS(n + 1) & \geqslant & 4 S(n) & + & 2 \\
	WS(n + 2) & \geqslant & 13 S(n) & + & 8 \\
	WS(n + 3) & \geqslant & 42 S(n) & + & 24 \\
	WS(n + 4) & \geqslant & 132 S(n) & + & 26 \\
\end{array}
\]

The first two inequalities were found by Rowley, they are detailed in \cite{RowleyWS}. The third inequality is optimal 
and was found with a SAT solver. It uses the first sophistication explained in the previous subsection in order to add 
the last number in the first color. The \hyperref[WSF-templates]{corresponding template} can be found in the appendix.
As for the fourth inequality, it was obtained by combining an optimal SF-template with width 33 with a WSF-template with width 4.
The best template we could get with a computer search gives the inequality \(\WS(n+4) \geqslant 127 S(n) + 68\).
It was also found with the SAT solver. In order to reduce the search space, we only looked for WSF-templates of
5 colors which start with a good \(\WS(4)\) partition and we assumed that the special color was the last by order of 
appearance. However, this approach most likely prevents us from finding the best WSF-templates
as we explain in the next subsection for weakly sum-free partitions. We highly suspect that there exists more efficient WSF-templates
with \(n \geqslant 5\) colors. One may try to go over a different search space using a Monte-Carlo method, as in \cite{Bouzy2015AnAP}.
This could be the suject of a future work. Further details about the encoding as a SAT problem can be found in \cite{Heule2017}.

Like in 3.3, we compute the lower bounds given by the previous inequalities for \( n \in [\![8,15]\!] \). The best lower bound
for each integer is highlighted.

\renewcommand{\arraystretch}{0.2}

\[
\begin{array}{c}
	\resetarraystretch
	\begin{NiceArray}{cwc{8ex}wc{10ex}wc{10ex}wc{11ex}}[hvlines]
	\CodeBefore
		\cellcolor{yellow}{2-4}
		\cellcolor{yellow}{2-5}
		\cellcolor{yellow}{3-2}
		\cellcolor{yellow}{4-3}
	\Body
		n & 8 & 9 & 10 & 11 \\
		4 \, S(n-1) + 2 & 6\,722 & 21\,146 & 71\,214 & 243\,794 \\
		13 \, S(n-2) + 8 & 6\,976 & 21\,848 & 68\,726 & 231\,447 \\
		42 \, S(n-3) + 24 & 6\,744 & 22\,536 & 70\,584 & 222\,036 \\
	\end{NiceArray}
	\\ \\
	\resetarraystretch
	\begin{NiceArray}{cwc{8ex}wc{10ex}wc{10ex}wc{11ex}}[hvlines]
	\CodeBefore
		\cellcolor{yellow}{2-2}
		\cellcolor{yellow}{2-5}
		\cellcolor{yellow}{3-3}
		\cellcolor{yellow}{4-4}
	\Body
		n & 12 & 13 & 14 & 15 \\
		4 \, S(n-1) + 2 & 815\,314 & 2\,554\,194 & 8\,045\,162 & 27\,061\,154 \\
		13 \, S(n-2) + 8 & 792\,332 & 2\,649\,772 & 8\,301\,132 & 26\,146\,778 \\
		42 \, S(n-3) + 24 & 747\,750 & 2\,559\,840 & 8\,560\,800 &  25\,886\,224 \\
	\end{NiceArray}
\end{array}
\]

\resetarraystretch

With \( S(9) \geqslant 17\,803 \), we found a new lower bound for \(\WS(10)\) using Rowley's inequality.
Moreover, the third inequality gives new lower bounds for \(\WS(9)\) and \(WS(14)\).


\subsection{Conclusion on WSF-templates}

\qquad In this section, we first gave a new construction which can be seen as an equivalent for weakly sum-free partitions of Abott
and Hanson's construction for sum-free partitions. We then  introduced WSF-templates and generalized this construction. This
allows us to find new lower bounds and new inequalities for weak Schur numbers. One may notice the significant difference
between the former lower bounds for weak Schur numbers obtained by conducting a computer search and the new lower bounds
obtained with WSF-templates (including Rowley's two inequalities). In the next section, we try to analyze this phenomenon.
