\section{Templates for Schur numbers}
\label{Schur}

\qquad In this section, we use Rowley's template-based constructions \cite{RowleyRamsey} in the context of Schur 
numbers. In order to improve lower bounds for Schur and Ramsey numbers, Rowley introduces special sum-free 
partitions verifying some additional properties which can be extended using a method generalizing Abbott and 
Hanson's construction \cite{AbbottHanson}. Rowley named these partitions "templates", and we keep this name in 
the entire article. We then find new templates and use them to provide new lower bounds for Schur numbers.

\subsection{Definition of \(S^+\)}

\begin{definition}
Let \((p,n) \in (\mathbb{N}^*)^2\). A SF-template with width \(p\) and \(n\) colors is defined as a partition of 
\([\![1,p]\!]\) into \(n\) sum-free subsets \(A_1, A_2, ..., A_n\) verifying
\[
\forall i \in [\![1, n-1]\!], \forall (x,y) \in A_i^2, x+y > p
\Longrightarrow x+y-p \notin A_i
\]
\end{definition}

Here \(n\) is the "special" color: it has less constraints than the other colors. However, please note that \(n\) 
is not necessarily the last color by order of appearance. \(SF\)-templates include Abbott and Hanson's construction \cite{AbbottHanson}
as a special case.

\begin{proposition}
	Let \(n \in [\![2, +\infty[\![\). We define \(S^+(n)\) as the maximal width of a SF-template with \(n\) colors. 
	\(S^+(n)\) is well defined and verifies
	\[
	2S(n-1)+1 \leqslant S^+(n) \leqslant S(n)
	\]
\end{proposition}

\begin{proof}[\textsc{Proof.}]
The lower bound comes from Abbott and Hanson's construction. The upper bound comes from the 
fact that a SF-template with width \(p\) and \(n\) colors is also a partition of \([\![1, p]\!]\) into \(n\) sum-free subsets. \\
\end{proof}

\begin{remark}
	\(S^+\) and \(S\) have the same asymptotic growth rate.
\end{remark}


\subsection{Construction of Schur partitions using SF-templates}

\qquad Here we state the main result on SF-templates stated by Rowley in the context of Ramsey numbers. It consists 
in the explicit construction of a sum-free partition using a SF-template and a sum-free partition.

\begin{theorem}
	Let \((p,k), (q,n) \in (\mathbb{N}^*)^2\). If there exists a \(SF\)-template with width \(q\) and \(n+1\) colors 
	and a partition of \([\![1,p]\!]\) into \(k\) sum-free subsets then there exists a partition of \([\![1,pq+m_{n+1}
	-1]\!]\) into \(n+k\) sum-free subsets. \(m_{n+1}\) denotes the minimum element in the special subset in the \(SF\)-template.
\end{theorem}

Setting \(q = S^+(n+1)\) and \(p = S(k)\) yields the following corollary.

\begin{corollary}
	Let \(n, k \in \mathbb{N}^*\). Then
	\[ S(n+k) \geqslant S^+(n+1)S(k) + m_{n+1} - 1 \]
\end{corollary}

The idea lying beneath this theorem is similar to Abbott and Hanson's contruction \cite{AbbottHanson}. They vertically 
extend a sum-free partition by repeating it and they use an other sum-free partition to color the other half according 
to the line number. This way the "blocks" act as safe areas for each other. We give here an example for \(p = 4\), 
\(q = 9\), \(n = 2\) and \(k = 2\) showing that \(S(2 + 2) \geqslant S(2) (2 S(2) + 1) + S(2)\), both with Abbott and 
Hanson's construction and with a SF-template which is not included in Abbott and Hanson's construction. In both cases, 
the special color is grey.

\renewcommand{\arraystretch}{1.7}

\begin{center}
\setlength{\tabcolsep}{1.5ex}

\textbf{Abbott and Hanson's construction}
\vspace{1.7ex}

\begin{NiceTabular}{|*{9}{c|}}[corners=SE,standard-cline,hlines]
\CodeBefore
	\cellcolor{red}{1-1}
	\cellcolor{green}{1-2}
	\cellcolor{green}{1-3}
	\cellcolor{red}{1-4}
	\cellcolor{cyan}{1-5}
	\cellcolor{cyan}{1-6}
	\cellcolor{cyan}{1-7}
	\cellcolor{cyan}{1-8}
	\cellcolor{cyan}{1-9}
	\cellcolor{red}{2-1}
	\cellcolor{green}{2-2}
	\cellcolor{green}{2-3}
	\cellcolor{red}{2-4}
	\cellcolor{yellow}{2-5}
	\cellcolor{yellow}{2-6}
	\cellcolor{yellow}{2-7}
	\cellcolor{yellow}{2-8}
	\cellcolor{yellow}{2-9}
	\cellcolor{red}{3-1}
	\cellcolor{green}{3-2}
	\cellcolor{green}{3-3}
	\cellcolor{red}{3-4}
	\cellcolor{yellow}{3-5}
	\cellcolor{yellow}{3-6}
	\cellcolor{yellow}{3-7}
	\cellcolor{yellow}{3-8}
	\cellcolor{yellow}{3-9}
	\cellcolor{red}{4-1}
	\cellcolor{green}{4-2}
	\cellcolor{green}{4-3}
	\cellcolor{red}{4-4}
	\cellcolor{cyan}{4-5}
	\cellcolor{cyan}{4-6}
	\cellcolor{cyan}{4-7}
	\cellcolor{cyan}{4-8}
	\cellcolor{cyan}{4-9}
	\cellcolor{red}{5-1}
	\cellcolor{green}{5-2}
	\cellcolor{green}{5-3}
	\cellcolor{red}{5-4}
\Body
	1 & 2 & 3 & 4 & 5 & 6 & 7 & 8 & 9 \\
	10 & 11 & 12 & 13 & 14 & 15 & 16 & 17 & 18 \\
	19 & 20 & 21 & 22 & 23 & 24 & 25 & 26 & 27 \\
	28 & 29 & 30 & 31 & 32 & 33 & 34 & 35 & 36 \\
	37 & 38 & 39 & 40 \\
\end{NiceTabular}

\vspace{1ex}
\setlength{\tabcolsep}{2ex}

\begin{tabular}{c c}
	\textbf{Corresponding SF-template} & \textbf{Corresponding sum-free partition} \\
	\begin{NiceTabular}{|*{9}{c|}}[standard-cline,hlines]
	\CodeBefore 
		\cellcolor{red}{1-1}
		\cellcolor{green}{1-2}
		\cellcolor{green}{1-3}
		\cellcolor{red}{1-4}
		\cellcolor{gray!40}{1-5}
		\cellcolor{gray!40}{1-6}
		\cellcolor{gray!40}{1-7}
		\cellcolor{gray!40}{1-8}
		\cellcolor{gray!40}{1-9}
	\Body
		1 & 2 & 3 & 4 & 5 & 6 & 7 & 8 & 9 \\
	\end{NiceTabular} &
	\begin{NiceTabular}{|*{4}{c|}}[standard-cline,hlines]
	\CodeBefore
		\cellcolor{cyan}{1-1}
		\cellcolor{yellow}{1-2}
		\cellcolor{yellow}{1-3}
		\cellcolor{cyan}{1-4}
	\Body
		1 & 2 & 3 & 4 \\
	\end{NiceTabular}
\end{tabular}

\setlength{\tabcolsep}{6pt}
\end{center}

In the general construction with SF-templates, the special color no longer necessarily contains consecutive 
numbers. However, the special color is still replaced by the colors of the sum-free partition according to the 
line number and the other colors are still vertically extended.

\begin{center}
\setlength{\tabcolsep}{1.5ex}

\textbf{SF-template construction}
\vspace{1.7ex}

\begin{NiceTabular}{|*{9}{c|}}[corners=SE,standard-cline,hlines]
\CodeBefore
	\cellcolor{red}{1-1}
	\cellcolor{green}{1-2}
	\cellcolor{green}{1-3}
	\cellcolor{red}{1-4}
	\cellcolor{cyan}{1-5}
	\cellcolor{cyan}{1-6}
	\cellcolor{red}{1-7}
	\cellcolor{cyan}{1-8}
	\cellcolor{cyan}{1-9}
	\cellcolor{red}{2-1}
	\cellcolor{green}{2-2}
	\cellcolor{green}{2-3}
	\cellcolor{red}{2-4}
	\cellcolor{yellow}{2-5}
	\cellcolor{yellow}{2-6}
	\cellcolor{red}{2-7}
	\cellcolor{yellow}{2-8}
	\cellcolor{yellow}{2-9}
	\cellcolor{red}{3-1}
	\cellcolor{green}{3-2}
	\cellcolor{green}{3-3}
	\cellcolor{red}{3-4}
	\cellcolor{yellow}{3-5}
	\cellcolor{yellow}{3-6}
	\cellcolor{red}{3-7}
	\cellcolor{yellow}{3-8}
	\cellcolor{yellow}{3-9}
	\cellcolor{red}{4-1}
	\cellcolor{green}{4-2}
	\cellcolor{green}{4-3}
	\cellcolor{red}{4-4}
	\cellcolor{cyan}{4-5}
	\cellcolor{cyan}{4-6}
	\cellcolor{red}{4-7}
	\cellcolor{cyan}{4-8}
	\cellcolor{cyan}{4-9}
	\cellcolor{red}{5-1}
	\cellcolor{green}{5-2}
	\cellcolor{green}{5-3}
	\cellcolor{red}{5-4}
\Body
	1 & 2 & 3 & 4 & 5 & 6 & 7 & 8 & 9 \\
	10 & 11 & 12 & 13 & 14 & 15 & 16 & 17 & 18 \\
	19 & 20 & 21 & 22 & 23 & 24 & 25 & 26 & 27 \\
	28 & 29 & 30 & 31 & 32 & 33 & 34 & 35 & 36 \\
	37 & 38 & 39 & 40 \\
\end{NiceTabular}

\vspace{1ex}
\setlength{\tabcolsep}{2ex}

\begin{tabular}{c c}
	\textbf{Corresponding SF-template} & \textbf{Corresponding sum-free partition} \\
	\begin{NiceTabular}{|*{9}{c|}}[standard-cline,hlines]
	\CodeBefore 
		\cellcolor{red}{1-1}
		\cellcolor{green}{1-2}
		\cellcolor{green}{1-3}
		\cellcolor{red}{1-4}
		\cellcolor{gray!40}{1-5}
		\cellcolor{gray!40}{1-6}
		\cellcolor{red}{1-7}
		\cellcolor{gray!40}{1-8}
		\cellcolor{gray!40}{1-9}
	\Body
		1 & 2 & 3 & 4 & 5 & 6 & 7 & 8 & 9 \\
	\end{NiceTabular} &
	\begin{NiceTabular}{|*{4}{c|}}[standard-cline,hlines]
	\CodeBefore
		\cellcolor{cyan}{1-1}
		\cellcolor{yellow}{1-2}
		\cellcolor{yellow}{1-3}
		\cellcolor{cyan}{1-4}
	\Body
		1 & 2 & 3 & 4 \\
	\end{NiceTabular}
\end{tabular}

\setlength{\tabcolsep}{6pt}
\end{center}

\resetarraystretch

We now proceed to prove the above mentioned theorem.

\begin{proof}[\textsc{Proof.}]
	Denote by \(f\) the coloring associated to the \(SF\)-template with width \(q\) and \(g\) the one associated 
	to the sum-free partition of \([\![1,p]\!]\); where \(f : [\![1,q]\!] \longrightarrow [\![1,n+1]\!]\) and 
	\(g : [\![1,p]\!] \longrightarrow [\![1,k]\!]\).
	
	\par
	NB: In the following three predicates, the conditions \(x + y \leqslant q\)  and \(x + y \leqslant p\) are omitted for readability.
	\par
	The sum-free condition is expressed as:
	\[\forall (x,y) \in [\![1,q]\!]^2, f(x) = f(y) \Longrightarrow f(x+y) \neq f(x)\],
	\[\forall (x,y) \in [\![1,p]\!]^2, g(x) = g(y) \Longrightarrow g(x+y) \neq g(x)\].
	
	The additionnal constraint for the SF-template is:
	\[
	\forall (x,y) \in [\![1,q]\!]^2, \left\{
	\begin{array}{l}
		f(x) = f(y) \leqslant n \\
		x + y > q
	\end{array}
	\right. \Longrightarrow f(x+y-q) \neq f(x)
	\].
	
	For \(x \in [\![1,pq+m_{n+1}-1]\!]\), write \(x = (\alpha - 1) + u\) for certain integers \(\alpha \in \mathbb{Z}\) and 
	\(u \in [\![1,q]\!]\). This decomposition is of course unique. \(\alpha\) (resp. \(u\)) can be interpreted as the row 
	(resp. column) number of \(x\). A new coloring \(h\) is defined as follows:
	\[
	\begin{array}{c c c l}
		h : & [\![1,pq+m_{n+1}-1]\!] & \longrightarrow & [\![1,n+k]\!], \\
		& x & \longmapsto & 
		\left\{ \begin{array}{l l}
			f(u), & \text{if}~f(u) \leqslant n, \\
			n + g(\alpha), & \text{if}~f(u) = n + 1. \\
		\end{array} \right.
	\end{array}
	\]
	
	Function \(h\) is well-defined since, by definition of \(m_{n+1}\), \(\forall x \in [\![p q + 1, p q + m_{n + 1} - 1 ]\!], f(u) 
	\leqslant n\) and therefore \(\forall x \in [\![1,pq+m_{n+1}-1]\!], f(u) = n + 1 \implies \alpha \in [\![1, p]\!]\).
	
	\par
	We now prove that \(h\) is a sum-free coloring. Let \(x,y \in [\![1,pq + m_{n+1}-1]\!]\) such that \(h(x) = h(y)\) 
	and \(x+y \leqslant pq+m_{n+1}-1\). We claim that \(h(x+y) \neq h(x)\). We write \(x = (\alpha - 1) q + u\) and 
	\(y = (\beta - 1) q + v\) where \(\alpha, \beta \in \mathbb{Z}\) and \(u, v \in [\![1,q]\!]\). Two cases are to be
	distinguished according to the value of \(h(x)\). \\
	
	\noindent \underline{\textbf{Case 1:} \(h(x) \leqslant n\)}
	\par
	Let us assume that \(h(x+y) \leqslant n\), otherwise \(h(x + y) \neq h(x)\) obviously holds. By definition of function 
	\(h\) and given that \(h(u) = h(v)\), we conclude \(f(u) = f(v)\). Two cases are to b distinguished according to the value of \(x + y\).
	
	\begin{itemize}
	\item If \(u + v > q\), we write \(w = u + v - q \in [\![1, q]\!]\). Consequentely \(x + y = (\alpha + \beta - 1) q + w\). By definition, 
		\(h(x + y) = f(w)\). Given that \(f(u) = f(v) \leqslant n\), the additionnal constraint on \(f\) implies \(f(w) 
		\neq f(u)\), that is \(h(x + y) \neq h(x)\).
	\item If \(u + v \leqslant p\), we write \(w = u + v \in [\![1, q]\!]\). Consequentely \(x+y = (\alpha + \beta- 2) q + w\). By definition, 
		\(h(x + y) = f(w)\). Given that \(f(u) = f(v) \leqslant n\), the sum-free property of \(f\) implies \(f(w) \neq f(u)\), 
		that is \(h(x + y) \neq h(x)\).
	\end{itemize} 
	  
	\noindent \underline{\textbf{Case 2:} \(h(x) \geqslant n + 1\)}
	\par
	Now we have \(h(x) = n + g(\alpha) = 
	k + g(\beta) = h(y)\), hence \(g(\alpha) = g(\beta)\). As in the first case, distinguish between two cases according 
	to the value of \(x + y\).
	
	\begin{itemize}
	\item \begin{sloppypar}
		If \(u + v > q\), write \(w = u + v - q \in [\![1, q]\!]\). Then \(x + y = (\alpha + \beta - 1) q + w\). Assume that 
		\({h(x+y) \geqslant n + 1}\),  otherwise \(h(x + y) \neq h(x)\) obviously holds. By definition, \({h(x + y) = n + g(\alpha + 
		\beta)}\). Given that \(g(\alpha) = g(\beta)\), the sum-free property of \(g\) implies \(g(\alpha + \beta) 
		\neq g(\alpha)\) that is \(h(x + y) \neq h(x)\).
		\end{sloppypar}
	\item  If \(u + v \leqslant q\), write \(w = u + v \in [\![1, q]\!]\). Then \(x+y = (\alpha + \beta- 2) q + w\). The sum-free 
		property of \(f\) implies \(f(w) \neq f(u)\). Therefore \(f(w) \leqslant k\) and thus \(h(x + y) \leqslant n\). In particular,
		given that \(h(x) \geqslant n + 1\), \(h(x + y) \neq h(x)\).
	\end{itemize}
\end{proof}

The following proposition may improve the additive constant of a SF-template. Although it does not allow us to 
improve the SF-templates we have found, the analogous of this proposition for WSF-templates (see next section) 
allows us to improve one of them.

\begin{proposition}
Let \((q, n) \in \mathbb{N}^*)^2\) and let \(f\) be a coloring associated to a SF-template with width \(q\) and \(n+1\) 
colors. Let \(b \in \mathbb{N}\) and assume there is a coloring \(g\) of 
\([\![1, b]\!]\) with \(n+1\) colors such that:
	
\begin{itemize}
\item \(\forall (x, y) \in [\![1, q]\!]^2, \left\{
	\begin{array}{l}
		f(x) = f(y) \\
		(x + y) \mod q \leqslant b
	\end{array}
	\right. \implies g((x + y) \mod q) \neq f(x)\),
\item \(\forall (x, y) \in [\![1, q]\!] \times  [\![1, b]\!],  \left\{
	\begin{array}{l}
		f(x) = g(y) \\
		x + y \leqslant b
	\end{array}
	\right. \implies g(x + y) \neq f(x)\).
\end{itemize}
	
Then, for every \(n \in \mathbb{N}^*\), by using on the last row the coloring \(x \longmapsto g(x - p S(n))\), we have
\[ S(n+k) \geqslant S^+(n+1)S(k) + b\].
\end{proposition}

This proposition corresponds to the fact that sometimes a column is not the sum of two columns of a given color, but 
adding this column to the color would create sums in the color when applying the extension procedure. However, the 
last row does not intereact with the right-most columns when it comes to creating new sums. As a result, the hypotheses 
made on the coloring of the last row can be weakened.

There is a construction theorem for SF-templates as well.

\begin{theorem}
	Let \((p,k), (q,n) \in (\mathbb{N}^*)^2\). If there is a SF-template with width \(q\) and \(n+1\) colors,
	and a SF-template with width \(p\) and \(k\) colors, then there also is a SF-template with width \(pq\) and \((n+k)\) 
	colors.
	\end{theorem}
	
	The inequality associated with this theorme is given by:
	
\begin{corollary}
	Let \(n, k \in \mathbb{N}^*\). Then
	\[ S^+(n+k) \geqslant S^+(n+1)S^+(k) \].
\end{corollary}

\begin{proof}[\textsc{Proof.}]
The idea is the same as in the previous theorem. The only difference is the SF-template property inherited 
from the second SF-template. \\
\end{proof}


\subsection{Inequalities and new lower bounds for Schur numbers}

\begin{definition}
A sum-free partition \(A_1, ..., A_n\) of \([\![1, p]\!]\) is said to be symmetric if for all \( x \in [\![1, p]\!]\), 
\(x\) and \(p + 1 - x\) belong to the same subset (except if \(x = p + 1 - x\)).

A SF-template with \(n\) colors is said to be symmetric if the partition into \(n\) sum-free subsets derived 
from this template by applying the extension procedure with a sum-free partition of length 1 is symmetric. 
\end{definition}

We produced SF-templates using a SAT solver, hence providing lower bound on \(S^+\) and inequalities 
of the type \(S(n+k) \geqslant a S(n) + b\). We sought templates providing the largest value of 
\((a, b)\) (in the lexicographic order). When the number of colors excceeded five, in order to reduce the search space we 
only looked for symmetric SF-templates, we assumed that the special color was the last color to appear and we constrained 
the \(m_c\)'s out of being too small. Further details about the encoding as a SAT problem can be found in \cite{Heule2017}.

\par
Here are the best inequalities on Schur numbers so far (the \hyperref[SF-templates]{templates} corresponding to 
the third, fourth and fifth inequalities can be found in the appendix).
\[
\begin{array}{r r r r r}
	S(n + 1) & \geqslant & 3 S(n) & + & 1 \\
	S(n + 2) & \geqslant & 9 S(n) & + & 4 \\
	S(n + 3) & \geqslant & 33 S(n) & + & 6 \\
	S(n + 4) & \geqslant & 111 S(n) & + & 43 \\
	S(n + 5) & \geqslant & 380 S(n) & + & 148 \\
	S(n + 6) & \geqslant & 1\,140 S(n) & + & 528 \\
\end{array}
\]

The first inequality comes from  Schur's original article\cite{Schur1917}. The second one is due to
Abott \cite{AbbottHanson}
and the third one to Rowley \cite{RowleyRamsey}. The three last inequalies are our result.

\par
The first three inequalities are optimal. The fourth one is optimal among symmetric SF-templates whose special color is 
the last in the order of apparition (and with a multiplicative factor less than or equal to 118). The fifth one is most 
likely not optimal but should not be too far from the optimum. 
Finally, the sixth one is obtained by combining a SF-template with width 380 and one with width 3. 
Although we could not find a better SF-template with seven colors, the last inequality is definitely very far from the 
optimal value. One may try to seek better templates by constraining less the search space and by using 
Monte-Carlo methods, as in \cite{Bouzy2015AnAP}. This could be the subject of a future work.

\par
The previous inequalities give new lower bounds for \(S(n)\) for
\( n \geqslant 9 \). We compute the lower
bounds for \( n \in [\![8,15]\!] \) using the four different inequalities, please notice that the best values for \( n
=8\) and \(n = 13\) were obtained thanks to the first one, found by Rowley. The best lower bounds are highlighted.

\renewcommand{\arraystretch}{0.2}

\[
\begin{array}{c}
	\resetarraystretch
	\begin{NiceArray}{cwc{8ex}wc{10ex}wc{10ex}wc{11ex}}[hvlines]
	\CodeBefore
		\cellcolor{yellow}{2-2}
		\cellcolor{yellow}{3-3}
		\cellcolor{yellow}{4-4}
		\cellcolor{yellow}{4-5}
	\Body
		n & 8 & 9 & 10 & 11 \\
		33 \, S(n-3) + 6 & 5\,286 & 17\,694 & 55\,446 & 174\,444 \\
		111 \, S(n-4) + 43 & 4927 & 17\,803 & 59\,539 & 186\,523 \\
		380 \, S(n-5) + 148 & 5\,088 & 16\,868 & 60\,948 & 203\,828 \\
		1\,140 \, S(n-6) + 528 & 5\,088 & 15\,348 & 50\,688 & 182\,928 \\
	\end{NiceArray}
	\\ \\
	\resetarraystretch
	\begin{NiceArray}{cwc{8ex}wc{10ex}wc{10ex}wc{11ex}}[hvlines]
	\CodeBefore
		\cellcolor{yellow}{2-3}
		\cellcolor{yellow}{4-2}
		\cellcolor{yellow}{4-4}
		\cellcolor{yellow}{4-5}
	\Body
		n & 12 & 13 & 14 & 15 \\
		33 \, S(n-3) + 6 & 587\,505 & 2\,011\,290 & 6\,726\,330 & 21\,072\,090 \\
		111 \, S(n-4) + 43 & 586\,789 & 1\,976\,176 & 6\,765\,271 & 22\,624\,951 \\
		380 \, S(n-5) + 148 & 638\,548 & 2\,008\,828 & 6\,765\,288 & 23\,160\,388 \\
		1\,140 \, S(n-6) + 528 & 611\,568 & 1\,915\,728 & 6\,026\,568 & 20\,295\,948 \\
	\end{NiceArray}
\end{array}
\]

\resetarraystretch

Except for 8, 9 and 13, the best lower bounds are obtained thanks to
the third inequality \( S(n+5) \geqslant 380S(n) + 148\). The table
doesn't go any further, but the same inequality allows to improve the
lower bounds for every \( n \geqslant 15 \).

\begin{corollary}
\begin{sloppypar}
The growth rate for Schur numbers (and Ramsey numbers \(R_n(3)\))  satisfies \({\gamma \geqslant \sqrt[5]{380} \approx 3.28}\).
\end{sloppypar}
\end{corollary}

\begin{proof}[\textsc{Proof.}]
It is a mere consequence of the inequality \( S(n+5) \geqslant 380S(n) + 148\). As for Ramsey
numbers, the following inequality holds \(S(n) \leqslant R_n(3) - 2\) (see \cite{AbbottHanson}) hence the result. \\
\end{proof}


\subsection{Conclusion on SF-templates}

\qquad In this section, we first formalized Rowley's template-based constructions \cite{RowleyRamsey} in the context of Schur numbers 
by introducing SF-templates as well as a new sequence, \(S^+\). We provided relations between \(S^+\) and \(S\) then 
stated Rowley's construction method in the context of Schur numbers. We found new SF-templates allowing us to obtain 
new lower bounds for schur numbers. One may notice that we mostly gave only lower bounds for \(S^+\). It should be possible to 
find better SF-templates by making different assumptions or using a different method (Monte-Carlo methods for instance).

\par
In the next section, we provide similar results for weak Schur numbers. We introduce WSF-templates and a corresponding sequence, 
\(\WS^+\). We then derive similar relations and a construction method allowing us to find new lower bounds for weak Schur numbers.
