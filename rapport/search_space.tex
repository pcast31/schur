\section{Analysis of the former search space}
\label{SearchSpace}

\qquad \hypertarget{sat}{In} this section, we first provide evidence which indicate that the main assumption made by papers which
found the previous best known lower bounds for weak Schur numbers using a computer may not be correct. 
This is done primarily by studying \(\WS(6)\). Then, in an effort to eliminate irrelevant search spaces, we 
obtain stronger results than those previously known for \(\WS (5)\) while gaining several orders of magnitude 
in computation time by giving additional information to the SAT solver without losing in generality. In this 
section, \textit{we assume that the subsets are ordered}.


\subsection{Limitation of the former assumption}

\qquad Rowley's new lower bound for \(\WS (6)\) (642) \cite{RowleyWS} was a significant improvement upon
the former best known lower bound (582) \cite{EliahouBook}. This previous lower bound has been found 
several times using a computer (often with Monte-Carlo methods) and by recursively making the assumption 
that a good partition for \(\WS (n+1)\) starts with a good partition for \(\WS (n)\) which is true for small
values of \(n\). Therefore, one may wonder whether the limiting factor are the assumptions or the 
methods used to search for partitions.Here we provide evidence indicating that the search space induced by 
these assumptions does not contain the optimal partitions.

\begin{computational theorem}
There is no weakly sum-free partition of \([\![1,583]\!]\) into 6 subsets such that
\begin{itemize}
	\item \(m_5 \geqslant 66\)
	\item \(m_6 \geqslant 186\)
	\item \([\![210,349]\!] \subset A_6\)
\end{itemize}
\end{computational theorem}

This result was obtained in 8 hours with the SAT solver plingeling \cite{Lingeling2017} on a 2.60 GHz Intel
i7 processor PC.
However, simply encoding the existence of such a partition as explained in the previous subsection would not result in a
reasonable
computation time. In order to help the SAT solver, we add additional information in the propositional formula. We did
not quantify the
speedup, but it most likely allowed us to gain several order of magnitude in computation time as we explain in the next
subsection.

\par
For every weakly sum-free coloring \(f\) of \([\![1,65]\!]\) with 4 colors, the sequence \(f(1), f(2), f(3), ...\) always
starts with the following subsequence 1121222133. Then 11 is always
either in subset 1 or 3, 12 is always in subset 3 and so on. For every integer in \([\![1,65]\!]\), we computed in which
subset it can appear.
By using this constraints, we could then compute for every integer in \([\![1,185]\!]\), in which subset it can appear
in a weakly sum-free
partition of \([\![1,185]\!]\) which starts with a weakly sum-free partition of \([\![1,65]\!]\) into 4 subsets. Adding
these constraints to the
formula corresponding to the above theorem gives additional information to the SAT solver without loss of generality.

\par
The above theorem shows that the previous lower bound for \(\WS (6)\) is optimal in the search space considered by the
papers which found it.
Therefore, finding a partition of \([\![1,n]\!]\) into 6 weakly sum-free subsets for some \(n \geqslant 590\) which does
not have a template-like structure
would be extremely interesting since it could give indications on a new search space for improving lower bounds with a
computer search. 

\par More generally, this theorem questions the search space previously used for finding lower bounds for \(\WS (n)\) 
with a computer. In particular, to our knowledge every paper that found the lower bound \(\WS (5) \geqslant 196\) 
used this assumption. Therefore one may wonder whether this is actually a good lower bound. In the next subsection, we give 
properties that a hypothetical partition of \([\![1,197]\!]\) into 5 weakly sum-free subsets has to verify.


\subsection{Investigating weak Schur number five}

\qquad As explained in the previous subsection, the search space used for showing that \(\WS (5) \geqslant 196\) may not contain
optimal solution. In this subsection,
we give necessary conditions for a hypothetical partition of \([\![1,197]\!]\) into 5 weakly sum-free subsets using the
same type of methods as in the
previous subsection.

\begin{notation}
Let \(P\) be a predicate over weakly sum-free partitions. We denote by \(\WS (n | P)\) the greatest number \(p\) such that
there exists a partition of
\([\![1,p]\!]\) into n weakly sum-free subsets which verifies P.
\end{notation}

\par
\cite{ELIAHOU2012175} verified with a SAT solver that there are no partition into 5 weakly sum-free subsets of
\([\![1,197]\!]\) with
\(A_5 = \{67, 68\} \cup [\![70,134]\!] \cup \{136\}\) in 17 hours and could not provide a similar result when only
assuming \(m_5 = 67\) even after several
weeks of runtime. By using the same method as above, we were able to verify that \(\WS (5 | m_5 = 67) = 196\) in 0.5
seconds with the SAT solver glucose \cite{Glucose}
on a 2.60 GHz Intel i7 processor PC (we used the non-parallel version here for the sake of comparison but in the rest of
this subsection, we used the parallel version of glucose).
The additional information we gave to the SAT solver is that every partition of \([\![1,66]\!]\) into 4 weakly sum-free
subsets starts with a partition of
\([\![1,23]\!]\) into 3 weakly sum-free subsets (this can be checked in a dozen of minutes with a SAT solver). Among
the 3 partitions of \([\![1,23]\!]\) into
3 weakly sum-free subsets, every number always appears in the same subset except for 16 and 17 which can appear in two
different subsets. We hardcoded
this external knowledge in the propositional formula which allowed us to gain several orders of magnitude in computation
time. Since \(\WS (4) = 66\), we have \(m_5 \leqslant 67\). We give the stronger following result.


\begin{computational theorem}
If there exists a partition of \([\![1,197]\!]\) into 5 weakly sum-free subsets then \(m_5 \leqslant 59\).
\end{computational theorem}

More precisely, we verified the following results (\(max~m_5\) is the greatest value of \(m_5\) for which we have not
verified that \(\WS (5 | m_5) \leq 196\)).

\renewcommand{\arraystretch}{1.5}
\setlength{\tabcolsep}{3pt}

\begin{center}
\begin{tabular}{| c | *{21}{c |}}
	\hline
	\(m_4\) & 4 & 5 & 6 & 7 & 8 & 9 & 10 & 11 & 12 & 13 & 14 & 15 & 16 & 17 & 18 & 19 & 20 & 21 & 22 & 23 & 24 \\
	\hline
	\(\WS (4 | m_4) + 1\) & 55 & 59 & 60 & 59 & 59 & 60 & 60 & 60 & 60 & 64 & 63 & 64 & 61 & 64 & 63 & 65 & 65 & 65 & 65 & 66
	& 67 \\
	\hline
	\(max~m_5\) & 49 & 51 & 54 & 53 & 54 & 54 & 55 & 55 & 55 & 55 & 55 & 56 & 57 & 57 & 59 & 59 & 59 & 59 & 59 & 58 & 53 \\
	\hline
\end{tabular}
\end{center}

\resetarraystretch
\setlength{\tabcolsep}{6pt}

For \(m_4 = 24\), the value \(m_5 \leqslant 53\) was obtained after 7 hours of runtime, and for  \(m_4 = 22\), the
value \(m_5 \leqslant 59\) was obtained after 11 hours of runtime. As for the other values, they were obtained after 
between 1 minute and 2 hours of runtime. To obtain these results, we once again provided additionnal information to 
the SAT solver. We added two different types information. The first one is the same as previously: we compute the 
subsets in which the first numbers can appear. The second type of information is the maximum length of a sequence 
using only a certain subset of the colors. For instance, if \(m_4 \geqslant 22\), then there cannot be more than 17 
consecutive numbers with color 1, 2 or 3.


\subsection{Conclusion on the search space}

\qquad As explained in the first subsection, the recursive assumption that a good partition for \(\WS (n+1)\) starts with a good partition for 
\(\WS (n)\) appears to be wrong. Given that no extensive search for \(\WS(5)\) has been conducted without making this assumption and 
that the size of the considered partitions is reasonable (the current lower bound is 196 and \(S(5) = 160\) was verified), it seems worth 
investigating this special case further. In a spirit of orientating this search, we then give necessary conditions on an hypothetical partition 
of \([\![1,197]\!]\) into 5 weakly sum-free subsets. Finding partitions that exceeds the lower bounds found with a computer (even if they 
do not exceed those obtained with templates) which are not as regular as those obtained with templates would be extremely interesing 
since it could designate a new search space for finding lower bounds with a computer.
