\documentclass[graphics]{beamer}
\usepackage[utf8]{inputenc}
\usepackage[french]{babel}
\usepackage{lmodern}
\usepackage{tcolorbox}
\usepackage{pgfpages}
\usepackage{graphicx}
\usepackage{mathdots}
\usepackage{subcaption}
\usepackage{xcolor}
\usepackage[export]{adjustbox}
\usepackage{appendixnumberbeamer}
\usepackage{stmaryrd} 
\usepackage{caption}
\usepackage{version}
\usepackage{nicematrix}
\captionsetup[figure]{labelformat=empty}

\newenvironment{changemargin}[2]{%
\begin{list}{}{%
\setlength{\topsep}{0pt}%
\setlength{\leftmargin}{#1}%
\setlength{\rightmargin}{#2}%
\setlength{\listparindent}{\parindent}%
\setlength{\itemindent}{\parindent}%
\setlength{\parsep}{\parskip}%
}%
\item[]}{\end{list}}



\usetheme{Warsaw}
\usecolortheme{dolphin}

\title[Weak Schur numbers]{Weak Schur numbers}
\subtitle{P05 - Formation à la recherche 1A}
\author[R. Ageron, P. Castéras, T. Pellerin, Y. Portella]{Romain Ageron, Paul Castéras, Thibaut Pellerin, Yann Portella}
\titlegraphic{\centering
	\includegraphics[scale=.08]{cs}
}
%\institute[]{\'CentraleSupélec}
\date{3 juin 2021}
	\setbeamersize{text margin left=15pt}
	\setbeamersize{text margin right=15pt}


% pour supprimer les symboles de navigation
\setbeamertemplate{navigation symbols}{}
% \setbeamertemplate{footline}[frame number]
\setbeamertemplate{caption}{\raggedright\insertcaption\par}

\newcommand\blfootnote[1]{%
	\begingroup
	\renewcommand\thefootnote{}\footnote{#1}%
	\addtocounter{footnote}{-1}%
	\endgroup
}

\begin{document}

\begin{frame}
\titlepage
%\begin{center}
%	\includegraphics[height=0.5cm]{logoens.pdf}
%\end{center}
\end{frame}

\section{Présentation}
\subsection{Un~problème~de~partition}
\begin{frame}
	En 1917, le russe \textbf{Issai Schur} pose le problème suivant :
	\pause
	\begin{itemize}
		\item On se donne \(n \geq 1\) un entier
		\item \(k \geq 1\) un autre entier, qui correspond au nombre de \textbf{couleurs}
	\end{itemize}
	\pause
	\begin{tcolorbox}[colback=green!5,colframe=green!40!black,title=Question]
		Peut-on colorier les entiers de \(1\) à \(n\) de sorte que si deux nombres ont la même couleur,
		leur somme n'est pas de cette couleur ? Si oui, un tel coloriage est dit \textbf{sans sommes}.
	\end{tcolorbox}
\end{frame}

\subsection{Les~nombres~de~Schur}

\begin{frame}
	Pour \(n = 13\) et \(k = 3\), le coloriage \\
	\begin{center}
	\begin{NiceTabular}{|*{13}{c|}}[standard-cline,hlines]
		\CodeBefore
			\cellcolor{red}{1-1}
			\cellcolor{blue}{1-2}
			\cellcolor{blue}{1-3}
			\cellcolor{red}{1-4}
			\cellcolor{green}{1-5}
			\cellcolor{green}{1-6}
			\cellcolor{blue}{1-7}
			\cellcolor{green}{1-8}
			\cellcolor{green}{1-9}
			\cellcolor{red}{1-10}
			\cellcolor{blue}{1-11}
			\cellcolor{blue}{1-12}
			\cellcolor{red}{1-13}
		\Body
			1 & 2 & 3 & 4 & 5 & 6 & 7 & 8 & 9 & 10 & 11 & 12 & 13\\
	\end{NiceTabular}
	\end{center}
	vérifie cette propriété.
	\pause
	\begin{tcolorbox}[colback=red!5,colframe=red!40!black,title=Définition]
		Pour \(k\) couleurs, on note \(S(k)\) le plus grand entier \(n\) tel qu'on puisse colorier les entiers de
		\(1\) à \(n\) en vérifiant cette propriété. C'est le \(k\)-ième \textbf{nombre de Schur}.
	\end{tcolorbox}
	\pause
	Sur l'exemple, on peut vérifier que \(S(3) = 13\) : on ne peut colorier \([\![1,14]\!]\) avec trois couleurs.
\end{frame}

\subsection{Weak~Schur}

\begin{frame}
	\begin{tcolorbox}[colback=red!5,colframe=red!40!black,title=Définition]
		Un coloriage est dit \textbf{faiblement sans sommes} lorsque pour deux nombres \textbf{différents} de même couleur,
		leur somme n'est pas de la même couleur. On définit de même \(WS(k)\) le plus grand entier \(n\) tel qu'on puisse colorier les entiers de
		\(1\) à \(n\) en vérifiant cette propriété plus faible.
	\end{tcolorbox}
	\pause
	Un coloriage sans sommes et en particulier faiblement sans somme, donc on a toujours \(WS(k) \geq S(k)\).\\
	\pause 
	\begin{center}
	\begin{NiceTabular}{|*{4}{c|}}[standard-cline,hlines]
		\CodeBefore
			\cellcolor{red}{1-1}
			\cellcolor{blue}{1-2}
			\cellcolor{blue}{1-3}
			\cellcolor{red}{1-4}
		\Body
			1 & 2 & 3 & 4 \\
	\end{NiceTabular}

	\begin{NiceTabular}{|*{8}{c|}}[standard-cline,hlines]
		\CodeBefore
			\cellcolor{red}{1-1}
			\cellcolor{red}{1-2}
			\cellcolor{blue}{1-3}
			\cellcolor{red}{1-4}
			\cellcolor{blue}{1-5}
			\cellcolor{blue}{1-6}
			\cellcolor{blue}{1-7}
			\cellcolor{red}{1-8}
		\Body
			1 & 2 & 3 & 4 & 5 & 6 & 7 & 8 \\
	\end{NiceTabular}
	\end{center}
	\(S(2) = 4\) mais \(WS(2) = 8\)
\end{frame}

\section{L'état~de~l'art}
\subsection{Calculer~ces~nombres}

\begin{frame}
	\begin{itemize}
	\item Pour montrer que \(S(k) = n\), il faut trouver un coloriage sans sommes de \([\![1,n]\!]\) à \(k\) couleurs
	\textbf{et} montrer qu'on ne peut pas colorier \([\![1,n+1]\!]\). 
	\pause
	\item En pratique, on se contente de \textbf{minorer} \(S(k\). Si on exhibe un coloriage à \(k\) couleurs de 
	\([\![1,n]\!]\), on a montré que \(S(k) \geq n\). 
	\pause
	\item Comment effectuer cette minoration ? On peut démontrer des inégalités récursives ou bien rechercher des coloriages
	par ordinateur.
	\end{itemize}
\end{frame}
\subsection{Approche~numérique}

\begin{frame}
	Les recherches récentes sur le sujet se focalisent sur les méthodes numériques.
	\begin{itemize}
		\item On fixe \(k\) et on essaye de colorier le plus loin possible
		\item Le problème peut s'encoder comme une exploration d'arbre, mais le nombre de coloriages possibles
		explose très vite 
		\item Plusieurs articles récents ont recourt à l'algorithme \textbf{Monte-Carlo Tree Search}, et ne considèrent 
		que certains type de coloriages
		\item On peut également encoder le problème avec un solveur \textbf{SAT} 
	\end{itemize}
	\pause
	Ces méthodes ont permis d'améliorer les bornes inférieures pour \(k \geq 5\) mais peuvent prendre du temps :
	récemment, le calcul exact de \(S(5)\) via un solveur SAT a demandé 20 années de calcul machine.
\end{frame}

\section{Amélioration~des~nombres~de~Schur}
\subsection{Un~article~fondateur~:~Abbott~et~Hanson}
\begin{frame}
La borne inférieure établie par I. Schur est :
\[S(n+1) \geqslant 3S(n) + 1 \Longrightarrow S(n) \geqslant \frac{3^n - 1}{2}
\]
Une première piste pour améliorer cette borne est proposée par H. L. Abbott et D. Hanson en 1972. Ils prouvent :
\[
S(n+m) \geqslant S(n) \left( 2S(m) + 1 \right) + S(m)
\]
Que font-ils concrètement ?
\end{frame}
\begin{frame}
Un exemple pour \(n = m = 2\) :
\renewcommand{\arraystretch}{1.7}
\begin{center}
\begin{NiceTabular}{|*{9}{c|}}[corners=SE,standard-cline,hlines]
\CodeBefore
	\cellcolor{red}{1-1}
	\cellcolor{green}{1-2}
	\cellcolor{green}{1-3}
	\cellcolor{red}{1-4}
	\cellcolor{cyan}{1-5}
	\cellcolor{cyan}{1-6}
	\cellcolor{cyan}{1-7}
	\cellcolor{cyan}{1-8}
	\cellcolor{cyan}{1-9}
	\cellcolor{red}{2-1}
	\cellcolor{green}{2-2}
	\cellcolor{green}{2-3}
	\cellcolor{red}{2-4}
	\cellcolor{yellow}{2-5}
	\cellcolor{yellow}{2-6}
	\cellcolor{yellow}{2-7}
	\cellcolor{yellow}{2-8}
	\cellcolor{yellow}{2-9}
	\cellcolor{red}{3-1}
	\cellcolor{green}{3-2}
	\cellcolor{green}{3-3}
	\cellcolor{red}{3-4}
	\cellcolor{yellow}{3-5}
	\cellcolor{yellow}{3-6}
	\cellcolor{yellow}{3-7}
	\cellcolor{yellow}{3-8}
	\cellcolor{yellow}{3-9}
	\cellcolor{red}{4-1}
	\cellcolor{green}{4-2}
	\cellcolor{green}{4-3}
	\cellcolor{red}{4-4}
	\cellcolor{cyan}{4-5}
	\cellcolor{cyan}{4-6}
	\cellcolor{cyan}{4-7}
	\cellcolor{cyan}{4-8}
	\cellcolor{cyan}{4-9}
	\cellcolor{red}{5-1}
	\cellcolor{green}{5-2}
	\cellcolor{green}{5-3}
	\cellcolor{red}{5-4}
\Body
	1 & 2 & 3 & 4 & 5 & 6 & 7 & 8 & 9 \\
	10 & 11 & 12 & 13 & 14 & 15 & 16 & 17 & 18 \\
	19 & 20 & 21 & 22 & 23 & 24 & 25 & 26 & 27 \\
	28 & 29 & 30 & 31 & 32 & 33 & 34 & 35 & 36 \\
	37 & 38 & 39 & 40 \\
	
\end{NiceTabular}
\end{center}
\[S(4) \geqslant S(2) \left( 2S(2) + 1 \right) + S(2) = 40
\]
\end{frame}
\subsection{Une~extension~par~Rowley~:~les~SF-templates}
\begin{frame}
\begin{itemize}
\item F. Rowley améliore cette approche théorique en 2020.
\pause
\item \textbf{Extension verticale} de structures plus générales : les SF-templates.
\pause
\item \textbf{Notre contribution} : recherche de SF-templates intéressants
\pause
\item Recette : SF-template = Partition sans somme + condition suivante :
\[
\forall i \in [\![1, n-1]\!], \forall (x,y) \in A_i^2, x+y > p
\Longrightarrow x+y-p \notin A_i
\]
\end{itemize}
\end{frame}
\begin{frame}
En fait, l'exemple précédent faisait déjà apparaître un SF-template, en voici un autre :
\begin{center}
\renewcommand{\arraystretch}{1.7}
\begin{NiceTabular}{|*{9}{c|}}[corners=SE,standard-cline,hlines]
\CodeBefore
	\cellcolor{red}{1-1}
	\cellcolor{green}{1-2}
	\cellcolor{green}{1-3}
	\cellcolor{red}{1-4}
	\cellcolor{cyan}{1-5}
	\cellcolor{cyan}{1-6}
	\cellcolor{red}{1-7}
	\cellcolor{cyan}{1-8}
	\cellcolor{cyan}{1-9}
	\cellcolor{red}{2-1}
	\cellcolor{green}{2-2}
	\cellcolor{green}{2-3}
	\cellcolor{red}{2-4}
	\cellcolor{yellow}{2-5}
	\cellcolor{yellow}{2-6}
	\cellcolor{red}{2-7}
	\cellcolor{yellow}{2-8}
	\cellcolor{yellow}{2-9}
	\cellcolor{red}{3-1}
	\cellcolor{green}{3-2}
	\cellcolor{green}{3-3}
	\cellcolor{red}{3-4}
	\cellcolor{yellow}{3-5}
	\cellcolor{yellow}{3-6}
	\cellcolor{red}{3-7}
	\cellcolor{yellow}{3-8}
	\cellcolor{yellow}{3-9}
	\cellcolor{red}{4-1}
	\cellcolor{green}{4-2}
	\cellcolor{green}{4-3}
	\cellcolor{red}{4-4}
	\cellcolor{cyan}{4-5}
	\cellcolor{cyan}{4-6}
	\cellcolor{red}{4-7}
	\cellcolor{cyan}{4-8}
	\cellcolor{cyan}{4-9}
	\cellcolor{red}{5-1}
	\cellcolor{green}{5-2}
	\cellcolor{green}{5-3}
	\cellcolor{red}{5-4}
\Body
	1 & 2 & 3 & 4 & 5 & 6 & 7 & 8 & 9 \\
	10 & 11 & 12 & 13 & 14 & 15 & 16 & 17 & 18 \\
	19 & 20 & 21 & 22 & 23 & 24 & 25 & 26 & 27 \\
	28 & 29 & 30 & 31 & 32 & 33 & 34 & 35 & 36 \\
	37 & 38 & 39 & 40 \\
\end{NiceTabular}
\end{center}
\vspace{1ex}
\renewcommand{\arraystretch}{1.7}
\begin{center}
\begin{NiceTabular}{|*{9}{c|}}[standard-cline,hlines]
	\CodeBefore 
		\cellcolor{red}{1-1}
		\cellcolor{green}{1-2}
		\cellcolor{green}{1-3}
		\cellcolor{red}{1-4}
		\cellcolor{cyan}{1-5}
		\cellcolor{cyan}{1-6}
		\cellcolor{red}{1-7}
		\cellcolor{cyan}{1-8}
		\cellcolor{cyan}{1-9}
	\Body
		1 & 2 & 3 & 4 & 5 & 6 & 7 & 8 & 9 \\
\end{NiceTabular}
\end{center}
\end{frame}
\subsection{Nouvelles~bornes}
\begin{frame}
\begin{changemargin}{-0.7cm}{-0.7cm}
\begin{center}
Quelques résultats !
\[
\begin{array}{c}
	\begin{NiceArray}{cwc{8ex}wc{10ex}wc{10ex}wc{11ex}}[hvlines]
	\CodeBefore
		\cellcolor{yellow}{2-2}
		\cellcolor{yellow}{3-3}
		\cellcolor{yellow}{4-4}
		\cellcolor{yellow}{4-5}
	\Body
		n & 8 & 9 & 10 & 11 \\
		33 \, S(n-3) + 6 & 5\,286 & 17\,694 & 55\,446 & 174\,444 \\
		111 \, S(n-4) + 43 & 4927 & 17\,803 & 59\,539 & 186\,523 \\
		380 \, S(n-5) + 148 & 5\,088 & 16\,868 & 60\,948 & 203\,828 \\
		1\,140 \, S(n-6) + 528 & 5\,088 & 15\,348 & 50\,688 & 182\,928 \\
	\end{NiceArray}
	\\ \\
	\begin{NiceArray}{cwc{8ex}wc{10ex}wc{10ex}wc{11ex}}[hvlines]
	\CodeBefore
		\cellcolor{yellow}{2-3}
		\cellcolor{yellow}{4-2}
		\cellcolor{yellow}{4-4}
		\cellcolor{yellow}{4-5}
	\Body
		n & 12 & 13 & 14 & 15 \\
		33 \, S(n-3) + 6 & 587\,505 & 2\,011\,290 & 6\,726\,330 & 21\,072\,090 \\
		111 \, S(n-4) + 43 & 586\,789 & 1\,976\,176 & 6\,765\,271 & 22\,624\,951 \\
		380 \, S(n-5) + 148 & 638\,548 & 2\,008\,828 & 6\,765\,288 & 23\,160\,388 \\
		1\,140 \, S(n-6) + 528 & 611\,568 & 1\,915\,728 & 6\,026\,568 & 20\,295\,948 \\
	\end{NiceArray}
\end{array}
\]

\end{center}
\end{changemargin}
\end{frame}

\section{Template~pour~les~nombres~de~Weak~Schur}


\subsection{Inégalités~entre~les~nombres~de~Schur~et~de~Weak~Schur}


\begin{frame}
	\begin{itemize}
		\item \textbf{premières inégalités obtenu par Rowley:}
		\vspace{3 mm}
			\begin{itemize}
			\item $WS(n+1) \geqslant 4S(n)+2$
			\item $WS(n+2) \geqslant 13S(n)+8$
			\end{itemize}
		\vspace{5 mm}
		\item \textbf{Inégalité généralisé:} 
		\vspace{5 mm} Soit $(n,k) \in \mathbb{N}^2$,
			\fbox{$WS (n+k) \geqslant S(k) \left (WS (n) + \left \lceil 					\displaystyle \frac{WS (n)}{2}
			\right \rceil +1 \right) + WS (n)$}
	\end{itemize}
\end{frame}

\begin{frame}
\begin{center}
$WS (n+k) \geqslant \underbrace{S(k)}_p \underbrace{\left (WS (n) +\overbrace{\left \lceil 					\displaystyle \frac{WS (n)}{2}
			\right \rceil}^{r=l} +1 \right)}_a + \underbrace{WS (n)}_b$
\end{center}

\begin{figure}
        \centering
        \includegraphics[scale=0.35]{tableau.png}
\end{figure}
\end{frame}

\subsection{Principe~général~du~template~Weak~Schur}
\begin{frame}


\begin{itemize}
\item On cherche un template à n couleurs de cardinal $WS^+(n)$ tel que:
\newline $WS(n+k) \geqslant S(k)WS^+(n)+b$
\vspace{8 mm}
\item Or $WS (n+k) \geqslant S(k) \left (WS (n) + \left \lceil 					\displaystyle \frac{WS (n)}{2}
			\right \rceil +1 \right) + WS (n)$
			\vspace{8 mm}
\item Par conséquent, $WS^+(n) \geqslant WS (n) + \left \lceil 					\displaystyle \frac{WS (n)}{2}
			\right \rceil +1 $
\end{itemize}

\end{frame}

\subsection{Nouvelles~valeurs~obtenus}

\begin{frame}
\begin{figure}
        \centering
        \includegraphics[scale=0.30]{tableau_resultat.png}
\end{figure}

\end{frame}
\end{document}