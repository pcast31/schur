\section{Definitions and notations}

We start by defining sum-free and weakly sum-free subsets to introduce regular and weak Schur numbers.

\begin{definition}
A subset \(A\) of \(\mathbb{N}\) is said to be \textit{sum-free} when:
\[ \forall (a,b) \in A^2 \text{, } a+b \notin A\]
\end{definition}

\begin{definition}
A subset \(B\) of \(\mathbb{N}\) is said to be \textit{weakly sum-free} when:
\[ \forall (a,b) \in B^2 \text{, } a \neq b \Longrightarrow a+b \notin B\]
\end{definition}

Let us notice that a sum-free subset is also weakly sum-free, hence justifying the name of \textit{weakly} sum-free
subsets. Given \(p\) and \(n\) two integers, we are interested in partitioning the set of integers from 1 to \(p\) into
\(n\) (weakly) sum-free subsets.

\begin{notation}
We denote by \([\![1,p]\!]\) the set of integers \(\{1, 2, ..., p\}\).
\end{notation}

Schur proved in \cite{Schur1917} that given a number of subsets \(n\), there exists a value of \(p\)
such that there exists no partition of \([\![1,q]\!]\) into \(n\) sum-free subsets for any \(q \geqslant p\). A similar
property holds for weakly sum-free subsets (reference necessaire). These observations lead to the following definitions.

\begin{definition}
Let \(n \in \mathbb{N}^*\). There exists a greatest integer that we denote \(S(n)\) (\textit{resp. \(\WS (n)\)}) such that
\([\![1,S(n)]\!]\) (resp. \([\![1, \WS (n)]\!]\)) can be partitioned into \(n\) sum-free subsets (resp. weakly sum-free
subsets). \(S(n)\) is called the \textit{\(n\)\textsuperscript{th} Schur number} and \textit{\(\WS (n)\) the
\(n\)\textsuperscript{th} weak
Schur number}.
\end{definition}

\begin{notation}
For a partition of \([\![1, p]\!]\) in \(n\) subsets, we generally denote these subsets \(A_1, ..., A_n\). We also denote
\(m_i = \min(A_i)\).
By ordering the subsets, we mean assuming that \(m_1 < ... < m_n\). However, if not specified we do not make this
hypothesis since we
do not always consider partitions in which every subset plays a symmetric role.
\end{notation}

\begin{definition}
We sometimes refer to a partitition as a colouring. The colouring associated to a partition \(A_1, ..., A_n\) of
\([\![1, p]\!]\) is the function \(f\) such that \(\forall x \in [\![1, p]\!], x \in A_{f(x)}\). Likewise, the partition associated to
a colouring \(f\) of \([\![1, p]\!]\) with \(n\) colors is \(\forall c \in [\![1, n]\!], A_c = f^{-1}(c)\).
\end{definition}
