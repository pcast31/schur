\section{Introduction}

\qquad We are interested in partioning the set of integers \(\{1, ..., p\}\) in \(n\) subsets such that there is no 
subset containing three integers \(x\), \(y\) and \(z\) verifying \(x + y = z\). We say these subsets are 
\textit{sum-free}. If we add the hypothesis \(x \neq y\), we say the subsets are \textit{weakly sum-free}. The 
greatest \(p\) for which there exists a partition into \(n\) sum-free subsets is called the \(n^{\text{th}}\) Schur 
number and is denoted \(S(n)\) \cite{Schur1917}. Likewise for weakly sum-free partitions we define \(\WS(n)\) 
the \(n^{\text{th}}\) weak Schur number \cite{Irving1973}. Only the first values of these sequences are known. 
This article focuses on the lower bounds for these numbers.


\subsection{State of the art}

\qquad Before Rowley's "template"-based approach for Schur and Ramsey numbers \cite{RowleyRamsey}, the 
previous generic construction for Schur numbers was given by Abbott and Hanson \cite{AbbottHanson} in 1972 
with a recursive construction. It used to give the best lower bounds for all sufficiently large numbers. No equivalent 
was known for weak Schur numbers and as a result the best known partitions for large weak Schur numbers 
did not utilize the weakly sum-free hypothesis. 

\par
As for smaller numbers, the best lower bounds were obtained by conducting a computer search. Eliahou \cite{ELIAHOU2012175}, 
Rafilipojaona \cite{Rafilipojaona} and Bouzy \cite{Bouzy2015AnAP} improved lower bounds with Monte-Carlo methods. It 
has been the main approch in the previous decade. 
This search for weakly sum-free partitions relied on the recursive assumption that a good weakly sum-free partition into \(n+1\) 
colors starts with a good weakly sum-free partition into \(n\) colors. This assumption was necessary in order for the Monte-Carlo
approch to yield results. However, we show in the \hyperlink{sat}{last section} how this search space may not countain
the optimal solution. Thus further work using similar methods might need to put that assumption aside and 
find a different search space.

\par
In 2020, Rowley introduced the notion of template for Schur and Ramsey numbers which generalizes Abbott and 
Hanson's construction and gives new lower bounds (and inequalties) for Schur numbers. Rowley also gives two 
inequalities for weak Schur numbers \cite{RowleyWS} that yield significant improvements over previous lower 
bounds which besides do utilize the \textit{weakly} sum-free hypothesis.

\setlength{\tabcolsep}{4pt}

\begin{center}

\textbf{Table 1 - Comparison of lower bounds for Schur numbers}

\begin{tabular}{|*{13}{c|}}
    \hline
    \(n\) & 1 & 2 & 3 & 4 & 5 & 6 & 7 & 8 & 9 & 10 & 11 & 12 \\
    \hline
    \multirow{2}{*}{before Rowley} & 1* & 4* & 13* & 44* & 160* & 536 & 1\,680 & 5\,041 & 15\,124 & 51\,120 & 172\,216 & 575\,664 \\
      & & & & & \cite{Heule2017} & \cite{Fredricksen} & \cite{Fredricksen} & \cite{ELIAHOU2012175} & 
    \cite{ELIAHOU2012175} & \cite{AbbottHanson} & \cite{AbbottHanson} & \cite{AbbottHanson} \\
    \hline
    \begin{tabular}{@{}c@{}}Rowley \\ \cite{RowleyRamsey}\end{tabular} & & & & & & & & 5\,286 & 17\,694 & 
    60\,320 & 201\,696 & 631\,840 \\
    \hline
    \hyperref[Schur]{\textbf{our results}}  & & & & & & & & & \textbf{17\,803} & \textbf{60\,948} & \textbf{203\,828} & \textbf{638\,548} \\
    \hline
\end{tabular}

\vspace{2.5ex}

\textbf{Table 2 - Comparison of lower bounds for weak Schur numbers}

\begin{tabular}{|*{13}{c|}}
    \hline
    \(n\) & 1 & 2 & 3 & 4 & 5 & 6 & 7 & 8 & 9 & 10 & 11 & 12 \\
    \hline
    \multirow{2}{*}{before Rowley} & 2* & 8* & 23* & 66* & 196 & 582 & 1\,740 & 5\,201 & 15\,596 & 51\,520 & 172\,216 & 575\,664 \\
      & & & & & \cite{ELIAHOU2012175} &\cite{EliahouBook} & \cite{Rafilipojaona} & \cite{Rafilipojaona} & 
    \cite{Rafilipojaona} & \cite{AbbottHanson} & \cite{AbbottHanson} & \cite{AbbottHanson} \\
    \hline
    \begin{tabular}{@{}c@{}}Rowley \\ \cite{RowleyWS}\end{tabular} & & & & & & 642 & 2\,146 & 6\,976 & 21\,848 
    & 70\,778 & 241\,282 & 806\,786 \\
    \hline
    \hyperref[WeakSchur]{\textbf{our results}} & & & & & & & & & \textbf{22\,536} & \textbf{71\,214} & \textbf{243\,794} & \textbf{815\,314} \\
    \hline
\end{tabular}

\vspace{1.5ex}
* denotes an exact value, not just a lower bound
\end{center}

\subsection{Structure of this article}

\par
The main contribution of this article is a generalization of the concept of template to weak Schur numbers. 
This gives new lower bounds (and inequalities) for weak Schur numbers. This construction also includes as a 
special case an analogous for weak Schur numbers of Abbott and Hanson's construction for Schur numbers.

\par
We first explain Rowley's template-based construction in the context of Schur numbers and then give 
new templates, thus providing new lowers bounds and inequalities as well as showing that the growth rates 
for both Schur and Ramsey numbers \(R_n(3)\) exceed 3.28. 

\par
We then  generalize the concept of templates to weak Schur numbers and provide new lower bounds for weak 
Schur numbers. 

\par
Finally, we analyze the significant difference between new lower bounds obtained with templates and the former 
lower bounds obtained by computer search and we provide evidence which indicate that the main assumption 
made in those articles removes the optimal partitions from the search space.

\par
We now introduce notations and definitions we use throughout this article.

\subsection{Definitions and notations}

We start by defining sum-free and weakly sum-free subsets to introduce regular and weak Schur numbers.

\begin{definition}
A subset \(A\) of \(\mathbb{N}\) is said to be \textit{sum-free} when:
\[ \forall (a,b) \in A^2 \text{, } a+b \notin A\]
\end{definition}

\begin{definition}
A subset \(B\) of \(\mathbb{N}\) is said to be \textit{weakly sum-free} when:
\[ \forall (a,b) \in B^2 \text{, } a \neq b \Longrightarrow a+b \notin B\]
\end{definition}

Let us notice that a sum-free subset is also weakly sum-free, hence justifying the name of \textit{weakly} sum-free
subsets. Given \(p\) and \(n\) two integers, we are interested in partitioning the set of integers \(\{1, 2, ..., p\}\), 
denoted by \([\![1,p]\!]\), into \(n\) (weakly) sum-free subsets.

\par
Schur proved in \cite{Schur1917} that given a number of subsets \(n\), there exists a value of \(p\)
such that there exists no partition of \([\![1,q]\!]\) into \(n\) sum-free subsets for any \(q \geqslant p\). A similar
property holds for weakly sum-free subsets (reference necessaire). These observations lead to the following definitions.

\begin{definition}
Let \(n \in \mathbb{N}^*\). There exists a greatest integer that we denote \(S(n)\) (\textit{resp. \(\WS (n)\)}) such that
\([\![1,S(n)]\!]\) (resp. \([\![1, \WS (n)]\!]\)) can be partitioned into \(n\) sum-free subsets (resp. weakly sum-free
subsets). \(S(n)\) is called the \textit{\(n\)\textsuperscript{th} Schur number} and \textit{\(\WS (n)\) the
\(n\)\textsuperscript{th} weak
Schur number}.
\end{definition}

Given a partition of \([\![1, p]\!]\) in \(n\) subsets, we generally denote these subsets \(A_1, ..., A_n\). We also denote
\(m_i = \min(A_i)\). By ordering the subsets, we mean assuming that \(m_1 < ... < m_n\). However, if not specified we do 
not make this hypothesis since we do not always consider partitions in which every subset plays a symmetric role.

\begin{definition}
We sometimes refer to a partitition as a coloring. The coloring associated to a partition \(A_1, ..., A_n\) of
\([\![1, p]\!]\) is the function \(f\) such that \(\forall x \in [\![1, p]\!], x \in A_{f(x)}\). Likewise, the partition associated to
a coloring \(f\) of \([\![1, p]\!]\) with \(n\) colors is \(\forall c \in [\![1, n]\!], A_c = f^{-1}(c)\).
\end{definition}
