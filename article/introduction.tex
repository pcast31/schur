\section{Introduction}

We are interested in partioning the set of integers \(\{1, ..., p\}\) in \(n\) subsets such that there is no subset 
containing three integers \(x\), \(y\) and \(z\) verifying \(x + y = z\). We say these subsets are sum-free. If we 
add the hypothesis \(x \neq y\), we say the subsets are weakly sum-free. The greatest \(p\) for which a 
partition into \(n\) sum-free exists is called the \(n^{\text{th}}\) Schur number and is denoted \(S(n)\) 
\cite{Schur1917}. Likewise for weakly sum-free partitions we define \(WS(n)\) the \(n^{\text{th}}\) weak Schur 
number \cite{Irving1973}. Only the first values of these sequences are known. This article concerns the lower 
bounds of these numbers.

\par Before Rowley's "template"-based approach for Schur and Ramsey numbers \cite{RowleyRamsey}, the 
previous generic construction for Schur numbers was given by Abbott and Hanson \cite{AbbottHanson} in 1972 
with a recursive construction. It gave the best lower bounds for all sufficiently large bounds. No equivalent 
was known for weak Schur numbers and as a result the best known partitions for large weak Schur numbers 
did not utilize the weakly sum-free hypothesis. As for smaller numbers, the best lower bounds were obtained 
by conducting a computer search. This search for weakly sum-free partitions relied on the recursive assumption 
that a good weakly sum-free partition into \(n+1\) colors starts with a good weakly sum-free partition into \(n\) 
colors. In 2020, Rowley introduced the notion of template for Schur and Ramsey numbers which generalizes 
Abbott and Hanson's construction and gives new lower bounds (and inequalties) for Schur numbers. Rowley also 
gives two inequalities for weak Schur numbers that yield significant improvements over previous lower bounds 
which besides do utilize the weakly sum-free hypothesis \cite{RowleyWS}.

\renewcommand{\arraystretch}{1.5}

\begin{center}

\textbf{Table 1 - Comparison of lower bounds for Schur numbers}

\begin{tabular}{|*{13}{c|}}
    \hline
    \(n\) & 1 & 2 & 3 & 4 & 5 & 6 & 7 & 8 & 9 & 10 & 11 & 12 \\
    \hline
    old & 1 & 4 & 13 & 44 & 160 & 536 & 1680 & 5041 & 15124 & 51120 & 172216 & 575664 \\
      & & & & & \cite{Heule2017} & \cite{Fredricksen} & \cite{Fredricksen} & \cite{ELIAHOU2012175} & 
    \cite{ELIAHOU2012175} & \cite{AbbottHanson} & \cite{AbbottHanson} & \cite{AbbottHanson} \\
    \hline
    Rowley & & & & & & & & 5286 & 17694 & 60320 & 201696 & 631840 \\
    \hline
    new & & & & & & & & & 17803 & 60948 & 203828 & 638548 \\
    \hline
\end{tabular}

\vspace{3ex}
\textbf{Table 2 - Comparison of lower bounds for weak Schur numbers}

\begin{tabular}{|*{13}{c|}}
    \hline
    \(n\) & 1 & 2 & 3 & 4 & 5 & 6 & 7 & 8 & 9 & 10 & 11 & 12 \\
    \hline
    old & 2 & 8 & 23 & 66 & 196 & 582 & 1740 & 5201 & 15596 & 51520 & 172216 & 575664 \\
      & & & & & \cite{ELIAHOU2012175} &\cite{EliahouBook} & \cite{Rafilipojaona} & \cite{Rafilipojaona} & 
    \cite{Rafilipojaona} & \cite{AbbottHanson} & \cite{AbbottHanson} & \cite{AbbottHanson} \\
    \hline
    Rowley & & & & & & 642 & 2146 & 6976 & 21848 & 70778 & 241282 & 806786 \\
    \hline
    new & & & & & & & & & 22536 & 71214 & 243794 & 815314 \\
    \hline
\end{tabular}

\end{center}

\par The main contribution of this article is a generalization of the concept of template to weak Schur numbers. 
This gives new lower bounds (and inequalities) for weak Schur numbers. This construction also includes as a 
special case an analogous for weak Schur numbers of Abbott and Hanson's construction.

\par We first explain Rowley's template-based construction in the context of Schur numbers and then give 
new templates, thus providing new lowers bounds and inequalities as well as showing that the growth rates 
for both Schur and Ramsey numbers exceed 3.28. Then, we generalize the concept of templates to weak Schur 
numbers and provide new lower bounds for weak Schur numbers. Finally, we analyze the significant difference 
between new lower bounds obtain with templates and the former lower bounds obtained by computer search 
and we provide evidence which indicate that the main assumption made in those articles removes the optimal 
partitions from the search space.

\begin{notation}
We denote by \([\![1,p]\!]\) the set of integers \(\{1, 2, ..., p\}\).
\end{notation}

\begin{notation}
For a partition of \([\![1, p]\!]\) in \(n\) subsets, we generally denote these subsets \(A_1, ..., A_n\). We 
also denote \(m_i = \min(A_i)\).
By ordering the subsets, we mean assuming that \(m_1 < ... < m_n\). However, if not specified we do not 
make this hypothesis since we do not always consider partitions in which every subset plays a symmetric role.
\end{notation}

\begin{definition}
We sometimes refer to a partitition as a colouring. The colouring associated to a partition \(A_1, ..., A_n\) of
\([\![1, p]\!]\) is the function \(f\) such that \(\forall x \in [\![1, p]\!], x \in A_{f(x)}\). Likewise, the partition 
associated to a colouring \(f\) of \([\![1, p]\!]\) with \(n\) colors is \(\forall c \in [\![1, n]\!], A_c = f^{-1}(c)\).
\end{definition}
