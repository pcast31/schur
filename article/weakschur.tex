\section{Weak Schur numbers}
\label{WeakSchur}

\qquad In this section, we generalize Rowley's constructions for weak Schur numbers \cite{RowleyWS} and give an analogous
for weak Schur numbers of Abbott and Hanson's construction for Schur numbers. By analogy with the previous section,
we then introduce WSF-templates as well as the sequence \(\WS^+(n)\). We find suitable templates and use them to
provide new lower bounds for weak Schur numbers.

\subsection{Lower bound for Weak Schur numbers using Schur and Weak Schur numbers}

\qquad Up to now, there was no equivalent for weak Schur numbers of Abbott and Hanson's construction for Schur numbers
\cite{AbbottHanson}. Here we give a general lower bound for weak Schur numbers as a function of both regular and
weak Schur numbers. The following theorem, inspired by Rowley's inequalities for \(\WS (n+1)\) and \(\WS (n+2)\),
was found and proved by Romain Ageron.

\begin{theorem}
Let \((p,q), (n,k) \in (\mathbb{N}^*)^2\). If there exists a partition of \([\![1,q]\!]\) into \(n\) weakly sum-free
subsets and a partition of \([\![1,p]\!]\) into \(k\) sum-free
subsets then there exists a partition of \([\![1,p(q+\left \lceil \frac{q}{2} \right \rceil + 1)+q]\!]\) into \(n+k\)
weakly sum-free subsets.
\end{theorem}

In particular, if we choose \(q = \WS (n)\) and \(p = S(k)\) in the last theorem, the next corollary follows.

\begin{corollary}
\( \forall (n,k) \in (\mathbb{N}^*)^2 \text{, } \WS (n+k) \geqslant S(k) \left (\WS (n) + \left \lceil \frac{\WS (n)}{2}
\right \rceil +1 \right) + \WS (n)\)
\end{corollary}

\begin{remark}
This can be seen as an equivalent for weak Schur numbers of Abott and Hanson's construction for Schur numbers.
\end{remark}

\begin{remark}
This formula includes the results of Rowley \cite{RowleyWS} as a special case. For \(n>2\), this formula does not give
new lower bounds.
\end{remark}

\begin{remark}
In the above ineqality, a "\(+1\)" can be added to the lower bound if \(\WS (n)\) is odd (more generally if \(q\) is
odd in the theorem). However, it makes the proof less clear and it is never useful in practice.
\end{remark}

Given that this theorem will appear as a particular case of a more general theorem after the introduction of
templates for weak Schur numbers, we only give here an intuitive explanation of the above theorem; a formal
proof can be found in the appendix.

Let \((p, q) \in \mathbb{N}^2\) such that there exists a partition of \([\![1,q]\!]\) into \(n\) weakly sum-free
subsets and a partition of \([\![1,p]\!]\) into \(k\) sum-free subsets. Let \(a \in \mathbb{N}\) with \(a > q\)
and let's try to build a colouring of \([\![1, ap + q]\!]\) into \(n + k\) weakly sum-free subsets. Let
\(l = a - b - 1\), \(r \in [\![1,q]\!]\) and \(w = a - l - r - 1 = b - r\).

First, put the integers of \([\![1, ap + q]\!]\) in the following table (with \(a\) columns and \(p + 1\) lines) and divide it into 3 blocks (the columns are numbered from \(-l\) to \(+q\)):

\begin{itemize}
	\item \(\mathcal{T}\) (the "tail"): the integers from 1 to \(q\). NB: this is line number 0.
	\item \(\mathcal{R}\) (the "rows"): the integers in columns \(-l\) to \(+r\) (excluding  \(\mathcal{T}\)).
	\item \(\mathcal{C}\) (the "columns"): the integers in the last \(w\) columns (excluding  \(\mathcal{T}\)).
\end{itemize}

Like for SF-templates, \(\mathcal{R}\) and \(\mathcal{C}\) play the role of security zones for each other. Note that with
this numbering of columns, the column of the sum of two numbers is the only integer in \([\![-l,q]\!]\) equal to two the
sum of the columns modulo \(a\).

\renewcommand{\arraystretch}{2}
\begin{center}
\[
\begin{array}{|c|c|c|c|c|c|c|c|c|>{\columncolor{blue}}c|>{\columncolor{blue}}c|>{\columncolor{blue}}c|>{\columncolor{green}} c|}
\cline{6-9}
   \multicolumn{5}{c|}{} & \cellcolor{green} 1 & \cellcolor{green} 2 & \cellcolor{blue} ... &\cellcolor{green} r & r + 1 & ... & b - 1 &  b \\
\hline
   \cellcolor{red} a - l & \cellcolor{red} a- l + 1 & \cellcolor{red} ... & \cellcolor{red} a - 1 & \cellcolor{red} a & \cellcolor{red} a + 1 & \cellcolor{red} ... &\cellcolor{red} a + r - 1 & \cellcolor{red} a + r & a + r + 1 & ... & a + b - 1 & a+ b \\
\hline
\cellcolor{yellow} 2 a - l &\cellcolor{yellow}...&\cellcolor{yellow}...&\cellcolor{yellow}...&\cellcolor{yellow} 2 a &\cellcolor{yellow} ... &\cellcolor{yellow} ... &\cellcolor{yellow} ... &\cellcolor{yellow} 2 a + r & ... & ... & ... & 2 a + b \\
\hline
...\cellcolor{yellow}&...\cellcolor{yellow}&...\cellcolor{yellow}&...\cellcolor{yellow}&...\cellcolor{yellow}&...\cellcolor{yellow}&...\cellcolor{yellow}&...\cellcolor{yellow}&...\cellcolor{yellow}&...&...&...&...\\
\hline
\cellcolor{red}...&\cellcolor{red}...&\cellcolor{red}...&\cellcolor{red}...&\cellcolor{red}...&\cellcolor{red}...&\cellcolor{red}...&\cellcolor{red}...&\cellcolor{red}...&...&...&...&...\\
\hline
p a - l & ... & ... & ... &p a & ... & ... & .... & p a + r & ... & ... & ... & p a + b\\
\hline
\end{array}
\]
\end{center}

\underline{\textbf{\(\mathcal{T}\) block}} \\
We color this block using the weakly sum-free colouring of \([\![1,q]\!]\) with colors \(1, ..., n\).

\underline{\textbf{\(\mathcal{R}\) block}} \\
In this block,  we use the colors \(n + 1, ..., n + k\). We colour an integer \(x\) according to its line number (written \(\lambda(x)\)).
For every \(x \in \mathcal{R}\), we colour \(x\) with \(n + c\) where \(c\) is the colour of \(\lambda(x)\) in the sum-free colouring of  \([\![1,p]\!]\).
Let \((x, y) \in \mathcal{R}^2\). The cases are twofold.

\begin{itemize}
	\item \underline{\(\lambda(x+y) = \lambda(x) + \lambda(y)\)} \\
	In this case, we use the sum-free property of the colouring of \([\![1,p]\!]\) (in block \(\mathcal{C}\), we only use colours \(1, ..., n\)).

	\item \underline{\(\lambda(x+y) \neq \lambda(x) + \lambda(y)\)} \\
	In this case, we do not have information about the colour of \(\lambda(x+y)\). Thereby, we want to have \(x+y \in \mathcal{C}\).
	A simple solution is to limit the horizontal movement, that is if the sum changes line, not to move too far so that it stays in \(\mathcal{C}\).
	There, the maximal displacement to the left (resp. to the right) is \(2l\) (resp. \(2r\)). Not crossing entirely \(\mathcal{C}\) by going to the left
	is then expressed as \(-2l > -a + r\). Likewise, not going to far to the right is expressed as \(2r < a - l\). It can then be written as
	 \(\text{max}(l, r) \leqslant w\).
\end{itemize}

\underline{\textbf{\(\mathcal{C}\) block}} \\
In this block,  we use colors \(1, ..., n\). We colour an integer \(x\) according to its column number (written \(\tilde{\pi}(x)\), it is linked to the
projection on the first line, written \(\pi\), by the relation \(\tilde{\pi}(x) = \pi(x) - a\)). A simple solution is to colour \(x\) with the same colour
as \(\tilde{\pi}(x)\) in the weakly sum-free colouring of \([\![1,q]\!]\). As long as \(2b \leqslant a + r\) (not going two far to the right) and there
is no \(x \in \tilde{\pi}(\mathcal{C})\) such that \(2x \in \tilde{\pi}(\mathcal{C})\) (so that we do not have a sum in \(\mathcal{C}\) when
taking two numbers in the same column), the colours \(1, ..., n\) are sum-free.

In particular, taking \(w = l = \lceil \frac{q}{2} \rceil\) and \(r = \lfloor \frac{q}{2} \rfloor\) works, thus obtaining the above theorem.\\
\par
As in the previous section, we now introduce WSF-templates and the sequence \(\WS^+\) in order to generalize the above construction.

\subsection{Definition of \(\WS^+\)}

\qquad In this subsection, we introduce WSF-templates and define the objects and prove the results needed
for the general theorem regarding templates for weak Schur numbers.

\begin{definition}

Let \((a,b) \in (\mathbb{N}^*)^2\),\(a>b\), \(n \in \mathbb{N}^*\), we will define \(\pi_{a,b}\) the projection:
\[ \pi_{a,b}:x \longmapsto (Id+a\mathds{1}_{ [\![0,b]\!]})(x \mod a))\]
\end{definition}

We will note  the projection \(\pi\) and not \(\pi_{a,b}\) when there is no doubt about the \(a\) and \(b\) we use.

\begin{remark}
\(\pi\) is the projection on the first line mentioned in the intuitive explanation.
\end{remark}

\begin{proposition}
Let x \(\in [\![1,b]\!]\), let  \(y \in \mathbb{N}^*\) such that \(x+\pi(y)\leqslant a+b\), then we have: \(\pi(x+y)=x+\pi(y)\)
\end{proposition}

\textsc{Proof :}Let  \(x\in [\![1,b]\!]\), let  \(y\in \mathbb{N}^*\) such that \(x+\pi(y)\leqslant a+b\)

if \(x+\pi(y)< a:\)we remark that \(\pi(y)>b\) and therefore \(x+\pi(y)>b\):

\begin{align*}
 \pi(x+y) & = (Id+a\mathds{1}_{ [\![0,b]\!]})(x+y \mod a)\\
& = (Id+a\mathds{1}_{ [\![0,b]\!]})(x+\pi(y) \mod a)\text{ since } \pi(y)=y \mod a \\
& =x+ \pi(y)
\end{align*}

if \(x+\pi(y)\geqslant a\):
\begin{align*}
 \pi(x+y) & = (Id+a\mathds{1}_{ [\![0,b]\!]})(x+y \mod a)\\
& = (Id+a\mathds{1}_{ [\![0,b]\!]})(x+\pi(y)\mod a)\text{ since } \pi(y)=y \mod a \\
& = (Id+a\mathds{1}_{ [\![0,b]\!]})(x+\pi(y) - a) \\
& = x+\pi(y) - a +a\mathds{1}_{ [\![0,b]\!]})(x+\pi(y) - a) \\
& = x+\pi(y) - a +a \text{ \quad since } x+\pi(y) - a \in [\![0,b]\!] \\
& = x+\pi(y)
\end{align*}


\begin{proposition}
Let \((x,y)\in (\mathbb{N}^*)^2\), \(\pi(\pi(x)+\pi(y))=\pi(x+y)\)
\end{proposition}

\textsc{Proof :}Let \((x,y)\in (\mathbb{N}^*)^2\),

\begin{align*}
 \\\pi(\pi(x)+\pi(y)) & = (Id+a\mathds{1}_{ [\![0,b]\!]})(\pi(x)+\pi(y)\text{ mod a})\\
& = (Id+a\mathds{1}_{ [\![0,b]\!]})((Id+a\mathds{1}_{ [\![0,b]\!]})(x \mod a)+(Id+a\mathds{1}_{ [\![0,b]\!]})(y \mod a)\ mod a) \\
& = (Id+a\mathds{1}_{ [\![0,b]\!]})(((x \mod a) + (y \mod a)) \mod a)\\
& =(Id+a\mathds{1}_{ [\![0,b]\!]})( x+y \mod a)\\
& =\pi(x+y)
\end{align*}


\begin{definition}

Let \((a,b) \in (\mathbb{N}^*)^2\), \(a>b\), let \(n \in\mathbb{N}\) we will define \(\lambda_{a,b}\) the projection:
\[ \lambda_{a,b}:x \longmapsto 1+ \left\lfloor\dfrac{x-b-1}{a}\right\rfloor\]
\end{definition}

We will note  the projection \(\lambda\) and not \(\lambda_{a,b}\) when there is no doubt about the \(a\) and \(b\) we use.

\begin{remark}
In the following theorem, \(\lambda\) is the function which maps an element x to its line number, as mentioned in the intuitive explanation.
\end{remark}

\begin{proposition}
Let \((a,b)\in (\mathbb{N}^*)^2\), \(a>b\), let  \(x\in \mathbb{N}^*\), \(x=a\lambda(x)+\pi(x)-a\)
\end{proposition}
\textsc{Proof :}Let \((a,b)\in (\mathbb{N}^*)^2\), \(a>b\), let  \(x\in \mathbb{N}^*\),
\\\\ \(a\lambda(x)+\pi(x)-a=a\left\lfloor\dfrac{x-b-1}{a}\right\rfloor+(x \mod a)+\mathds{1}_{ [\![0,b]\!]}(x \mod a)\)

if \(x \mod a>b\): \(a\lambda(x)+\pi(x)-a=a\left\lfloor\dfrac{x}{a}\right\rfloor+x \mod a=x\)

if \(x \mod a \leqslant b\):
\(a\lambda(x)+\pi(x)-a=a \left( \left \lfloor \dfrac{x}{a} \right \rfloor - 1 \right)+\text{x mod a}+a=x\)


\begin{definition}
Let \( (p,n,b) \in (\mathbb{N}^*)^3\), Let \((A_1,...,A_n)\) a partition of  \([\![1, p]\!]\).
This partition is said to be a b-weakly-sum-free template (b-WSF-template) of \(n\) colors and lenght \(p\) when:
\\\\
\underline{\(\forall i \in [\![1, n]\!], \quad A_i\) is weakly-sum-free}
\\\\
\underline{\(\forall i \in [\![1, n]\!], \quad A_i\backslash [\![1, b]\!]\) is sum-free}
\\\\
\underline{For \(A_n\) (the special subset):} \quad \(\forall (x,y) \in A_n^2,\)
\\
\[if \quad x+y>b+2(p-b),\quad x+y-2(p-b)\notin A_n\]
\\
\underline{For the others subsets:}\quad \(\forall i \in [\![0,n-1]\!], \forall(x,y) \in A_i^2,\)
\\
\[
x+y>p \implies \pi(x+y) \notin A_i
\]
\end{definition}


\begin{remark}
	Please note that the special color \(n\) is not necessarily the last color by order of appearance.
\end{remark}

\begin{definition}
Let \( (k,b) \in (\mathbb{N}^*)^2\). If there exist \(p\) such that exists a partition of \([\![1, p]\!]\) into \(k\)
subsets which is a \(b\)-WSF-template of \(k\) colors and lenght \(p\), we note:
\\\\\(\WS_b^+=-b+\max \{p\in \mathbb{N}^*\)/ there exists a partition of \([\![1, p]\!]\) into \(k\) subsets which is a
\(b\)-WSF-template of \(k\) colors and lenght \(p\) \}
\\\\
If this \(p\) does not exist, we set \(\WS_b^+= 0\)
\end{definition}

\begin{definition}
Let \( n \in \mathbb{N}^*\), we define \(\WS^+(n)=\max_{b\in \mathbb{N}^*} \{\WS_b^+(n)\}\)
\end{definition}

\begin{proposition}
	Let \(n \in [\![2, +\infty]\!]\), we have :
	\[
	\frac{3}{2} \WS (n-1)+1 \leqslant \WS^+(n) \leqslant \WS (n)
	\]
\end{proposition}

\textsc{Proof :} The lower bound comes from the analogous of Abott and Hanson's construction for weak Schur numbers.
The upper bound comes from the fact that a WSF-template of length \(p\) with \(n\) colors is also a partition of
\([\![1, p]\!]\) into \(n\) sum-free subsets.

\begin{remark}
	\(\WS^+\) and \(\WS\) have the same asymptotic growth rate.
\end{remark}

We now proceed to state and prove the main result of this article.


\subsection{Construction of weak Schur partitions using WSF-templates}

\begin{theorem}
Let \((q,n,b) \in (\mathbb{N}^*)^3\), let \( (p,k) \in (\mathbb{N}^*)^2\). If there exists a partition of \(k\)
sum-free subsets of \([\![1,p]\!]\) and a partition of \(n\) subsets \((A_1,....,A_n)\) of \([\![1, q]\!]\) which is a
\(b\)-WSF-template of \(n\) colors and lenght \(q\),
 then there exists a partition of \([\![1, b+p \times (q-b)]\!]\) into \((k+n-1)\) weakly sum-free subsets.
\end{theorem}

In particular, if we choose \(p = S(k)\) and \(q = \WS^+(n)\) in the last theorem, the next corollary follows.

\begin{corollary}
\( \forall (n,k) \in (\mathbb{N}^*)^2 \text{, } let ~ b_{max}=max \{b\in \mathbb{N}^*/ \WS_b^+(n+1) = \WS^+(n+1)\},\) \[
\WS(n+k) \geqslant S(k) \WS^+(n+1)+b_{max}\]
\end{corollary}

\begin{remark}
In the SF-template construction for Schur numbers, the additive constant comes from the fact that the special color does
not necessarily appear right at the begining of the repeating pattern. Likewise, \(b_max\) can actually be replaced by \\
\[\max \left\{ b + \min (A_{n+1} \backslash [\![1, b ]\!]) - 1~|~ \WS_b^+(n+1) = \WS^+(n+1) \right\}\]
\end{remark}

\begin{remark}
Acutally, like for SF-templates, the additive constant of a WSF-template (in the form given by the above remark) can be improved
by weakening the hypotheses made on the last row. The principle behind it is the same as in the analogous proposition for SF-templates.
\end{remark}

\begin{proposition}
Let \((b, k, p) \in \mathbb{N}^*)^3\) and let \(f\) be a colouring associated to a \(b\)-WSF-template of length \(p\) with \(k\) colors. Let
\(c \in \mathbb{N}\) (\(c = \min (A_{k+1} \backslash [\![1, b ]\!]) - 1\) works) and assume there there exists a colouring \(g\) of
\([\![b + 1, b + c]\!]\) with \(k\) colors such that:

\begin{itemize}
	\item \(\forall c \in [\![1, k]\!], \forall (x, y) \in  [\![1, a + b]\!] \times  [\![b + 1, a + b]\!], (f(x) = f(y) \text{ and } \pi(x + y) \leqslant b + c)
	\implies g(\pi(x + y)) \neq f(x)\)
	\item \(\forall c \in [\![1, k]\!], \forall (x, y) \in  [\![1, a + b]\!] \times  [\![b + 1, b + c]\!],  (f(x) = g(y) \text{ and } \pi(x + y) \leqslant b + c)
\implies g(\pi(x + y)) \neq f(x)\)
\end{itemize}

Then, for every \(n \in \mathbb{N}^*\), by using on the last row the colouring \(i \longmapsto g(i - p S(n)\), we have\\
\[ \WS(n+k) \geqslant \WS^+(k+1) S(n) + b + c\]
\end{proposition}

\textsc{Proof:} Let \((q,n,b) \in (\mathbb{N}^*)^3\), let \( (p,k) \in (\mathbb{N}^*)^2\), let \(a=q-b\). \\
We denote by \(g\) the colouring associated to the partition of \([\![1,q]\!]\) and \(h\) the
one associated to the partition of \([\![1,p]\!]\).
\[ g : [\![1,q]\!] \longrightarrow [\![1,n]\!] \text{ and } (A_{g^{-1}(1)},...A_{g^{-1}(q)})\text{ is a \(b\)-WSF-template.}
\]
\[h : [\![1,p]\!] \longrightarrow [\![1,k]\!] \text{ and } \forall (x,y) \in [\![1,q]\!]^2 \text{, } h(x) = h(y)
\Longrightarrow h(x+y) \neq h(x)
\]

Define \( f : [\![1,b+pa]\!] \longrightarrow [\![1,n]\!] \) as follows:

\begin{itemize}
\item if  \(x\leqslant b \text{ (we will note }x \in \mathcal{T}): f(x)=g(x)\)
\item if \( x \in [\![1,b+pa]\!] \text{ and } \pi(x) \notin A_n \text{ (we will note x} \in \mathcal{C}): f(x)=g(\pi(x))\)
\item if \( x \in [\![1,b+pa]\!]  \text{ and }\pi(x) \in A_n \text{ (we will note x} \in \mathcal{R}): f(x)=n-1+h(\lambda(x))\)
\end{itemize}

\(f\) is well defined since \(\pi\) is defined for \(x>b\) and \(\forall x \in \![1,b+pa]\!], f(x)\leqslant n+k-1\) because \(h(\lambda(x))\leqslant k\)

We have parted the integers of \([\![1,b+pa]\!]\) in three disjoints subsets \(\mathcal{T},\mathcal{C} \text{ and } \mathcal{R}\).

We have to verify that f induces weakly-sum-free templates:\\
\\
\underline{if \((x,y) \in \mathcal{T}^2\) such that \(f(x)=f(y)\), \(x \neq y\), then \(f(x+y)\neq f(x)\)  :}\\
\\\(x+y<a+b\) since \(b<a\) and \(g(x)=f(x)=f(y)=g(y)\).
\\Hence \(f(x+y)=g(x+y)\neq g(x)=f(x)\)
\\\\
\underline{if \((x,y) \in \mathcal{T} \times \mathcal{C}\) such that \(f(x)=f(y)\), \(x \neq y\), then \(f(x+y)\neq f(x)\)  :}\\
We distinguish two cases:


\begin{itemize}
\item If \(x+\pi(y)\leqslant a+b\)
\\\(g(x)=f(x)=f(y)=g(\pi(y)\). Hence \(g(x)\neq g(x+\pi(y))=g(\pi(x+y))\) (qv previous proposition)
\\if \(g(\pi(x+y))=n, f(x+y)\geqslant n > f(x)\)
\\else, \(f(x+y)=g(\pi(x+y))\neq g(x)=f(x)\)
\item If \(x+\pi(y)> a+b\), \(x+y>a+b\) and by definition of \(g\), \(g(\pi(x+y))\neq g(x)\)
\\if \(g(\pi(x+y))=n, f(x+y)\geqslant n > f(x)\)
\\else, \(f(x+y)=g(\pi(x+y))\neq g(\pi(x))=f(x)\)
\end{itemize}


\underline{if \((x,y) \in \mathcal{T} \times \mathcal{R}\) such that \(f(x)=f(y)\), \(x \neq y\), then \(f(x+y)\neq f(x)\)  :}\\
\\Then, \(f(x)=f(y)=n\). We distinguish two cases:


\begin{itemize}
\item If \(\lambda(y)=\lambda(x+y)\),
\\\(g(x)=g(\pi(y))=n \text{ since } g(y)=g(\pi(y))\)
\\Therefore \(g(\pi(x+y))=g(x+\pi(y)) \neq g(x)=n\) (qv previous proposition)
\\Hence \(f(x+y)=g(\pi(x+y))\neq n\)
\item If  \(\lambda(y)\neq \lambda(x+y)\), \(\lambda(y)+1= \lambda(x+y)\)\\
\(n=f(y)=n-1+h(\lambda(y))\). Hence \(h(\lambda(y))=1\).
\\Moreover \(h(1)=1\), therefore \(h(\lambda(y)+1) \neq 1\)
\\if \(\pi(x+y) \in A_n\), \(f(x+y)=n-1+h(\lambda(x+y))>n\)
\end{itemize}

\underline{if \((x,y) \in \mathcal{C}^2\) such that \(f(x)=f(y)\), \(x \neq y\), then \(f(x+y)\neq f(x)\)  :}\\
\\Then  \(g(\pi(x))=f(x)=f(y)=g(\pi(y))\). We distinguish two cases:

\begin{itemize}
\item If \(\pi(x)+\pi(y)>q\), \(g(\pi(\pi(x)+\pi(y)) \neq g(\pi(x))\) (qv previous proposition)
\\Hence \(g(\pi(x+y))= g(\pi(\pi(x)+\pi(y))\neq g(\pi(x))\)
\\if \(g(\pi(x+y))=n, f(x+y)\geqslant n > f(x)\)
\\else, \(f(x+y)=g(\pi(x+y))\neq g(\pi(x))=f(x)\)
\item If  \(\pi(x)+\pi(y)\leqslant q\), \(g(\pi(\pi(x)+\pi(y)) \neq g(\pi(x))\) since \(g\) is sum-free for \(x>b\)
\\if \(g(\pi(x+y))=n, f(x+y)\geqslant n > f(x)\)
\\else, \(f(x+y)=g(\pi(x+y))=g(\pi(\pi(x)+\pi(y))\neq g(\pi(x))=f(x)\)


\end{itemize}



\underline{if \((x,y) \in \mathcal{C} \times \mathcal{R} \), \(f(x)\neq f(y)\)}\\

\underline{if \((x,y) \in (\mathcal{R})^2\) such that \(f(x)=f(y)\), \(x \neq y\), then \(f(x+y)\neq f(x)\)  :}\\
\\Let \(r(x)=\pi(x)-a\) and \(r(y)=\pi(y)-a\),
\\We proved that \(x=a\lambda(x)+\pi(x)-a\), therefore \(x=a\lambda(x)+\pi(x)\)
\\\(x+y=a(\lambda(x)+\lambda(y))+r(x)+r(y)\). We distinguish three cases:

\begin{itemize}
\item If \(r(x)+r(y) \in [\![b-a+1,b]\!]\), \(h(\lambda(x))=f(x)+1-n=f(y)+1-n=h(\lambda(y))\)
Hence, \(h(\lambda(x)+\lambda(y)) \neq h(\lambda(x))\).
\begin{align*}
 \\\lambda(x+y) & =1+\left\lfloor\dfrac{a(\lambda(x)+\lambda(y))+r(x)+r(y)-b-1}{a}\right\rfloor+1\\
& = \lambda(x)+\lambda(y)+\left\lfloor\dfrac{r(x)+r(y)-b-1}{a}\right\rfloor+1 \\
& = \lambda(x)+\lambda(y) -1 +1 \text{ since } r(x)+r(y) \in [\![b-a+1,b]\!] \\
& =\lambda(x)+\lambda(y)
\end{align*}


\begin{align*}
 \\\text{if \(f(x+y) \geqslant n\), }f(x+y) & =n-1+h(\lambda(x+y))\\
& =n-1+h(\lambda(x)+\lambda(y)) \\
& \neq n-1+h(\lambda(x))\\
& =f(x)
\end{align*}


\item If \(r(x)+r(y)>b\), \(\pi(x)+\pi(y)>2a+b\)
\\Since \(g\) is a \(b\)-WSF template, \(g(\pi(\pi(x)+\pi(y))) \neq n\)
\\Therefore, \(g(\pi(x+y)) \neq n\) ie \(x+y \in \mathcal{C}\)
\\Hence \(f(x+y) <n\leqslant f(x)\)
\item If \(r(x)+r(y) \leqslant b - a\), \(\pi(x)+\pi(y)\leqslant b+a\)
\\Since \(g\) is sum-free for \(x>b\), since \(g(\pi(x)) = g(\pi(y))=n\), \(g(\pi(x+y)) \neq g(\pi(x))=n\)
\\Hence, \(f(x+y) <n\leqslant f(x)\)
\end{itemize}

\begin{corollary}
The general lower bound for weak Schur numbers in function of both regular and weak Schur numbers can be seen as a particular
case of WSF-template in the same way Abott and Hanson's construction can be seen as a particular case of SF-template.
\end{corollary}

\begin{remark}
The WSF-templates can actually be fine-tuned further. However, it gives only minor improvements (most likely only additive
constant) at the cost of dramatically increasing the size of the search space while. Therefore, it does not seem relevant to
use this sophistications given that we could not even find good WSF-templates with 5 colors using a computer (here good
means better than those obtain by combining smaller templates).

\par These modifications work as follows. One may notice that the first row (excluding the "tail") has constraints that other rows
do not have because of the tail, especially if the special color appears in the tail as well. Thus allowing to have a colouring on the
first row different from the colouring of the other rows would weaken the constraints. Acutally, one may even go further by
noticing that on the one hand the first (ordered) color of the sum-free partition used for the extension procedure has more
more constraints than the other colors of the sum-free partition since the first row is of this color and is more constrained than
the other rows, but that on the other hand it has more degrees of freedom than the other colors of the sum-free partition since in
the sum-free partition there cannot be two consecutive numbers of this color. As a result, it removes  some constraints imposed
 by the first row on the other rows.

\par To sum up, one can look for a generalised WSF-template that uses a special colouring for the tail and the first row, a
colouring dedicated to the rows whose number is not 1 but is in the first color in the sum-free partition, a colouring for all
the other rows and a special colouring for the last numbers (as previously explained for the improvement of the additive
constant of WSF-templates).
\end{remark}

We also have a similar theorem where only \(\WS^+\) is involved.

\begin{theorem}
	Let \((n,k), (p,q) \in (\mathbb{N}^*)^2\). If there exists a \(SF\)-template of \(k+1\) colors and lenght \(p\),
	and \(WSF\)-template of \(n\) color and lenght \(q\), then there exists \(WSF\)-template of \((n+k)\) and lenght \(pq\).
\end{theorem}

And the associated inequality :

\begin{corollary}
	Let \(n, k \in \mathbb{N}^*\), we have \\
	\[ \WS^+(n+k) \geqslant S^+(k+1) \WS^+(n) \]
\end{corollary}

\textsc{Proof :} The idea is the same as in the previous theorem. The only difference is the \(SF\) property inherited
from the second \(SF\)-template.



\subsection{New lower bounds for Weak Schur numbers}

\qquad Having found suitable templates, which can be found in the appendix, with a SAT solver, we claim that for all \(n \in
\mathbb{N}^*\):

\[
\WS(n+1) \geqslant 4S(n) + 1
\]
\[
\WS(n+2) \geqslant 13S(n) + 8
\]
\[
\WS(n+3) \geqslant 42S(n) + 24
\]
\[
\WS(n+4) \geqslant 132S(n) + 26
\]

The first two inequalities were found by Rowley, they are detailed in [2]. The third inequality is optimal and was found with a SAT solver.
It uses the first sophistication explained in the previous subsection in order to add the last number in the first color.
As for the fourth inequality, it was obtained by combining an optimal SF-template of length 33 with a WSF-template of length 4.
The best template we could get with a computer search gives the inequality \(\WS(n+4) \geqslant 127 S(n) + 68\).
It was also found with the SAT solver. In order to reduce the search space, we only looked for WSF-templates of
5 colors which start with a good \(\WS(4)\) partition. However, this approach most likely prevents us from finding the best WSF-templates
as we explain in the next subsection for weakly sum-free partitions. We highly suspect that there exists more efficient WSF-templates
with \(n \geqslant 5\) colors. One may try to go over a different search space using a Monte-Carlo method, as in \cite{Bouzy2015AnAP}.
This could be the suject of a future work. Further details about the encoding as a SAT problem can be found in the
\hyperref[SAT]{SAT section}.

Like in 3.3, we compute the lower bounds given by the previous inequalities for \( n \in [\![8,15]\!] \). The best lower bound
for each integer is highlighted.

\begin{center}
\begin{tabular}{|*{5}{c|}}
    \hline
	n & 8 & 9 & 10 & 11 \\
	\hline
	\(4S(n-1) + 2 \) & 6722 & 21146 & \cellcolor{yellow} 71214 & \cellcolor{yellow} 243794\\
	\hline
	\(13S(n-2) + 8 \) & \cellcolor{yellow} 6976 & 21848 & 68726 & 231447\\
	\hline
	\(42S(n-3) + 24 \) & 6744 & \cellcolor{yellow} 22536 & 70584 & 222036 \\
	\hline
	\(127S(n-4) + 68 \) & 5656 & 20388 & 68140 & 213428\\
	\hline
	\hline
	n & 12 & 13 & 14 & 15 \\
	\hline
	\(4S(n-1) + 2 \) & \cellcolor{yellow} 815314 & 2554194 & 8045162 & \cellcolor{yellow} 27061154\\
	\hline
	\(13S(n-2) + 8 \) & 792332 & \cellcolor{yellow}2649772 & 8301132 & 26146778 \\
	\hline
	\(42S(n-3) + 24 \) & 747750 & 2559840 & \cellcolor{yellow} 8560800 &  25886224 \\
	\hline
	\(127S(n-4) + 68 \) & 671390 & 2261049 & 7740464 & 25886224 \\
	\hline
\end{tabular}
\end{center}

With \( S(9) \geqslant 17803 \), we found a new lower bound for \(\WS(10)\) using Rowley's inequality.
Moreover, the third inequality gives new lower bounds for \(\WS(9)\) and \(WS(14)\).


\subsection{Conclusion on WSF-templates}

\qquad In this section, we first gave a new construction which can be seen as an equivalent for weakly sum-free partitions of Abott
and Hanson's construction for sum-free partitions. We then  introduced WSF-templates and generalized this construction. This
allows us to find new lower bounds and new inequalities for weak Schur numbers. One may notice the significant difference
between the former lower bounds for weak Schur numbers obtained by conducting a computer search and the new lower bounds
obtained with WSF-templates (including Rowley's two inequalities). In the next section, we try to analyze this phenomenon.
