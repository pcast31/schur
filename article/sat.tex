\section{About the construction of lower bounds for weak Schur numbers using a computer}

In this section, we first reframe the question of the existence of (weakly) sum-free partitions as a boolean
satisfiability (SAT) problem. We then provide evidence which indicates that the main assumption made by papers which
found the previous best known lower bounds for weak Schur numbers may not be correct. Finally, we obtain stronger
results than those previously known for \(WS(5)\) while gaining several orders of magnitude in computation time by
giving additional information to the SAT solver without losing in generality. In this section, we assume that the
subsets are ordered.


\subsection{Reformulation as a SAT problem}
\label{SAT}

We encode the existence of (weakly) sum-free partitions as propositional formulae like in
\cite{Heule2017} and then use
SAT solvers to determine whether these formulae are satisfiable.

\begin{definition}
A \textit{literal} is either a variable \(v\) (a positive literal) or the negation \(\bar{v}\) of a variable \(v\) (a
negative literal) where \(v\)
takes a truth value: \(true\) or \(false\). A \textit{clause} is a disjunction of literals and a \textit{formula} is a
conjunction of clauses: it
is a propositional formula in \textit{conjonctive normal form} (CNF).
\end{definition}

\begin{definition}
An \textit{assignment} is a function from a set of variables to the truth values \(true\) (1) and \(false\) (0). A
literal \(l\) is
\textit{satisfied} (\textit{falsified}) by an assignment \(\alpha\) if l is positive and \(\alpha(var (l)) = 1\)
(resp. \(\alpha(var (l)) = 0\)) or if it is negative and \(\alpha(var (l)) = 0\) (resp. \(\alpha(var (l)) = 1\)). A
clause is \textit{satisfied}
by an assignment \(\alpha\) if it contains a literal that is satisfied by \(\alpha\). Finally, a formula is
\textit{satisfied} by an assignment
\(\alpha\) if all its clauses are satisfied by \(\alpha\). A formula is \textit{satisfiable} if there exists an
assignment that satisfies it;
otherwise it is \textit{unsatisfiable}.
\end{definition}

We then encode the existence of a partition of \([\![1,p]\!]\) in \(k\) weakly sum-free subsets as follows: for every
integer
\(i \in [\![1,p]\!]\), take \(k\) variables \(x^{(i)}_{1}, ..., x^{(i)}_{k}\) and for every \(\forall c \in [\![1,k]\!],
x^{(i)}_c = 1 \iff i \in A_c\).
The corresponding clauses are:

\begin{itemize}
\item \textbf{sumfree:} \(\forall c \in [\![1,k]\!], \forall (i, j) \in [\![1,p]\!]^2, (i \neq j ~ \text{and} ~ i + j
\leq n) \implies \lnot x^{(i)}_c
\lor  \lnot x^{(i)}_c \lor \lnot x^{(i+j)}_c\)
\item \textbf{union:} \(\forall i \in [\![1,p]\!], x^{(i)}_1 \lor ... \lor x^{(i)}_k\)
\item \textbf{disjoint:} \(\forall i \in [\![1,p]\!],\forall (c_1, c_2) \in [\![1,k]\!]^2, c_1 \neq c_2 \implies \neg
x^{(i)}_{c_1} \lor \neg x^{(i)}_{c_2}\)
\end{itemize}

In the above formula, every color plays a symmetric role. Hence the search space can reduced by \(k!\) by ordering the
subsets, that is by
enforcing that \(m_1 < ... < m_k\). The corresponding clauses are: \linebreak
\textbf{symmetry breaking:} \(x^{(1)}_1 = 1\) and \(\forall c \in [\![2,k-1]\!], \forall i \in [\![1,WS(c - 1)+1]\!],
x^{(1)}_c \lor ... \lor x^{(i)}_c \lor \neg x^{(i+1)}_{c+1}\)

\begin{remark}
For a given problem, it can be interesting to try out different SAT solvers because the relative performance can vary
significantly according to the problem.
For instance, we used two different SAT solvers in the next two next subsections.
\end{remark}

\begin{remark}
Using a parallel SAT solver usually reduces the computation time, especially when trying to show that a formula is
unsatisfiable. However, most of the
parallel SAT solver do not have a deterministic behaviour and it can results in a strong variation of running times.
\end{remark}


\subsection{The search space previously used in computer search for lower bounds may not contain the optimal partitions}
Rowley's new lower bound for \(WS(6)\) (642) \cite{RowleyWS} was a quite significant improvement upon
the former
best known lower bound (582) \cite{EliahouBook}. This previous lower bound was found using a computer
(often with Monte-Carlo methods) and by making the
assumption that a good partition for \(WS(n+1)\) starts with a good partition for \(WS(n)\) which is true for small
values of \(n\).
Therefore, one may wonder whether the limiting factor are the assumptions or the methods used to search for partitions.
It appears that the search space
induced by these assumptions does not contain the optimal partitions.

\begin{computational theorem}
There is no weakly sum-free partition of \([\![1,583]\!]\) in 6 parts such that:
\begin{itemize}
	\item \(m_5 \geqslant 66\)
	\item \(m_6 \geqslant 186\)
	\item \([\![210,349]\!] \subset A_6\)
\end{itemize}
\end{computational theorem}

This result was obtained in 8 hours with the SAT solver plingeling \cite{Lingeling2017} on a 2.60 GHz Intel
i7 processor PC.
However, simply encoding the existence of such a partition as explained in the previous subsection would not result in a
reasonable
computation time. In order to help the SAT solver, we add additional information in the propositional formula. We did
not quantify the
speedup, but it most likely allowed us to gain several order of magnitude in computation time as we explain in the next
subsection.

\par
Every weakly sum-free partition of \([\![1,65]\!]\) in 4 subsets starts with the following sequence 1121222133. Then 11
is always
either in subset 1 or 3, 12 is always in subset 3 and so on. For every integer in \([\![1,65]\!]\), we computed in which
subset it can appear.
By using this constraints, we could then compute for every integer in \([\![1,185]\!]\), in which subset it can appear
in a weakly sum-free
partition of \([\![1,185]\!]\) which starts with a weakly sum-free partition of \([\![1,65]\!]\) in 4 subsets. Adding
these constraints to the
formula corresponding to the above theorem gives additional information to the SAT solver without losing in generality.

\par
The above theorem shows that the previous lower bound for \(WS(6)\) is optimal in the search space considered by the
papers which found it.
Therefore, finding a partition of \([\![1,n]\!]\) in 6 weakly sum-free subsets for some \(n \geqslant 590\) which does
not have a template-like structure
would be extremely interesting since it could give indications on a new search space for improving lower bounds with a
computer. More generally,
it questions the search space previously used for finding lower bounds for \(WS(n)\) with a computer. In particular, to
our knowledge every paper
that found the lower bound \(WS(5) \geqslant 196\) used this assumption. Therefore one may wonder
if this actually a good lower
bound. In the next subsection, we give properties that a partition of \([\![1,197]\!]\) in 5 weakly sum-free subsets has
to verify.


\subsection{Weak Schur number five}
As explained in the previous subsection, the search space used for showing that \(WS(5) \geqslant 196\) may not contain
optimal solution. In this subsection,
we give necessary conditions for a hypothetical partition of \([\![1,197]\!]\) in 5 weakly sum-free subsets using the
same type of methods as in the
previous subsection.

\begin{notation}
Let \(P\) be a predicate over weakly sum-free partitions. We denote by \(WS(n | P)\) the greatest number \(p\) such that
there exists a partition of
\([\![1,p]\!]\) in n weakly sum-free subsets which verifies P.
\end{notation}

\par
\cite{ELIAHOU2012175} verified with a SAT solver that there are no partition in 5 weakly sum-free subsets of
\([\![1,197]\!]\) with
\(A_5 = \{67, 68\} \cup [\![70,134]\!] \cup \{136\}\) in 17 hours and could not provide a similar result when only
assuming \(m_5 = 67\) even after several
weeks of runtime. By using the same method as above, we were able to verify that \(WS(5 | m_5 = 67) = 196\) in 0.5
seconds with the SAT solver glucose \cite{Glucose}
on a 2.60 GHz Intel i7 processor PC (we used the non-parallel version here for the sake of comparison but in the rest of
this subsection, we used the parallel version of glucose).
The additional information we gave to the SAT solver is that every partition of \([\![1,66]\!]\) in 4 weakly sum-free
subsets starts with a partition of
\([\![1,23]\!]\) in 3 weakly sum-free subsets (this can be checked in a few dozens of minutes with a SAT solver). Among
the 3 partitions of \([\![1,23]\!]\) in
3 weakly sum-free subsets, every number always appears in the same subset except for 16 and 17 which can appear in two
different subsets. We hardcoded
this external knowledge in the propositional formula which allowed us to gain several orders of magnitude in computation
time. We also give the stronger
following result.

\begin{computational theorem}
If there exists a partition of \([\![1,197]\!]\) in 5 weakly sum-free subsets then \(m_5 \leq\). \\
NB : on doit finir d'obtenir les valeurs.
\end{computational theorem}

More precisely, we verified the following results (\(max~m_5\) is the greatest value of \(m_5\) for which we have not
verified that \(WS(5 | m_5) \leq 196\)).

\begin{tabular}{| c | *{21}{ p{2mm} |}}
	\hline
	\(m_4\) & 4 & 5 & 6 & 7 & 8 & 9 & 10 & 11 & 12 & 13 & 14 & 15 & 16 & 17 & 18 & 19 & 20 & 21 & 22 & 23 & 24 \\
	\hline
	\(WS(4 | m_4) + 1\) & 55 & 59 & 60 & 59 & 59 & 60 & 60 & 60 & 60 & 64 & 63 & 64 & 61 & 64 & 63 & 65 & 65 & 65 & 65 & 66
	& 67 \\
	\hline
	\(max~m_5\) & 49 & 51 & 54 & 53 & 54 & 54 & & & & & & & & & & & & & & 57 & 53 \\
	\hline
\end{tabular}

To obtain these results, we once again provided additionnal information to the SAT solver. We also gave other types of
information to the SAT solver.
(pas encore fini)
