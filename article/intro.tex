\section{Abstract}

\qquad This paper offers new lower bounds for both Schur and weak Schur numbers. These results were produced by pushing forward
Rowley's "templates"-based approach for Ramsey numbers in 2020. Finding suitable templates allows us to apply Rowley's construction 
to get explicit partitions improving lower bounds. Furthermore, this paper tries to analyse former work on the subject based on
the principle that good partitions into \(n+1\) subsets start with good partitions into \(n\) subsets. We show that exceeding
the previous lower bound \(WS(6) \geqslant 582\) is impossible with such a method upon imposing certain conditions on the good
5-subsets partition. The new lower bounds includes \(S(10) \geqslant 60948\), \(WS(9) \geqslant 22536 \) and \(WS(10) \geqslant 71214 \).

\section{Introduction, context and notations}

We start by defining sum-free and weakly sum-free subsets to introduce regular and weak Schur numbers.

\begin{definition}
A subset \(A\) of \(\mathbb{N}\) is said to be \textit{sum-free} when:
\[ \forall (a,b) \in A^2 \text{, } a+b \notin A\]
\end{definition}

\begin{definition}
A subset \(B\) of \(\mathbb{N}\) is said to be \textit{weakly sum-free} when:
\[ \forall (a,b) \in B^2 \text{, } a \neq b \Longrightarrow a+b \notin B\]
\end{definition}

Let us notice that a sum-free subset is also weakly sum-free, hence justifying the name of \textit{weakly} sum-free
subsets. Given \(p\) and \(n\) two integers, we are interested in partitioning the set of integers from 1 to \(p\) into
\(n\) weakly sum-free subsets.

\begin{notation}
We denote by \([\![1,p]\!]\) the set of integers \(\{1, 2, ..., p\}\).
\end{notation}

Schur proved in \cite{Schur1917} that given a number of subsets \(n\), there exists a value of \(p\)
such that there exists no partition of \([\![1,q]\!]\) into \(n\) sum-free subsets for any \(q \geqslant p\). A similar
property holds for weakly sum-free subsets (reference necessaire). These observations lead to the following definitions.
\begin{definition}
Let \(n \in \mathbb{N}^*\). There exists a greatest integer that we note \(S(n)\) (\textit{resp. \(WS(n)\)}) such that
\([\![1,S(n)]\!]\) (resp. \([\![1,WS(n)]\!]\)) can be partitioned into \(n\) sum-free subsets (resp. weakly sum-free
subsets). \(S(n)\) is called the \textit{\(n\)\textsuperscript{th} Schur number} and \textit{\(WS(n)\) the
\(n\)\textsuperscript{th} weak
Schur number}.
\end{definition}

\begin{notation}
For a partition of \([\![1, p]\!]\) in \(n\) subsets, we generally note these subsets \(A_1, ..., A_n\). We also note
\(m_i = \text{min}(A_i)\).
By ordering the subsets, we mean assuming that \(m_1 < ... < m_n\). However, if not specified we do not make this
hypothesis since we
do not always consider partitions in which every subset plays a symmetric role.
\end{notation}