\section{Abstract}



\section{Introduction, context and notations}

We start by defining sum-free and weakly sum-free subsets to introduce regular and weak Schur numbers.

\begin{definition}
A subset \(A\) of \(\mathbb{N}\) is said to be \textit{sum-free} when:
\[ \forall (a,b) \in A^2 \text{, } a+b \notin A\]
\end{definition}

\begin{definition}
A subset \(B\) of \(\mathbb{N}\) is said to be \textit{weakly sum-free} when:
\[ \forall (a,b) \in B^2 \text{, } a \neq b \Longrightarrow a+b \notin B\]
\end{definition}

Let us notice that a sum-free subset is also weakly sum-free, hence justifying the name of \textit{weakly} sum-free
subsets. Given \(m\) and \(n\) two integers, we are interested in partitioning the set of integers from 1 to \(m\) into
\(n\) weakly sum-free subsets.

\begin{notation}
We denote by \([\![1,n]\!]\) the set of integers \(\{1, 2, ..., n\}\).
\end{notation}

Schur proved in \hyperlink{label1}{\textbf{[1]}} that given a number of subsets \(n\), there exists a value of \(m\)
such that there exists no partition of \([\![1,p]\!]\) into \(n\) sum-free subsets for any \(p \geqslant m\). A similar
property holds for weakly sum-free subsets (reference necessaire). These observations lead to the following definitions.

\begin{definition}
Let \(n \in \mathbb{N}^*\). There exists a greatest integer that we note \(S(n)\) (\textit{resp. \(WS(n)\)}) such that
\([\![1,S(n)]\!]\) (resp. \([\![1,WS(n)]\!]\)) can be partitioned into \(n\) sum-free subsets (resp. weakly sum-free
subsets). \(S(n)\) is called the \textit{\(n\)\textsuperscript{th} Schur number} and \textit{\(WS(n)\) the \(n\)\textsuperscript{th} weak
Schur number}.
\end{definition}

\begin{notation}
For a partition of \([\![1, n]\!]\) in \(k\) subsets, we generally note these subsets \(A_1, ..., A_k\). We also note \(m_i = \text{min}(A_i)\). 
By ordering the subsets, we mean assuming that \(m_1 < ... < m_k\). However, if not specified we do not make this hypothesis since we 
do not always consider partitions in which every subset plays a symmetric role.
\end{notation}