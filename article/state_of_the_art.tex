\section{State of the art}

\[
\begin{array}{|*{13}{c|}}
    \hline
    k & 1 & 2 & 3 & 4 & 5 & 6 & 7 & 8 & 9 & 10 & 11 & 12 \\
    \hline
    S(n) & & & & & & & & & & & & \\
    \text{old} & 1 & 4 & 13 & 44 & 160 \cite{Heule2017} & 536 \cite{Fredricksen} & 1680 \cite{Fredricksen} & 
    5041 \cite{ELIAHOU2012175} & 15124 \cite{ELIAHOU2012175} & 51120 \cite{AbbottHanson} & 172216 \cite{AbbottHanson} & 
    575664 \cite{AbbottHanson} \\
    \text{Rowley} & & & & & & & & 5286 & 17694 & 60320 & 201696 & 631840 \\
    \text{new} & & & & & & & & & 17803 & 60948 & 203828 & 638548 \\
    \hline
    WS(n) & & & & & & & & & & & & \\
    \text{old} & 2 & 8 & 23 & 66 & 196 \cite{ELIAHOU2012175} & 582 \cite{EliahouBook} & 1740 \cite{Malgache} & 5201 \cite{Malgache} & 
    15596 \cite{Malgache} & 51520 & 172216 & 575664 \\
    \text{Rowley} & & & & & & 642 & 2146 & 6976 & 21848 & 70778 & 241282 & 806786 \\
    \text{new} & & & & & & & & & 22536 & 71214 & 243794 & 815314 \\
    \hline
\end{array}
\]
Before Rowley, the best lower bounds were obtained thanks to a random
tree search. Indeed, the postulate stating that every good weakly sum-
free partition into \(n+1\) colours starts with a weakly sum-free
partition into \(n\) colors leads to a recursive method. However,
these values were improved by Rowley using a theoretical and explicit
construction for special cases. It's interesting to notice that before
that there were no theoretical approach such as Abbott and Hanson's
\cite{AbbottHanson} for weak Schur.
