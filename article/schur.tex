\section{Schur numbers}

\qquad In this section, we use Rowley's constructions [7] in a Schur context. To improve lower bounds for Ramsay's numbers, Rowley 
introduces partitions verifying some properties which can be extended using method which generalizes Abbott-Hanson's [8] construction. 
Rolwey named these partitions "templates", and we will keep this name in the entire article. Finding suitable templates and applying his 
method to these will allow us to find new lower bounds for Schur methods.

\subsection{Definition of SE\(_+\)}

\begin{definition}
	We call SF-template of \(k\) colors and lenght \(n\) a partition of \( [\![1,n]\!]\) into \(k\) sum-free subsets \(A_1, A_2, ..., A_k\) which verify :
	\[
	\forall i \in [\![1, k-1]\!], \forall (x,y) \in A_i^2, x+y > n
	\Longrightarrow x+y-n \notin A_i
	\]
	We note \(SE_+(k)\) the maximal lenght for a SF-template of \(k\) colors. 
\end{definition}

\begin{theorem}
	Let \(n \in [\![2, +\infty]\!]\), we have :
	\[
	2S(n-1)+1 \leqslant SE_+(n) \leqslant S(n)
	\]
\end{theorem}

\begin{remark}
	\(SE_+\) and \(S\) have the same asymptotic growth rate.
\end{remark}


\subsection{Inequaties on \(SE_+\)} 

The main result on \(SE_+\) follows. It allows us to improve lower bounds on Schur numbers by computing \(SE_+\).

\begin{theorem}
	Let $(n,k), (r,s) \in (\mathbb{N}^*)^2$. If there exists a \(SF\)-template of $k+1$ colors and lenght \(r\),
	and a partition of $n$ sum-free subsets of $[\![1,s]\!]$ then there exists a partition of $n+k$ sum-free subsets of $[\![1,rs]\!]$
\end{theorem}

Setting $r = SE_+(k+1)$ and $s = S(n)$ yields the following corollary. 

\begin{corollary}
	Let \(n, k \in \mathbb{N}^*\), we have \\
	\[ S(n+k) \geqslant SE_+(k+1)S(n) \]
\end{corollary}

Using a SAT solvers, we are able to exhibit SF-template,
hence providing lower bound on \(SE_+\). Moreover, when using 
a SF-template in the above formula, it is possible to get an additive constant. We seek templates providing "affine"
inequalities : \(aS(n+k) + b \) where \(a\) is a lower bound for \(SE_+(k+1)\). Further details can be found in the \hyperref[SAT]{SAT section}.
Here are the best inequalities on Schur numbers so far :
\[ S(n+1) \geqslant 3S(n) + 1 \]
\[ S(n+2) \geqslant 9S(n) + 4 \]
\[ S(n+3) \geqslant 33S(n) + 6 \]
\[ S(n+4) \geqslant 111S(n) + 43 \]
\[ S(n+5) \geqslant 380S(n) + 148 \]
\[ S(n+6) \geqslant 1140S(n) + 528 \]

The first inequality comes from the original Schur's paper \hyperlink{label1}{\textbf{[1]}}. The second and third ones were found by Rowley (reference nécessaire), 
while the other ones are new. \\

We also have a similar theorem where only \(SE_+\) is involved. 

\begin{theorem}
	Let $(n,k), (r,s) \in (\mathbb{N}^*)^2$. If there exists a \(SF\)-template of $k+1$ colors and lenght \(r\),
	and \(SF\)-template of \(n\) color and lenght \(s\), then there exists \(SF\)-template of \((n+k)\) and lenght \(rs\).
\end{theorem}

And the associated inequality :

\begin{corollary}
	Let \(n, k \in \mathbb{N}^*\), we have \\
	\[ SE_+(n+k) \geqslant SE_+(k+1)SE_+(n) \]
\end{corollary}


\subsection{New lower bounds for Schur numbers}

The previous inequalities give new lower bounds for \(S(n)\) for
\( n \in [\![9,+\infty]\!] \backslash \{13\} \). We compute the lower
bounds for \( n \in [\![8,15]\!] \) using the four different inequalities, please notice that the best values for \( n = 8\) and \(n = 13\) were already obtained thanks to the first one, found by Rowley. The best lower bounds are highlighted.\\
\\
\begin{center}
\begin{tabular}{|*{5}{c|}}
    \hline
	 & 8 & 9 & 10 & 11 \\
	\hline
	\(33S(n-3) + 6 \) & \cellcolor{yellow} 5286 & 17694 & 55446 & 174444\\
	\hline
	\(111S(n-4) + 43 \) & 4927 & \cellcolor{yellow} 17803 & 59539 & 186523\\
	\hline
	\(380S(n-5) + 148 \) & 5088 & 16868 & \cellcolor{yellow} 60948 & \cellcolor{yellow} 203828 \\
	\hline
	\(1140S(n-6) + 528 \) & 5088 & 15348 & 50688 & 182928\\
	\hline
	\hline
	& 12 & 13 & 14 & 15 \\
	\hline
	\(33S(n-3) + 6 \) & 587505 & \cellcolor{yellow} 2011290 & 6726330 & 21072090\\
	\hline
	\(111S(n-4) + 43 \) & 586789 & 1976176 & 6765271 & 22624951 \\
	\hline
	\(380S(n-5) + 148 \) & \cellcolor{yellow} 638548 & 2008828 & \cellcolor{yellow} 6765288 & \cellcolor{yellow} 23160388 \\
	\hline
	\(1140S(n-6) + 528 \) & 611568 & 1915728 & 6026568 & 20295948 \\
	\hline
\end{tabular}
\end{center}
Except for 8, 9 and 13, the best lower bounds are obtained thanks to the third inequality \( 380S(n-5) + 148 \leqslant S(n) \). The table doesn't go any further, but the same inequality allows to improve the lower bounds for every \( n \in [\![15,+\infty]\!] \).