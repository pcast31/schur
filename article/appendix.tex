\appendix
\renewcommand{\arraystretch}{1}


\section{SF-templates}

\begin{center}
SF-template partitionning \([\![1, 33]\!]\) into 4 subsets \\
\begin{tabular}{|*{2}{c|}}
	\hline
	1 & 1, 6, 9, 13, 16, 20, 24, 27, 31 \\
	\hline
	2 & 2, 5, 14, 15, 25, 26 \\
	\hline
	3 & 3, 4, 10, 11, 12, 28, 29, 30 \\
	\hline
	4 & 7, 8, 17, 18, 19, 21, 22, 23, 32, 33 \\
	\hline
\end{tabular}
\end{center}

\begin{center}
SF-template partitionning \([\![1, 111]\!]\) into 5 subsets
\begin{tabular}{|*{2}{c|}}
	\hline
	1 & 1, 5, 18, 12, 14 ,21, 23, 30, 32, 36, 39, 43, 45, 52, 103 \\
	 & 106, 110 \\
	\hline
	2 & 2, 6, 7, 10, 15, 18, 26, 29, 34, 37, 38, 42, 46, 51, 54 \\
	& 101, 104, 109 \\
	\hline
	3 & 3, 4, 9, 11, 17, 19, 25, 27, 33, 35, 40, 41, 47, 48, 55 \\
	& 100, 107, 108 \\
	\hline
	4 & 13, 16, 20, 22, 24, 28, 31, 58, 61, 67, 88, 94, 97 \\
	\hline
	5 & 44, 50, 53, 56, 57, 59, 60, 62, 63, 64, 65, 66, 68, 69, 70\\
	& 71, 72, 73, 74, 75, 76, 77, 78, 79, 80, 81, 82, 83, 84, 85\\
	& 86, 87, 89, 90, 91, 92, 93, 95, 96, 98, 99, 102, 105, 111 \\
	\hline
\end{tabular}
\end{center}

\begin{center}
SF-template partitionning \([\![1, 380]\!]\) into 6 subsets
\begin{tabular}{|*{2}{c|}}
	\hline
	1 & 1, 5, 8, 11, 15, 17, 29, 33, 36, 39, 43, 57, 61, 88, 92 \\
	& 106, 110, 113, 116, 120, 132, 134, 138, 141, 144, 148, 150, 154, 157, 160 \\
	& 164, 178, 182, 185, 188, 341, 344, 347, 351, 365, 369, 372, 375, 379\\
	\hline
	2 & 2, 9, 13, 16, 20, 23, 24, 27, 28, 31, 34, 35, 38, 42, 45\\
	& 49, 53, 60, 67, 71, 78, 82, 89, 96, 100, 104, 107, 111, 114, 115\\
	& 118, 121, 122, 125, 126, 129, 133, 136, 140, 147, 158, 162, 165, 169, 172\\
	& 176, 183, 187, 194, 201, 328, 335, 342, 346, 353, 357, 360, 364, 367, 371\\
	\hline
	3 & 3, 4, 12, 14, 19, 25, 30, 32, 40, 41, 47, 48, 58, 91, 101\\
	& 102, 108, 109, 117, 119, 124, 130, 135, 137, 145, 146, 152, 153, 161, 163\\
	& 168, 179, 181, 190, 339, 348, 350, 361, 366, 368, 376, 377\\
	\hline
	4 & 6, 7, 10, 18, 21, 22, 26, 37, 46, 50, 51, 54, 65, 70, 79\\
	& 84, 95, 98, 99, 103, 112, 123, 127, 128, 131, 139, 142, 143, 151, 155\\
	& 156, 159, 167, 170, 171, 175, 186, 343, 354, 358, 359, 362, 370, 373, 374\\
	& 378\\
	\hline
	5 & 44, 52, 55, 56, 59, 62, 63, 64, 66, 68, 69, 72, 73, 74, 75\\
	& 76, 77, 80, 81, 83, 85, 86, 87, 90, 93, 94, 97, 105, 189, 196\\
	& 197, 200, 203, 206, 207, 209, 214, 219, 231, 298, 310, 315, 320, 322, 323\\
	& 326, 329, 332, 333, 340\\
	\hline
	6 & 149, 166, 173, 174, 177, 180, 184, 191, 192, 193, 195, 198, 199, 202, 204\\
	& 205, 208, 210, 211, 212, 213, 215, 216, 217, 218, 220, 221, 222, 223, 224\\
	& 225, 226, 227, 228, 229, 230, 232, 233, 234, 235, 236, 237, 238, 239, 240\\
	& 241, 242, 243, 244, 245, 246, 247, 248, 249, 250, 251, 252, 253, 254, 255\\
	& 256, 257, 258, 259, 260, 261, 262, 263, 264, 265, 266, 267, 268, 269, 270\\
	& 271, 272, 273, 274, 275, 276, 277, 278, 279, 280, 281, 282, 283, 284, 285\\
	& 286, 287, 288, 289, 290, 291, 292, 293, 294, 295, 296, 297, 299, 300, 301\\
	& 302, 303, 304, 305, 306, 307, 308, 309, 311, 312, 313, 314, 316, 317, 318\\
	& 319, 321, 324, 325, 327, 330, 331, 334, 336, 337, 338, 345, 349, 352, 355\\
	& 356, 363, 380\\
	\hline

\end{tabular}
\end{center}


\section{WSF-templates}

\begin{center}
WSF-template partitionning \([\![1, 42]\!]\) into 4 subsets
\begin{tabular}{|*{2}{c|}}
	\hline
	1 & 1, 2, 4, 8, 11, 22, 25, \(\mathbf{(N+1)}\)\\
	\hline
	2 & 5, 6, 7, 19, 21, 23, 36\\
	\hline
	3 & 9, 10, 12, 13, 14, 15, 16, 17, 18, 20\\
	\hline
	4 & 24, 26, 27, 28, 29, 30, 31, 32, 33, 34, 35, 37, 38, 39, 40\\
	& 41, 42\\
	\hline
\end{tabular}
\end{center}

This template provides the inequality \(\WS (n+3) \geqslant 42S(n) + 24\)
by placing one last number, here represented by \(\mathbf{(N+1)}\), in the first subset.



\section{Reformulation as a SAT problem}
\label{SAT}

In this section, we reframe the question of the existence of (weakly) sum-free partitions as a boolean
satisfiability (SAT) problem. We encode the existence of (weakly) sum-free partitions as propositional formulae like in
\cite{Heule2017} and then use SAT solvers to determine whether these formulae are satisfiable.

\begin{definition}
A \textit{literal} is either a variable \(v\) (a positive literal) or the negation \(\bar{v}\) of a variable \(v\) (a
negative literal) where \(v\)
takes a truth value: \(true\) or \(false\). A \textit{clause} is a disjunction of literals and a \textit{formula} is a
conjunction of clauses: it
is a propositional formula in \textit{conjonctive normal form} (CNF).
\end{definition}

\begin{definition}
An \textit{assignment} is a function from a set of variables to the truth values \(true\) (1) and \(false\) (0). A
literal \(l\) is
\textit{satisfied} (\textit{falsified}) by an assignment \(\alpha\) if l is positive and \(\alpha(var (l)) = 1\)
(resp. \(\alpha(var (l)) = 0\)) or if it is negative and \(\alpha(var (l)) = 0\) (resp. \(\alpha(var (l)) = 1\)). A
clause is \textit{satisfied}
by an assignment \(\alpha\) if it contains a literal that is satisfied by \(\alpha\). Finally, a formula is
\textit{satisfied} by an assignment
\(\alpha\) if all its clauses are satisfied by \(\alpha\). A formula is \textit{satisfiable} if there exists an
assignment that satisfies it;
otherwise it is \textit{unsatisfiable}.
\end{definition}

We then encode the existence of a partition of \([\![1,p]\!]\) in \(k\) weakly sum-free subsets as follows: for every
integer
\(i \in [\![1,p]\!]\), take \(k\) variables \(x^{(i)}_{1}, ..., x^{(i)}_{k}\) and for every \(\forall c \in [\![1,k]\!],
x^{(i)}_c = 1 \iff i \in A_c\).
The corresponding clauses are:

\begin{itemize}
\item \textbf{weakly sum-free:} \(\forall c \in [\![1,k]\!], \forall (i, j) \in [\![1,p]\!]^2, (i \neq j ~ \text{and} ~ i + j
\leq n) \implies \lnot x^{(i)}_c
\lor  \lnot x^{(i)}_c \lor \lnot x^{(i+j)}_c\)
\item \textbf{union:} \(\forall i \in [\![1,p]\!], x^{(i)}_1 \lor ... \lor x^{(i)}_k\)
\item \textbf{disjoint:} \(\forall i \in [\![1,p]\!],\forall (c_1, c_2) \in [\![1,k]\!]^2, c_1 \neq c_2 \implies \neg
x^{(i)}_{c_1} \lor \neg x^{(i)}_{c_2}\)
\end{itemize}

In the above formula, every color plays a symmetric role. Hence the search space can reduced by \(k!\) by ordering the
subsets, that is by
enforcing that \(m_1 < ... < m_k\). The corresponding clauses are: \linebreak
\textbf{symmetry breaking:} \(x^{(1)}_1 = 1\) and \(\forall c \in [\![2,k-1]\!], \forall i \in [\![1, \WS (c - 1)+1]\!],
x^{(1)}_c \lor ... \lor x^{(i)}_c \lor \neg x^{(i+1)}_{c+1}\)

\begin{remark}
For a given problem, it can be interesting to try out different SAT solvers because the relative performance can vary
significantly according to the problem.
For instance, we used two different SAT solvers in the next two next subsections.
\end{remark}

\begin{remark}
Using a parallel SAT solver usually reduces the computation time, especially when trying to show that a formula is
unsatisfiable. However, most of the
parallel SAT solver do not have a deterministic behaviour and it can results in a strong variation of running times.
\end{remark}



\section{Proof of theorem j'arrive pas a afficher le numero du theoreme avec un lien}

\begin{proof}[\textsc{Proof.}]
Let \((p,q), (n,k) \in (\mathbb{N}^*)^2\), \(N = p(q+\left \lceil \frac{q}{2} \right \rceil + 1)+q\),
\(\alpha = \left \lceil \frac{q}{2} \right \rceil > 0\) and \(\beta = q + \alpha + 1\).
We denote by \(f\) the colouring associated to the partition of \([\![1,q]\!]\) and \(g\) the
one associated to the partition of \([\![1,p]\!]\).
\[ f : [\![1,q]\!] \longrightarrow [\![1,n]\!] \text{ and } \forall (x,y) \in [\![1,q]\!]^2, \left\{
\begin{array}{ll}
	x \neq y \\
	f(x) = f(y)
\end{array}
\right.
\Longrightarrow f(x+y) \neq f(x)
\]
\[g : [\![1,p]\!] \longrightarrow [\![1,k]\!] \text{ and } \forall (x,y) \in [\![1,q]\!]^2 \text{, } f(x) = f(y)
\Longrightarrow f(x+y) \neq f(x)
\]
Let us start by parting the integers of \([\![1,N]\!]\) in two subsets \(\mathcal{A}\) and \(\mathcal{B}\) where
\(\mathcal{A} = [\![1,\alpha]\!] \cup \{a\beta + u \mid (a,u) \in [\![0,p]\!] \times [\![\alpha + 1,q]\!]\}\) and
\(\mathcal{B} = \{a\beta + u \mid (a,u) \in [\![1,p]\!] \times [\![-\alpha,\alpha]\!]\}\).\\
\\
First, \underline{\(\mathcal{A} \cap \mathcal{B} = \varnothing\)} : \\
By contradiction, suppose there exists \(x \in \mathcal{A} \cap \mathcal{B} \neq \varnothing \). Then there are \((a,u)
\in [\![0,p]\!] \times [\![\alpha + 1,q]\!]\) and \((b,v) \in [\![1,p]\!] \times [\![-\alpha,\alpha]\!]\) such that \(x
= a\beta + u = b\beta +v\). By definition of \(\alpha\) and \(\beta\) we have \(u \in [\![\alpha + 1,q]\!] \subset
[\![0,\beta - 1]\!]\).
From there, we distinguish two cases :
\begin{itemize}
\item If \(v \in [\![0,\alpha]\!]\) then \(v \in [\![0,\beta - 1]\!]\) and \(v \neq u\) because \(v < \alpha + 1
\leqslant u\)
\item If \(v \in [\![-\alpha,-1]\!]\), we note \(\tilde{v} = \beta + v\) and thus have \(x = (b-1)\beta + \tilde{v}\)
with \(\tilde{v} \in [\![\beta - \alpha,\beta - 1]\!] \subset [\![0,\beta - 1]\!]\) and \(\tilde{v} \neq u\) because
\(u< q+1 = \beta - \alpha \leqslant \tilde{v}\).
\end{itemize}
In either cases, we run into a contradiction because of the remainder's uniqueness in the euclidean division of \(x\)
by\(\beta\).\\
\\
Then, we have \underline{\(\mathcal{A} \cup \mathcal{B} = [\![1,N]\!]\)}:
\begin{itemize}
	\item On the one hand : \(1 = \text{min}(\mathcal{A}) \leqslant \text{max}(\mathcal{A}) = p\beta + q = N\) and
\(1 \leqslant \beta - \alpha = \text{min}(\mathcal{B}) \leqslant \text{max}(\mathcal{B}) = p\beta + \alpha \leqslant
N\),
	which gives \(\mathcal{A} \cup \mathcal{B} \subset [\![1,N]\!]\).
\item On the other hand, let \(x \in [\![1,N]\!]\). If \(x \leqslant \alpha\), we directly have \(x \in \mathcal{A}\),
let us then suppose that \(x > \alpha\) and write \(x = a\beta + u\)
the euclidean division of \(x\) by \(\beta\). We have \(x \leqslant N\), thus \(a \leqslant p\). We distinguish three
cases : \\
- If \(u \in [\![0,\alpha]\!]\) then we necessarily have \(a \geqslant 1\) because \(x > \alpha\), and so \(x \in
\mathcal{B}\).\\
	- If \(u \in [\![\alpha + 1,q]\!]\), then \(x \in \mathcal{A}\). \\
- If \(u \in [\![q + 1,\beta - 1]\!]\) then \(x = (a+1)\beta - (\beta - u)\) with \(-\alpha \leqslant \beta - u
\leqslant 0\).
	Furthermore, \(a \leqslant p - 1\), else we would have \(x > N\), and so \(x \in \mathcal{B}\) \\
In any case, \(x \in \mathcal{A} \cup \mathcal{B}\) and we can thus conclude that \([\![1,N]\!] \subset \mathcal{A}
\cup\mathcal{B}\).
\end{itemize}
This first partition of \([\![1,N]\!]\) will help us to define our final partition by the projection of its equivalence
relation.
We thereby define \(h : [\![1,N]\!] \longrightarrow [\![1,n+k]\!]\) as such :\\
- If \(x \in \mathcal{A}\) then \(h(x) = f(x \text{ mod } \beta)\) (well defined because \(x \text{ mod } \beta \in
[\![1,N]\!]\))\\
- If \(x \in \mathcal{B}\) then \(x = a\beta + u\) with a unique \((a,u) \in [\![1,p]\!] \times
[\![-\alpha,\alpha]\!]\)and we define \(h(x) = n + g(a)\)\\
The fact that \((\mathcal{A}, \mathcal{B})\) is a partition of \([\![1,N]\!]\) ensures that this definition of \(h\) is
valid. We then have to verify that \(h\) induces weakly sum-free subsets.\\
\\
\underline{The classes of equivalence \(h(x)\) for \(x \in \mathcal{A}\) are weakly sum-free :}
\\
\\
Let \((x,y) \in \mathcal{A}^2\) such that \(h(x) = h(y)\), \(x \neq y\) and \(x + y \leqslant N\)
\begin{itemize}
	\item If \((x,y) \in [\![1,\alpha]\!]^2\) :\\
	We have \(x + y \leqslant 2\alpha \leqslant q\) and \(x + y = 0\beta + x + y\), therefore \(x + y \in \mathcal{A}\).
Then, by definition : \(h(x) = f(x)\), \(h(y) = f(y)\) and \(h(x+y) = f(x+y)\), which gives us, thanks to the property
verified by \(f\), that \(h(x+y) \neq h(x)\).
	\item If \((x,y) \in [\![1,\alpha]\!] \times ( \mathcal{A} \text{ \textbackslash} ~ [\![1,\alpha]\!] )\) :\\
We write \(y = a\beta + u\) with \((a,u) \in [\![0,p]\!] \times [\![\alpha + 1,q]\!]\). Then \(x+y = a\beta + x + u =
(a+1)\beta + x + u - \beta\),
and if \(x + u > q\) it follows that \(a \leqslant p-1\) since \(x+y \leqslant N\), and \(-\alpha \leqslant x + u -
\beta \leqslant -1\).
Therefore \(x+y \in \mathcal{B}\) and \(h(x+y) \neq h(x) = f(x)\) by definition of h. On the contrary, if \(x - u
\leqslant n\),
then \(x+y \in \mathcal{A}\) and \(h(x+y) = f(x+u)\) because \(x+u\) is actually the remainder of the euclidean
divisionof \(x+y\) by \(\beta\).
Moreover, \(h(x) = f(x)\), \(x < u\) and, with our initial hypothesis, \(h(x) = h(y) = f(u)\). The property verified by
\(f\) gives us \(f(x+u) \neq f(x)\) which can be rewritten as \(h(x+y) \neq h(x)\).
	\item If \((x,y) \in ( \mathcal{A} \text{ \textbackslash} ~ [\![1,\alpha]\!] ) \times [\![1,\alpha]\!]\) : \\
	This case is handled exactly like the previous one by swaping the roles of \(x\) and \(y\).
	\item If \((x,y) \in ( \mathcal{A} \text{ \textbackslash} ~ [\![1,\alpha]\!] )^2\) : \\
We write \(x = a\beta + u\) and \(y = b\beta + v\) with \((a,u)\) and \((b,v)\) in \([\![0,p]\!] \times [\![\alpha +
1,q]\!]\). Then \(x+y = (a+b)\beta + u+v = (a+b+1)\beta + u + v - \beta\)
with \(a+b \leqslant p-1\) (else we would have \(x+y > N\) because \(u+v > q\)) and \(-\alpha \leqslant u + v - \beta
\leqslant \alpha\), therefore \(x+y \in \mathcal{B}\) and by definition \(h(x+y) \neq h(x)\).
\end{itemize}
In any case, \(h(x+y) \neq h(x)\) and the classes of equivalence \(h(x)\) for \(x \in \mathcal{A}\) are weakly
sum-free.\\
\\
\underline{The classes of equivalence \(h(x)\) for \(x \in \mathcal{B}\) are weakly sum-free :}
\\
\\
Let \((x,y) \in \mathcal{B}^2\) such that \(h(x) = h(y)\), \(x \neq y\) and \(x + y \leqslant N\).\\
We write \(x = a\beta + u\) and \(y = b\beta + v\) with \((a,u)\) and \((b,v)\) in \([\![1,p]\!] \times
[\![-\alpha,\alpha]\!]\).
We have \(h(x) = q + g(a)\) and \(h(y) = q + g(b)\), therefore \(g(a) = g(b)\). We also have \(x+y = (a+b)\beta + u
+v\).\\
If \(u + v \in [\![-\alpha,\alpha]\!]\), then \(x+y \in \mathcal{B}\) and \(h(x+y) = g(a+b)\), hence we can deduce that
\(h(x+y) \neq h(x)\) because of the property verified by \(g\). On the contrary, if \(u+v \notin
[\![-\alpha,\alpha]\!]\), then necessarily \(x+y \in \mathcal{A}\). Suppose \(x+y \in \mathcal{B}\), then \(x+y =
c\beta+ w\) with \((c,w) \in [\![1,p]\!] \times [\![-\alpha,\alpha]\!]\). Thus, \(c\beta + w = (a+b)\beta + u + v\) and
\((a+b-c)\beta = w-u-v\). Furthermore \(a+b-c \neq 0\), else we would have \(u+v = w \in [\![-\alpha,\alpha]\!]\). This
finally leads to the following inequality :
\[\beta \leqslant |a+b-c|\beta = |w-u-v| \leqslant |w| + |u| + |v| \leqslant 3\alpha \leqslant q + \alpha < \beta
\]
which is absurd. We can therefore conclude that \(x+y \in \mathcal{A}\) and by definition of \(h\), \(h(x+y) \neq
h(x)\), proving that the classes of equivalence \(h(x)\) for \(x \in \mathcal{B}\) are weakly sum-free.\\
\\
Finally, we have showed that every classe of equivalence induced by \(h\) is weakly sum-free, which ends the proof.
\end{proof}
