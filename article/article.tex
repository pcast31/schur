\documentclass[3p]{elsarticle}
\title{New lower bounds for Schur and weak Schur numbers}
\author{Romain Ageron, Paul Casteras, Thibaut Pellerin, Yann Portella}

\usepackage{amsmath}
\usepackage{amssymb}
\usepackage{amsthm}
\usepackage{dsfont}
\usepackage{MyMnSymbol}

\usepackage[english]{babel}
\usepackage[hyphens]{url}
\usepackage{hyperref}
\usepackage[utf8]{inputenc}
\usepackage{siunitx}
\usepackage{float}

\usepackage[table]{xcolor}
\usepackage{multirow}
\usepackage{nicematrix}

\renewcommand{\qedsymbol}{\(\square\)}

\newtheorem{definition}{Definition}[section]
\newtheorem{notation}[definition]{Notation}
\newtheorem{theorem}[definition]{Theorem}
\newtheorem{computational theorem}[definition]{Computational Theorem}
\newtheorem{corollary}[definition]{Corollary}
\newtheorem{remark}[definition]{Remark}
\newtheorem{proposition}[definition]{Proposition}

\newcommand{\WS}{\mathit{WS}}
\newcommand{\docarraystretch}{1.2}
\newcommand{\resetarraystretch}{\renewcommand{\arraystretch}{\docarraystretch}}
\resetarraystretch


\begin{document}

\begin{abstract}

This article provides new lower bounds for both Schur and weak Schur numbers. These results were obtained by 
continuing on Rowley's "template"-based approach for Schur and Ramsey numbers. Finding templates allows us to 
get explicit partitions improving lower bounds as well as the growth rate for both Schur numbers and Ramsey 
numbers \(R_n(3)\). We also developed a method to improve lower bounds for weak Schur numbers. 
Furthermore, we try to analyze former works on weak Schur numbers based on the principle that \textit{good} 
partitions into \(n+1\) weakly sum-free subsets start with a \textit{good} partition into \(n\) weakly sum-free subsets. 
We show that exceeding the previous lower bound \(\WS (6) \geqslant 582\) is impossible with such an assumption 
upon imposing certain conditions on the \textit{good} 5-subsets partitions. The new lower bounds include 
\(S(9) \geqslant 17\,803\), \(S(10) \geqslant 60\,948\), \(\WS (9) \geqslant 22\,536\) and 
\(\WS (10) \geqslant 71\,214 \).

\end{abstract}

\maketitle

\section{Introduction}

\qquad We are interested in partioning the set of integers \(\{1, ..., p\}\) in \(n\) subsets such that there is no 
subset containing three integers \(x\), \(y\) and \(z\) verifying \(x + y = z\). We say these subsets are 
\textit{sum-free}. If we add the hypothesis \(x \neq y\), we say the subsets are \textit{weakly sum-free}. The 
greatest \(p\) for which there exists a partition into \(n\) sum-free subsets is called the \(n^{\text{th}}\) Schur 
number and is denoted \(S(n)\) \cite{Schur1917}. Likewise for weakly sum-free partitions we define \(\WS(n)\) 
the \(n^{\text{th}}\) weak Schur number \cite{Irving1973}. Only the first values of these sequences are known. 
This article focuses on the lower bounds for these numbers.


\subsection{State of the art}

\qquad Before Rowley's "template"-based approach for Schur and Ramsey numbers \cite{RowleyRamsey}, the 
previous generic construction for Schur numbers was given by Abbott and Hanson \cite{AbbottHanson} in 1972 
with a recursive construction. It used to give the best lower bounds for all sufficiently large numbers. No equivalent 
was known for weak Schur numbers and as a result the best known partitions for large weak Schur numbers 
did not utilize the weakly sum-free hypothesis. 

\par
As for smaller numbers, the best lower bounds were obtained by conducting a computer search. Eliahou \cite{ELIAHOU2012175}, 
Rafilipojaona \cite{Rafilipojaona} and Bouzy \cite{Bouzy2015AnAP} improved lower bounds with Monte-Carlo methods. It 
has been the main approch in the previous decade. 
This search for weakly sum-free partitions relied on the recursive assumption that a good weakly sum-free partition into \(n+1\) 
colors starts with a good weakly sum-free partition into \(n\) colors. This assumption was necessary in order for the Monte-Carlo
approch to yield results. However, we show in the \hyperlink{sat}{last section} how this search space may not countain
the optimal solution. Thus further work using similar methods might need to put that assumption aside and 
find a different search space.

\par
In 2020, Rowley introduced the notion of template for Schur and Ramsey numbers which generalizes Abbott and 
Hanson's construction and gives new lower bounds (and inequalties) for Schur numbers. Rowley also gives two 
inequalities for weak Schur numbers \cite{RowleyWS} that yield significant improvements over previous lower 
bounds which besides do utilize the \textit{weakly} sum-free hypothesis.

\renewcommand{\arraystretch}{1.2}
\setlength{\tabcolsep}{4pt}

\begin{center}

\textbf{Table 1 - Comparison of lower bounds for Schur numbers}

\begin{tabular}{|*{13}{c|}}
    \hline
    \(n\) & 1 & 2 & 3 & 4 & 5 & 6 & 7 & 8 & 9 & 10 & 11 & 12 \\
    \hline
    \multirow{2}{*}{old} & 1 & 4 & 13 & 44 & 160 & 536 & 1680 & 5041 & 15124 & 51120 & 172216 & 575664 \\
      & & & & & \cite{Heule2017} & \cite{Fredricksen} & \cite{Fredricksen} & \cite{ELIAHOU2012175} & 
    \cite{ELIAHOU2012175} & \cite{AbbottHanson} & \cite{AbbottHanson} & \cite{AbbottHanson} \\
    \hline
    \begin{tabular}{@{}c@{}}Rowley \\ \cite{RowleyRamsey}\end{tabular} & & & & & & & & 5286 & 17694 & 
    60320 & 201696 & 631840 \\
    \hline
    \hyperref[Schur]{new}  & & & & & & & & & 17803 & 60948 & 203828 & 638548 \\
    \hline
    \(\displaystyle \frac{\text{new} - \text{old}}{\text{old}}\) & 0\% & 0\% & 0\% & 0\% & 0\% & 0\% & 
    0\% & 0\%
     & 17.7\% & 19.2\% & 18.4\% & 11.0\% \\
    \hline
\end{tabular}

\vspace{3ex}

\textbf{Table 2 - Comparison of lower bounds for weak Schur numbers}

\begin{tabular}{|*{13}{c|}}
    \hline
    \(n\) & 1 & 2 & 3 & 4 & 5 & 6 & 7 & 8 & 9 & 10 & 11 & 12 \\
    \hline
    \multirow{2}{*}{old} & 2 & 8 & 23 & 66 & 196 & 582 & 1740 & 5201 & 15596 & 51520 & 172216 & 575664 \\
      & & & & & \cite{ELIAHOU2012175} &\cite{EliahouBook} & \cite{Rafilipojaona} & \cite{Rafilipojaona} & 
    \cite{Rafilipojaona} & \cite{AbbottHanson} & \cite{AbbottHanson} & \cite{AbbottHanson} \\
    \hline
    \begin{tabular}{@{}c@{}}Rowley \\ \cite{RowleyWS}\end{tabular} & & & & & & 642 & 2146 & 6976 & 21848 
    & 70778 & 241282 & 806786 \\
    \hline
    \hyperref[WeakSchur]{new} & & & & & & & & & 22536 & 71214 & 243794 & 815314 \\
    \hline
    \(\displaystyle \frac{\text{new} - \text{old}}{\text{old}}\) & 0\% & 0\% & 0\% & 0\% & 0\% & 10.3\% & 
    23.3\% & 34.1\% & 44.5\% & 38.2\% & 41.6\% & 41.6\% \\
    \hline
\end{tabular}

\end{center}

\subsection{Structure of this article}

\par
The main contribution of this article is a generalization of the concept of template to weak Schur numbers. 
This gives new lower bounds (and inequalities) for weak Schur numbers. This construction also includes as a 
special case an analogous for weak Schur numbers of Abbott and Hanson's construction for Schur numbers.

\par
We first explain Rowley's template-based construction in the context of Schur numbers and then give 
new templates, thus providing new lowers bounds and inequalities as well as showing that the growth rates 
for both Schur and Ramsey numbers \(R_n(3)\) exceed 3.28. 

\par
We then  generalize the concept of templates to weak Schur numbers and provide new lower bounds for weak 
Schur numbers. 

\par
Finally, we analyze the significant difference between new lower bounds obtained with templates and the former 
lower bounds obtained by computer search and we provide evidence which indicate that the main assumption 
made in those articles removes the optimal partitions from the search space.

\par
We now introduce notations and definitions we use throughout this article.

\subsection{Definitions and notations}

We start by defining sum-free and weakly sum-free subsets to introduce regular and weak Schur numbers.

\begin{definition}
A subset \(A\) of \(\mathbb{N}\) is said to be \textit{sum-free} when:
\[ \forall (a,b) \in A^2 \text{, } a+b \notin A\]
\end{definition}

\begin{definition}
A subset \(B\) of \(\mathbb{N}\) is said to be \textit{weakly sum-free} when:
\[ \forall (a,b) \in B^2 \text{, } a \neq b \Longrightarrow a+b \notin B\]
\end{definition}

Let us notice that a sum-free subset is also weakly sum-free, hence justifying the name of \textit{weakly} sum-free
subsets. Given \(p\) and \(n\) two integers, we are interested in partitioning the set of integers \(\{1, 2, ..., p\}\), 
denoted by \([\![1,p]\!]\), into \(n\) (weakly) sum-free subsets.

\par
Schur proved in \cite{Schur1917} that given a number of subsets \(n\), there exists a value of \(p\)
such that there exists no partition of \([\![1,q]\!]\) into \(n\) sum-free subsets for any \(q \geqslant p\). A similar
property holds for weakly sum-free subsets (reference necessaire). These observations lead to the following definitions.

\begin{definition}
Let \(n \in \mathbb{N}^*\). There exists a greatest integer that we denote \(S(n)\) (\textit{resp. \(\WS (n)\)}) such that
\([\![1,S(n)]\!]\) (resp. \([\![1, \WS (n)]\!]\)) can be partitioned into \(n\) sum-free subsets (resp. weakly sum-free
subsets). \(S(n)\) is called the \textit{\(n\)\textsuperscript{th} Schur number} and \textit{\(\WS (n)\) the
\(n\)\textsuperscript{th} weak
Schur number}.
\end{definition}

Given a partition of \([\![1, p]\!]\) in \(n\) subsets, we generally denote these subsets \(A_1, ..., A_n\). We also denote
\(m_i = \min(A_i)\). By ordering the subsets, we mean assuming that \(m_1 < ... < m_n\). However, if not specified we do 
not make this hypothesis since we do not always consider partitions in which every subset plays a symmetric role.

\begin{definition}
We sometimes refer to a partitition as a coloring. The coloring associated to a partition \(A_1, ..., A_n\) of
\([\![1, p]\!]\) is the function \(f\) such that \(\forall x \in [\![1, p]\!], x \in A_{f(x)}\). Likewise, the partition associated to
a coloring \(f\) of \([\![1, p]\!]\) with \(n\) colors is \(\forall c \in [\![1, n]\!], A_c = f^{-1}(c)\).
\end{definition}


\section{Templates for Schur numbers}
\label{Schur}

\qquad In this section, we use Rowley's template-based constructions \cite{RowleyRamsey} in the context of Schur 
numbers. In order to improve lower bounds for Schur and Ramsey numbers, Rowley introduces special sum-free 
partitions verifying some additional properties which can be extended using a method generalizing Abbott and 
Hanson's construction \cite{AbbottHanson}. Rowley named these partitions "templates", and we keep this name in 
the entire article. We then find new templates and use them to provide new lower bounds for Schur numbers.

\subsection{Definition of \(S^+\)}

\begin{definition}
A SF-template with length \(p\) and \(n\) colors is defined as a partition of \( [\![1,p]\!]\) into \(n\) sum-free subsets 
\(A_1, A_2, ..., A_n\) verifying:
\[
\forall i \in [\![1, n-1]\!], \forall (x,y) \in A_i^2, x+y > p
\Longrightarrow x+y-p \notin A_i
\]
\end{definition}

Here \(n\) is the "special" color: it has less constraints than the other colors. However, please note that \(n\) 
is not necessarily the last color by order of appearance. \(SF\)-templates include Abbott and Hanson's construction \cite{AbbottHanson}
as a special case.

\begin{proposition}
	Let \(n \in [\![2, +\infty[\![\). We define \(S^+(n)\) as the maximal lenght of a SF-template with \(n\) colors. 
	\(S^+(n)\) is well defined and verifies:
	\[
	2S(n-1)+1 \leqslant S^+(n) \leqslant S(n)
	\]
\end{proposition}

\begin{proof}[\textsc{Proof.}]
The lower bound comes from Abbott and Hanson's construction. The upper bound comes from the 
fact that a SF-template with length \(p\) and \(n\) colors is also a partition of \([\![1, p]\!]\) into \(n\) sum-free subsets.
\end{proof}

\begin{remark}
	\(S^+\) and \(S\) have the same asymptotic growth rate.
\end{remark}


\subsection{Construction of Schur partitions using SF-templates}

\qquad Here we state the main result on SF-templates stated by Rowley in the context of Ramsey numbers. It consists 
in the explicit construction of a sum-free partition using a SF-template and a sum-free partition.

\begin{theorem}
	Let \((n,k), (p,q) \in (\mathbb{N}^*)^2\). If there exists a \(SF\)-template with lenght \(p\) and \(k+1\) colors 
	and a partition of \([\![1,q]\!]\) into \(n\) sum-free subsets then there exists a partition of \([\![1,pq+m_{k+1}
	-1]\!]\) into \(n+k\) sum-free subsets. \(m_{k+1}\) is the minimum of the special subset in the \(SF\)-template.
\end{theorem}

Setting \(p = S^+(k+1)\) and \(q = S(n)\) yields the following corollary.

\begin{corollary}
	Let \(n, k \in \mathbb{N}^*\). Then
	\[ S(n+k) \geqslant S^+(k+1)S(n) + m_{k+1} - 1 \]
\end{corollary}

The idea lying beneath this theorem is similar to Abbott and Hanson's contruction \cite{AbbottHanson}. They vertically 
extend a sum-free partition by repeating it and they use an other sum-free partition to color the other half according 
to the line number. This way the "blocks" act as safe areas for each other. We give here an example for \(p = 9\), 
\(q = 4\), \(n = 2\) and \(k = 2\) showing that \(S(2 + 2) \geqslant S(2) (2 S(2) + 1) + S(2)\), both with Abbott 
and Hanson's construction and with a SF-template which is not of this type. In both cases, the special color is grey.

\renewcommand{\arraystretch}{1.7}

\begin{center}
\setlength{\tabcolsep}{1.5ex}

\textbf{Abbott and Hanson's construction}
\vspace{1.7ex}

\begin{NiceTabular}{|*{9}{c|}}[corners=SE,standard-cline,hlines]
\CodeBefore
	\cellcolor{red}{1-1}
	\cellcolor{green}{1-2}
	\cellcolor{green}{1-3}
	\cellcolor{red}{1-4}
	\cellcolor{cyan}{1-5}
	\cellcolor{cyan}{1-6}
	\cellcolor{cyan}{1-7}
	\cellcolor{cyan}{1-8}
	\cellcolor{cyan}{1-9}
	\cellcolor{red}{2-1}
	\cellcolor{green}{2-2}
	\cellcolor{green}{2-3}
	\cellcolor{red}{2-4}
	\cellcolor{yellow}{2-5}
	\cellcolor{yellow}{2-6}
	\cellcolor{yellow}{2-7}
	\cellcolor{yellow}{2-8}
	\cellcolor{yellow}{2-9}
	\cellcolor{red}{3-1}
	\cellcolor{green}{3-2}
	\cellcolor{green}{3-3}
	\cellcolor{red}{3-4}
	\cellcolor{yellow}{3-5}
	\cellcolor{yellow}{3-6}
	\cellcolor{yellow}{3-7}
	\cellcolor{yellow}{3-8}
	\cellcolor{yellow}{3-9}
	\cellcolor{red}{4-1}
	\cellcolor{green}{4-2}
	\cellcolor{green}{4-3}
	\cellcolor{red}{4-4}
	\cellcolor{cyan}{4-5}
	\cellcolor{cyan}{4-6}
	\cellcolor{cyan}{4-7}
	\cellcolor{cyan}{4-8}
	\cellcolor{cyan}{4-9}
	\cellcolor{red}{5-1}
	\cellcolor{green}{5-2}
	\cellcolor{green}{5-3}
	\cellcolor{red}{5-4}
\Body
	1 & 2 & 3 & 4 & 5 & 6 & 7 & 8 & 9 \\
	10 & 11 & 12 & 13 & 14 & 15 & 16 & 17 & 18 \\
	19 & 20 & 21 & 22 & 23 & 24 & 25 & 26 & 27 \\
	28 & 29 & 30 & 31 & 32 & 33 & 34 & 35 & 36 \\
	37 & 38 & 39 & 40 \\
\end{NiceTabular}

\vspace{1ex}
\setlength{\tabcolsep}{2ex}

\begin{tabular}{c c}
	\textbf{Corresponding SF-template} & \textbf{Corresponding sum-free partition} \\
	\begin{NiceTabular}{|*{9}{c|}}[standard-cline,hlines]
	\CodeBefore 
		\cellcolor{red}{1-1}
		\cellcolor{green}{1-2}
		\cellcolor{green}{1-3}
		\cellcolor{red}{1-4}
		\cellcolor{gray!40}{1-5}
		\cellcolor{gray!40}{1-6}
		\cellcolor{gray!40}{1-7}
		\cellcolor{gray!40}{1-8}
		\cellcolor{gray!40}{1-9}
	\Body
		1 & 2 & 3 & 4 & 5 & 6 & 7 & 8 & 9 \\
	\end{NiceTabular} &
	\begin{NiceTabular}{|*{4}{c|}}[standard-cline,hlines]
	\CodeBefore
		\cellcolor{cyan}{1-1}
		\cellcolor{yellow}{1-2}
		\cellcolor{yellow}{1-3}
		\cellcolor{cyan}{1-4}
	\Body
		1 & 2 & 3 & 4 \\
	\end{NiceTabular}
\end{tabular}

\setlength{\tabcolsep}{6pt}
\end{center}

In the general construction with SF-templates, the special color no longer necessarily contains consecutive 
numbers. However, the special color is still replaced by the colors of the sum-free partition according to the 
line number and the other colors are still vertically extended.

\begin{center}
\setlength{\tabcolsep}{1.5ex}

\textbf{SF-template construction}
\vspace{1.7ex}

\begin{NiceTabular}{|*{9}{c|}}[corners=SE,standard-cline,hlines]
\CodeBefore
	\cellcolor{red}{1-1}
	\cellcolor{green}{1-2}
	\cellcolor{green}{1-3}
	\cellcolor{red}{1-4}
	\cellcolor{cyan}{1-5}
	\cellcolor{cyan}{1-6}
	\cellcolor{red}{1-7}
	\cellcolor{cyan}{1-8}
	\cellcolor{cyan}{1-9}
	\cellcolor{red}{2-1}
	\cellcolor{green}{2-2}
	\cellcolor{green}{2-3}
	\cellcolor{red}{2-4}
	\cellcolor{yellow}{2-5}
	\cellcolor{yellow}{2-6}
	\cellcolor{red}{2-7}
	\cellcolor{yellow}{2-8}
	\cellcolor{yellow}{2-9}
	\cellcolor{red}{3-1}
	\cellcolor{green}{3-2}
	\cellcolor{green}{3-3}
	\cellcolor{red}{3-4}
	\cellcolor{yellow}{3-5}
	\cellcolor{yellow}{3-6}
	\cellcolor{red}{3-7}
	\cellcolor{yellow}{3-8}
	\cellcolor{yellow}{3-9}
	\cellcolor{red}{4-1}
	\cellcolor{green}{4-2}
	\cellcolor{green}{4-3}
	\cellcolor{red}{4-4}
	\cellcolor{cyan}{4-5}
	\cellcolor{cyan}{4-6}
	\cellcolor{red}{4-7}
	\cellcolor{cyan}{4-8}
	\cellcolor{cyan}{4-9}
	\cellcolor{red}{5-1}
	\cellcolor{green}{5-2}
	\cellcolor{green}{5-3}
	\cellcolor{red}{5-4}
\Body
	1 & 2 & 3 & 4 & 5 & 6 & 7 & 8 & 9 \\
	10 & 11 & 12 & 13 & 14 & 15 & 16 & 17 & 18 \\
	19 & 20 & 21 & 22 & 23 & 24 & 25 & 26 & 27 \\
	28 & 29 & 30 & 31 & 32 & 33 & 34 & 35 & 36 \\
	37 & 38 & 39 & 40 \\
\end{NiceTabular}

\vspace{1ex}
\setlength{\tabcolsep}{2ex}

\begin{tabular}{c c}
	\textbf{Corresponding SF-template} & \textbf{Corresponding sum-free partition} \\
	\begin{NiceTabular}{|*{9}{c|}}[standard-cline,hlines]
	\CodeBefore 
		\cellcolor{red}{1-1}
		\cellcolor{green}{1-2}
		\cellcolor{green}{1-3}
		\cellcolor{red}{1-4}
		\cellcolor{gray!40}{1-5}
		\cellcolor{gray!40}{1-6}
		\cellcolor{red}{1-7}
		\cellcolor{gray!40}{1-8}
		\cellcolor{gray!40}{1-9}
	\Body
		1 & 2 & 3 & 4 & 5 & 6 & 7 & 8 & 9 \\
	\end{NiceTabular} &
	\begin{NiceTabular}{|*{4}{c|}}[standard-cline,hlines]
	\CodeBefore
		\cellcolor{cyan}{1-1}
		\cellcolor{yellow}{1-2}
		\cellcolor{yellow}{1-3}
		\cellcolor{cyan}{1-4}
	\Body
		1 & 2 & 3 & 4 \\
	\end{NiceTabular}
\end{tabular}

\setlength{\tabcolsep}{6pt}
\end{center}

\resetarraystretch

We now proceed to prove the above mentioned theorem.\\

\begin{proof}[\textsc{Proof.}]
Denote by \(f\) the colouring associated to the \(SF\)-template of lenght \(p\) and \(g\) the one associated 
to the sum-free partition of \([\![1,q]\!]\); where \(f : [\![1,p]\!] \longrightarrow [\![1,k+1]\!]\) and 
\(g : [\![1,q]\!] \longrightarrow [\![1,n]\!]\).

\par
The sum-free condition is expressed as
\[\forall (x,y) \in [\![1,p]\!]^2, f(x) = f(y) \Longrightarrow f(x+y) \neq f(x)\]
\[\forall (x,y) \in [\![1,q]\!]^2 \text{, } g(x) = g(y) \Longrightarrow g(x+y) \neq g(x)\]

The additionnal constraint for the SF-template is
\[
\forall (x,y) \in [\![1,p]\!]^2, \left\{
\begin{array}{ll}
	f(x) = f(y) \leqslant k \\
	x + y > p
\end{array}
\right. \Longrightarrow f(x+y-p) \neq f(x)
\]

For \(x \in [\![1,pq+m_{k+1}-1]\!]\), write \(x = (\alpha - 1) + u\) for some integers \(\alpha \in \mathbb{Z}\) and 
\(u \in [\![1,p]\!]\). This decomposition is of course unique. \(\alpha\) (resp. \(u\)) can be interpreted as the row 
(resp. column) number of \(x\). Define a new coloring \(h\) as follows
\[
\begin{array}{c c c l}
	h : & [\![1,pq+m_{k+1}-1]\!] & \longrightarrow & [\![1,n+k]\!] \\
	& x & \longmapsto & 
	\left\{ \begin{array}{l l}
		f(u) & \text{if}~f(u) \leqslant k \\
		k + g(\alpha) & \text{if}~f(u) = k + 1 \\
	\end{array} \right.
\end{array}
\]

\(h\) is well-defined since, by definition of \(m_{k+1}\), \(\forall x \in [\![p q + 1, p q + m_{k + 1} - 1 ]\!], f(u) 
\leqslant k\) and therefore \(\forall x \in [\![1,pq+m_{k+1}-1]\!], f(u) = k + 1 \implies \alpha \in [\![1, q]\!]\).

\par
We now prove that \(h\) is a sum-free coloring. Let \(x,y \in [\![1,pq + m_{k+1}-1]\!]\) such that \(h(x) = h(y)\) 
and \(x+y \leqslant pq+m_{k+1}-1\). We claim that \(h(x+y) \neq h(x)\). Write \(x = (\alpha - 1) p + u\) and 
\(y = (\beta - 1) p + v\) where \(\alpha, \beta \in \mathbb{Z}\) and \(u, v \in [\![1,p]\!]\). Distinguish between 
two cases according to the value of \(h(x)\). \\

\noindent \underline{\textbf{First case:} \(h(x) \leqslant k\)} \\
\qquad Assume that \(h(x+y) \leqslant k\), otherwise \(h(x + y) \neq h(x)\) obviously holds. By definition of \(h\) 
and given that \(h(u) = h(v)\), \(f(u) = f(v)\). Distinguish between two cases according to the value of \(x + y\).

\begin{itemize}
\item If \(u + v > p\), write \(w = u + v - p \in [\![1, p]\!]\). Then \(x + y = (\alpha + \beta - 1) p + w\). By definition, 
	\(h(x + y) = f(w)\). Given that \(f(u) = f(v) \leqslant k\), the additionnal constraint on \(f\) implies \(f(w) 
	\neq f(u)\), that is \(h(x + y) \neq h(x)\).
\item If \(u + v \leqslant p\), write \(w = u + v \in [\![1, p]\!]\). Then \(x+y = (\alpha + \beta- 2) p + w\). By definition, 
	\(h(x + y) = f(w)\). Given that \(f(u) = f(v) \leqslant k\), the sum-free property of \(f\) implies \(f(w) \neq f(u)\), 
	that is \(h(x + y) \neq h(x)\).
\end{itemize} 
  
\noindent \underline{\textbf{Second case:} \(h(x) \geqslant k + 1\)} \\
\qquad \(h(x) = k + g(\alpha) = 
k + g(\beta) = h(y)\), hence \(g(\alpha) = g(\beta)\). As in the first case, distinguish between two cases according 
to the value of \(x + y\).

\begin{itemize}
\item If \(u + v > p\), write \(w = u + v - p \in [\![1, p]\!]\). Then \(x + y = (\alpha + \beta - 1) p + w\). Assume that 
	\(h(x+y) \geqslant k + 1\),  otherwise \(h(x + y) \neq h(x)\) obviously holds. By definition, \(h(x + y) = k + g(\alpha + 
	\beta)\). Given that \(g(\alpha) = g(\beta)\), the sum-free property of \(g\) implies \(g(\alpha + \beta) 
	\neq g(\alpha)\) that is \(h(x + y) \neq h(x)\).
\item  If \(u + v \leqslant p\), write \(w = u + v \in [\![1, p]\!]\). Then \(x+y = (\alpha + \beta- 2) p + w\). The sum-free 
	property of \(f\) implies \(f(w) \neq f(u)\). Therefore \(f(w) \leqslant k\) and thus \(h(x + y) \leqslant k\). In particular,
	given that \(h(x) \geqslant k + 1\), \(h(x + y) \neq h(x)\).
\end{itemize}
\end{proof}

The following proposition can help improve the additive constant of a SF-template. Although it does not allow us to 
improve the SF-templates we have found, the analogous of this proposition for WSF-templates (see next section) 
allows us to improve one of them.

\begin{proposition}
Let \((k, p) \in \mathbb{N}^*)^2\) and let \(f\) be a colouring associated to a SF-template of length \(p\) with \(k+1\) 
colors. Let \(b \in \mathbb{N}\) (\(b = m_{k+1} - 1\) works) and assume there there exists a colouring \(g\) of 
\([\![1, b]\!]\) with \(k+1\) colors such that:

\begin{itemize}
	\item \(\forall (x, y) \in [\![1, p]\!]^2,(f(x) = f(y) \text{ and } (x + y) \mod p \leqslant b) 
	\implies g((x + y) \mod p) \neq f(x)\)
	\item \(\forall (x, y) \in [\![1, p]\!] \times  [\![1, b]\!], (f(x) = g(y) \text{ and } x + y \leqslant b) 
	\implies g(x + y) \neq f(x)\)
\end{itemize}

Then, for every \(n \in \mathbb{N}^*\), by using on the last row the colouring \(i \longmapsto g(i - p S(n))\), we have
\[ S(n+k) \geqslant S^+(k+1)S(n) + b\]
\end{proposition}

This proposition corresponds to the fact that sometimes a column is not the sum of two columns of a given color, but adding this 
column to the color would create sums in the color when applying the extension procedure. However, the last line does not 
intereact will all the columns when it comes to creating new sums. As a result, the hypotheses made on the colouring of the last 
row can be weakened.

We also have a similar construction theorem where only \(S^+\) is involved.

\begin{theorem}
	Let \((n,k), (p,q) \in (\mathbb{N}^*)^2\). If there exists a \(SF\)-template of \(k+1\) colors and lenght \(p\),
	and \(SF\)-template of \(n\) color and lenght \(q\), then there exists \(SF\)-template of \((n+k)\) and lenght \(pq\).
\end{theorem}

And the associated inequality :

\begin{corollary}
	Let \(n, k \in \mathbb{N}^*\), we have \\
	\[ S^+(n+k) \geqslant S^+(k+1)S^+(n) \]
\end{corollary}

\begin{proof}[\textsc{Proof.}]
The idea is the same as in the previous theorem. The only difference is the SF property inherited 
from the second SF-template.
\end{proof}


\subsection{Inequalities and new lower bounds for Schur numbers}

\begin{definition}
A SF-template with \(n\) colors is said to be symmetric if the partition in \(n\) sum-free subsets derived (with the additive constant) from this template is symmetric. 
A sum-free partition \(A_1, ..., A_n\) of \([\![1, p]\!]\) is said to be symmetric if for all \( x \in [\![1, p]\!]\), \(x\) and \(p + 1 - x\) belong to the same subset 
(except if \(x = p + 1 - x\)).
\end{definition}

Using a SAT solver, we exhibited SF-templates, hence providing lower bound on \(S^+\) and inequalities 
of the type \(S(n+k) \geqslant a S(n) + b\). We have sought templates providing the greatest value of 
\((a, b)\) (for the lexicographic order). When the number of colors excceeded 5, in order to reduce the search space we 
looked for symmetric SF-templates, we assumed that the special color was the last color to appear and we constrained 
the \(m_c\)'s out of being too small. Further details about the encoding as a SAT problem can be found in \cite{Heule2017}.

\par
Here are the best inequalities on Schur numbers so far (the templates corresponding to the third, fourth and fifth 
inequalities can be found in the appendix):
\[
\begin{array}{r r r r r}
	S(n + 1) & \geqslant & 3 S(n) & + & 1 \\
	S(n + 2) & \geqslant & 9 S(n) & + & 4 \\
	S(n + 3) & \geqslant & 33 S(n) & + & 6 \\
	S(n + 4) & \geqslant & 111 S(n) & + & 43 \\
	S(n + 5) & \geqslant & 380 S(n) & + & 148 \\
	S(n + 6) & \geqslant & 1\,140 S(n) & + & 528 \\
\end{array}
\]

The first inequality comes from  Schur's original article\cite{Schur1917}. The second one is due to
Abott \cite{AbbottHanson}
and the third one to Rowley \cite{RowleyRamsey}. The other ones are new.

\par
The first 3 inequalities are optimal. The fourth one is optimal among symmetric SF-templates whose special color is 
the last in the order of apparition (and with a multiplicative factor less than or equal to 118). The fifth one is most 
likely not optimal but should not be too far from the optimal. 
Finally, the sixth one is obtained by combining (see below) the SF-template of length 380 and the one of length 3. 
Although we could not find a better SF-template with 7 colors, the last inequality is definetely very far from the 
optimal value. One may try to seek better templates by constraining less the search space and by using 
Monte-Carlo methods, as in \cite{Bouzy2015AnAP}. This could be the suject of a future work.

\par
The previous inequalities give new lower bounds for \(S(n)\) for
\( n \geqslant 9 \). We compute the lower
bounds for \( n \in [\![8,15]\!] \) using the four different inequalities, please notice that the best values for \( n
=8\) and \(n = 13\) were obtained thanks to the first one, found by Rowley. The best lower bounds are highlighted.

\begin{center}
\begin{tabular}{|*{5}{c|}}
	\hline
	\(n\) & 8 & 9 & 10 & 11 \\
	\hline
	\(33S(n-3) + 6 \) & \cellcolor{yellow} 5\,286 & 17\,694 & 55\,446 & 174\,444\\
	\hline
	\(111S(n-4) + 43 \) & 4927 & \cellcolor{yellow} 17\,803 & 59\,539 & 186\,523\\
	\hline
	\(380S(n-5) + 148 \) & 5\,088 & 16\,868 & \cellcolor{yellow} 60\,948 & \cellcolor{yellow} 203\,828 \\
	\hline
	\(1\,140S(n-6) + 528 \) & 5\,088 & 15\,348 & 50\,688 & 182\,928\\
	\hline
	\hline
	\(n\) & 12 & 13 & 14 & 15 \\
	\hline
	\(33S(n-3) + 6 \) & 587\,505 & \cellcolor{yellow} 2\,011\,290 & 6\,726\,330 & 21\,072\,090\\
	\hline
	\(111S(n-4) + 43 \) & 586\,789 & 1\,976\,176 & 6\,765\,271 & 22\,624\,951 \\
	\hline
	\(380S(n-5) + 148 \) & \cellcolor{yellow} 638\,548 & 2\,008\,828 & \cellcolor{yellow} 6\,765\,288 & \cellcolor{yellow} 23\,160\,388 \\
	\hline
	\(1\,140S(n-6) + 528 \) & 611\,568 & 1\,915\,728 & 6\,026\,568 & 20\,295\,948 \\
	\hline
\end{tabular}
\end{center}

Except for 8, 9 and 13, the best lower bounds are obtained thanks to
the third inequality \( S(n+5) \geqslant 380S(n) + 148\). The table
doesn't go any further, but the same inequality allows to improve the
lower bounds for every \( n \geqslant 15 \).

\begin{corollary}
The growth rate for Schur numbers (and Ramsey numbers \(R_n(3)\))  satisfies \(\gamma \geqslant \sqrt[5]{380} \approx 3.28\).
\end{corollary}

\begin{proof}[\textsc{Proof.}]
It is a mere consequence of the inequality \( S(n+5) \geqslant 380S(n) + 148\). As for Ramsey
numbers, the following inequality holds \(S(n) \leqslant R_n(3) - 2\) (see \cite{AbbottHanson}).
\end{proof}


\subsection{Conclusion on SF-templates}

\qquad In this section, we first formalized Rowley's template-based constructions \cite{RowleyRamsey} in the context of Schur numbers 
by introducing SF-templates as well as a new sequence, \(S^+\). We provided relations between \(S^+\) and \(S\) then 
stated Rowley's construction method in the context of Schur numbers. We found new SF-templates allowing us to obtain 
new lower bounds for schur numbers. One may notice that we mostly gave only lower bounds for \(S^+\). It should be possible to 
find better SF-templates by making different assumptions or using a different method (Monte-Carlo methods for instance).

\par
In the next section, we provide similar results for weak Schur numbers. We introduce WSF-templates and a corresponding sequence, 
\(WS^+\). We then derive similar relations and construction method allowing us to find new lower bounds for weak Schur numbers.


\section{Templates for weak Schur numbers}
\label{WeakSchur}

In this section, we generalize Rowley's constructions for weak Schur numbers \cite{RowleyWS} and give an analogous
for weak Schur numbers of Abbott and Hanson's construction for Schur numbers \cite{AbbottHanson}. By analogy with the previous section,
we then introduce \textit{WS-templates}, standing for \textit{"weak Schur templates"}, as well as an associated sequence \(\WS^+(n)\). 
We find templates and use them to provide new lower bounds for weak Schur numbers. Finally, we give a short explanation for 
the new lower bound \(\WS(6) \geqslant 646\) which was not directly obtained with a template contrary to the other lower bounds 
given in this paper.

\subsection{Inequality for weak Schur numbers using Schur and weak Schur numbers}

Up to now, no equivalent for weak Schur numbers of Abbott and Hanson's construction for Schur numbers
\cite{AbbottHanson} was known. Here we give a general lower bound for weak Schur numbers as a function of both
regular and weak Schur numbers.

\begin{theorem}
\label{thm:WS-ab}
\begin{sloppypar}
Let \((p,k), (q,n) \in (\mathbb{N}^*)^2\). If there exists a partition of \([\![1,q]\!]\) into \(n\) weakly sum-free
subsets and a partition of \([\![1,p]\!]\) into \(k\) sum-free
subsets then there exists a partition of \({[\![1,p(q+\left \lceil \frac{q}{2} \right \rceil + 1)+q]\!]}\) into \(n+k\)
weakly sum-free subsets.
\end{sloppypar}
\end{theorem}

In particular, by setting \(q = \WS (n)\) and \(p = S(k)\) in Theorem \ref{thm:WS-ab}, one obtains the following corollary.

\begin{corollary}
\label{cor:ineqWS-S}
\( \forall (n,k) \in (\mathbb{N}^*)^2 \text{, } \WS (n+k) \geqslant S(k) \left (\WS (n) + \left \lceil \displaystyle \frac{\WS (n)}{2}
\right \rceil +1 \right) + \WS (n)\)
\end{corollary}

This can be seen as an equivalent for weak Schur numbers of Abott and Hanson's construction for Schur numbers. This formula includes
the results of Rowley \cite{RowleyWS} as a special case. For \(n>2\), this formula does not give new lower bounds.

\begin{remark}
The inequality from Corollary \ref{cor:ineqWS-S} can be improved by adding 1 to the lower bound if \(\WS (n)\) is odd (more generally if \(q\) is
odd in Theorem \ref{thm:WS-ab}). However, it lengthens the proof and it is never useful in practice.
\end{remark}

Given that Theorem \ref{thm:WS-ab} will appear as a particular case of a more general theorem after the introduction of
templates for weak Schur numbers, we only give here an intuitive explanation of the demonstration; a formal
\hyperref[PreuveThm]{proof} using templates for weak Schur numbers is provided in Subsection \ref{ConstructionWS}.

Let \((p, k), (q, n) \in (\mathbb{N}^*)^2\) such that there exists a partition of \([\![1,q]\!]\) into \(n\) weakly sum-free
subsets and a partition of \([\![1,p]\!]\) into \(k\) sum-free subsets. Let \(a \in \mathbb{N}\) with \(a > q\)
and let us try to build a coloring of \([\![1, ap + q]\!]\) into \(n + k\) weakly sum-free subsets. Let
\(l = a - b - 1\), \(r \in [\![1,q]\!]\) and \(w = a - l - r - 1 = b - r\).

First, we put the integers of \([\![1, ap + q]\!]\) in the following table (with \(a\) columns and \(p + 1\) lines)
and divide it into 3 blocks (the columns are numbered from \(-l\) to \(+q\)).

\begin{itemize}
	\item \(\mathcal{T}\) (the "tail"): the integers from 1 to \(q\). NB: this is line number 0.
	\item \(\mathcal{R}\) (the "rows"): the integers in columns \(-l\) to \(+r\) (excluding  \(\mathcal{T}\)).
	\item \(\mathcal{C}\) (the "columns"): the integers in the last \(w\) columns (excluding  \(\mathcal{T}\)).
\end{itemize}

Like S-templates, \(\mathcal{R}\) and \(\mathcal{C}\) play the role of security zones for each other. Note that with
this numbering of columns, the column of the sum of two numbers is the only integer in \([\![-l,q]\!]\) equal to two the
sum of the columns modulo \(a\).

\renewcommand{\arraystretch}{1.7}
\setlength{\arraycolsep}{3pt}

\begin{figure}[H]
\[
\begin{NiceArray}{*{13}{c}}[corners={NW,SW},hvlines,first-row,last-row,first-col]
\CodeBefore
	\cellcolor{green}{1-6}
	\cellcolor{green}{1-7}
	\cellcolor{cyan}{1-8}
	\cellcolor{green}{1-9}
	\cellcolor{cyan}{1-10}
	\cellcolor{cyan}{1-11}
	\cellcolor{cyan}{1-12}
	\cellcolor{green}{1-13}
	\cellcolor{red}{2-1}
	\cellcolor{red}{2-2}
	\cellcolor{red}{2-3}
	\cellcolor{red}{2-4}
	\cellcolor{red}{2-5}
	\cellcolor{red}{2-6}
	\cellcolor{red}{2-7}
	\cellcolor{red}{2-8}
	\cellcolor{red}{2-9}
	\cellcolor{cyan}{2-10}
	\cellcolor{cyan}{2-11}
	\cellcolor{cyan}{2-12}
	\cellcolor{green}{2-13}
	\cellcolor{yellow}{3-1}
	\cellcolor{yellow}{3-2}
	\cellcolor{yellow}{3-3}
	\cellcolor{yellow}{3-4}
	\cellcolor{yellow}{3-5}
	\cellcolor{yellow}{3-6}
	\cellcolor{yellow}{3-7}
	\cellcolor{yellow}{3-8}
	\cellcolor{yellow}{3-9}
	\cellcolor{cyan}{3-10}
	\cellcolor{cyan}{3-11}
	\cellcolor{cyan}{3-12}
	\cellcolor{green}{3-13}
	\cellcolor{yellow}{4-1}
	\cellcolor{yellow}{4-2}
	\cellcolor{yellow}{4-3}
	\cellcolor{yellow}{4-4}
	\cellcolor{yellow}{4-5}
	\cellcolor{yellow}{4-6}
	\cellcolor{yellow}{4-7}
	\cellcolor{yellow}{4-8}
	\cellcolor{yellow}{4-9}
	\cellcolor{cyan}{4-10}
	\cellcolor{cyan}{4-11}
	\cellcolor{cyan}{4-12}
	\cellcolor{green}{4-13}
	\cellcolor{red}{5-1}
	\cellcolor{red}{5-2}
	\cellcolor{red}{5-3}
	\cellcolor{red}{5-4}
	\cellcolor{red}{5-5}
	\cellcolor{red}{5-6}
	\cellcolor{red}{5-7}
	\cellcolor{red}{5-8}
	\cellcolor{red}{5-9}
	\cellcolor{cyan}{5-10}
	\cellcolor{cyan}{5-11}
	\cellcolor{cyan}{5-12}
	\cellcolor{green}{5-13}
	\cellcolor{cyan}{6-10}
	\cellcolor{cyan}{6-11}
	\cellcolor{cyan}{6-12}
	\cellcolor{green}{6-13}
\Body
	& & & & & & \Block{1-8}{\overbrace{\hphantom{---------------------------}}^{\mathcal{T}}} \\
	& & & & & & 1 & 2 & ... & r & r + 1 & ... & b - 1 & b \\
	\Block{5-1}{\mathcal{R} \left\{ \vphantom{\begin{array}{l} . \\ . \\ . \\ . \\ . \\ . \\ . \\ . \\ . \end{array}} \right.}
	& a - l & a - l + 1 & ... & a - 1 & a & a + 1 & ... & a + r - 1 & a + r & a + r + 1 & ... & a + b - 1 & a + b \\
	& 2 a - l & ... & ... & ... & 2 a & ... & ... & ... & 2 a + r & ... & ... & ... & 2 a + b \\
	& ... & ... & ... & ... & ... & ... & ... & ... & ... & ... & ... & ... & ... \\
	& ... & ... & ... & ... & ... & ... & ... & ... & ... & ... & ... & ... & ... \\
	& p a - l & ... & ... & ... & p a & ... & ... & ... & p a + r & ... & ... & ... & p a + b \\
	& & & & & & & & & & \Block{1-4}{\underbrace{\hphantom{--------------}}_{\mathcal{C}}} \\
\end{NiceArray}
\]
\label{SchemaWS}
\caption{Construction of the weakly sum-free coloring}
\end{figure}

\resetarraystretch
\setlength{\arraycolsep}{6pt}

\noindent \underline{\textbf{\(\mathcal{T}\) block}}
\par
We color this block using the weakly sum-free coloring of \([\![1,q]\!]\) with colors \(1, ..., n\).

\noindent \underline{\textbf{\(\mathcal{R}\) block}}
\par
In this block,  we use the colors \(n + 1, ..., n + k\). We color an integer \(x\) according to its line number (written \(\lambda(x)\)).
For every \(x \in \mathcal{R}\), we color \(x\) with \(n + c\) where \(c\) is the color of \(\lambda(x)\) in the sum-free coloring of  \([\![1,p]\!]\).
Let \((x, y) \in \mathcal{R}^2\). The cases are twofold.

\begin{itemize}
	\item \underline{\(\lambda(x+y) = \lambda(x) + \lambda(y)\)} \\
	In this case, we use the sum-free property of the coloring of \([\![1,p]\!]\) (in block \(\mathcal{C}\), we only
	use colors \(1, ..., n\)).
	\item \underline{\(\lambda(x+y) \neq \lambda(x) + \lambda(y)\)} \\
	In this case, we do not have information about the color of \(\lambda(x+y)\). Thereby, we want to have
	\(x+y \in \mathcal{C}\). A simple solution is to limit the horizontal movement, that is if the sum changes line
	(that is its line number is different from \(\lambda(x) + \lambda(y)\)), not to move too far from \((\lambda(x)
	+ \lambda(y)) a\) so that the sum stays in \(\mathcal{C}\). There, the maximal displacement to the left (resp.
	to the right) is \(2l\) (resp. \(2r\)). Not crossing entirely \(\mathcal{C}\) by going to the left is then expressed
	as \(-2l > -a + r\). Likewise, not going to far to the right is expressed as \(2r < a - l\). It can then be written
	as \(\max(l, r) \leqslant w\).
\end{itemize}

\noindent \underline{\textbf{\(\mathcal{C}\) block}}
\par
In this block,  we use colors \(1, ..., n\). We color an integer \(x\) according to its column number, denoted by \(\tilde{\pi}(x)\). It is linked to the
projection on the first line, denoted by \(\pi\), by the relation \(\tilde{\pi}(x) = \pi(x) - a\). A simple solution is to color \(x\) with the same color
as \(\tilde{\pi}(x)\) in the weakly sum-free coloring of \([\![1,q]\!]\). As long as \(2b \leqslant a + r\) (not going two far to the right) and there
is no \(x \in \tilde{\pi}(\mathcal{C})\) such that \(2x \in \tilde{\pi}(\mathcal{C})\) (so that we do not have a sum in \(\mathcal{C}\) when
taking two numbers in the same column), the colors \(1, ..., n\) are sum-free.

In particular, taking \(w = l = \left\lceil \displaystyle \frac{q}{2} \right\rceil\) and \(r = \left\lfloor \displaystyle \frac{q}{2} \right\rfloor\) works, thus obtaining the above theorem.\\
\par
As in the previous section, we now introduce WS-templates and the sequence \(\WS^+\) in order to generalize the above construction.

\subsection{Definition of \([\![ 1, \WS^+ ]\!]\)}
\label{DefinitionWS+}

In this subsection, we introduce WS-templates and prove calculative results for the general construction
theorem on templates for weak Schur numbers.

\begin{definition}
Let \((a,b) \in (\mathbb{N}^*)^2\) such that \(a>b\). We define :
\[ \pi_{a,b}:x \longmapsto (\operatorname{Id}+a\mathds{1}_{ [\![0,b]\!]})(x \mod a).\]
\end{definition}

\begin{sloppypar}
If there is no confusion on the \(a\) and \(b\) to use, \(\pi_{a, b}\) is denoted by \(\pi\). Notice that for all \(x \in \mathbb{Z}\), 
\({\pi(x) = x \mod a}\) and for all \(x \in [\![b + 1, a + b]\!], b + 1 \leqslant \pi(x) \leqslant a + b\).
\end{sloppypar}

\begin{sloppypar}
\(\pi\) is the projection on the first line mentioned in the intuitive explanation. The following four propositions are
calculative properties on \(\pi\) reflecting the behaviour of an element's column number in \hyperref[SchemaWS]{Figure 3} and
that we will use later when we introduce WS-templates.
\end{sloppypar}

\begin{proposition}
\label{prop1}
\[
\forall x \in [\![b + 1, a + b]\!], \pi(x) = x.
\]
\end{proposition}

\begin{proof}[\textsc{Proof.}]
\begin{sloppypar}
Let \(x \in [\![b + 1, a + b]\!]\). If \(x < a\) then \(x \mod a = x \notin [\![1, b]\!]\). Hence \(\pi(x) = x\).
Otherwise, \({x \mod a = x - a \in [\![1, b]\!]}\). Hence \({\pi(x) = x - a + a = x}\).
\end{sloppypar}
\end{proof}

\begin{proposition}
\label{prop2}
Let x \(\in [\![1,b]\!]\) and \(y \in \mathbb{N}^*\). Then
\[
\pi(x+\pi(y)) = \pi(x+y).
\]
\end{proposition}

\begin{proof}[\textsc{Proof.}]
It is a direct consequence of \(\pi(x) = x \mod a\). \\
\end{proof}

\begin{proposition}
\label{prop3}
Let x \(\in [\![1,b]\!]\) and \(y \in \mathbb{N}^*\) such that \(x+\pi(y) \leqslant a+b\). Then
\[
\pi(x+y)=x+\pi(y).
\]
\end{proposition}

\begin{proof}[\textsc{Proof.}]
\(\pi(y) \geqslant b + 1\) and thus \(b + 1 \leqslant x + \pi(y) \leqslant a + b\).
\[
\begin{array}{l l l l}
	\pi(x + y) & = & \pi(x + \pi(y)) & \text{by Proposition \ref{prop2}} \\
	 & = & x + \pi(y) & \text{by Proposition \ref{prop1}}
\end{array}
\]
\end{proof}

\begin{proposition}
\label{prop4}
Let \((x,y)\in (\mathbb{N}^*)^2\). Then
\[
\pi(\pi(x)+\pi(y))=\pi(x+y).
\]
\end{proposition}

\begin{proof}[\textsc{Proof.}]
It is a direct consequence of \(\pi(x) = x \mod a\). \\
\end{proof}

After defining the function related to the column number of each element in \hyperref[SchemaWS]{Figure 3},
we introduce the function related to its line number.

\begin{definition}
Let \((a,b) \in (\mathbb{N}^*)^2\) such that \(a>b\). Define
\[ \lambda_{a,b}:x \longmapsto 1+ \left\lfloor\dfrac{x-b-1}{a}\right\rfloor.\]
\end{definition}

If there is no confusion on the \(a\) and \(b\) to use, \(\lambda_{a, b}\) is denoted by \(\lambda\).

Function \(\lambda\) maps an element \(x\) to its line number as mentioned in the intuitive explanation.
As we just did with \(\pi\), we prove three more calculative results on both \(\pi\) and \(\lambda\) that are used in
Subsection \ref{ConstructionWS}.

\begin{proposition}
\label{prop5}
Let \(x\in \mathbb{N}^*\). Then
\[
x=a\lambda(x)+\pi(x)-a.
\]
\end{proposition}

\begin{proof}[\textsc{Proof.}]
Let \((a,b)\in (\mathbb{N}^*)^2\) such that \(a>b\) and let \(x\in \mathbb{N}^*\). \\
We have \(a\lambda(x)+\pi(x)-a=a\left\lfloor\dfrac{x-b-1}{a}\right\rfloor+(x \mod a)+ a \mathds{1}_{ [\![0,b]\!]}(x \mod a)\).

\begin{itemize}
\item If \(x \mod a>b\) then \(a\lambda(x)+\pi(x)-a=a\left\lfloor\dfrac{x}{a}\right\rfloor+x \mod a=x\).
\item If \(x \mod a \leqslant b\) then \(a\lambda(x)+\pi(x)-a=a \left( \left \lfloor \dfrac{x}{a} \right \rfloor - 1 \right)+x \mod a +a=x\).
\end{itemize}
\end{proof}

\begin{proposition}
\label{prop6}
Let \(x, y \in \mathbb{Z}\) such that \(\lambda(x + y) = \lambda(y)\). Then
\[
\pi(x + y) = x + \pi(y).
\]
\end{proposition}

\begin{proof}[\textsc{Proof.}]
By applying Proposition \ref{prop5} twice, we get \(a \lambda(x + y) + \pi(x + y) - a = x + y = x + a \lambda(y) + \pi(y) - a\).
The result is then obtained by simplifying the equality. \\
\end{proof}

\begin{proposition}
\label{prop7}
Let \(x, y \in \mathbb{Z}\) such that \(\pi(x) + \pi(y) \in [\![a + b + 1, 2 a + b]\!]\). Then
\[
\lambda(x + y) = \lambda(x) + \lambda(y)
\]
\end{proposition}

\begin{proof}[\textsc{Proof.}]
By Proposition \ref{prop5}, \(x + y = a (\lambda(x) + \lambda(y)) + \pi(x) + \pi(y) - 2 a\). Then
\[
\begin{array}{l l l}
	\lambda(x + y) & = & \left \lfloor \dfrac{x + y - b - 1}{a} \right \rfloor + 1 \vspace{1mm} \\
	 & = & \left \lfloor \dfrac{a (\lambda(x) + \lambda(y)) + \pi(x) + \pi(y) - 2 a - b - 1}{a} \right \rfloor + 1 \vspace{1mm} \\
	 & = & \lambda(x) + \lambda(y) - 1 \left \lfloor \dfrac{\pi(x) + \pi(y) - b - 1}{a}\right\rfloor \\
	 & = & \lambda(x)+\lambda(y) -1 +1 ~~~~\text{since } \pi(x) + \pi(y) \in [\![a + b + 1, 2 a + b]\!] \\
	 & =& \lambda(x)+\lambda(y)
\end{array}
\]
\end{proof}

\begin{definition}
Let \( (a,n,b) \in (\mathbb{N}^*)^3\) with \(a > b\). Let \((A_1,...,A_n)\) a partition of  \([\![1, a + b]\!]\).
This partition is said to be a \(b\)-weakly-sum-free template (\(b\)-WS-template) with width \(a\) and \(n\) colors when :

\begin{itemize}
\item \(\forall i \in [\![1, n]\!], A_i\) is weakly-sum-free,
\item \(\forall i \in [\![1, n]\!], A_i\backslash [\![1, b]\!]\) is sum-free,
\item For \(A_n\) (the special subset) :
	\[
	\forall (x,y) \in A_n^2, \,x+y>b+2a \implies x+y-2a\notin A_n,
	\]
\item For the others subsets
	\[
	\forall i \in [\![1,n-1]\!], \, \forall(x,y) \in A_i^2, \, x+y>a+b \implies \pi(x+y) \notin A_i.
	\]
\end{itemize}
\end{definition}

Note that the special color \(n\) is not necessarily the last color by order of appearance, any color can play this role.
We now introduce the number \(\WS^+(n)\) that plays the same role as \(S^+(n)\) for S-templates.
However, WS-templates being more sophisticated than S-templates, the definition of \(\WS^+(n)\)
is slighty more complicted.

\begin{definition}
Let \( (n,b) \in (\mathbb{N}^*)^2\). If there exists \(a\) such that there exists a \(b\)-WS-template with width \(a\)
and \(n\) colors, we define :
\[
\WS_b^+(n)= \max \{a \in \mathbb{N}^*/ \text{there exists a } b \text{-WS-template with width } a \text{ and } n \text{ colors} \}.
\]
If no such \(a\) exists, we set \(\WS_b^+(n) = 0\).
\end{definition}

\begin{definition}
Let \( n \in \mathbb{N}^*\). We define :
\[
\WS^+(n)=\max_{b\in \mathbb{N}^*} \WS_b^+(n).
\]
\end{definition}
The following proposition briefly shows how \(\WS^+(n)\) compares to weak Schur numers.
\begin{proposition}
Let \(n \in [\![2, +\infty]\!]\). Then :
\[
\frac{3}{2} \WS (n-1)+1 \leqslant \WS^+(n) \leqslant \WS (n).
\]
\end{proposition}

\begin{proof}[\textsc{Proof.}]
The lower bound comes from Corollary \ref{cor:ineqWS-S}.
The upper bound comes from the fact that a WS-template with width \(a\) and \(n\) colors is in particular a partition of
\([\![1, a]\!]\) into \(n\) weakly sum-free subsets. \\
\end{proof}

Having established these properties, we now have everything we need to state and prove our main result. We proceed to do so
in the next subsection.

\subsection{Construction of weak Schur partitions using WS-templates}
\label{ConstructionWS}

Theorem \ref{thm:Stemp} provides a way to extend a S-template into a larger Schur partitions using other Schur partitions. Likewise, a WS-template can 
be extended into larger weak Schur partitions by using Schur partitions. This is the object of the following theorem.

\begin{theorem}
\label{thm:WStemp}
Let \((a,n,b) \in (\mathbb{N}^*)^3\) with \(a > b\) and \( (p,k) \in (\mathbb{N}^*)^2\). If there exists a partition of \([\![1,p]\!]\)
into \(k\) sum-free subsets and a \(b\)-WS-template \((A_1,...,A_{n+1})\) with width \(a\) and \(n+1\) colors,
then there exists a partition of \([\![1, p a + b]\!]\) into \(k+n\) weakly sum-free subsets.
\end{theorem}

In particular, by setting \(p = S(k)\) and \(a = \WS^+(n + 1)\) in Theorem \ref{thm:WStemp}, the next corollary follows.

\begin{corollary}
\label{cor:ineqWS+}
Let \(n,k \in \mathbb{N}^*\) and set \( b_{max} = \max \{b\in \mathbb{N}^*/ \WS_b^+(n+1) = \WS^+(n+1)\}\).
Then :
\[
\WS(n+k) \geqslant S(k) \WS^+(n+1)+b_{max}.
\]
\end{corollary}

\begin{remark}
In the S-template construction for Schur numbers, the additive constant comes from the fact that the special color does
not necessarily appear right at the begining of the repeating pattern. Likewise, \(b_{max}\) can actually be replaced by
\[
\max_{b \in \mathbb{N}^*} \left\{\min (A_{n+1} \backslash [\![1, b ]\!]) - 1~|~ \WS_b^+(n+1) = \WS^+(n+1) \right\}.
\]
\end{remark}

\begin{proof}[\textsc{Proof.}]
\begin{sloppypar}
Let \((a,n,b) \in (\mathbb{N}^*)^3\) and \((p,k) \in (\mathbb{N}^*)^2\). Denote by \(f\) the coloring
associated to the \(b\)-WS-template  and \(g\) the one associated to the sum-free partition of \([\![1,p]\!]\); where
\({f : [\![1, a + b]\!] \longrightarrow [\![1,n+1]\!]}\) and \({g : [\![1, p]\!]  \longrightarrow [\![1, k]\!]}\). Moreover, assume
that the sum-free coloring of \([\![1, p]\!]\) is ordered.
\end{sloppypar}

\par
NB: To keep the notation short, the conditions \(x + y \leqslant p\)  and \(x + y \leqslant a + b\) are omitted in the following five predicates.
\par
The (weakly) sum-free conditions are expressed as:
\begin{equation}
\forall (x,y) \in [\![1,a + b]\!]^2, \left\{
\begin{array}{l}
	f(x) = f(y) \\
	x \neq y
\end{array}
\right. \Longrightarrow f(x+y) \neq f(x),
\end{equation}
\begin{equation}
\forall (x,y) \in [\![b+1,a + b]\!]^2, f(x) = f(y) \Longrightarrow f(x+y) \neq f(x),
\end{equation}
\begin{equation}
\forall (x,y) \in [\![1,p]\!]^2, g(x) = g(y) \Longrightarrow g(x+y) \neq g(x).
\end{equation}

The additionnal constraints for the WS-template are:
\begin{equation}
\forall (x,y) \in [\![1,a + b]\!]^2, \left\{
\begin{array}{l}
	f(x) = f(y) \leqslant n \\
	x + y > a + b
\end{array}
\right. \Longrightarrow f(\pi(x+y)) \neq f(x),
\end{equation}
\begin{equation}
\forall (x,y) \in [\![1,a + b]\!]^2, \left\{
\begin{array}{l}
	f(x) = f(y) = n + 1 \\
	x + y > 2 a + b
\end{array}
\right. \Longrightarrow f(x+y - 2 a) \neq f(x).
\end{equation}

Split \([\![1, p a + b]\!]\) into three subsets. \\
NB: To keep the notation short, the restriction to \([\![b + 1, p a + b]\!]\) of \(\pi\) defined in Subsection \ref{DefinitionWS+}
is denoted by \(\pi\) in the hereunder equations.

\begin{itemize}
	\item \(\mathcal{T} = [\![1, b]\!]\)
	\item \(\mathcal{C} = \pi^{-1}(f^{-1}([\![1, n]\!]))\)
	\item \(\mathcal{R} = \pi^{-1}(f^{-1}(\{n + 1\}))\)
\end{itemize}

A new coloring \(h\) is defined as follows:
\[
\begin{array}{c c c l}
	h : & [\![1, p a + b]\!] & \longrightarrow & [\![1,n+k]\!] \\
	& x & \longmapsto &
	\left\{ \begin{array}{l l}
		f(x) & \text{if}~x \in \mathcal{T} \\
		f(\pi(x)) & \text{if}~x \in \mathcal{C} \\
		n + g(\lambda(x)) & \text{if}~x \in \mathcal{R}
	\end{array} \right.
\end{array}
\]

Function \(h\) is well defined since \((\mathcal{T}, \mathcal{C}, \mathcal{R})\) is a partition of  \([\![1, p a + b]\!]\).
We now prove that \(h\) is a weakly sum-free coloring. Let \(x,y \in [\![1, p a + b]\!]\) be such that \(x \neq y\),
\(h(x) = h(y)\) and \(x+y \leqslant p a+ b\). We claim that \(h(x+y) \neq h(x)\). Nine cases are to be distinguished
according to the subsets \((\mathcal{T}, \mathcal{C}, \mathcal{R})\) to which \(x\) and \(y\) belong. It is
sufficient to check only six cases out of nine since \(x\) and \(y\) play symmetric roles. \\

\noindent \underline{\textbf{Case 1:} \((x,y) \in \mathcal{T}^2\)}
\par
If \(x + y \leqslant b\) then \(h(x+y)=f(x+y)\). Otherwise, \(b < x+y < a+b\) since \(b < a\) and
therefore \(\pi(x + y) = x +y\) (Proposition \ref{prop1}). Hence in both cases \(h(x+y)=f(x+y)\). Given that
\(f\) is a weakly sum-free coloring, \(f(x + y) \neq f(x)\) since \(f(x)=h(x)=h(y)=f(y)\) and \(x \neq y \). That
is \(h(x + y) \neq h(x)\). \\

\noindent \underline{\textbf{Case 2:} \((x,y) \in \mathcal{T} \times \mathcal{C}\)}
\par
Given that \(h(x) = h(y)\) and by definition of \(h\), \(f(x) = f(\pi(y))\). Besides, \(f(\pi(y)) \leqslant n\) since
\(y \in \mathcal{C}\). Two cases are to be distinguished according to the value of \(x + \pi(y)\).
\begin{itemize}
\item \begin{sloppypar}
	If \(x + \pi(y) \leqslant a + b\) then \(f(x + \pi(y)) = f(\pi(x + y))\) (Proposition \ref{prop3}). Given that \(f\) is
	a weakly sum-free coloring, \(f(x+\pi(y)) \neq f(x)\) since \(f(x)=f(\pi(y))\) and \(x \neq \pi(y)\) since
	\({x \leqslant b < \pi(y)}\).
	\end{sloppypar}
\item \begin{sloppypar}
	If \(x+\pi(y)> a+b\) then given that \(f\) is a WS-template and since \({f(x) = f(\pi(y)) \leqslant n}\),
	\({f(\pi(x+\pi(y))) \neq f(x)}\).~Furthermore \({f(\pi(x+\pi(y))) = f(\pi(x+y))}\) (Proposition \ref{prop2}), such that
	\({f(\pi(x+ y)) \neq f(x)}\).
	\end{sloppypar}
\end{itemize}
\par
Hence in both cases \(f(\pi(x+y)) \neq f(x)\). If  \(f(\pi(x+y)) \leqslant n\) then \(h(x+y) = f(\pi(x+y))\). Therefore
\(h(x+y) \neq h(x)\) since \(f(x) = h(x)\). Otherwise, \(f(\pi(x+y)) = n + 1\) and thus \(h(x+y) > n\). In particular,
\(h(x + y) \neq h(x)\) since \(h(x) = h(y) \leqslant n\). \\

\noindent \underline{\textbf{Case 3:} \((x,y) \in \mathcal{T} \times \mathcal{R}\)}
\par
Necessarily \(h(x) = h(y) = n + 1\). Two cases are to be distinguished according to the value of \(\lambda(x+y)\).
\begin{itemize}
\item If \(\lambda(y)=\lambda(x+y)\) then \(\pi(x + y) = x + \pi(y)\) (Proposition \ref{prop6}). By definition of
	\(h\), \(f(x) = f(\pi(y))\). Given that \(f\) is a weakly sum-free coloring, \(f(x + \pi(y)) \neq f(x)\) since
	\(f(x) = f(\pi(y))\) and \(x \neq \pi(y)\) since \({x \leqslant b < \pi(y)}\). Hence \(h(x + y) \neq h(x)\).
\item If \(\lambda(y) \neq \lambda(x + y)\) then \(\lambda(x + y) = \lambda(y) + 1\) since \(x \leqslant b < a\).
	Besides, \(n + 1 = h(y) = n +  g(\lambda(y))\). Hence \(g(\lambda(y)) = 1\). Moreover \(g(1) = 1\) since \(g\)
	is an ordered coloring. Therefore, given that \(g\) is sum-free, \(g(\lambda(y) + 1) \neq 1\). If \(\pi(x + y) \in
	A_{n + 1}\) then \(h(x + y) = n + g(\lambda(x + y)) \neq n + 1\). Otherwise, \(h(x + y) \leqslant n\). Hence
	in both cases \(h(x + y) \neq h(x)\).
\end{itemize}

\noindent \underline{\textbf{Case 4:} \((x,y) \in \mathcal{C}^2\)}
\par
By definition of \(h\) and since \(h(x)=h(y)\), \(f(\pi(x)) = f(\pi(y))\). Two cases are to be distinguished according
to the value of \(\pi(x)+\pi(y)\).
\begin{itemize}
\item If \(\pi(x) + \pi(y) \leqslant a+b\) then \(\pi(x)+\pi(y) = \pi(x + y)\). Hence \(f(\pi(x + y)) \neq f(\pi(x))\) since
	\(f\) is sum-free for \(x>b\)
\item \begin{sloppypar}
	If \(\pi(x)+\pi(y)>a+b\) then given that \(f\) is a WS-template, \({f(\pi(\pi(x)+\pi(y))) \neq f(\pi(x))}\) since
	\({f(\pi(x)) = f(\pi(y))}\). Besides,  \({f(\pi(\pi(x)+\pi(y))) = f(\pi(x + y))}\) (Proposition \ref{prop4}). Hence \({f(\pi(x + y))
	\neq  f(\pi(x))}\).
	\end{sloppypar}
\end{itemize}
\par
Hence in both cases \(f(\pi(x+y)) \neq f(x)\). If  \(f(\pi(x+y)) \leqslant n\) then \(h(x+y) = f(\pi(x+y))\). Therefore
\(h(x+y) \neq h(x)\) since \(f(x) = h(x)\). Otherwise, \(f(\pi(x+y)) = n + 1\) and thus \(h(x+y) > n\). In particular,
\(h(x + y) \neq h(x)\) since \(h(x) = h(y) \leqslant n\). \\

\noindent \underline{\textbf{Case 5:} \((x,y) \in \mathcal{C} \times \mathcal{R}\)}
\par
By definition of \(h\), \(h(x) \neq h(y)\).\\

\noindent \underline{\textbf{Case 6:} \((x,y) \in \mathcal{R}^2\)}
\par
In particular \(f(\pi(x)) = f(\pi(y))=n + 1\). Three cases are to be distinguished according to the value of \(\pi(x) + \pi(y)\).
\begin{itemize}
\item If \(\pi(x) + \pi(y) \in [\![a + b + 1, 2 a + b]\!]\) then \(\lambda(x + y) = \lambda(x) + \lambda(y)\)
	(Proposition \ref{prop7}). By definition of \(h\) and since \(h(x) = h(y)\), \(g(\lambda(x)) = g(\lambda(y))\).
	Hence, \(h(\lambda(x + y)) \neq h(\lambda(x))\) since \(h\) is a sum-free coloring. If \(f(x+y) \geqslant n + 1\)
	then \(h(x + y) = n + g(\lambda(x + y))\). And \(h(x) = n + g(\lambda(x))\). Therefore, \(h(x + y)  \neq h(x)\).
	Otherwise \(h(x+y) \leqslant n < h(x)\). In particular \(h(x + y) \neq h(x)\).
\item If \(\pi(x)+\pi(y)>2a+b\) then \(f(\pi(\pi(x)+\pi(y))) \neq f(\pi(x)) = n + 1\) since \(f\) is a \(b\)-WS template and
	\(f(\pi(x)) = f(\pi(y))\). Given that \(\pi(\pi(x)+\pi(y)) = \pi(x+y)\) (Proposition \ref{prop4}), \(f(\pi(x+y)) \neq n + 1\).
\item \begin{sloppypar}
	If \(\pi(x)+\pi(y)\leqslant b+a\) then, given that \(\pi(x)+\pi(y) \geqslant b\) and \(f_{| [\![b, a + b ]\!]}\) is
	sum-free, \({f(\pi(x) + \pi(y)) \neq f(\pi(x)) = n + 1}\). That is \({f(\pi(x + y)) \neq n + 1}\) (Proposition \ref{prop1}).
	\end{sloppypar}
\end{itemize}
\par
In both of the last two cases, \({f(\pi(x + y)) \neq n + 1}\) that is \(x+y \in \mathcal{C}\). Therefore \(h(x+y) < n \leqslant h(x)\).
In particular, \(h(x + y) \neq h(x)\). \\
\end{proof}

There is a construction theorem for WS-templates as well.

\begin{theorem}
\label{thm:compWStemp}
Let \((k,p) \in (\mathbb{N}^*)^2\) and \((a, n, b) \in (\mathbb{N}^*)^3\). If there exists a S-template with width
\(p\) and \(k+1\) colors and a \(b\)-WS-template with width \(a\) and \(n\) colors, then there exists a \(pb\)-WS-template
with width \(pq\) and \((n+k)\) colors.
\end{theorem}

Theorem \ref{thm:compWStemp} yields the following corollary.

\begin{corollary}
Let \(n, k \in \mathbb{N}^*\). Then
\[
\WS^+(n+k) \geqslant S^+(k+1) \WS^+(n).
\]
\end{corollary}

\begin{proof}[\textsc{Proof.}]
The idea is the same as in the previous theorem. The only difference is the WS property inherited
from both the S-template and the WS-template. \\
\end{proof}

Theorem \ref{thm:WS-ab} can be seen as a particular case of WS-template in the same way Abott and Hanson's 
construction \cite{AbbottHanson} can be seen as a particular case of S-template.

\begin{proof}[\textsc{Proof of Theorem \ref{thm:WS-ab}.}]
\label{PreuveThm}
Let \((q, n) \in (\mathbb{N}^*)^2\) such that there exists a partition of \([\![1, q]\!]\) into n weakly sum-free subsets. 
Let \(f: [\![1, q]\!] \rightarrow [\!]1, n]\!]\) a weakly sum-free colouring.
Let \(b=q\) and \(a = q + \left \lceil \dfrac{q}{2} \right \rceil + 1\). A new colouring \(g\) is defined as follows:

\[
\begin{array}{c c c l}
	g : & [\![1, a + b]\!] & \longrightarrow & [\![1, n + 1]\!] \\
	& x & \longmapsto &
		\left\{ \begin{array}{l l}
			f(x) & \text{if}~x \in [\![1, b]\!]  \\
			n + 1 & \text{if}~x \in [\![b + 1, 2 b + 1]\!] \\
			f(x - a) & \text{if}~x \in  [\![2 b + 2, a + b]\!]
		\end{array} \right.
\end{array}
\]

We claim that \(g\) is a \(b\)-WSF-template with width \(a\) and \(n + 1\) colours.

\begin{itemize}
\item Function \(g_{|[\![b + 1, a + b]\!]}\) is a sum-free colouring. Indeed, let \((x, y) \in 
	[\![b + 1, a + b]\!]^2\) such that \(g(x) = g(y)\). If \(g(x) = n + 1\) then \(z = x + y > 2 b + 1\) 
	and therefore, either \(z > a + b\) or \(f(z) \neq n + 1\). Otherwise, \(x + y > a + b\).
\item Function \(g\) is a weakly sum-free colouring. Indeed, let \((x, y) \in [\![1, a + b]\!]^2\) such that 
	\(z = x + y \leqslant a + b\) and \(g(x) = g(y)\). Given that \(x\) and \(y\) have symmetric roles, 
	we can assume that \(x \leqslant y\). If \(x > b\) then \(g(z) \neq g(x)\) as seen above. If \(y \leqslant b\) then 
	\(f(x) = g(x) = g(y) = f(y) \leqslant n\) and either \(z \leqslant a + b\) and \(f(z) \neq f(x)\) since \(f\) is a weakly 
	sum-free colouring or \(a + b + 1 \leqslant z \leqslant 2 b\) and \(g(z) = n + 1\); therefore \(g(z) \neq g(x)\). If 
	\(x \leqslant b\) and \(y > b\) then \(g(x) = f(x)\), \(g(y) = f(y - a)\) and \(g(z) = f(z - a)\).We have \(x \neq y - a\) 
	(otherwise, we would have \(a + b \geqslant z = 2 y  - a \geqslant 4 b + 4 - a > a + b\)) and thus \(f(x + y - a) \neq f(x)\),
	that is \(g(z) \neq g(x)\).
\item Colour \(n + 1\) verifies the additionnal constraints for the special colour. Indeed, \(b + 2 a \geqslant 4 b + 2\). 
	Hence, \(\forall (x, y) \in g^{-1}(\{n + 1\}), x + y \leqslant b + 2 a\).
\item Colours \(1, ..., n\) verify the additionnal constraints for the regular colours. Indeed, let 
	\((x, y) \in g^{-1}([\![1, n]\!])\) such that \(x + y > a + b\). Given that \(x\) and \(y\) have symmetric roles, 
	we can assume that \(x \leqslant y\). Necessarily \(y \geqslant 2 b + 2\). If \(x \leqslant b\) then 
	\(z = x + y \in [\![a + b + 1, a + 2 b]\!]\) and therefore \(\pi(z) \in [\![b + 1, 2 b]\!]\). Otherwise, 
	\(z = x + y \in [\![4 b + 4, 2 a + 2 b]\!]\) and therefore \(\pi(z) \in [\![b + 1, 2 b]\!]\). Hence, in both cases, 
	\(g(\pi(z)) \neq g(x)\).
\end{itemize}

The result is then obtained by applying Theorem \ref{thm:WStemp}. \\
\end{proof}

As in Corollary \ref{cor:ineqS}, the additive constant Theorem \ref{thm:WStemp} can be improved 
by weakening the hypotheses made on the last row. The principle behind it is the same as in Proposition \ref{prop:rafStemp}.

\begin{proposition}
Let \((b, k, a) \in (\mathbb{N}^*)^3\) and let \(f\) be a coloring associated to a \(b\)-WS-template with width \(p\) and \(k\) colors. Let
\(c \in \mathbb{N}\) and assume there there exists a coloring \(g\) of \([\![b + 1, b + c]\!]\) with \(k\) colors such that for all \(c \in [\![1, k]\!]\),

\begin{itemize}
\item \(\forall (x, y) \in  [\![1, a + b]\!] \times  [\![b + 1, a + b]\!],  \left\{
	\begin{array}{l}
		f(x) = f(y) \\
		\pi(x + y) \leqslant b + c
	\end{array}
	\right. \implies g(\pi(x + y)) \neq f(x)\),
\item \(\forall (x, y) \in  [\![1, a + b]\!] \times  [\![b + 1, b + c]\!], \left\{
	\begin{array}{l}
		f(x) = g(y) \\
		\pi(x + y) \leqslant b + c
	\end{array}
	\right. \implies g(\pi(x + y)) \neq f(x)\).
\end{itemize}

Then, for every \(n \in \mathbb{N}^*\), by using on the last row the coloring \(x \longmapsto g(x - p S(n))\), we have\\
\[ \WS(n+k) \geqslant \WS^+(k+1) S(n) + b + c.\]
\end{proposition}

The WS-templates can actually be fine-tuned further. However, it only gives minor improvements (most likely only an additive
constant) at the cost of dramatically increasing the size of the search space. Therefore, it does not seem relevant to
use this sophistications given that we could not even find good WS-templates with five colors using a computer (here good
means better than those obtain by combining smaller templates).

\par These modifications work as follows. One may notice that the first row (excluding the "tail") has constraints that other rows
do not have because of the tail, especially if the special color appears in the tail as well. Thus allowing to have a coloring on the
first row different from the coloring of the other rows would weaken the constraints. Acutally, one may even go further by
noticing that on the one hand the first (ordered) color of the sum-free partition used for the extension procedure has more
more constraints than the other colors of the sum-free partition since the first row is of this color and is more constrained than
the other rows, but that on the other hand it has more degrees of freedom than the other colors of the sum-free partition since in
the sum-free partition there cannot be two consecutive numbers of this color. As a result, it removes  some constraints imposed
 by the first row on the other rows.

\par
To sum up, one can look for a generalised WS-template that uses a special coloring for the tail and the first row, a
coloring dedicated to the rows whose number is not 1 but is in the first color in the sum-free partition, a coloring for all
the other rows and a special coloring for the last numbers (as previously explained for the improvement of the additive
constant of WS-templates).

\subsection{New lower bounds for Weak Schur numbers}

We produced WS-templates using a SAT solver, hence providing lower bound on \(WS^+\) and inequalities
of the type \(WS(n+k) \geqslant a S(n) + b\). We sought templates providing the greatest value of \((a, b)\) (in the 
lexicographic order). Details concerning the encoding as a SAT problem can be found in \cite{Heule2017}.

\par
Here are the inequalities given by the current best WS-templates. The \hyperref[WS-templates]{template} 
corresponding to the third inequality can be found in the appendix.

\begin{align}
	WS(n + 1) &\geqslant  4\,S(n)  +  2 \label{WS(n+1)} \\
	WS(n + 2) &\geqslant  13\,S(n)  +  8 \label{WS(n+2)} 
\end{align}

Inequalities (\ref{WS(n+1)}) and (\ref{WS(n+2)}) were found by Rowley, they are detailed in \cite{RowleyWS}.

\begin{align}
	WS(n + 3) &\geqslant  42\,S(n)  +  24 \label{WS(n+3)} \\
	WS(n + 4) &\geqslant  132\,S(n)  +  26 \label{WS(n+4)}
\end{align}

Inequality (\ref{WS(n+3)}) cannot be further improved and was found with a SAT solver. It uses the first 
sophistication explained in Subsection \ref{ConstructionWS} in order to add
the last number in the first color. As for inequality (\ref{WS(n+4)}), it was obtained by combining a S-template with 
width 33 with a WS-template with width 4. The best template we could find with a computer search gives the 
inequality \(\WS(n+4) \geqslant 127 S(n) + 68\). It was also found with the SAT solver. In order to reduce the search space, 
we only looked for WS-templates of five colors which start with a near-optimal \(\WS(4)\) partition and we assumed that the 
special color was the last by order of appearance. We highly suspect that better WS-templates with \(n \geqslant 5\) 
colors can be found but one would have not to use the above assumptions. One may try to go over a different search space 
using a Monte-Carlo method, as in \cite{Bouzy2015AnAP}. This could be the suject of a future work.

Like in  Subsection \ref{subsec:lowS}, we compute the lower bounds given by inequalities (\ref{WS(n+1)}),
(\ref{WS(n+2)}) and (\ref{WS(n+3)}) for \( n \in [\![8,15]\!] \). The best lower bound for each value of \(n\) is highlighted.

\renewcommand{\arraystretch}{0.2}

\begin{table}[H]

\label{LowerBoundsWS}
\[
\begin{array}{c}
	\resetarraystretch
	\begin{NiceArray}{cwc{8ex}wc{10ex}wc{10ex}wc{11ex}}[hvlines]
	\CodeBefore
		\cellcolor{yellow}{2-5}
		\cellcolor{yellow}{3-2}
		\cellcolor{yellow}{4-3}
		\cellcolor{yellow}{4-4}
	\Body
		n & 8 & 9 & 10 & 11 \\
		4 \, S(n-1) + 2 & 6\,786 & 21\,146 & 71\,214 & 243\,794 \\
		13 \, S(n-2) + 8 & 6\,976 & 22\,056 & 68\,726 & 231\,447 \\
		42 \, S(n-3) + 24 & 6\,744 & 22\,536 & 71\,256 & 222\,036 \\
	\end{NiceArray}
	\\ \\
	\resetarraystretch
	\begin{NiceArray}{cwc{8ex}wc{10ex}wc{10ex}wc{11ex}}[hvlines]
	\CodeBefore
		\cellcolor{yellow}{2-2}
		\cellcolor{yellow}{3-3}
		\cellcolor{yellow}{4-4}
		\cellcolor{yellow}{4-5}
	\Body
		n & 12 & 13 & 14 & 15 \\
		4 \, S(n-1) + 2 & 815\,314 & 2\,578\,514 & 8\,045\,162 & 27\,061\,154 \\
		13 \, S(n-2) + 8 & 792\,332 & 2\,649\,772 & 8\,380\,172 & 26\,146\,778 \\
		42 \, S(n-3) + 24 & 747\,750 & 2\,559\,840 & 8\,560\,800 &  27\,074\,400 \\
	\end{NiceArray}
\end{array}
\]
\caption{New lower bounds for \( n \in [\![8,15]\!] \)}
\end{table}

\resetarraystretch

With \( S(9) \geqslant 17\,803 \), we found a new lower bound for \(\WS(10)\) using (\ref{WS(n+1)}).
Moreover, inequality (\ref{WS(n+3)}) gives new lower bounds for \(\WS(9)\), \(\WS(10)\), \(\WS(14)\) and \(\WS(15)\).


\subsection{Another approach: \(\WS(6) \geqslant 646\)}

The weak Schur partitions obtained with WS-templates have an extremely regular structure. Therefore, one may 
expect to find larger partitions by allowing more freedom in the structure of the partition while still preserving a structure 
somewhat close to a template-based partition thus reducing the search space to a manageable size.

When applying the inequality \(\WS(n+1) \geqslant 4 \, S(n) + 2\), one may realize that it is possible to build a weakly 
sum-free partition of length \(4 \, S(n) + 3\) for small values of \(n\) (\(n \leqslant 4\)) by using the contruction of 
Corollary \ref{cor:ineqWS+} for the integers 1, 2, \(4i\) and \(4i+1\) for \(i \in [\![1, S(n)]\!]\) but not 
constraining the other integers. We did the same for the Schur partitions corresponding to Schur number five. More precisely, 
we imposed these constraints only for \(i \leqslant 50\) in order to have more degrees of freedom. The number 50 was chosen 
arbitrarly so that all of the Schur number 5-partitions could be tested in a few hours.

However, trying out all of the 2\,447\,113\,088 Schur number 5-partitions (\cite{Heule2017}) one by one would not result in a 
reasonable computation time: it is necessary to test the construction on several partitions at once. We encoded the problem as a 
satisfiability problem and used the 1616 backdoors that were used in \cite{Heule2017} in order to encode a group of partitions in 
a compact and efficient way. Among all of the 1616 backdoors, only the 911\textsuperscript{th} backdoor gave a weakly sum-free 
partition of length 643. This backdoor gave a weakly sum-free partition of length 646 and cannot give a weakly sum-free 
partition of length 647. A weakly sum-free partition of length 646 can be found in the \hyperref[WS(6)]{appendix}.


\subsection{Conclusion on WS-templates}

We started by giving a new construction which can be seen as an equivalent for weakly sum-free partitions of Abott
and Hanson's construction for sum-free partitions. We then generalized this construction by introducing WS-templates. This
allows us to find new lower bounds and new inequalities for weak Schur numbers. One may notice the significant gap
between the former lower bounds for weak Schur numbers obtained by conducting a computer search and the new lower bounds
obtained with WS-templates. We reckon better WS-templates with \(n \geqslant 5\) colors can be found by making different assumptions 
and using a different method (Monte-Carlo methods for instance).


\section{Conclusions and future work}

\qquad These new results come from an extension of Rowley's template-based approach for Ramsey graphs and 
Schur numbers which is relatively new. Therefore, we will not be surprised if lower bounds are later improved 
using better templates. Moreover, studying specifically \(S^+(n)\) and \(\WS^+(n)\) might be of interest as they
 are closely related to Schur and weak Schur numbers. In order to find new templates, algorithms based on 
randomness such as Monte-Carlo algorithms may prove to be very useful. This could be the subject of a future work.

\section{Acknowledgments}

\qquad We would first like to thank our professors, Joanna Tomasik and Arpad Rimmel, who guided and advised
us during the whole research and writing. Last but not least, we would like to address warm thanks
to Fred Rowley who kindly answered our many questions and with who it was a pleasure to exchange.


\bibliography{biblio}
\bibliographystyle{IEEEtran}

\appendix
\renewcommand{\arraystretch}{1}


\section{SF-templates}

\begin{center}
SF-template partitionning \([\![1, 33]\!]\) into 4 subsets \\
\begin{tabular}{|*{2}{c|}}
	\hline
	1 & 1, 6, 9, 13, 16, 20, 24, 27, 31 \\
	\hline
	2 & 2, 5, 14, 15, 25, 26 \\
	\hline
	3 & 3, 4, 10, 11, 12, 28, 29, 30 \\
	\hline
	4 & 7, 8, 17, 18, 19, 21, 22, 23, 32, 33 \\
	\hline
\end{tabular}
\end{center}

\begin{center}
SF-template partitionning \([\![1, 111]\!]\) into 5 subsets
\begin{tabular}{|*{2}{c|}}
	\hline
	1 & 1, 5, 18, 12, 14 ,21, 23, 30, 32, 36, 39, 43, 45, 52, 103 \\
	 & 106, 110 \\
	\hline
	2 & 2, 6, 7, 10, 15, 18, 26, 29, 34, 37, 38, 42, 46, 51, 54 \\
	& 101, 104, 109 \\
	\hline
	3 & 3, 4, 9, 11, 17, 19, 25, 27, 33, 35, 40, 41, 47, 48, 55 \\
	& 100, 107, 108 \\
	\hline
	4 & 13, 16, 20, 22, 24, 28, 31, 58, 61, 67, 88, 94, 97 \\
	\hline
	5 & 44, 50, 53, 56, 57, 59, 60, 62, 63, 64, 65, 66, 68, 69, 70\\
	& 71, 72, 73, 74, 75, 76, 77, 78, 79, 80, 81, 82, 83, 84, 85\\
	& 86, 87, 89, 90, 91, 92, 93, 95, 96, 98, 99, 102, 105, 111 \\
	\hline
\end{tabular}
\end{center}

\begin{center}
SF-template partitionning \([\![1, 380]\!]\) into 6 subsets
\begin{tabular}{|*{2}{c|}}
	\hline
	1 & 1, 5, 8, 11, 15, 17, 29, 33, 36, 39, 43, 57, 61, 88, 92 \\
	& 106, 110, 113, 116, 120, 132, 134, 138, 141, 144, 148, 150, 154, 157, 160 \\
	& 164, 178, 182, 185, 188, 341, 344, 347, 351, 365, 369, 372, 375, 379\\
	\hline
	2 & 2, 9, 13, 16, 20, 23, 24, 27, 28, 31, 34, 35, 38, 42, 45\\
	& 49, 53, 60, 67, 71, 78, 82, 89, 96, 100, 104, 107, 111, 114, 115\\
	& 118, 121, 122, 125, 126, 129, 133, 136, 140, 147, 158, 162, 165, 169, 172\\
	& 176, 183, 187, 194, 201, 328, 335, 342, 346, 353, 357, 360, 364, 367, 371\\
	\hline
	3 & 3, 4, 12, 14, 19, 25, 30, 32, 40, 41, 47, 48, 58, 91, 101\\
	& 102, 108, 109, 117, 119, 124, 130, 135, 137, 145, 146, 152, 153, 161, 163\\
	& 168, 179, 181, 190, 339, 348, 350, 361, 366, 368, 376, 377\\
	\hline
	4 & 6, 7, 10, 18, 21, 22, 26, 37, 46, 50, 51, 54, 65, 70, 79\\
	& 84, 95, 98, 99, 103, 112, 123, 127, 128, 131, 139, 142, 143, 151, 155\\
	& 156, 159, 167, 170, 171, 175, 186, 343, 354, 358, 359, 362, 370, 373, 374\\
	& 378\\
	\hline
	5 & 44, 52, 55, 56, 59, 62, 63, 64, 66, 68, 69, 72, 73, 74, 75\\
	& 76, 77, 80, 81, 83, 85, 86, 87, 90, 93, 94, 97, 105, 189, 196\\
	& 197, 200, 203, 206, 207, 209, 214, 219, 231, 298, 310, 315, 320, 322, 323\\
	& 326, 329, 332, 333, 340\\
	\hline
	6 & 149, 166, 173, 174, 177, 180, 184, 191, 192, 193, 195, 198, 199, 202, 204\\
	& 205, 208, 210, 211, 212, 213, 215, 216, 217, 218, 220, 221, 222, 223, 224\\
	& 225, 226, 227, 228, 229, 230, 232, 233, 234, 235, 236, 237, 238, 239, 240\\
	& 241, 242, 243, 244, 245, 246, 247, 248, 249, 250, 251, 252, 253, 254, 255\\
	& 256, 257, 258, 259, 260, 261, 262, 263, 264, 265, 266, 267, 268, 269, 270\\
	& 271, 272, 273, 274, 275, 276, 277, 278, 279, 280, 281, 282, 283, 284, 285\\
	& 286, 287, 288, 289, 290, 291, 292, 293, 294, 295, 296, 297, 299, 300, 301\\
	& 302, 303, 304, 305, 306, 307, 308, 309, 311, 312, 313, 314, 316, 317, 318\\
	& 319, 321, 324, 325, 327, 330, 331, 334, 336, 337, 338, 345, 349, 352, 355\\
	& 356, 363, 380\\
	\hline

\end{tabular}
\end{center}


\section{WSF-templates}

\begin{center}
WSF-template partitionning \([\![1, 42]\!]\) into 4 subsets
\begin{tabular}{|*{2}{c|}}
	\hline
	1 & 1, 2, 4, 8, 11, 22, 25, \(\mathbf{(N+1)}\)\\
	\hline
	2 & 5, 6, 7, 19, 21, 23, 36\\
	\hline
	3 & 9, 10, 12, 13, 14, 15, 16, 17, 18, 20\\
	\hline
	4 & 24, 26, 27, 28, 29, 30, 31, 32, 33, 34, 35, 37, 38, 39, 40\\
	& 41, 42\\
	\hline
\end{tabular}
\end{center}

This template provides the inequality \(WS(n+3) \geqslant 42S(n) + 24\)
by placing one last number, here represented by \(\mathbf{(N+1)}\), in the first subset.


\section{Proof of theorem j'arrive pas a afficher le numero du theoreme avec un lien}

\textsc{Proof :} Let \((p,q), (n,k) \in (\mathbb{N}^*)^2\), \(N = p(q+\left \lceil \frac{q}{2} \right \rceil + 1)+q\),
\(\alpha = \left \lceil \frac{q}{2} \right \rceil > 0\) and \(\beta = q + \alpha + 1\).
We denote by \(f\) the colouring associated to the partition of \([\![1,q]\!]\) and \(g\) the
one associated to the partition of \([\![1,p]\!]\).
\[ f : [\![1,q]\!] \longrightarrow [\![1,n]\!] \text{ and } \forall (x,y) \in [\![1,q]\!]^2, \left\{
\begin{array}{ll}
	x \neq y \\
	f(x) = f(y)
\end{array}
\right.
\Longrightarrow f(x+y) \neq f(x)
\]
\[g : [\![1,p]\!] \longrightarrow [\![1,k]\!] \text{ and } \forall (x,y) \in [\![1,q]\!]^2 \text{, } f(x) = f(y)
\Longrightarrow f(x+y) \neq f(x)
\]
Let us start by parting the integers of \([\![1,N]\!]\) in two subsets \(\mathcal{A}\) and \(\mathcal{B}\) where
\(\mathcal{A} = [\![1,\alpha]\!] \cup \{a\beta + u \mid (a,u) \in [\![0,p]\!] \times [\![\alpha + 1,q]\!]\}\) and
\(\mathcal{B} = \{a\beta + u \mid (a,u) \in [\![1,p]\!] \times [\![-\alpha,\alpha]\!]\}\).\\
\\
First, \underline{\(\mathcal{A} \cap \mathcal{B} = \varnothing\)} : \\
By contradiction, suppose there exists \(x \in \mathcal{A} \cap \mathcal{B} \neq \varnothing \). Then there are \((a,u)
\in [\![0,p]\!] \times [\![\alpha + 1,q]\!]\) and \((b,v) \in [\![1,p]\!] \times [\![-\alpha,\alpha]\!]\) such that \(x
= a\beta + u = b\beta +v\). By definition of \(\alpha\) and \(\beta\) we have \(u \in [\![\alpha + 1,q]\!] \subset
[\![0,\beta - 1]\!]\).
From there, we distinguish two cases :
\begin{itemize}
\item If \(v \in [\![0,\alpha]\!]\) then \(v \in [\![0,\beta - 1]\!]\) and \(v \neq u\) because \(v < \alpha + 1
\leqslant u\)
\item If \(v \in [\![-\alpha,-1]\!]\), we note \(\tilde{v} = \beta + v\) and thus have \(x = (b-1)\beta + \tilde{v}\)
with \(\tilde{v} \in [\![\beta - \alpha,\beta - 1]\!] \subset [\![0,\beta - 1]\!]\) and \(\tilde{v} \neq u\) because
\(u< q+1 = \beta - \alpha \leqslant \tilde{v}\).
\end{itemize}
In either cases, we run into a contradiction because of the remainder's uniqueness in the euclidean division of \(x\)
by\(\beta\).\\
\\
Then, we have \underline{\(\mathcal{A} \cup \mathcal{B} = [\![1,N]\!]\)}:
\begin{itemize}
	\item On the one hand : \(1 = \text{min}(\mathcal{A}) \leqslant \text{max}(\mathcal{A}) = p\beta + q = N\) and
\(1 \leqslant \beta - \alpha = \text{min}(\mathcal{B}) \leqslant \text{max}(\mathcal{B}) = p\beta + \alpha \leqslant
N\),
	which gives \(\mathcal{A} \cup \mathcal{B} \subset [\![1,N]\!]\).
\item On the other hand, let \(x \in [\![1,N]\!]\). If \(x \leqslant \alpha\), we directly have \(x \in \mathcal{A}\),
let us then suppose that \(x > \alpha\) and write \(x = a\beta + u\)
the euclidean division of \(x\) by \(\beta\). We have \(x \leqslant N\), thus \(a \leqslant p\). We distinguish three
cases : \\
- If \(u \in [\![0,\alpha]\!]\) then we necessarily have \(a \geqslant 1\) because \(x > \alpha\), and so \(x \in
\mathcal{B}\).\\
	- If \(u \in [\![\alpha + 1,q]\!]\), then \(x \in \mathcal{A}\). \\
- If \(u \in [\![q + 1,\beta - 1]\!]\) then \(x = (a+1)\beta - (\beta - u)\) with \(-\alpha \leqslant \beta - u
\leqslant 0\).
	Furthermore, \(a \leqslant p - 1\), else we would have \(x > N\), and so \(x \in \mathcal{B}\) \\
In any case, \(x \in \mathcal{A} \cup \mathcal{B}\) and we can thus conclude that \([\![1,N]\!] \subset \mathcal{A}
\cup\mathcal{B}\).
\end{itemize}
This first partition of \([\![1,N]\!]\) will help us to define our final partition by the projection of its equivalence
relation.
We thereby define \(h : [\![1,N]\!] \longrightarrow [\![1,n+k]\!]\) as such :\\
- If \(x \in \mathcal{A}\) then \(h(x) = f(x \text{ mod } \beta)\) (well defined because \(x \text{ mod } \beta \in
[\![1,N]\!]\))\\
- If \(x \in \mathcal{B}\) then \(x = a\beta + u\) with a unique \((a,u) \in [\![1,p]\!] \times
[\![-\alpha,\alpha]\!]\)and we define \(h(x) = n + g(a)\)\\
The fact that \((\mathcal{A}, \mathcal{B})\) is a partition of \([\![1,N]\!]\) ensures that this definition of \(h\) is
valid. We then have to verify that \(h\) induces weakly sum-free subsets.\\
\\
\underline{The classes of equivalence \(h(x)\) for \(x \in \mathcal{A}\) are weakly sum-free :}
\\
\\
Let \((x,y) \in \mathcal{A}^2\) such that \(h(x) = h(y)\), \(x \neq y\) and \(x + y \leqslant N\)
\begin{itemize}
	\item If \((x,y) \in [\![1,\alpha]\!]^2\) :\\
	We have \(x + y \leqslant 2\alpha \leqslant q\) and \(x + y = 0\beta + x + y\), therefore \(x + y \in \mathcal{A}\).
Then, by definition : \(h(x) = f(x)\), \(h(y) = f(y)\) and \(h(x+y) = f(x+y)\), which gives us, thanks to the property
verified by \(f\), that \(h(x+y) \neq h(x)\).
	\item If \((x,y) \in [\![1,\alpha]\!] \times ( \mathcal{A} \text{ \textbackslash} ~ [\![1,\alpha]\!] )\) :\\
We write \(y = a\beta + u\) with \((a,u) \in [\![0,p]\!] \times [\![\alpha + 1,q]\!]\). Then \(x+y = a\beta + x + u =
(a+1)\beta + x + u - \beta\),
and if \(x + u > q\) it follows that \(a \leqslant p-1\) since \(x+y \leqslant N\), and \(-\alpha \leqslant x + u -
\beta \leqslant -1\).
Therefore \(x+y \in \mathcal{B}\) and \(h(x+y) \neq h(x) = f(x)\) by definition of h. On the contrary, if \(x - u
\leqslant n\),
then \(x+y \in \mathcal{A}\) and \(h(x+y) = f(x+u)\) because \(x+u\) is actually the remainder of the euclidean
divisionof \(x+y\) by \(\beta\).
Moreover, \(h(x) = f(x)\), \(x < u\) and, with our initial hypothesis, \(h(x) = h(y) = f(u)\). The property verified by
\(f\) gives us \(f(x+u) \neq f(x)\) which can be rewritten as \(h(x+y) \neq h(x)\).
	\item If \((x,y) \in ( \mathcal{A} \text{ \textbackslash} ~ [\![1,\alpha]\!] ) \times [\![1,\alpha]\!]\) : \\
	This case is handled exactly like the previous one by swaping the roles of \(x\) and \(y\).
	\item If \((x,y) \in ( \mathcal{A} \text{ \textbackslash} ~ [\![1,\alpha]\!] )^2\) : \\
We write \(x = a\beta + u\) and \(y = b\beta + v\) with \((a,u)\) and \((b,v)\) in \([\![0,p]\!] \times [\![\alpha +
1,q]\!]\). Then \(x+y = (a+b)\beta + u+v = (a+b+1)\beta + u + v - \beta\)
with \(a+b \leqslant p-1\) (else we would have \(x+y > N\) because \(u+v > q\)) and \(-\alpha \leqslant u + v - \beta
\leqslant \alpha\), therefore \(x+y \in \mathcal{B}\) and by definition \(h(x+y) \neq h(x)\).
\end{itemize}
In any case, \(h(x+y) \neq h(x)\) and the classes of equivalence \(h(x)\) for \(x \in \mathcal{A}\) are weakly
sum-free.\\
\\
\underline{The classes of equivalence \(h(x)\) for \(x \in \mathcal{B}\) are weakly sum-free :}
\\
\\
Let \((x,y) \in \mathcal{B}^2\) such that \(h(x) = h(y)\), \(x \neq y\) and \(x + y \leqslant N\).\\
We write \(x = a\beta + u\) and \(y = b\beta + v\) with \((a,u)\) and \((b,v)\) in \([\![1,p]\!] \times
[\![-\alpha,\alpha]\!]\).
We have \(h(x) = q + g(a)\) and \(h(y) = q + g(b)\), therefore \(g(a) = g(b)\). We also have \(x+y = (a+b)\beta + u
+v\).\\
If \(u + v \in [\![-\alpha,\alpha]\!]\), then \(x+y \in \mathcal{B}\) and \(h(x+y) = g(a+b)\), hence we can deduce that
\(h(x+y) \neq h(x)\) because of the property verified by \(g\). On the contrary, if \(u+v \notin
[\![-\alpha,\alpha]\!]\), then necessarily \(x+y \in \mathcal{A}\). Suppose \(x+y \in \mathcal{B}\), then \(x+y =
c\beta+ w\) with \((c,w) \in [\![1,p]\!] \times [\![-\alpha,\alpha]\!]\). Thus, \(c\beta + w = (a+b)\beta + u + v\) and
\((a+b-c)\beta = w-u-v\). Furthermore \(a+b-c \neq 0\), else we would have \(u+v = w \in [\![-\alpha,\alpha]\!]\). This
finally leads to the following inequality :
\[\beta \leqslant |a+b-c|\beta = |w-u-v| \leqslant |w| + |u| + |v| \leqslant 3\alpha \leqslant q + \alpha < \beta
\]
which is absurd. We can therefore conclude that \(x+y \in \mathcal{A}\) and by definition of \(h\), \(h(x+y) \neq
h(x)\), proving that the classes of equivalence \(h(x)\) for \(x \in \mathcal{B}\) are weakly sum-free.\\
\\
Finally, we have showed that every classe of equivalence induced by \(h\) is weakly sum-free, which ends the proof.


\end{document}
