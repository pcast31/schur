\documentclass{article}
\title{New lower bounds for Schur and weak Schur numbers}
\author{Romain Ageron, Paul Casteras, Thibaut Pellerin, Yann Portella}

\usepackage{amsmath}
\usepackage{amssymb}

\usepackage[english]{babel}
\usepackage[hyphens]{url}
\usepackage{hyperref}
\usepackage[utf8]{inputenc}

\newtheorem{definition}{Definition}[section]
\newtheorem{notation}[definition]{Notation}
\newtheorem{theorem}{Theorem}[section]
\newtheorem{computational theorem}{Computational Theorem}[section]
\newtheorem{corollary}{Corollary}[theorem]




\begin{document}
\maketitle

\section{Abstract}



\section{Introduction, context and notations}

We start by defining sum-free and weakly sum-free subsets to introduce regular and weak Schur numbers.

\begin{definition}
A subset \(A\) of \(\mathbb{N}\) is said to be \textit{sum-free} when :
\[ \forall (a,b) \in A^2 \text{, } a+b \notin A\]
\end{definition}

\begin{definition}
A subset \(B\) of \(\mathbb{N}\) is said to be \textit{weakly sum-free} when:
\[ \forall (a,b) \in B^2 \text{, } a \neq b \Longrightarrow a+b \notin B\]
\end{definition}

Let us notice that a sum-free subset is also weakly sum-free, hence justifying the name of \textit{weakly} sum-free
subsets. Given \(m\) and \(n\) two integers, we are interested in partitioning the set of integers from 1 to \(m\) into
\(n\) weakly sum-free subsets.

\begin{notation}
We denote by \([\![1,n]\!]\) the set of integers \(\{1, 2, ..., n\}\).
\end{notation}

Schur proved in \hyperlink{label1}{\textbf{[1]}} that given a number of subsets \(n\), there exists a value of \(m\)
such that there exists no partition of \([\![1,p]\!]\) into \(n\) sum-free subsets for any \(p \geqslant m\). A similar
property holds for weakly sum-free subsets (reference necessaire). These observations lead to the following definitions.

\begin{definition}
Let \(n \in \mathbb{N}^*\). There exists a greatest integer that we note \(S(n)\) (\textit{resp. \(WS(n)\)}) such that
\([\![1,S(n)]\!]\) (resp. \([\![1,WS(n)]\!]\)) can be partitioned into \(n\) sum-free subsets (resp. weakly sum-free
subsets). \(S(n)\) is called the \textit{\(n\)\textsuperscript{th} Schur number} and \textit{\(WS(n)\) the \(n\)\textsuperscript{th} weak
Schur number}.
\end{definition}

\begin{notation}
For a partition of \([\![1, n]\!]\) in \(k\) subsets, we generally note these subsets \(A_1, ..., A_k\). We also note \(m_i = \text{min}(A_i)\). 
By ordering the subsets, we mean assuming that \(m_1 < ... < m_k\). However, if not specified we do not make this hypothesis since we 
do not always consider partitions in which every subset plays a symmetric role.
\end{notation}



\section{Schur numbers}



\section{Weak Schur numbers}



\section{About the construction of lower bounds for weak Schur numbers using a computer}

In this section, we first reframe the question of the existence of (weakly) sum-free partitions as a boolean
satisfiability (SAT) problem. We then provide evidence which indicates that the main assumption made by papers which
found the previous best known lower bounds for weak Schur numbers may not be correct. Finally, we obtain stronger
results than those previously known for \(WS(5)\) while gaining several orders of magnitude in computation time by 
giving additional information to the SAT solver without losing in generality. In this section, we assume that the subsets are ordered.


\subsection{Reformulation as a SAT problem}

We encode the existence of (weakly) sum-free partitions as propositional formulae like in \hyperlink{label3}{\textbf{[3]}} and then use 
SAT solvers to determine whether these formulae are satisfiable.

\begin{definition}
A \textit{literal} is either a variable \(v\) (a positive literal) or the negation \(\bar{v}\) of a variable \(v\) (a negative literal) where \(v\) 
takes a truth value: \(true\) or \(false\). A \textit{clause} is a disjunction of literals and a \textit{formula} is a conjunction of clauses: it 
is a propositional formula in \textit{conjonctive normal form} (CNF).
\end{definition}

\begin{definition}
An \textit{assignment} is a function from a set of variables to the truth values \(true\) (1) and \(false\) (0). A literal \(l\) is 
\textit{satisfied} (\textit{falsified}) by an assignment \(\alpha\) if l is positive and \(\alpha(var (l)) = 1\) 
(resp. \(\alpha(var (l)) = 0\)) or if it is negative and \(\alpha(var (l)) = 0\) (resp. \(\alpha(var (l)) = 1\)). A clause is \textit{satisfied} 
by an assignment \(\alpha\) if it contains a literal that is satisfied by \(\alpha\). Finally, a formula is \textit{satisfied} by an assignment 
\(\alpha\) if all its clauses are satisfied by \(\alpha\). A formula is \textit{satisfiable} if there exists an assignment that satisfies it; 
otherwise it is \textit{unsatisfiable}.
\end{definition}

We then encode the existence of a partition of \([\![1,n]\!]\) in \(k\)  weakly sum-free subsets as follows: for every integer 
\(i \in [\![1,n]\!]\), take \(k\) variables \(x^{(i)}_{1}, ..., x^{(i)}_{k}\) and for every \(\forall c \in [\![1,k]\!], x^{(i)}_c = 1 \iff i \in A_c\). 
The corresponding clauses are:

\begin{itemize}
\item \textbf{sumfree:} \(\forall c \in  [\![1,k]\!], \forall (i, j) \in [\![1,n]\!]^2, (i \neq j ~ \text{and} ~ i + j \leq n) \implies \lnot x^{(i)}_c 
\lor  \lnot x^{(i)}_c \lor \lnot x^{(i+j)}_c\)
\item \textbf{union:} \(\forall i \in [\![1,n]\!], x^{(i)}_1 \lor ... \lor x^{(i)}_k\)
\item \textbf{disjoint:} \(\forall i \in [\![1,n]\!],\forall (c_1, c_2) \in  [\![1,k]\!]^2, c_1 \neq c_2 \implies \neg x^{(i)}_{c_1} \lor \neg x^{(i)}_{c_2}\)
\end{itemize} 

In the above formula, every color plays a symmetric role. Hence the search space can reduced by \(k!\) by ordering the subsets, that is by 
enforcing that \(m_1 < ... < m_k\). The correspond clauses are: \linebreak
\textbf{symmetry breaking:}  \(x^{(1)}_1 = 1\) and \(\forall c \in [\![2,k-1]\!], \forall i \in [\![1,WS(c - 1)+1]\!], x^{(1)}_c \lor ... \lor x^{(i)}_c \lor \neg x^{(i+1)}_{c+1}\)


\subsection{The search space previously used in computer search for lower bound may not contain the optimal partitions}

Rowley's new lower bound for \(WS(6)\) (642)  \hyperlink{label2}{\textbf{[2]}} was a quite significant improvement upon the former 
best known lower bound (582). This previous lower bound was found using a computer with Monte-Carlo methods and by making the 
assumption that a good partition for \(WS(n+1)\) starts with a good partition for \(WS(n)\) which is true for small values of \(n\).
Therefore, one may wonder whether the limiting factor is the assumption or the methods used to search for partitions. It appears that 
the latter is correct.

\begin{computational theorem}
There is no weakly sum-free partition of \([\![1,583]\!]\) in 6 parts such that:
\begin{itemize}
\item \(m_5 \ge 66\)
\item \(m_6 \ge 186\)
\item \([\![210,349]\!] \subset A_6\)
\end{itemize}  
\end{computational theorem}

This result was obtained in 8 hours with the SAT solver plingeling \hyperlink{label5}{\textbf{[5]}} on a 2.60 GHz Intel i7 processor PC.
However, simply encoding the existence of such a partition as explained in the previous subsection would not result in a reasonable 
computation time. In order to help the SAT solver, we add additional information in the propositional formula. We did not quantify the 
speedup, but it most likely allows us to gain several order of magnitude in computation time as we explain in the next subsection.

\par
Every weakly sum-free partition of \([\![1,65]\!]\) in 4 subsets starts with the following sequence 1121222133. Then 11 is always 
either in subset 1 or 3, 12 is always in subset 3 and so. For every integer in \([\![1,65]\!]\), we computed in which subset it can appear. 
By using this constraints, we could then compute for every integer in \([\![1,185]\!]\), in which subset it can appear in a weakly sum-free 
partition of  \([\![1,185]\!]\) which starts with a  weakly sum-free partition of \([\![1,65]\!]\) in 4 subsets. Adding these constraints to the 
formula corresponding to the above theorem gives additional information to the SAT solver without losing in generality. 


\subsection{Weak Schur number five}



\section{Conclusions and future work}



\section{Acknowledgments}



\section{References}

\hypertarget{label1}{\textbf{[1]}} Schur, I. 1917. Uber die Kongruenz \(x^m + y^m = z^m~(\text{mod}~p)\). Jahresbericht der 
Deutschen Mathematikervereinigung 25:114–117.: 114-116. \url{http://eudml.org/doc/145475}

\hypertarget{label2}{\textbf{[2]}} Rowley, F. 2020. New Lower Bounds for Weak Schur Partitions. arXiv 2011.11292, 
\url{https://arxiv.org/pdf/2011.11292.pdf}

\hypertarget{label3}{\textbf{[3]}} Heule, M.J.H. 2017. Schur Number Five. arXiv 1711.08076, \url{https://arxiv.org/pdf/1711.08076.pdf}

\hypertarget{label4}{\textbf{[4]}} S. Eliahou, J.M. Marín, M.P. Revuelta and M.I. Sanz 2012. Weak Schur numbers and the search for G.W. 
Walker’s lost partitions. Computers \& Mathematics with Applications Vol. 63, \url{https://www.sciencedirect.com/science/article/pii/S0898122111009722}

\hypertarget{label5}{\textbf{[5]}} Bierem, A. 2017. CaDiCaL, Lingeling, Plingeling, Treengeling, YalSAT Entering the SAT Competition 2017. 
Proceedinfs of SAT Competition 2017 -- Solver and Benchmark Descriptions \url{http://fmv.jku.at/papers/Biere-SAT-Competition-2017-solvers.pdf}

\end{document}
