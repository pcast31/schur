\documentclass{article}
\title{New lower bounds for Schur and weak Schur numbers}
\author{Romain Ageron, Paul Casteras, Thibaut Pellerin, Yann Portella}

\usepackage{amsmath}
\usepackage{amssymb}

\usepackage[english]{babel}
\usepackage[hyphens]{url}
\usepackage{hyperref}
\usepackage[utf8]{inputenc}

\newtheorem{definition}{Definition}[section]
\newtheorem{notation}[definition]{Notation}
\newtheorem{theorem}{Theorem}[section]
\newtheorem{computational theorem}{Computational Theorem}[theorem]
\newtheorem{corollary}{Corollary}[theorem]
\newtheorem{remark}{Remark}[section]



\begin{document}
\maketitle

\section{Abstract}



\section{Introduction, context and notations}

We start by defining sum-free and weakly sum-free subsets to introduce regular and weak Schur numbers.

\begin{definition}
A subset \(A\) of \(\mathbb{N}\) is said to be \textit{sum-free} when:
\[ \forall (a,b) \in A^2 \text{, } a+b \notin A\]
\end{definition}

\begin{definition}
A subset \(B\) of \(\mathbb{N}\) is said to be \textit{weakly sum-free} when:
\[ \forall (a,b) \in B^2 \text{, } a \neq b \Longrightarrow a+b \notin B\]
\end{definition}

Let us notice that a sum-free subset is also weakly sum-free, hence justifying the name of \textit{weakly} sum-free
subsets. Given \(m\) and \(n\) two integers, we are interested in partitioning the set of integers from 1 to \(m\) into
\(n\) weakly sum-free subsets.

\begin{notation}
We denote by \([\![1,n]\!]\) the set of integers \(\{1, 2, ..., n\}\).
\end{notation}

Schur proved in \hyperlink{label1}{\textbf{[1]}} that given a number of subsets \(n\), there exists a value of \(m\)
such that there exists no partition of \([\![1,p]\!]\) into \(n\) sum-free subsets for any \(p \geqslant m\). A similar
property holds for weakly sum-free subsets (reference necessaire). These observations lead to the following definitions.

\begin{definition}
Let \(n \in \mathbb{N}^*\). There exists a greatest integer that we note \(S(n)\) (\textit{resp. \(WS(n)\)}) such that
\([\![1,S(n)]\!]\) (resp. \([\![1,WS(n)]\!]\)) can be partitioned into \(n\) sum-free subsets (resp. weakly sum-free
subsets). \(S(n)\) is called the \textit{\(n\)\textsuperscript{th} Schur number} and \textit{\(WS(n)\) the \(n\)\textsuperscript{th} weak
Schur number}.
\end{definition}

\begin{notation}
For a partition of \([\![1, n]\!]\) in \(k\) subsets, we generally note these subsets \(A_1, ..., A_k\). We also note \(m_i = \text{min}(A_i)\). 
By ordering the subsets, we mean assuming that \(m_1 < ... < m_k\). However, if not specified we do not make this hypothesis since we 
do not always consider partitions in which every subset plays a symmetric role.
\end{notation}



\section{Schur numbers}

\qquad In this section, we use Rowley's constructions [7] in a Schur context. To improve lower bounds for Ramsay's numbers, Rowley 
introduces partitions verifying some properties which can be extended using method which generalizes Abbott-Hanson's [8] construction. 
Rolwey named these partitions "templates", and we will keep this name in the entire article. Finding suitable templates and applying his 
method to these will allow us to find new lower bounds for Schur methods.

\subsection{Definition of SE\(_+\)}

\begin{definition}
Let \( n \in \mathbb{N}^*\), there exists a greatest integer that we note \(SE_+(n)\) (resp. \(SE_-(n)\)) such that \( [\![1, SE_+(n)]\!]\) 
(resp. \( [\![1, SE_-(n)]\!]\)) can be partitionned (resp. can not be partionned) into \(n\) sum-free subsets \(A_1, A_2, ..., A_n\) which verify :
\[
\forall k \in [\![1, n-1]\!], \forall (x,y) \in A_k^2, x+y > SE_+(n)
\Longrightarrow x+y-SE_+(n) \notin A_k
\]
\end{definition}
We notice that in general, unlike regular Schur numbers, we don't have \(SE_+(n) + 1 \neq SE_-(n)\)

\begin{theorem}
Let \(n \in [\![2, +\infty]\!]\), we have :
\[
2S(n-1)+1 \leqslant SE_+(n) \leqslant S(n)
\]
\end{theorem}

\begin{remark}
\(SE_+\) and \(S\) have the same asymptotic growth rate.
\end{remark}


\subsection{Inequaties on SE_+} 

The main result on \(SE_+\) follows. It allows us to improve lower bounds on Schur numbers by computing \(SE_+\).

\begin{theorem}
Let \(n, k \in \mathbb{N}^*\), we have \\
\[ S(n+k) \geq SE_+(k+1)S(n) \]
\end{theorem}

Using a SAT solvers, we are able to exhibit adequate Schur partitions, i.e satisfying \(SE_+\) constraint, 
hence providing lower bound on \(SE_+\). We refer to such partition as \textit{templates}. Moreover, when using 
a template in the above formula, it is possible to get an additive constant. We seek templates providing "affine"
inequalities : \(aS(n+k) + b \) where \(a\) is a lower bound for \(SE_+(k+1)\). Further details can be found in the \hyperlink{SAT}{SAT section}. \\

The special case \(k \in {1, 2} \) provides Rowley's inequalities \hyperlink{label2}{\textbf{[2]}}, \(S(n+1) \geq 4S(n)+2\) and \(S(n+2) \geq 13S(n)+8\). 
In this case, \(SE_+(2) = 4 = S(2)\) and \(SE_+(3) = 13 = S(3)\). We found other inequalities as well.


\subsection{}


\section{Weak Schur numbers}

In this section, we generalize Rowley's constructions in [2]. We then introduce, by analogy with the third section, the integer \(WE_+(n)\) 
to build suitable templates.

\subsection{Lower bound for Weak Schur numbers using Schur and Weak Schur numbers}

The following theorem, inspired by Rowley's work, was found and proved by Romain Ageron.
\begin{theorem}
Let $(n,p), (r,s) \in (\mathbb{N}^*)^2$. If there exists a partition of $r$ weakly sum-free subsets of $[\![1,n]\!]$ and a partition of $s$ sum-free 
subsets of $[\![1,p]\!]$ then there exists a partition of $r+s$ weakly sum-free subsets of $[\![1,p(n+\left \lceil \frac{n}{2} \right \rceil + 1)+n]\!]$
\end{theorem}
In particular, if we choose $n = WS(r)$ and $p = WS(s)$ in the last theorem, the next corollary follows.
\begin{corollary}
$ \forall (n,m) \in (\mathbb{N}^*)^2 \text{, } WS(n+m) \geqslant S(m) \left (WS(n) + \left \lceil \frac{WS(n)}{2} \right \rceil +1 \right) + WS(n)$
\end{corollary}
\textsc{Proof :} Let $(n,p), (r,s) \in (\mathbb{N}^*)^2$,  $N = p(n+\left \lceil \frac{n}{2} \right \rceil + 1)+n$, $\alpha = \left \lceil \frac{n}{2} \right \rceil > 0$ and $\beta = n + \alpha + 1$. We denote by $f$ the projection of the equivalence relation induced by the partition of $[\![1,n]\!]$ and $g$ the one induced by the partition of $[\![1,p]\!]$. Each equivalence class is represented by a single integer, therefore :
\[ f : [\![1,n]\!] \longrightarrow [\![1,r]\!] \text{ and } \forall (x,y) \in [\![1,n]\!]^2, \left\{
    \begin{array}{ll}
        x \neq y \\
        f(x) = f(y)
    \end{array}
\right.
\Longrightarrow f(x+y) \neq f(x)
\]
\[g : [\![1,p]\!] \longrightarrow [\![1,s]\!] \text{ and } \forall (x,y) \in [\![1,n]\!]^2 \text{, } f(x) = f(y) \Longrightarrow f(x+y) \neq f(x)
\]
Let us start by parting the integers of $[\![1,N]\!]$ in two subsets $\mathcal{A}$ and $\mathcal{B}$ where $\mathcal{A} = [\![1,\alpha]\!] \cup \{a\beta + u \mid (a,u) \in [\![0,p]\!] \times [\![\alpha + 1,n]\!]\}$ and $\mathcal{B} = \{a\beta + u \mid (a,u) \in [\![1,p]\!] \times [\![-\alpha,\alpha]\!]\}$.\\
\\
First, \underline{$\mathcal{A} \cap \mathcal{B} = \varnothing$} : \\
Suppose there exists $x \in \mathcal{A} \cap \mathcal{B} = \varnothing$. Then there are $(a,u) \in [\![0,p]\!] \times [\![\alpha + 1,n]\!]$ and $(b,v) \in [\![1,p]\!] \times [\![-\alpha,\alpha]\!]$ such that $x = a\beta + u = b\beta +v$. By definition of $\alpha$ and  $\beta$ we have $u \in [\![\alpha + 1,n]\!] \subset [\![0,\beta - 1]\!]$. From there, we distinguish two cases :
\begin{itemize}
\item If $v \in [\![0,\alpha]\!]$ then $v \in [\![0,\beta - 1]\!]$ and $v \neq u$ because $v < \alpha + 1 \leqslant u$
\item If $v \in [\![-\alpha,-1]\!]$, we note $\tilde{v} = \beta + v$ and thus have $x = (b-1)\beta + \tilde{v}$ with $\tilde{v} \in [\![\beta - \alpha,\beta - 1]\!] \subset [\![0,\beta - 1]\!]$ and $\tilde{v} \neq u$ because $u < n+1 = \beta - \alpha \leqslant \tilde{v}$.
\end{itemize}
In either cases, we run into a contradiction because of the remainder's uniqueness in the euclidean division of $x$ by $\beta$.\\
\\
Then, we have \underline{$\mathcal{A} \cup \mathcal{B} = [\![1,N]\!]$}:
\begin{itemize}
\item On one hand : $1 = \text{min($\mathcal{A}$)} \leqslant \text{max($\mathcal{A}$)} = p\beta + n = N$ and $1 \leqslant \beta - \alpha = \text{min($\mathcal{B}$)} \leqslant \text{max($\mathcal{B}$)} = p\beta + \alpha \leqslant N$, which gives $\mathcal{A} \cup \mathcal{B} \subset [\![1,N]\!]$.
\item On the other hand, let $x \in  [\![1,N]\!]$. If $x \leqslant \alpha$, we directly have $x \in \mathcal{A}$, let us then suppose that $x > \alpha$ and write $x = a\beta + u$ the euclidean division of $x$ by $\beta$. We have $x \leqslant N$, thus $a \leqslant p$. We distinguish three cases : \\
- If $u \in [\![0,\alpha]\!]$ then we necessarily have $a \geqslant 1$ because $x > \alpha$, and so $x \in \mathcal{B}$.\\
- If $u \in [\![\alpha + 1,n]\!]$, then $x \in \mathcal{A}$. \\
- If $u \in [\![n + 1,\beta - 1]\!]$ then $x = (a+1)\beta - (\beta - u)$ with $-\alpha \leqslant \beta - u \leqslant 0$. Furthermore, $a \leqslant p - 1$, else we would have $x > N$, and so $x \in \mathcal{B}$ \\
In any case, $x \in \mathcal{A} \cup \mathcal{B}$ and we can thus conclude that $[\![1,N]\!] \subset \mathcal{A} \cup \mathcal{B}$.
\end{itemize}
This first partition of $[\![1,N]\!]$ will help us to define our final partition by the projection of its equivalence relation. We thereby define $h : [\![1,N]\!] \longrightarrow [\![1,r+s]\!]$ as such :\\
- If $x \in \mathcal{A}$ then $h(x) = f(x \text{ mod } \beta)$ (well defined because $x \text{ mod } \beta \in [\![1,N]\!]$)\\
- If $x \in \mathcal{B}$ then $x = a\beta + u$ with a unique $(a,u) \in [\![1,p]\!] \times [\![-\alpha,\alpha]\!]$ and we define $h(x) = r + g(a)$\\
The fact that $(\mathcal{A}, \mathcal{B})$ is a partition of $[\![1,N]\!]$ ensures that this definition of $h$ is valid. We then have to verify that $h$ induces weakly sum-free subsets.\\
\\
\underline{The classes of equivalence $h(x)$ for $x \in \mathcal{A}$ are weakly sum-free :}
\\
\\
Let $(x,y) \in \mathcal{A}^2$ such that $h(x) = h(y)$, $x \neq y$ and $x + y \leqslant N$
\begin{itemize}
\item If $(x,y) \in [\![1,\alpha]\!]^2$ :\\
We have $x + y \leqslant 2\alpha \leqslant n$ and $x + y = 0\beta + x + y$, therefore $x + y \in \mathcal{A}$. Then, by definition : $h(x) = f(x)$, $h(y) = f(y)$ and $h(x+y) = f(x+y)$, which gives us, thanks to the property verified by $f$, that $h(x+y) \neq h(x)$.
\item If $(x,y) \in [\![1,\alpha]\!] \times ( \mathcal{A} \text{ \textbackslash} \text{ } [\![1,\alpha]\!] )$ :\\
We write $y = a\beta + u$ with $(a,u) \in [\![0,p]\!] \times [\![\alpha + 1,n]\!]$. Then $x+y = a\beta + x + u = (a+1)\beta + x + u - \beta$, and if $x + u > n$ it follows that $a \leqslant p-1$ since $x+y \leqslant N$, and $-\alpha \leqslant x + u - \beta \leqslant -1$. Therefore $x+y \in \mathcal{B}$ and $h(x+y) \neq h(x) = f(x)$ by definition of h. On the contrary, if $x - u \leqslant n$, then $x+y \in \mathcal{A}$ and $h(x+y) = f(x+u)$ because $x+u$ is actually the remainder of the euclidean division of $x+y$ by $\beta$. Moreover, $h(x) = f(x)$, $x < u$ and, with our initial hypothesis, $h(x) = h(y) = f(u)$. The property verified by $f$ gives us $f(x+u) \neq f(x)$ which can be rewritten as $h(x+y) \neq h(x)$.
\item If $(x,y) \in ( \mathcal{A} \text{ \textbackslash} \text{ } [\![1,\alpha]\!] ) \times [\![1,\alpha]\!]$ : \\
This case is handled exactly like the previous one by swaping the roles of $x$ and $y$.
\item If $(x,y) \in ( \mathcal{A} \text{ \textbackslash} \text{ } [\![1,\alpha]\!] )^2$ : \\
We write $x = a\beta + u$ and $y = b\beta + v$ with $(a,u)$ and $(b,v)$ in $[\![0,p]\!] \times [\![\alpha + 1,n]\!]$. Then $x+y = (a+b)\beta + u+v = (a+b+1)\beta + u + v - \beta$ with $a+b \leqslant p-1$ (else we would have $x+y > N$ because $u+v > n$) and $-\alpha \leqslant u + v - \beta \leqslant \alpha$, therefore $x+y \in \mathcal{B}$ and by definition $h(x+y) \neq h(x)$.
\end{itemize}
In any case, $h(x+y) \neq h(x)$ and the classes of equivalence $h(x)$ for $x \in \mathcal{A}$ are weakly sum-free.\\
\\
\underline{The classes of equivalence $h(x)$ for $x \in \mathcal{B}$ are weakly sum-free :}
\\
\\
Let $(x,y) \in \mathcal{B}^2$ such that $h(x) = h(y)$, $x \neq y$ and $x + y \leqslant N$.\\
We write $x = a\beta + u$ and $y = b\beta + v$ with $(a,u)$ and $(b,v)$ in $[\![1,p]\!] \times [\![-\alpha,\alpha]\!]$. We have $h(x) = r + g(a)$ and $h(y) = r + g(b)$, therefore $g(a) = g(b)$. We also have $x+y = (a+b)\beta + u +v$.\\
If $u + v \in [\![-\alpha,\alpha]\!]$, then $x+y \in \mathcal{B}$ and $h(x+y) = g(a+b)$, hence we can deduce that $h(x+y) \neq h(x)$ because of the property verified by $g$. On the contrary, if $u+v \notin [\![-\alpha,\alpha]\!]$, then necessarily $x+y \in \mathcal{A}$. Suppose $x+y \in \mathcal{B}$, then $x+y = c\beta + w$ with $(c,w) \in [\![1,p]\!] \times [\![-\alpha,\alpha]\!]$. Thus, $c\beta + w = (a+b)\beta + u + v$ and $(a+b-c)\beta = w-u-v$. Furthermore $a+b-c \neq 0$, else we would have $u+v = w \in [\![-\alpha,\alpha]\!]$. This finally leads to the following inequality :
\[\beta \leqslant |a+b-c|\beta = |w-u-v| \leqslant |w| + |u| + |v| \leqslant 3\alpha \leqslant n + \alpha < \beta
\]
which is absurd. We can therefore conclude that $x+y \in \mathcal{A}$ and by definition of $h$, $h(x+y) \neq h(x)$, proving that the classes of equivalence $h(x)$ for $x \in \mathcal{B}$ are weakly sum-free.\\
\\
Finally, we have showed that every classe of equivalence induced by $h$ is weakly sum-free, which ends the proof.



\section{About the construction of lower bounds for weak Schur numbers using a computer}

In this section, we first reframe the question of the existence of (weakly) sum-free partitions as a boolean
satisfiability (SAT) problem. We then provide evidence which indicates that the main assumption made by papers which
found the previous best known lower bounds for weak Schur numbers may not be correct. Finally, we obtain stronger
results than those previously known for \(WS(5)\) while gaining several orders of magnitude in computation time by 
giving additional information to the SAT solver without losing in generality. In this section, we assume that the subsets are ordered.


\subsection{Reformulation as a SAT problem}
\label{SAT}

We encode the existence of (weakly) sum-free partitions as propositional formulae like in \hyperlink{label3}{\textbf{[3]}} and then use 
SAT solvers to determine whether these formulae are satisfiable.

\begin{definition}
A \textit{literal} is either a variable \(v\) (a positive literal) or the negation \(\bar{v}\) of a variable \(v\) (a negative literal) where \(v\) 
takes a truth value: \(true\) or \(false\). A \textit{clause} is a disjunction of literals and a \textit{formula} is a conjunction of clauses: it 
is a propositional formula in \textit{conjonctive normal form} (CNF).
\end{definition}

\begin{definition}
An \textit{assignment} is a function from a set of variables to the truth values \(true\) (1) and \(false\) (0). A literal \(l\) is 
\textit{satisfied} (\textit{falsified}) by an assignment \(\alpha\) if l is positive and \(\alpha(var (l)) = 1\) 
(resp. \(\alpha(var (l)) = 0\)) or if it is negative and \(\alpha(var (l)) = 0\) (resp. \(\alpha(var (l)) = 1\)). A clause is \textit{satisfied} 
by an assignment \(\alpha\) if it contains a literal that is satisfied by \(\alpha\). Finally, a formula is \textit{satisfied} by an assignment 
\(\alpha\) if all its clauses are satisfied by \(\alpha\). A formula is \textit{satisfiable} if there exists an assignment that satisfies it; 
otherwise it is \textit{unsatisfiable}.
\end{definition}

We then encode the existence of a partition of \([\![1,n]\!]\) in \(k\)  weakly sum-free subsets as follows: for every integer 
\(i \in [\![1,n]\!]\), take \(k\) variables \(x^{(i)}_{1}, ..., x^{(i)}_{k}\) and for every \(\forall c \in [\![1,k]\!], x^{(i)}_c = 1 \iff i \in A_c\). 
The corresponding clauses are:

\begin{itemize}
\item \textbf{sumfree:} \(\forall c \in  [\![1,k]\!], \forall (i, j) \in [\![1,n]\!]^2, (i \neq j ~ \text{and} ~ i + j \leq n) \implies \lnot x^{(i)}_c 
\lor  \lnot x^{(i)}_c \lor \lnot x^{(i+j)}_c\)
\item \textbf{union:} \(\forall i \in [\![1,n]\!], x^{(i)}_1 \lor ... \lor x^{(i)}_k\)
\item \textbf{disjoint:} \(\forall i \in [\![1,n]\!],\forall (c_1, c_2) \in  [\![1,k]\!]^2, c_1 \neq c_2 \implies \neg x^{(i)}_{c_1} \lor \neg x^{(i)}_{c_2}\)
\end{itemize} 

In the above formula, every color plays a symmetric role. Hence the search space can reduced by \(k!\) by ordering the subsets, that is by 
enforcing that \(m_1 < ... < m_k\). The corresponding clauses are: \linebreak
\textbf{symmetry breaking:}  \(x^{(1)}_1 = 1\) and \(\forall c \in [\![2,k-1]\!], \forall i \in [\![1,WS(c - 1)+1]\!], x^{(1)}_c \lor ... \lor x^{(i)}_c \lor \neg x^{(i+1)}_{c+1}\)

\begin{remark}
For a given problem, it can be interesting to try out different SAT solvers because the relative performance can vary significantly according to the problem. 
For instance, we used two different SAT solvers in the next two next subsections.
\end{remark}


\subsection{The search space previously used in computer search for lower bounds may not contain the optimal partitions}

Rowley's new lower bound for \(WS(6)\) (642)  \hyperlink{label2}{\textbf{[2]}} was a quite significant improvement upon the former 
best known lower bound (582). This previous lower bound was found using a computer (often with Monte-Carlo) methods and by making the 
assumption that a good partition for \(WS(n+1)\) starts with a good partition for \(WS(n)\) which is true for small values of \(n\).
Therefore, one may wonder whether the limiting factor is the assumption or the methods used to search for partitions. It appears that 
the latter is correct.

\begin{computational theorem}
There is no weakly sum-free partition of \([\![1,583]\!]\) in 6 parts such that:
\begin{itemize}
	\item \(m_5 \ge 66\)
	\item \(m_6 \ge 186\)
	\item \([\![210,349]\!] \subset A_6\)
\end{itemize}  
\end{computational theorem}

This result was obtained in 8 hours with the SAT solver plingeling \hyperlink{label5}{\textbf{[5]}} on a 2.60 GHz Intel i7 processor PC.
However, simply encoding the existence of such a partition as explained in the previous subsection would not result in a reasonable 
computation time. In order to help the SAT solver, we add additional information in the propositional formula. We did not quantify the 
speedup, but it most likely allows us to gain several order of magnitude in computation time as we explain in the next subsection.

\par
Every weakly sum-free partition of \([\![1,65]\!]\) in 4 subsets starts with the following sequence 1121222133. Then 11 is always 
either in subset 1 or 3, 12 is always in subset 3 and so on. For every integer in \([\![1,65]\!]\), we computed in which subset it can appear. 
By using this constraints, we could then compute for every integer in \([\![1,185]\!]\), in which subset it can appear in a weakly sum-free 
partition of  \([\![1,185]\!]\) which starts with a  weakly sum-free partition of \([\![1,65]\!]\) in 4 subsets. Adding these constraints to the 
formula corresponding to the above theorem gives additional information to the SAT solver without losing in generality.

\par
The above theorem shows that the previous lower bound for \(WS(6)\) is optimal in the search space considered by the papers which found it. 
Therefore, finding a partition of \([\![1,n]\!]\) in 6 weakly sum-free subsets for some \(n \geq 590\) which does not have a template-like structure 
would be extremely interesting since it could give indications on a new search space for improving lower bounds with a computer. More generally, 
it questions the search space previously used for finding lower bounds for \(WS(n)\) with a computer. In particular, to our knowledge every paper 
that found the lower bound \(WS(5) \geq 196\) used this assumption [reference necessaire]. Therefore one may wonder if this actually a good lower 
bound. In the next subsection, we give properties that a partition of \([\![1,197]\!]\) in 5 weakly sum-free subsets has to verify.


\subsection{Weak Schur number five}
As explained in the previous subsection, the search space used for showing that  \(WS(5) \geq 196\) may not contain optimal solution. In this subsection, 
we give necessary conditions for a hypothetical partition of \([\![1,197]\!]\) in 5 weakly sum-free subsets using the same type of methods as in the
 previous subsection.

\par
\hyperlink{label4}{\textbf{[4]}} verified with a SAT solver that there are no partition in 5 weakly sum-free subsets of \([\![1,197]\!]\) with  
\(A_5 = \{67, 68\} \cup [\![70,134]\!] \cup \{136\}\) in 17 hours and could not provide a similar result when only assuming \(m_5 = 67\) even after several 
weeks of runtime. By using the same method as above, we were able to verify that \(WS(5 | m_5 = 67) = 196\) in 0.5 seconds with the SAT solver glucose 
on a 2.60 GHz Intel i7 processor PC (we used the non-parallel version here but in the rest of this subsection, we used the parallel version of glucose). 
The additional information we gave to the SAT solver is that every partition of \([\![1,66]\!]\) in 4 weakly sum-free subsets starts with a partition of 
\([\![1,23]\!]\) in 3 weakly sum-free subsets (this can be checked in a few dozens of minutes with a SAT solver). Among the 3 partitions of \([\![1,23]\!]\) in 
3 weakly sum-free subsets, every number always appears in the same subset except for 16 and 17 which can appear in two different subsets. We hardcoded 
this external knowledge in the propositional formula which allowed us to gain several orders of magnitude in computation time.

\begin{computational theorem}
If there exists a partition of \([\![1,197]\!]\) in 5 weakly sum-free subsets then \(m_5 \leq\). 
\end{computational theorem}

More precisely, we verified the following results.

\begin{tabular}{| c | *{21}{ p{2mm} |}}
	\hline
	\(m_4\)                   &   4   &   5   &   6  &   7   &   8   &   9   &  10  &  11  &  12  &  13  &  14  &  15  &  16  &  17  &  18  &  19  &  20  &  21  &  22  &  23  &  24  \\ 
	\hline
	\(WS(4 | m_4) + 1\) &  55  &  59  &  60  &  59  &  59  &  60  &  60  &  60  &  60  &  64  &  63  &  64  &  61  &  64  &  63  &  65  &  65  &  65  &  65  &  66  &  67  \\
	\hline
	\(max~m_5\)           &        &       &        &        &       &        &       &        &        &       &        &       &        &       &        &        &       &        &       &        &  53  \\
	\hline
\end{tabular}



\section{Conclusions and future work}



\section{Acknowledgments}



\section{References}

\hypertarget{label1}{\textbf{[1]}} Schur, I. 1917. Uber die Kongruenz \(x^m + y^m = z^m~(\text{mod}~p)\). Jahresbericht der 
Deutschen Mathematikervereinigung 25:114–117.: 114-116. \url{http://eudml.org/doc/145475}

\hypertarget{label2}{\textbf{[2]}} Rowley, F. 2020. New Lower Bounds for Weak Schur Partitions. arXiv 2011.11292, 
\url{https://arxiv.org/pdf/2011.11292.pdf}

\hypertarget{label3}{\textbf{[3]}} Heule, M.J.H. 2017. Schur Number Five. arXiv 1711.08076, \url{https://arxiv.org/pdf/1711.08076.pdf}

\hypertarget{label4}{\textbf{[4]}} S. Eliahou, J.M. Marín, M.P. Revuelta and M.I. Sanz 2012. Weak Schur numbers and the search for G.W. 
Walker’s lost partitions. Computers \& Mathematics with Applications Vol. 63, \url{https://www.sciencedirect.com/science/article/pii/S0898122111009722}

\hypertarget{label5}{\textbf{[5]}} Bierem, A. 2017. CaDiCaL, Lingeling, Plingeling, Treengeling, YalSAT Entering the SAT Competition 2017. 
Proceedinfs of SAT Competition 2017 -- Solver and Benchmark Descriptions \url{http://fmv.jku.at/papers/Biere-SAT-Competition-2017-solvers.pdf}

\end{document}