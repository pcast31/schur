\documentclass[final,onefignum,onetabnum]{siamart190516}


\usepackage{amssymb}
\usepackage{dsfont}
\usepackage{MyMnSymbol}

\usepackage{float}

\usepackage{nicematrix}

\newsiamremark{remark}{Remark}

\newcommand{\WS}{\mathit{WS}}



\title{New lower bounds for Schur and weak Schur numbers \thanks{Submitted to the editors 02/24/2022.}}

\author{Romain Ageron\thanks{CentraleSup\'elec, Universit\'e{} Paris-Saclay, Gif-sur-Yvette, France
	(\email{romain.ageron@student-cs.fr}, \email{paul.casteras@student-cs.fr},
	\email{thibaut.pellerin@student-cs.fr}, \email{yann.portella@student-cs.fr}).}
\and Paul Casteras\footnotemark[2]
\and Thibaut Pellerin\footnotemark[2]
\and Yann Portella\footnotemark[2]
\and Arpad Rimmel\footnotemark[2]\ \thanks{LISN, CentraleSup\'elec, Universit\'e{} Paris-Saclay, Orsay, France
	(\email{arpad.rimmel@centralesupelec.fr}).}
\and Joanna Tomasik\footnotemark[2]\ \footnotemark[3]\ \thanks{Corresponding author
	(\email{joanna.tomasik@centralesupelec.fr}).}}

\headers{New lower bounds for Schur and weak Schur numbers}{R. Ageron, P. Casteras,
T. Pellerin, Y. Portella, A. Rimmel, J. Tomasik}


\begin{document}
\maketitle

\begin{abstract}
This article provides new lower bounds for both Schur and weak Schur numbers by exploiting a "template"-based approach. 
The concept of "template" is also generalized to weak Schur numbers. Finding new templates leads to explicit partitions 
improving lower bounds as well as the growth rate for Schur numbers, weak Schur numbers, and multicolor Ramsey numbers \(R_n(3)\). 
The new lower bounds include \(S(9) \geqslant 17\,803\), \(S(10) \geqslant 60\,948\), \(\WS(6) \geqslant 646\), 
\(\WS (9) \geqslant 22\,536\) and \(\WS (10) \geqslant 71\,256 \).
\end{abstract}

\begin{keywords}
	Schur number, Weak Schur number, Ramsey theory, Sum-free partition
\end{keywords}

\begin{AMS}
	05D10, 11P81, 05A17, 11B75
\end{AMS}

\section{Introduction}

We are interested in partioning the set of integers \(\{1, ..., p\}\) in \(n\) subsets such that there is no subset 
containing three integers \(x\), \(y\) and \(z\) verifying \(x + y = z\). We say these subsets are sum-free. If we 
add the hypothesis \(x \neq y\), we say the subsets are weakly sum-free. The greatest \(p\) for which a 
partition into \(n\) sum-free exists is called the \(n^{\text{th}}\) Schur number and is denoted \(S(n)\) 
\cite{Schur1917}. Likewise for weakly sum-free partitions we define \(WS(n)\) the \(n^{\text{th}}\) weak Schur 
number \cite{Irving1973}. Only the first values of these sequences are known. This article concerns the lower 
bounds of these numbers.

\par Before Rowley's "template"-based approach for Schur and Ramsey numbers \cite{RowleyRamsey}, the 
previous generic construction for Schur numbers was given by Abbott and Hanson \cite{AbbottHanson} in 1972 
with a recursive construction. It gave the best lower bounds for all sufficiently large bounds. No equivalent 
was known for weak Schur numbers and as a result the best known partitions for large weak Schur numbers 
did not utilize the weakly sum-free hypothesis. As for smaller numbers, the best lower bounds were obtained 
by conducting a computer search. This search for weakly sum-free partitions relied on the recursive assumption 
that a good weakly sum-free partition into \(n+1\) colors starts with a good weakly sum-free partition into \(n\) 
colors. In 2020, Rowley introduced the notion of template for Schur and Ramsey numbers which generalizes 
Abbott and Hanson's construction and gives new lower bounds (and inequalties) for Schur numbers. Rowley also 
gives two inequalities for weak Schur numbers that yield significant improvements over previous lower bounds 
which besides do utilize the weakly sum-free hypothesis \cite{RowleyWS}.

\renewcommand{\arraystretch}{1.5}

\begin{center}

\textbf{Table 1 - Comparison of lower bounds for Schur numbers}

\begin{tabular}{|*{13}{c|}}
    \hline
    \(n\) & 1 & 2 & 3 & 4 & 5 & 6 & 7 & 8 & 9 & 10 & 11 & 12 \\
    \hline
    old & 1 & 4 & 13 & 44 & 160 & 536 & 1680 & 5041 & 15124 & 51120 & 172216 & 575664 \\
      & & & & & \cite{Heule2017} & \cite{Fredricksen} & \cite{Fredricksen} & \cite{ELIAHOU2012175} & 
    \cite{ELIAHOU2012175} & \cite{AbbottHanson} & \cite{AbbottHanson} & \cite{AbbottHanson} \\
    \hline
    Rowley & & & & & & & & 5286 & 17694 & 60320 & 201696 & 631840 \\
    \hline
    new & & & & & & & & & 17803 & 60948 & 203828 & 638548 \\
    \hline
\end{tabular}

\vspace{3ex}
\textbf{Table 2 - Comparison of lower bounds for weak Schur numbers}

\begin{tabular}{|*{13}{c|}}
    \hline
    \(n\) & 1 & 2 & 3 & 4 & 5 & 6 & 7 & 8 & 9 & 10 & 11 & 12 \\
    \hline
    old & 2 & 8 & 23 & 66 & 196 & 582 & 1740 & 5201 & 15596 & 51520 & 172216 & 575664 \\
      & & & & & \cite{ELIAHOU2012175} &\cite{EliahouBook} & \cite{Rafilipojaona} & \cite{Rafilipojaona} & 
    \cite{Rafilipojaona} & \cite{AbbottHanson} & \cite{AbbottHanson} & \cite{AbbottHanson} \\
    \hline
    Rowley & & & & & & 642 & 2146 & 6976 & 21848 & 70778 & 241282 & 806786 \\
    \hline
    new & & & & & & & & & 22536 & 71214 & 243794 & 815314 \\
    \hline
\end{tabular}

\end{center}

\par The main contribution of this article is a generalization of the concept of template to weak Schur numbers. 
This gives new lower bounds (and inequalities) for weak Schur numbers. This construction also includes as a 
special case an analogous for weak Schur numbers of Abbott and Hanson's construction.

\par We first explain Rowley's template-based construction in the context of Schur numbers and then give 
new templates, thus providing new lowers bounds and inequalities as well as showing that the growth rates 
for both Schur and Ramsey numbers exceed 3.28. Then, we generalize the concept of templates to weak Schur 
numbers and provide new lower bounds for weak Schur numbers. Finally, we analyze the significant difference 
between new lower bounds obtain with templates and the former lower bounds obtained by computer search 
and we provide evidence which indicate that the main assumption made in those articles removes the optimal 
partitions from the search space.

\begin{notation}
We denote by \([\![1,p]\!]\) the set of integers \(\{1, 2, ..., p\}\).
\end{notation}

\begin{notation}
For a partition of \([\![1, p]\!]\) in \(n\) subsets, we generally denote these subsets \(A_1, ..., A_n\). We 
also denote \(m_i = \min(A_i)\).
By ordering the subsets, we mean assuming that \(m_1 < ... < m_n\). However, if not specified we do not 
make this hypothesis since we do not always consider partitions in which every subset plays a symmetric role.
\end{notation}

\begin{definition}
We sometimes refer to a partitition as a colouring. The colouring associated to a partition \(A_1, ..., A_n\) of
\([\![1, p]\!]\) is the function \(f\) such that \(\forall x \in [\![1, p]\!], x \in A_{f(x)}\). Likewise, the partition 
associated to a colouring \(f\) of \([\![1, p]\!]\) with \(n\) colors is \(\forall c \in [\![1, n]\!], A_c = f^{-1}(c)\).
\end{definition}


\section{Schur numbers}

\qquad In this section, we use Rowley's constructions \cite{RowleyRamsey} in the context of Schur numbers. To improve lower bounds
for Ramsay's numbers, Rowley introduces partitions verifying some properties which can be extended using a method which generalizes 
Abbott and Hanson's  construction \cite{AbbottHanson}. Rowley named these partitions "templates", and we will keep this name in the 
entire article. We find suitable templates and use them to find new lower bounds for Schur numbers.

\subsection{Definition of \(S^+\)}

\begin{definition}
We call SF-template of \(n\) colors and lenght \(p\) a partition of \( [\![1,p]\!]\) into \(n\) sum-free subsets \(A_1,
A_2, ..., A_n\) which verify :
	\[
	\forall i \in [\![1, n-1]\!], \forall (x,y) \in A_i^2, x+y > p
	\Longrightarrow x+y-p \notin A_i
	\]
	We note \(S^+(n)\) the maximal lenght for a SF-template of \(n\) colors.
\end{definition}

\begin{remark}
	Here, \(n\) is the "special" color: it has less constraints than the other colors. However, please note that \(n\) 
	is not necessarily the last color by order of appearance.
\end{remark}

\begin{remark}
	\(SF\)-templates include Abott and Hanson's construction \cite{AbbottHanson} as a special case.
\end{remark}

\begin{proposition}
	Let \(n \in [\![2, +\infty]\!]\), we have :
	\[
	2S(n-1)+1 \leqslant S^+(n) \leqslant S(n)
	\]
\end{proposition}

\textsc{Proof :} The lower bound comes from Abott and Hanson's construction. The upper bound comes
from the fact that a SF-template of length \(p\) with \(n\) colors is also a partition of
\([\![1, p]\!]\) into \(n\) sum-free subsets.

\begin{remark}
	\(S^+\) and \(S\) have the same asymptotic growth rate.
\end{remark}


\subsection{Inequalities using \(S^+\)}

The main result on \(S^+\) follows. It allows us to improve lower bounds on Schur numbers by computing \(S^+\).

\begin{theorem}
	Let \((n,k), (p,q) \in (\mathbb{N}^*)^2\). If there exists a \(SF\)-template of \(k+1\) colors and lenght \(p\),
	and a partition of \(n\) sum-free subsets of \([\![1,q]\!]\) then there exists a partition of \(n+k\) sum-free subsets
	of \([\![1,pq+m_{k+1}-1]\!]\). \(m_{k+1}\) is the first number colored with the \(k+1\)-th color in the \(SF\)-template.
\end{theorem}

Setting \(p = S^+(k+1)\) and \(q = S(n)\) yields the following corollary.

\begin{corollary}
	Let \(n, k \in \mathbb{N}^*\), we have \\
	\[ S(n+k) \geqslant S^+(k+1)S(n) + m_{k+1} - 1 \]
\end{corollary}

The idea lying beneath this inequality is similar to Abbott and Hanson's contruction \cite{AbbottHanson}. 
They extend vertically a sum-free partition, and horizontally an other sum-free partition. This way each "block" 
acts like a security zone for the other one. Here, the horizontal partition is no longer to the side of the 
vertical one, but it occupies the column of the special color of the \(SF\)-template, i.e the one without the 
extra condition. We give the following example for \(p = 7, q = 4, n = 2 ~\textrm{and}~ k = 2\). \\

This shows the inequality \(S(2+2) \geqslant S^+(3)S(2) + 4\). The special color is blue and is first used
to color 5, hence \(4 = m_{3} - 1\). 
We prooved \(S^+(3) = 9\). The special color is blue. 

\begin{center}
\[
\begin{array}{|*{9}{c|}}
	\multicolumn{9}{c}{\overset{S^+(3)}{\overbrace{\rule{6cm}{0pt}}}} \\
	\hline
	\cellcolor{red} 1 & \cellcolor{green} 2 & \cellcolor{green} 3 & \cellcolor{red} 4 & \cellcolor{blue} 5 & \cellcolor{blue} 6 & \cellcolor{red} 7 & \cellcolor{blue} 8 & \cellcolor{blue} 9 \\
	\hline
	\cellcolor{red} 10 & \cellcolor{green} 11 & \cellcolor{green} 12 & \cellcolor{red} 13 & \cellcolor{yellow} 14 & \cellcolor{yellow} 15 & \cellcolor{red} 16 & \cellcolor{yellow} 17 & \cellcolor{yellow} 18\\
	\hline
	\cellcolor{red} 19 & \cellcolor{green} 20 & \cellcolor{green} 21 & \cellcolor{red} 22 & \cellcolor{yellow} 23 & \cellcolor{yellow} 24 & \cellcolor{red} 25 & \cellcolor{yellow} 26 & \cellcolor{yellow} 27\\
	\hline
	\cellcolor{red} 28 & \cellcolor{green} 29 & \cellcolor{green} 30 & \cellcolor{red} 31 & \cellcolor{blue} 32 & \cellcolor{blue} 33 & \cellcolor{red} 34 & \cellcolor{blue} 35 & \cellcolor{blue} 36\\
	\hline
	\cellcolor{red} 37 & \cellcolor{green} 38 & \cellcolor{green} 39 & \cellcolor{red} 40 \\
	\cline{1-4}
\end{array}
\]
\end{center}


\begin{center}
\[
S^+(3) 
\begin{array}{|*{9}{c|}}
	\hline 
	\cellcolor{red} 1 & \cellcolor{green} 2 & \cellcolor{green} 3 & \cellcolor{red} 4 & \cellcolor{blue} 5 & \cellcolor{blue} 6 & \cellcolor{red} 7 & \cellcolor{blue} 8 & \cellcolor{blue} 9\\
	\hline
\end{array}  \]
\[
S(2) 
\begin{array}{|*{4}{c|}}
	\hline 
	\cellcolor{blue} 1 & \cellcolor{yellow} 2 & \cellcolor{yellow} 3 & \cellcolor{blue} 4 \\
	\hline
\end{array}
\]
\end{center}

To compare, here is a classic Abbott and Hanson partition : \(S(2+2) \geqslant S(2)(2S(2)+1)+S(2) \). 
\begin{center}
\[
\begin{array}{|*{9}{c|}}
	\multicolumn{4}{c}{\overset{S(2)}{\overbrace{\rule{2.6cm}{0pt}}}} \\
	\hline
	\cellcolor{red} 1 & \cellcolor{green} 2 & \cellcolor{green} 3 & \cellcolor{red} 4 & \cellcolor{blue} 5 & \cellcolor{blue} 6 & \cellcolor{blue} 7 & \cellcolor{blue} 8 & \cellcolor{blue} 9\\
	\hline
	\cellcolor{red} 10 & \cellcolor{green} 11 & \cellcolor{green} 12 & \cellcolor{red} 13 & \cellcolor{yellow} 14 & \cellcolor{yellow} 15 & \cellcolor{yellow} 16 & \cellcolor{yellow} 17 & \cellcolor{yellow} 18\\
	\hline
	\cellcolor{red} 19 & \cellcolor{green} 20 & \cellcolor{green} 21 & \cellcolor{red} 22 & \cellcolor{yellow} 23 & \cellcolor{yellow} 24 & \cellcolor{yellow} 25 & \cellcolor{yellow} 26 & \cellcolor{yellow} 27\\
	\hline
	\cellcolor{red} 28 & \cellcolor{green} 29 & \cellcolor{green} 30 & \cellcolor{red} 31 & \cellcolor{blue} 32 & \cellcolor{blue} 33 & \cellcolor{blue} 34 & \cellcolor{blue} 35 & \cellcolor{blue} 36 \\
	\hline
	\cellcolor{red} 37 & \cellcolor{green} 38 & \cellcolor{green} 39 & \cellcolor{red} 40 \\
	\cline{1-4}
\end{array}
\]
\end{center}

\textsc{Proof :} We denote by \(f\) the colouring associated to the \(SF\)-template of lenght \(p\)
and \(g\) the one associated to the sum-free partition of \([\![1,q]\!]\).

\[ f : [\![1,p]\!] \longrightarrow [\![1,k+1]\!] \text{ and } \forall (x,y) \in [\![1,p]\!]^2, \left\{
\begin{array}{ll}
	f(x) = f(y) \leqslant k \\
	x + y > p
\end{array}
\right.
\Longrightarrow f(x+y-p) \neq f(x)
\]
with also the sum-free condition : \[ \forall (x,y) \in [\![1,p]\!]^2 \text{, } f(x) = f(y)
\Longrightarrow f(x+y) \neq f(x) \]
Moreover, by definition of \(m_{k+1}\), if \(x < m_{k+1}\), \(f(x) \leqslant k\).

\[g : [\![1,q]\!] \longrightarrow [\![1,n]\!] \text{ and } \forall (x,y) \in [\![1,q]\!]^2 \text{, } g(x) = g(y)
\Longrightarrow g(x+y) \neq g(x)
\]

We now define \(h : [\![1,pq+m_{k+1}-1]\!] \longrightarrow [\![1,n+k]\!] \) as follows :
\( \forall x \in [\![1,pq + m_{k+1}-1]\!] \), we write \(x = \alpha p - u\) where \(\alpha \in [\![1,q+1]\!] \) and \( u \in
[\![0,p-1]\!] \).
This decomposition if of course unique. \\

If \(f(p-u) \leqslant k\), we set \(h(x) = f(p-u)\), else (i.e \(f(p-u) = k + 1)\) \(h(x) = k + g(\alpha) \). \\
If \(\alpha = q+1\), we have a problem plugging it into \(g\). Hopefully, this issue never arises : if 
\(\alpha = q+1\), we have \(u > p - m_{k+1}\) since \(x \in [\![1,pq + m_{k+1}-1]\!]\), hence \(p-u < m_{k+1}\) and \(f(p-u) \leqslant k\). \\

Prooving that the partition of \([\![1,pq]\!]\) induced by \(h\) is sum-free will complete the proof.\\

Let \(x,y \in [\![1,pq + m_{k+1}-1]\!]\), with \(h(x) = h(y)\) and \(x+y \leqslant pq+m_{k+1}-1\). We write \(x = \alpha p - u\) and \(y =
\beta p - v\). We simply need \(h(x+y) \neq h(x)\). \\

We first examine the case where \(h(x) \leqslant k\). We assume \(h(x+y) \leqslant k\), otherwise we are fine. 
Thus by definition of \(h\), we have \(h(x) = f(p-u) = f(p-v) = h(y)\).
\begin{itemize}
\item If \(u+v < p\), then \(x+y = (\alpha+\beta) - (u+v) \) with \(u+v \in [\![0,p-1]\!] \).
We have \(f(p-u) = f(p-v) \leqslant k\) and \(p-u + p-v > p\), the extra condition on \(f\) provides \(f(p-u-v) \neq
f(p-u)\).
We assumed \(h(x+y) \leqslant k\) hence \(h(x+y) = f(p-u-v)\). Since \(h(x) = f(p-u)\), we have at last \(h(x+y) \neq
h(x)\).
\item If \(u+v \geqslant p\), then \(x+y = (\alpha+\beta-1) - (u+v-p) \) with \(u+v-p \in [\![0,p-1]\!] \).
We assumed \(h(x+y) \leqslant k \) hence \(h(x+y) = f( p- (p-u-v)) = f(p-u + p-v)\). Since \(f(p-u+p-v) \neq f(p-u)\),
we get \(h(x+y) \neq h(x)\).
\end{itemize} 
  
We now assume \(h(x) > k\). Then \(h(x) = k + g(\alpha) = k + g(\beta) = h(y)\), hence \(g(\alpha) = g(\beta)\). 
We have the two same cases as before.

\begin{itemize}
\item If \(u+v < p\), then \(x+y = (\alpha+\beta) - (u+v) \) with \(u+v \in [\![0,p-1]\!] \).
We assume \(h(x+y) > k\) otherwise the expected result is trivial.
Since \(g(\alpha) = g(\beta)\), the sum-free condition garanties \(g(\alpha) \neq g(\alpha+\beta)\), 
thus \(h(x+y) = k + g(\alpha + \beta) \neq k + g(\alpha) = h(x)\).
\item If \(u+v \geqslant p\), then \(x+y = (\alpha+\beta-1) - (u+v-p) \) with \(u+v-p \in [\![0,p-1]\!] \).
Because of the assumption \(h(x) > k\), we necessarily have \(f(p-u) = k+1 = f(p-v)\). 
Then \(f(p-(u+v-p)) = f(p-u + p-v) \leqslant k \) with the sum-free condition on \(f\), 
hence \(h(x+y) = f(p-(u+v-p)) \) by construction of \(h\). Thus \(h(x+y) \leqslant k < h(x)\).
\end{itemize}

\begin{definition}
A SF-template with \(n\) colors is said to be symmetric if the partition in \(n\) sum-free subsets derived (with the additive constant) from this template is symmetric. 
A sum-free partition \(A_1, ..., A_n\) of \([\![1, p]\!]\) is said to be symmetric if for all \( x \in [\![1, p]\!]\), \(x\) and \(p + 1 - x\) belong to the same subset 
(except if \(x = p + 1 - x\)).
\end{definition}

Using a SAT solver, we exhibited SF-templates, hence providing lower bound on \(S^+\) and inequalities 
of the type \(S(n+k) \geqslant a S(n) + b\). We have sought templates providing the greatest value of 
\((a, b)\) (for the lexicographic order). When the number of colors excceeded 5,in order to reduce the search space we 
looked for symmetric SF-templates, we assumed that the special color was the last color to appear and we constrained 
the \(m_c\)'s out of being too small. Further details about the encoding as a SAT problem can be found in the
\hyperref[SAT]{SAT section}. \\

Here are the best inequalities on Schur numbers so far (the templates corresponding to the third, fourth and fifth 
inequalities can be found in the appendix):
\[ S(n+1) \geqslant 3S(n) + 1 \]
\[ S(n+2) \geqslant 9S(n) + 4 \]
\[ S(n+3) \geqslant 33S(n) + 6 \]
\[ S(n+4) \geqslant 111S(n) + 43 \]
\[ S(n+5) \geqslant 380S(n) + 148 \]
\[ S(n+6) \geqslant 1140S(n) + 528 \]

The first inequality comes from the original Schur's paper \cite{Schur1917}. The second one is due to
Abott \cite{AbbottHanson}
and the third one to Rowley \cite{RowleyRamsey}. The other ones are new. \\

The first 3 inequalities are optimal. The fourth one is optimal among symmetric SF-templates whose special color is 
the last in the order of apparition (and with a multiplicative factor less than or equal to 118). The fifth one is most 
likely not optimal but should not be too far from the optimal SF-template. 
Finally, the sixth one is obtained by combining (see below) the SF-template of length 380 and the one of length 3. Although we 
could not find a better SF-template with 7 colors, this one definetely very far from the optimal value. One may try to seek better 
templates by constraining less the search space and by using a Monte-Carlo method, as in \cite{Bouzy2015AnAP}. 
This could be the suject of a future work.\\

We also have a similar theorem where only \(S^+\) is involved.

\begin{theorem}
	Let \((n,k), (p,q) \in (\mathbb{N}^*)^2\). If there exists a \(SF\)-template of \(k+1\) colors and lenght \(p\),
	and \(SF\)-template of \(n\) color and lenght \(q\), then there exists \(SF\)-template of \((n+k)\) and lenght \(pq\).
\end{theorem}

And the associated inequality :

\begin{corollary}
	Let \(n, k \in \mathbb{N}^*\), we have \\
	\[ S^+(n+k) \geqslant S^+(k+1)S^+(n) \]
\end{corollary}

\textsc{Proof :} The idea is the same as in the previous theorem. The only difference is the \(SF\) property inherited 
from the second \(SF\)-template.



\subsection{New lower bounds for Schur numbers}

The previous inequalities give new lower bounds for \(S(n)\) for
\( n \geqslant 9 \). We compute the lower
bounds for \( n \in [\![8,15]\!] \) using the four different inequalities, please notice that the best values for \( n
=8\) and \(n = 13\) were obtained thanks to the first one, found by Rowley. The best lower bounds are highlighted.\\
\\
\begin{center}
\begin{tabular}{|*{5}{c|}}
    \hline
	\(n\) & 8 & 9 & 10 & 11 \\
	\hline
	\(33S(n-3) + 6 \) & \cellcolor{yellow} 5286 & 17694 & 55446 & 174444\\
	\hline
	\(111S(n-4) + 43 \) & 4927 & \cellcolor{yellow} 17803 & 59539 & 186523\\
	\hline
	\(380S(n-5) + 148 \) & 5088 & 16868 & \cellcolor{yellow} 60948 & \cellcolor{yellow} 203828 \\
	\hline
	\(1140S(n-6) + 528 \) & 5088 & 15348 & 50688 & 182928\\
	\hline
	\hline
	\(n\) & 12 & 13 & 14 & 15 \\
	\hline
	\(33S(n-3) + 6 \) & 587505 & \cellcolor{yellow} 2011290 & 6726330 & 21072090\\
	\hline
	\(111S(n-4) + 43 \) & 586789 & 1976176 & 6765271 & 22624951 \\
	\hline
\(380S(n-5) + 148 \) & \cellcolor{yellow} 638548 & 2008828 & \cellcolor{yellow} 6765288 & \cellcolor{yellow} 23160388 \\\hline
	\(1140S(n-6) + 528 \) & 611568 & 1915728 & 6026568 & 20295948 \\
	\hline
\end{tabular}
\end{center}
Except for 8, 9 and 13, the best lower bounds are obtained thanks to
the third inequality \( S(n+5) \geqslant 380S(n) + 148\). The table
doesn't go any further, but the same inequality allows to improve the
lower bounds for every \( n \geqslant 15 \).

\begin{corollary}
The growth rate for Schur numbers (and Ramsey numbers \(R_n(3)\))  satisfies \(\gamma \geqslant \sqrt[5]{380} \approx 3.28 \).
\end{corollary}
\textsc{Proof :} It is a mere consequence of the inequality \( S(n+5) \geqslant 380S(n) + 148\). As for Ramsey's
numbers growth rate, a lower bound can be found using Schur's one, thanks to \(S(n) \leqslant R_n(3)-2 \)
(see \cite{Schur1917}).


\section{Weak Schur numbers}

In this section, we generalize Rowley's constructions in \cite{RowleyWS}. We then introduce, by analogy with the third
section, the
integer \(WS^+(n)\)
to build suitable templates.

\subsection{Lower bound for Weak Schur numbers using Schur and Weak Schur numbers}

Up to now, there was no equivalent for weak Schur numbers of Abott and Hanson's construction \cite{AbbottHanson}. Here
we
give a general lower bound for weak Schur numbers as a function of both regular and weak Schur numbers.
The following theorem, inspired by Rowley's inequalities for \(WS(n+1)\) and \(WS(n+2)\), was found and proved
by Romain Ageron.

\begin{theorem}
Let \((p,q), (n,k) \in (\mathbb{N}^*)^2\). If there exists a partition of \([\![1,q]\!]\) into \(n\) weakly sum-free
subsets and a partition of \([\![1,p]\!]\) into \(k\) sum-free
subsets then there exists a partition of \([\![1,p(q+\left \lceil \frac{q}{2} \right \rceil + 1)+q]\!]\) into \(n+k\)
weakly sum-free subsets.
\end{theorem}
In particular, if we choose \(q = WS(n)\) and \(p = S(k)\) in the last theorem, the next corollary follows.
\begin{corollary}
\( \forall (n,k) \in (\mathbb{N}^*)^2 \text{, } WS(n+k) \geqslant S(k) \left (WS(n) + \left \lceil \frac{WS(n)}{2}
\right \rceil +1 \right) + WS(n)\)
\end{corollary}
\textsc{Proof :} Let \((p,q), (n,k) \in (\mathbb{N}^*)^2\), \(N = p(q+\left \lceil \frac{q}{2} \right \rceil + 1)+q\),
\(\alpha = \left \lceil \frac{q}{2} \right \rceil > 0\) and \(\beta = q + \alpha + 1\).
We denote by \(f\) the projection of the equivalence relation induced by the partition of \([\![1,q]\!]\) and \(g\) the
one induced by the partition of \([\![1,p]\!]\). Each equivalence class is represented by a single integer, therefore :
\[ f : [\![1,q]\!] \longrightarrow [\![1,n]\!] \text{ and } \forall (x,y) \in [\![1,q]\!]^2, \left\{
\begin{array}{ll}
	x \neq y \\
	f(x) = f(y)
\end{array}
\right.
\Longrightarrow f(x+y) \neq f(x)
\]
\[g : [\![1,p]\!] \longrightarrow [\![1,k]\!] \text{ and } \forall (x,y) \in [\![1,q]\!]^2 \text{, } f(x) = f(y)
\Longrightarrow f(x+y) \neq f(x)
\]
Let us start by parting the integers of \([\![1,N]\!]\) in two subsets \(\mathcal{A}\) and \(\mathcal{B}\) where
\(\mathcal{A} = [\![1,\alpha]\!] \cup \{a\beta + u \mid (a,u) \in [\![0,p]\!] \times [\![\alpha + 1,q]\!]\}\) and
\(\mathcal{B} = \{a\beta + u \mid (a,u) \in [\![1,p]\!] \times [\![-\alpha,\alpha]\!]\}\).\\
\\
First, \underline{\(\mathcal{A} \cap \mathcal{B} = \varnothing\)} : \\
By contradiction, suppose there exists \(x \in \mathcal{A} \cap \mathcal{B} \neq \varnothing \). Then there are \((a,u)
\in [\![0,p]\!] \times [\![\alpha + 1,q]\!]\) and \((b,v) \in [\![1,p]\!] \times [\![-\alpha,\alpha]\!]\) such that \(x
= a\beta + u = b\beta +v\). By definition of \(\alpha\) and \(\beta\) we have \(u \in [\![\alpha + 1,q]\!] \subset
[\![0,\beta - 1]\!]\).
From there, we distinguish two cases :
\begin{itemize}
\item If \(v \in [\![0,\alpha]\!]\) then \(v \in [\![0,\beta - 1]\!]\) and \(v \neq u\) because \(v < \alpha + 1
\leqslant u\)
\item If \(v \in [\![-\alpha,-1]\!]\), we note \(\tilde{v} = \beta + v\) and thus have \(x = (b-1)\beta + \tilde{v}\)
with \(\tilde{v} \in [\![\beta - \alpha,\beta - 1]\!] \subset [\![0,\beta - 1]\!]\) and \(\tilde{v} \neq u\) because
\(u< q+1 = \beta - \alpha \leqslant \tilde{v}\).
\end{itemize}
In either cases, we run into a contradiction because of the remainder's uniqueness in the euclidean division of \(x\)
by\(\beta\).\\
\\
Then, we have \underline{\(\mathcal{A} \cup \mathcal{B} = [\![1,N]\!]\)}:
\begin{itemize}
	\item On the one hand : \(1 = \text{min}(\mathcal{A}) \leqslant \text{max}(\mathcal{A}) = p\beta + q = N\) and
\(1 \leqslant \beta - \alpha = \text{min}(\mathcal{B}) \leqslant \text{max}(\mathcal{B}) = p\beta + \alpha \leqslant
N\),
	which gives \(\mathcal{A} \cup \mathcal{B} \subset [\![1,N]\!]\).
\item On the other hand, let \(x \in [\![1,N]\!]\). If \(x \leqslant \alpha\), we directly have \(x \in \mathcal{A}\),
let us then suppose that \(x > \alpha\) and write \(x = a\beta + u\)
the euclidean division of \(x\) by \(\beta\). We have \(x \leqslant N\), thus \(a \leqslant p\). We distinguish three
cases : \\
- If \(u \in [\![0,\alpha]\!]\) then we necessarily have \(a \geqslant 1\) because \(x > \alpha\), and so \(x \in
\mathcal{B}\).\\
	- If \(u \in [\![\alpha + 1,q]\!]\), then \(x \in \mathcal{A}\). \\
- If \(u \in [\![q + 1,\beta - 1]\!]\) then \(x = (a+1)\beta - (\beta - u)\) with \(-\alpha \leqslant \beta - u
\leqslant 0\).
	Furthermore, \(a \leqslant p - 1\), else we would have \(x > N\), and so \(x \in \mathcal{B}\) \\
In any case, \(x \in \mathcal{A} \cup \mathcal{B}\) and we can thus conclude that \([\![1,N]\!] \subset \mathcal{A}
\cup\mathcal{B}\).
\end{itemize}
This first partition of \([\![1,N]\!]\) will help us to define our final partition by the projection of its equivalence
relation.
We thereby define \(h : [\![1,N]\!] \longrightarrow [\![1,n+k]\!]\) as such :\\
- If \(x \in \mathcal{A}\) then \(h(x) = f(x \text{ mod } \beta)\) (well defined because \(x \text{ mod } \beta \in
[\![1,N]\!]\))\\
- If \(x \in \mathcal{B}\) then \(x = a\beta + u\) with a unique \((a,u) \in [\![1,p]\!] \times
[\![-\alpha,\alpha]\!]\)and we define \(h(x) = n + g(a)\)\\
The fact that \((\mathcal{A}, \mathcal{B})\) is a partition of \([\![1,N]\!]\) ensures that this definition of \(h\) is
valid. We then have to verify that \(h\) induces weakly sum-free subsets.\\
\\
\underline{The classes of equivalence \(h(x)\) for \(x \in \mathcal{A}\) are weakly sum-free :}
\\
\\
Let \((x,y) \in \mathcal{A}^2\) such that \(h(x) = h(y)\), \(x \neq y\) and \(x + y \leqslant N\)
\begin{itemize}
	\item If \((x,y) \in [\![1,\alpha]\!]^2\) :\\
	We have \(x + y \leqslant 2\alpha \leqslant q\) and \(x + y = 0\beta + x + y\), therefore \(x + y \in \mathcal{A}\).
Then, by definition : \(h(x) = f(x)\), \(h(y) = f(y)\) and \(h(x+y) = f(x+y)\), which gives us, thanks to the property
verified by \(f\), that \(h(x+y) \neq h(x)\).
	\item If \((x,y) \in [\![1,\alpha]\!] \times ( \mathcal{A} \text{ \textbackslash} ~ [\![1,\alpha]\!] )\) :\\
We write \(y = a\beta + u\) with \((a,u) \in [\![0,p]\!] \times [\![\alpha + 1,q]\!]\). Then \(x+y = a\beta + x + u =
(a+1)\beta + x + u - \beta\),
and if \(x + u > q\) it follows that \(a \leqslant p-1\) since \(x+y \leqslant N\), and \(-\alpha \leqslant x + u -
\beta \leqslant -1\).
Therefore \(x+y \in \mathcal{B}\) and \(h(x+y) \neq h(x) = f(x)\) by definition of h. On the contrary, if \(x - u
\leqslant n\),
then \(x+y \in \mathcal{A}\) and \(h(x+y) = f(x+u)\) because \(x+u\) is actually the remainder of the euclidean
divisionof \(x+y\) by \(\beta\).
Moreover, \(h(x) = f(x)\), \(x < u\) and, with our initial hypothesis, \(h(x) = h(y) = f(u)\). The property verified by
\(f\) gives us \(f(x+u) \neq f(x)\) which can be rewritten as \(h(x+y) \neq h(x)\).
	\item If \((x,y) \in ( \mathcal{A} \text{ \textbackslash} ~ [\![1,\alpha]\!] ) \times [\![1,\alpha]\!]\) : \\
	This case is handled exactly like the previous one by swaping the roles of \(x\) and \(y\).
	\item If \((x,y) \in ( \mathcal{A} \text{ \textbackslash} ~ [\![1,\alpha]\!] )^2\) : \\
We write \(x = a\beta + u\) and \(y = b\beta + v\) with \((a,u)\) and \((b,v)\) in \([\![0,p]\!] \times [\![\alpha +
1,q]\!]\). Then \(x+y = (a+b)\beta + u+v = (a+b+1)\beta + u + v - \beta\)
with \(a+b \leqslant p-1\) (else we would have \(x+y > N\) because \(u+v > q\)) and \(-\alpha \leqslant u + v - \beta
\leqslant \alpha\), therefore \(x+y \in \mathcal{B}\) and by definition \(h(x+y) \neq h(x)\).
\end{itemize}
In any case, \(h(x+y) \neq h(x)\) and the classes of equivalence \(h(x)\) for \(x \in \mathcal{A}\) are weakly
sum-free.\\
\\
\underline{The classes of equivalence \(h(x)\) for \(x \in \mathcal{B}\) are weakly sum-free :}
\\
\\
Let \((x,y) \in \mathcal{B}^2\) such that \(h(x) = h(y)\), \(x \neq y\) and \(x + y \leqslant N\).\\
We write \(x = a\beta + u\) and \(y = b\beta + v\) with \((a,u)\) and \((b,v)\) in \([\![1,p]\!] \times
[\![-\alpha,\alpha]\!]\).
We have \(h(x) = q + g(a)\) and \(h(y) = q + g(b)\), therefore \(g(a) = g(b)\). We also have \(x+y = (a+b)\beta + u
+v\).\\
If \(u + v \in [\![-\alpha,\alpha]\!]\), then \(x+y \in \mathcal{B}\) and \(h(x+y) = g(a+b)\), hence we can deduce that
\(h(x+y) \neq h(x)\) because of the property verified by \(g\). On the contrary, if \(u+v \notin
[\![-\alpha,\alpha]\!]\), then necessarily \(x+y \in \mathcal{A}\). Suppose \(x+y \in \mathcal{B}\), then \(x+y =
c\beta+ w\) with \((c,w) \in [\![1,p]\!] \times [\![-\alpha,\alpha]\!]\). Thus, \(c\beta + w = (a+b)\beta + u + v\) and
\((a+b-c)\beta = w-u-v\). Furthermore \(a+b-c \neq 0\), else we would have \(u+v = w \in [\![-\alpha,\alpha]\!]\). This
finally leads to the following inequality :
\[\beta \leqslant |a+b-c|\beta = |w-u-v| \leqslant |w| + |u| + |v| \leqslant 3\alpha \leqslant q + \alpha < \beta
\]
which is absurd. We can therefore conclude that \(x+y \in \mathcal{A}\) and by definition of \(h\), \(h(x+y) \neq
h(x)\), proving that the classes of equivalence \(h(x)\) for \(x \in \mathcal{B}\) are weakly sum-free.\\
\\
Finally, we have showed that every classe of equivalence induced by \(h\) is weakly sum-free, which ends the proof.

\begin{remark}
This formula includes the results of Rowley \cite{RowleyWS} as a special case. For \(n>2\),
this formula does not give new lower bounds but in the same way as we introduced \(S^+\)\hyperref[SE]{(Definition
3.1)},we will define \(WS^+\) and find inequalities between \(WS^+\),\(WS\) and \(S\)
\end{remark}

\subsection{Definition of \(WS^+\)}

We will now define notations and results, we will use in the following theorem. 

\begin{notation}

Let (a,b) \(n \in (\mathbb{N}^*)^2\),\(a>b\), we will define \(\pi_{a,b}\) the projection:
\[ \pi_{a,b}:x->(Id+a\mathbf{1}_{ [\![0,b]\!]})(\text{x mod a}))\]
\end{notation}
 
We will note  the projection \(\pi\) and not \(\pi_{a,b}\) when there is no doubt about the a and b we use.

\begin{proposition}
Let x \(\in [\![1,b]\!]\), let y \(\in \mathbb{N}^*\) such that \(x+\pi(y)\leqslant a+b\), then we have: \(\pi(x+y)=x+\pi(y)\)
\end{proposition}

\textsc{Proof :}Let x \(\in [\![1,b]\!]\), let y \(\in \mathbb{N}^*\) such that \(x+\pi(y)\leqslant a+b\)

if \(x+\pi(y)< a:\)we remark that \(\pi(y)>b\) and therefore \(x+\pi(y)>b\):

\begin{align*}
 \pi(x+y) & = (Id+a\mathbf{1}_{ [\![0,b]\!]})(\text{x+y mod a})\\
& = (Id+a\mathbf{1}_{ [\![0,b]\!]})(x+\pi(y)\text{ mod a) since } \pi(y)\text{=y mod a} \\
& =x+ \pi(y)
\end{align*}

if \(x+\pi(y\geqslant a\):
\begin{align*}
 \pi(x+y) & = (Id+a\mathbf{1}_{ [\![0,b]\!]})(\text{x+y mod a})\\
& = (Id+a\mathbf{1}_{ [\![0,b]\!]})(x+\pi(y)\text{ mod a) since } \pi(y)\text{=y mod a} \\
& = (Id+a\mathbf{1}_{ [\![0,b]\!]})(x+\pi(y) - a) \\
& = x+\pi(y) - a +a\mathbf{1}_{ [\![0,b]\!]})(x+\pi(y) - a) \\
& = x+\pi(y) - a +a \text{ \quad since } x+\pi(y) - a \in [\![0,b]\!] \\
& = x+\pi(y)
\end{align*}


\begin{proposition}
Let (x,y)\(\in (\mathbb{N}^*)^2\), \(\pi(\pi(x)+\pi(y))=\pi(x+y)\)
\end{proposition}

\textsc{Proof :}Let (x,y)\(\in (\mathbb{N}^*)^2\),

\begin{align*}
 \\\pi(\pi(x)+\pi(y)) & = (Id+a\mathbf{1}_{ [\![0,b]\!]})(\pi(x)+\pi(y)\text{ mod a})\\
& = (Id+a\mathbf{1}_{ [\![0,b]\!]})((Id+a\mathbf{1}_{ [\![0,b]\!]})(\text{x mod a})+(Id+a\mathbf{1}_{ [\![0,b]\!]})(\text{y mod a})\text{ mod a}) \\
& = (Id+a\mathbf{1}_{ [\![0,b]\!]})(\text{ ((x mod a) + (y mod a)) mod a})\\
& =(Id+a\mathbf{1}_{ [\![0,b]\!]})(\text{ x+y mod a})\\
& =\pi(x+y)
\end{align*}


\begin{notation}

Let (a,b) \(n \in (\mathbb{N}^*)^2\),\(a>b\), we will define \(\lambda_{a,b}\) the projection:
\[ \lambda_{a,b}:x->1+ \left\lfloor\dfrac{x-b-1}{a}\right\rfloor\]
\end{notation}

We will note  the projection \(\lambda\) and not \(\lambda_{a,b}\) when there is no doubt about the a and b we use.

\begin{remark}
In the following theorem, \(\lambda\) is the function which return the line number of an element x.
\end{remark}

\begin{proposition}
Let (a,b)\(\in (\mathbb{N}^*)^2\), \(a>b\), let x \(\in \mathbb{N}^*\), \(x=a\lambda(x)+\pi(x)-a\)
\end{proposition}
\textsc{Proof :}Let (a,b)\(\in (\mathbb{N}^*)^2\), \(a>b\), let x \(\in \mathbb{N}^*\),
\\\\ \(a\lambda(x)+\pi(x)-a=a\left\lfloor\dfrac{x-b-1}{a}\right\rfloor+(\text{x mod a})+\mathbf{1}_{ [\![0,b]\!]}(\text{x mod a})\)

if x mod a \(>b\): \(a\lambda(x)+\pi(x)-a=a\left\lfloor\dfrac{x}{a}\right\rfloor+\text{x mod a}=x\)

if x mod a \(\leqslant b\):
\(a\lambda(x)+\pi(x)-a=a(\left\lfloor\dfrac{x}{a}\right\rfloor-1)+\text{x mod a}+a=x\)
\begin{definition}
Let \( (p,k,b) \in (\mathbb{N}^*)^3\), Let \((A_1,....,A_k)\) a partition of  \([\![1, p]\!]\).
This partition is said to be a b-weakly-sum-free template (b-WSF-template) of \(k\) colors and lenght \(p\) when:
\\\\
\underline{\(\forall i \in [\![1, k]\!], \quad A_i\) is weakly-sum-free}
\\\\
\underline{\(\forall i \in [\![1, k]\!], \quad A_i\backslash [\![1, b]\!]\) is sum-free}
\\\\
\underline{For \(A_k\) (the special subset):} \quad \(\forall (x,y) \in A_k^2,\)
\\
\[if \quad x+y>b+2(p-b),\quad x+y-2(p-b)\notin A_k\]
\\
\underline{For the others subsets:}\quad \(\forall i \in [\![0,k-1]\!], \forall(x,y) \in A_i^2\)
\\
\[
(Id+(p-b)\mathbf{1}_{ [\![0,b]\!]})(x+y ~\text{mod}~ (p-b)) \notin A_i
\]
\end{definition}

\begin{definition}
Let \( (k,b) \in (\mathbb{N}^*)^2\). If there exist \(p\) such that exists a partition of \([\![1, p]\!]\) into \(k\)
subsets which is a \(b\)-WS-template of \(k\) colors and lenght \(p\), we note:
\\\\\(WS_b^+=-b+\max \{p\in \mathbb{N}^*\)/ there exists a partition of \([\![1, p]\!]\) into \(k\) subsets which is a
\(b\)-WSF-template of \(k\) colors and lenght \(p\) \}
\\\\
If this \(p\) does not exist, we set \(WS_b^+= 0\)
\end{definition}

\begin{definition}
Let \( n \in \mathbb{N}^*\), we define \(WS^+(n)\)=\(\max_{b\in \mathbb{N}^*} \{WS_b^+(n)\}\)
\end{definition}


\subsection{Lower bound for Weak schur numbers using Schur and Weak Schur template numbers}


\begin{theorem}
Let \((q,n,b) \in (\mathbb{N}^*)^3\), let \( (p,k) \in (\mathbb{N}^*)^2\). If there exists a partition of \(k\)
sum-free subsets of \([\![1,p]\!]\) and a partition of n subsets \((A_1,....,A_n)\) of \([\![1, q]\!]\) which is a
\(b\)-WSF of \(n\) colors and lenght \(q\),
 then there exists a partition of \([\![1, b+p \times (q-b)]\!]\) into \((k+n-1)\) weakly sum-free subsets.
\end{theorem}

In particular, if we choose \(p = S(k)\) and \(q = WS^+(n)\) in the last theorem, the next corollary follows.


\begin{corollary}
\( \forall (n,k) \in (\mathbb{N}^*)^2 \text{, } let ~ b_{max}=max \{b\in \mathbb{N}^*/WS^+(n+1)=WS_b^+(n+1)\},\) \[
WS(n+k) \geqslant S(k) WS^+(n+1)+b_{max}\]
\end{corollary}

\renewcommand{\arraystretch}{2}
\begin{center}
\[
\begin{array}{|c|c|c|c|c|c|c|c|c|>{\columncolor{blue}}c|>{\columncolor{blue}}c|>{\columncolor{blue}}c|>{\columncolor{green}} c|}
\cline{6-9}
   \multicolumn{5}{c|}{} & \cellcolor{green}1 & \cellcolor{green}2 & \cellcolor{blue} 3&\cellcolor{green}4&5&6&7&8 \\
\hline
   \cellcolor{red}9 & \cellcolor{red}10 & \cellcolor{red}11 & \cellcolor{red}12 &\cellcolor{red}13&\cellcolor{red}14&\cellcolor{red}15&\cellcolor{red}16&\cellcolor{red}17&18&19&20&21\\
\hline
\cellcolor{yellow}22&\cellcolor{yellow}23&\cellcolor{yellow}24&\cellcolor{yellow}25&\cellcolor{yellow}26&\cellcolor{yellow}27&\cellcolor{yellow}28&\cellcolor{yellow}29&\cellcolor{yellow}30&31&32&33&34\\
\hline
...\cellcolor{yellow}&...\cellcolor{yellow}&...\cellcolor{yellow}&...\cellcolor{yellow}&...\cellcolor{yellow}&...\cellcolor{yellow}&...\cellcolor{yellow}&...\cellcolor{yellow}&...\cellcolor{yellow}&...&...&...&...\\
\hline
\cellcolor{red}...&\cellcolor{red}...&\cellcolor{red}...&\cellcolor{red}...&\cellcolor{red}...&\cellcolor{red}...&\cellcolor{red}...&\cellcolor{red}...&\cellcolor{red}...&...&...&...&...\\
\hline
...&...&...&...&...&...&...&...&...&...&...&...&...\\
\hline
\end{array}
\]
\end{center}
\textsc{Proof :}Let \((q,n,b) \in (\mathbb{N}^*)^3\), let \( (p,k) \in (\mathbb{N}^*)^2\), let a=q-b. \\
We denote by \(g\) the projection of the equivalence relation induced by the partition of \([\![1,q]\!]\) and \(h\) the
one induced by the partition of \([\![1,p]\!]\). Each equivalence class is represented by a single integer, therefore :
\[ g : [\![1,q]\!] \longrightarrow [\![1,n]\!] \text{ and } (A_{g^{-1}(1)},...A_{g^{-1}(q)})\text{ is a b-WSF-template.}
\]
\[h : [\![1,p]\!] \longrightarrow [\![1,k]\!] \text{ and } \forall (x,y) \in [\![1,q]\!]^2 \text{, } h(x) = h(y)
\Longrightarrow h(x+y) \neq h(x)
\]

Let \( f : [\![1,b+pa]\!] \longrightarrow [\![1,n]\!] \) such as:
\\\\-if x \(\leqslant b \text{ (we will note x} \in \mathcal{T}): f(x)=g(x)\)
\\\\-if \( x \in [\![1,b+pa]\!] \text{ and } \pi(x) \notin A_n \text{ (we will note x} \in \mathcal{C}): f(x)=g(\pi(x))\)
\\\\-if \( x \in [\![1,b+pa]\!]  \text{ and }\pi(x) \in A_n \text{ (we will note x} \in \mathcal{L}): f(x)=n-1+h(\lambda(x))\)
\\\\
f is well defined because \(\pi\) is defined for x\(>b\) and \(\forall x \in \![1,b+pa]\!], f(x)\leqslant n+k-1\) because h(\(\lambda(x))\leqslant k\)

We have parted the integers of \(\![1,b+pa]\!]\) in three disjoints subsets \(\mathcal{T},\mathcal{C} \text{ and } \mathcal{L}\).

We have to verify that f induces weakly-sum-free templates:\\
\\
\underline{if (x,y) \(\in (\mathcal{T})^2\) such that f(x)=f(y), \(x \neq y\), then \(f(x+y)\neq f(x)\)  :}\\
\\x+y\(<\)a+b since b\(<\)a and g(x)=f(x)=f(y)=g(y). 
\\Hence f(x+y)=g(x+y)\(\neq\)g(x)=f(x)
\\\\
\underline{if (x,y) \(\in \mathcal{T} \times \mathcal{C}\) such that f(x)=f(y), \(x \neq y\), then \(f(x+y)\neq f(x)\)  :}\\
We distinguish two cases:


\begin{itemize}
\item If x+\(\pi(y)\leqslant a+b\)
\\g(x)=f(x)=f(y)=g(\(\pi(y)\). Hence g(x)\(\neq g(x+\pi(y))=g(\pi(x+y))\) (qv previous proposition)
\\if \(g(\pi(x+y))=n, f(x+y)\geqslant n > f(x)\)
\\else, \(f(x+y)=g(\pi(x+y))\neq g(x)=f(x)\)
\item If x+\(\pi(y)> a+b\), x+y\(>\)a+b and by definition of g, \(g(\pi(x+y))\neq g(x)\)
\\if \(g(\pi(x+y))=n, f(x+y)\geqslant n > f(x)\)
\\else, \(f(x+y)=g(\pi(x+y))\neq g(\pi(x))=f(x)\)
\end{itemize}


\underline{if (x,y) \(\in \mathcal{T} \times \mathcal{L}\) such that f(x)=f(y), \(x \neq y\), then \(f(x+y)\neq f(x)\)  :}\\
\\Then, f(x)=f(y)=n. We distinguish two cases:


\begin{itemize}
\item If \(\lambda(y)=\lambda(x+y)\),
\\\(g(x)=g(\pi(y))=n \text{ since } g(y)=g(\pi(y))\)
\\Therefore \(g(\pi(x+y))=g(x+\pi(y)) \neq g(x)=n\)(qv previous proposition)
\\Hence \(f(x+y)=g(\pi(x+y))\neq n\)
\item If  \(\lambda(y)\neq \lambda(x+y)\), \(\lambda(y)+1= \lambda(x+y)\)\\
n=f(y)=n-1+h(\(\lambda(y)\)). Hence h(\(\lambda(y)\))=1.
\\Moreover h(1)=1, therefore \(h(\lambda(y)+1) \neq 1\)
\\if \(\pi(x+y) \in A_n\), \(f(x+y)=n-1+h(\lambda(x+y))>n\)
\end{itemize}

\underline{if (x,y) \(\in (\mathcal{C})^2\) such that f(x)=f(y), \(x \neq y\), then \(f(x+y)\neq f(x)\)  :}\\
\\Then  \(g(\pi(x))=f(x)=f(y)=g(\pi(y))\). We distinguish two cases:

\begin{itemize}
\item If \(\pi(x)+\pi(y)>q\), \(g(\pi(\pi(x)+\pi(y)) \neq g(\pi(x))\) (qv previous proposition)
\\Hence \(g(\pi(x+y))= g(\pi(\pi(x)+\pi(y))\neq g(\pi(x))\) 
\\if \(g(\pi(x+y))=n, f(x+y)\geqslant n > f(x)\)
\\else, \(f(x+y)=g(\pi(x+y))\neq g(\pi(x))=f(x)\)
\item If  \(\pi(x)+\pi(y)\leqslant q\), \(g(\pi(\pi(x)+\pi(y)) \neq g(\pi(x))\) since g is sum-free for x\(>b\)
\\if \(g(\pi(x+y))=n, f(x+y)\geqslant n > f(x)\)
\\else, \(f(x+y)=g(\pi(x+y))=g(\pi(\pi(x)+\pi(y))\neq g(\pi(x))=f(x)\)


\end{itemize}



\underline{if (x,y) \(\in \mathcal{C} \times \mathcal{L} \), \(f(x)\neq f(y)\)}\\

\underline{if (x,y) \(\in (\mathcal{L})^2\) such that f(x)=f(y), \(x \neq y\), then \(f(x+y)\neq f(x)\)  :}\\
\\Let r(x)=\(\pi(x)-a\) and r(y)=\(\pi(y)-a\),
\\We proved that \(x=a\lambda(x)+\pi(x)-a\), therefore \(x=a\lambda(x)+\pi(x)\)
\\\(x+y=a(\lambda(x)+\lambda(y))+r(x)+r(y)\). We distinguish three cases:

\begin{itemize}
\item If \(r(x)+r(y) \in [\![b-a+1,b]\!]\), \(h(\lambda(x))=f(x)+1-n=f(y)+1-n=h(\lambda(y))\)
Hence, \(h(\lambda(x)+\lambda(y)) \neq h(\lambda(x))\).
\begin{align*}
 \\\lambda(x+y) & =1+\left\lfloor\dfrac{a(\lambda(x)+\lambda(y))+r(x)+r(y)-b-1}{a}\right\rfloor+1\\
& = \lambda(x)+\lambda(y)+\left\lfloor\dfrac{r(x)+r(y)-b-1}{a}\right\rfloor+1 \\
& = \lambda(x)+\lambda(y) -1 +1 \text{ since } r(x)+r(y) \in [\![b-a+1,b]\!] \\
& =\lambda(x)+\lambda(y)
\end{align*}


\begin{align*}
 \\\text{if f(x+y)\(\geqslant\)n, }f(x+y) & =n-1+h(\lambda(x+y))\\
& =n-1+h(\lambda(x)+\lambda(y)) \\
& \neq n-1+h(\lambda(x))\\
& =f(x)
\end{align*}


\item If \(r(x)+r(y)>b\), \(\pi(x)+\pi(y)>2a+b\)
\\Since g is a b-WSF template, \(g(\pi(\pi(x)+\pi(y))) \neq n\)
\\Therefore, \(g(\pi(x+y)) \neq n\) ie \(x+y \in \mathcal{C}\)
\\Hence \(f(x+y) <n\leqslant f(x)\)
\item If \(r(x)+r(y)\leqslant b-a,\), \(\pi(x)+\pi(y)\leqslant b+a\)
\\Since g is sum-free for x\(>b\), since \(g(\pi(x)) = g(\pi(y))=n\), \(g(\pi(x+y)) \neq g(\pi(x))=n\)
\\Hence, \(f(x+y) <n\leqslant f(x)\)
\end{itemize}

\begin{remark}
There exists a \(c(b) \geqslant \text{min}(A_{k+1} \backslash [\![1,b] - b - 1\) such that \(c(b)\) numbers can be
added at the end of the extended partition.Therefore, we can get a better lower bound by finding a couple \((b,c(b))\)
which maximizes the sum \(b+c(b)\) such that \(WS^+(k+1)=WS_b^+(k+1)\). Hence we have:
	\[ WS(n+k) \geqslant S(n) WS^+(k+1)+b+c(b)\]

\end{remark}


\subsection{New lower bounds for Weak Schur numbers}
Having found suitable templates, which can be found in the appendix, with a SAT solver, we claim that for all \(n \in
\mathbb{N}^*\):
\[
4S(n) + 1 \leqslant WS(n+1)
\]
\[
13S(n) + 8 \leqslant WS(n+2)
\]
\[
42S(n) + 24 \leqslant WS(n+3)
\]
\[
127S(n) + 68 \leqslant WS(n+4)
\]
The first two inequalities are due to Rowley, they are detailed in [2]. Like in 3.3, we compute the lower bounds given
by the previous inequalities for \( n \in [\![8,15]\!] \). The best lower bound for each integer is highlighted.\\
\\
\begin{center}
\begin{tabular}{|*{5}{c|}}
    \hline
	n & 8 & 9 & 10 & 11 \\
	\hline
	\(4S(n-1) + 2 \) & 6722 & 21146 & \cellcolor{yellow} 71214 & \cellcolor{yellow} 243794\\
	\hline
	\(13S(n-2) + 8 \) & \cellcolor{yellow} 6976 & 21848 & 68726 & 231447\\
	\hline
	\(42S(n-3) + 24 \) & 6744 & \cellcolor{yellow} 22536 & 70584 & 222036 \\
	\hline
	\(127S(n-4) + 68 \) & 5656 & 20388 & 68140 & 213428\\
	\hline
	\hline
	n & 12 & 13 & 14 & 15 \\
	\hline
	\(4S(n-1) + 2 \) & \cellcolor{yellow} 815314 & 2554194 & 8045162 & \cellcolor{yellow} 27061154\\
	\hline
	\(13S(n-2) + 8 \) & 792332 & \cellcolor{yellow}2649772 & 8301132 & 26146778 \\
	\hline
	\(42S(n-3) + 24 \) & 747750 & 2559840 & \cellcolor{yellow} 8560800 &  25886224 \\
	\hline
	\(127S(n-4) + 68 \) & 671390 & 2261049 & 7740464 & 25886224 \\
	\hline
\end{tabular}
\end{center}

With \( S(9) \geqslant 17803 \), we found a new lower bound for
\(WS(10)\) using Rowley's inequality. Moreover, the third inequality
gives new lower bounds for \(WS(9)\) and \(WS(14)\). However, the last
inequality doesn't give any better lower bound, even beyond \( n = 15 \)
: the best bounds are always provided by the first three. We highly suspect that these values can be improved by
investigating the search
space further, which would provide new, more effective templates. One
may try to go over this search space using a Monte-Carlo method, as in \cite{Bouzy2015AnAP},
but with a different search space (as we explain in \hyperref[SAT]{SAT section}).
This could be the suject of a future work.


\section{Conclusion and future work}

We have produced new templates for Schur and Ramsey numbers. We also have generalized the concept of template to weak Schur numbers. 
These templates provide new general inequalities of the form \(S(n+k) \geqslant a S(n) + b\) and \(\WS(n+k) \geqslant a S(n) + b\) 
as well as new lower bounds for both Schur and weak Schur numbers. We have given a new lower bound \(\WS(6) \geqslant 646\) with 
a method that can yield slight improvements for fixed values of \(n\) over lower bounds obtained with the inequality 
\(\WS(n+1) \geqslant 4 S(n) + 2\).

Given that the introduction of templates is quite recent, we expect the bounds to be improved as this special type of partition 
becomes better understood and larger templates are found. One may try to produce better templates by using heuristics or 
approximation algorithms, such as Monte-Carlo algorithms for instance.

The best lower bound \(\WS(6) \geqslant 583\) achieved with a computer search using Monte-Carlo methods 
\cite{EliahouBook} is significantly lower than those obtained with a template-like structure: \(\WS(6) \geqslant 642\) \cite{RowleyWS}, 
\(\WS(6) \geqslant 646\). We have evidence that 583 is the maximal value in the search space considered by 
\cite{EliahouBook,Bouzy2015AnAP,Rafilipojaona}. Therefore, it questions the assumption that large partitions for \(\WS(n+1)\) 
start with large partitions for \(\WS(n)\). It also indicates that \(\WS(5)\) might need further investigation since the current lower 
bound \(\WS(5) \geqslant 196\) \cite{ELIAHOU2012175} was obtained by considering the same type of search space.


\section{Acknowledgments}

This study results from a long-term project performed within the framework of the Project Cluster "Training for Research" of CentraleSupélec. 
We would like to address warm thanks to Fred Rowley who kindly answered our many questions
and with who it was a pleasure to exchange. Last but not least, we would like to thank Marijn Heule for providing the backdoors
used in the computation of the new lower bound for \(\WS(6)\).


\bibliography{biblio}
\bibliographystyle{siamplain}

\appendix


\section{S-templates}
\label{S-templates}

\renewcommand{\arraystretch}{1}
\vspace{-1.5ex}

\begin{table}[H]
\[
\begin{array}{|*{2}{c|}}
	\hline
	1 & 1, 6, 9, 13, 16, 20, 24, 27, 31 \\
	\hline
	2 & 2, 5, 14, 15, 25, 26 \\
	\hline
	3 & 3, 4, 10, 11, 12, 28, 29, 30 \\
	\hline
	4 & 7, 8, 17, 18, 19, 21, 22, 23, 32, 33 \\
	\hline
\end{array}
\]
\vspace{-3ex}
\caption{S-template with width 33 and 4 colors}
\vspace{-2ex}
\end{table}

\begin{table}[H]
\[
\begin{array}{|*{2}{c|}}
	\hline
	1 & 1, 5, 18, 12, 14 ,21, 23, 30, 32, 36, 39, 43, 45, 52, 103 \\
	 & 106, 110 \\
	\hline
	2 & 2, 6, 7, 10, 15, 18, 26, 29, 34, 37, 38, 42, 46, 51, 54 \\
	& 101, 104, 109 \\
	\hline
	3 & 3, 4, 9, 11, 17, 19, 25, 27, 33, 35, 40, 41, 47, 48, 55 \\
	& 100, 107, 108 \\
	\hline
	4 & 13, 16, 20, 22, 24, 28, 31, 58, 61, 67, 88, 94, 97 \\
	\hline
	5 & 44, 50, 53, 56, 57, 59, 60, 62, 63, 64, 65, 66, 68, 69, 70\\
	& 71, 72, 73, 74, 75, 76, 77, 78, 79, 80, 81, 82, 83, 84, 85\\
	& 86, 87, 89, 90, 91, 92, 93, 95, 96, 98, 99, 102, 105, 111 \\
	\hline
\end{array}
\]
\vspace{-3ex}
\caption{S-template with width 111 and 5 colors}
\vspace{-2ex}
\end{table}

\begin{table}[H]
\[
\begin{array}{|*{2}{c|}}
	\hline
	1 & 1, 5, 8, 11, 15, 17, 29, 33, 36, 39, 43, 57, 61, 88, 92 \\
	& 106, 110, 113, 116, 120, 132, 134, 138, 141, 144, 148, 150, 154, 157, 160 \\
	& 164, 178, 182, 185, 188, 341, 344, 347, 351, 365, 369, 372, 375, 379\\
	\hline
	2 & 2, 9, 13, 16, 20, 23, 24, 27, 28, 31, 34, 35, 38, 42, 45\\
	& 49, 53, 60, 67, 71, 78, 82, 89, 96, 100, 104, 107, 111, 114, 115\\
	& 118, 121, 122, 125, 126, 129, 133, 136, 140, 147, 158, 162, 165, 169, 172\\
	& 176, 183, 187, 194, 201, 328, 335, 342, 346, 353, 357, 360, 364, 367, 371\\
	\hline
	3 & 3, 4, 12, 14, 19, 25, 30, 32, 40, 41, 47, 48, 58, 91, 101\\
	& 102, 108, 109, 117, 119, 124, 130, 135, 137, 145, 146, 152, 153, 161, 163\\
	& 168, 179, 181, 190, 339, 348, 350, 361, 366, 368, 376, 377\\
	\hline
	4 & 6, 7, 10, 18, 21, 22, 26, 37, 46, 50, 51, 54, 65, 70, 79\\
	& 84, 95, 98, 99, 103, 112, 123, 127, 128, 131, 139, 142, 143, 151, 155\\
	& 156, 159, 167, 170, 171, 175, 186, 343, 354, 358, 359, 362, 370, 373, 374\\
	& 378\\
	\hline
	5 & 44, 52, 55, 56, 59, 62, 63, 64, 66, 68, 69, 72, 73, 74, 75\\
	& 76, 77, 80, 81, 83, 85, 86, 87, 90, 93, 94, 97, 105, 189, 196\\
	& 197, 200, 203, 206, 207, 209, 214, 219, 231, 298, 310, 315, 320, 322, 323\\
	& 326, 329, 332, 333, 340\\
	\hline
	6 & 149, 166, 173, 174, 177, 180, 184, 191, 192, 193, 195, 198, 199, 202, 204\\
	& 205, 208, 210, 211, 212, 213, 215, 216, 217, 218, 220, 221, 222, 223, 224\\
	& 225, 226, 227, 228, 229, 230, 232, 233, 234, 235, 236, 237, 238, 239, 240\\
	& 241, 242, 243, 244, 245, 246, 247, 248, 249, 250, 251, 252, 253, 254, 255\\
	& 256, 257, 258, 259, 260, 261, 262, 263, 264, 265, 266, 267, 268, 269, 270\\
	& 271, 272, 273, 274, 275, 276, 277, 278, 279, 280, 281, 282, 283, 284, 285\\
	& 286, 287, 288, 289, 290, 291, 292, 293, 294, 295, 296, 297, 299, 300, 301\\
	& 302, 303, 304, 305, 306, 307, 308, 309, 311, 312, 313, 314, 316, 317, 318\\
	& 319, 321, 324, 325, 327, 330, 331, 334, 336, 337, 338, 345, 349, 352, 355\\
	& 356, 363, 380\\
	\hline
\end{array}
\]
\vspace{-3ex}
\caption{S-template with width 380 and 6 colors}
\vspace{-2ex}
\end{table}

\section{WS-templates}
\label{WS-templates}

\vspace{-1.5ex}

\begin{table}[H]
\[
\begin{array}{|*{2}{c|}}
	\hline
	1 & 1, 2, 4, 8, 11, 22, 25, \mathbf{(N+1)}\\
	\hline
	2 & 5, 6, 7, 19, 21, 23, 36\\
	\hline
	3 & 9, 10, 12, 13, 14, 15, 16, 17, 18, 20\\
	\hline
	4 & 24, 26, 27, 28, 29, 30, 31, 32, 33, 34, 35, 37, 38, 39, 40\\
	& 41, 42\\
	\hline
\end{array}
\]
\vspace{-3ex}
\caption{23-WS-template with width 42 and 4 colors}
\vspace{-1ex}
\end{table}

This template provides the inequality \(\WS (n+3) \geqslant 42 \, S(n) + 24\)
by placing one last number, here represented by \(\mathbf{(N+1)}\), in the first subset.

\resetarraystretch

\section{\(WS(6) \geqslant 646\)}
\label{WS(6)}

\begin{table}[H]
	\[
	\begin{array}{|*{2}{c|}}
		\hline
		1 & 1, 2, 6, 10, 14, 18, 26, 30, 34, 42, 46, 50, 54, 62, 70, 79, 82, 90, 95, 99, 111, \\
		& 115, 119, 123, 131, 135, 139, 143, 151, 155, 159, 163, 171, 175, 179, 183, 187, \\
		& 195, 199, 203, 207, 211, 215, 220, 224, 228, 232, 236, 239, 244, 252, 256, 260, 264, \\
		& 267, 272, 275, 280, 284, 292, 296, 300, 304, 308, 312, 316, 320, 328, 340, 344, 348, \\
		& 353, 360, 364, 368, 372, 381, 385, 388, 393, 397, 404, 408, 413, 417, 425, 428, 433, \\
		& 441, 445, 449, 453, 457, 461, 465, 469, 473, 485, 489, 493, 497, 502, 505, 509, 513, \\
		& 517, 521, 525, 529, 533, 537, 541, 546, 549, 553, 558, 562, 566, 569, 574, 578, 586, \\
		& 590, 593, 598, 602, 606, 610, 614, 618, 622, 626, 630, 634, 638, 642, 646 \\
		\hline
		2 & 3, 4, 5, 15, 16, 17, 27, 28, 29, 39, 40, 41, 47, 48, 49, 112, 113, 114, 120, 121, \\
		& 122, 132, 133, 134, 156, 157, 158, 176, 177, 178, 200, 201, 202, 221, 222, 258, 259, \\
		& 281, 282, 283, 301, 302, 303, 345, 346, 347, 365, 366, 367, 389, 426, 427, 446, 447, \\
		& 448, 470, 471, 472, 490, 491, 492, 514, 515, 516, 526, 527, 528, 534, 535, 536, 599, \\
		& 600, 601, 607, 608, 609, 619, 620, 621, 631, 632, 633, 643, 644, 645 \\
		\hline
		3 & 7, 8, 9, 19, 20, 21, 22, 23, 24, 25, 35, 36, 37, 38, 87, 88, 89, 136, 137, 138, 150, \\
		& 152, 153, 154, 180, 181, 182, 196, 197, 198, 208, 209, 210, 212, 213, 214, 261, 262, 263, \\
		& 265, 266, 276, 277, 278, 279, 309, 321, 322, 323, 324, 325, 326, 327, 338, 339, 369, 370, \\
		& 371, 382, 384, 386, 387, 434, 435, 436, 437, 438, 439, 440, 450, 451, 452, 466, 467, 468, \\
		& 482, 494, 495, 496, 499, 500, 510, 511, 512, 559, 560, 561, 611, 612, 613, 623, 624, 625, \\
		& 627, 628, 629, 639, 640, 641 \\
		\hline
		4 & 11, 12, 13, 31, 32, 33, 51, 100, 101, 102, 103, 104, 105, 106, 107, 108, 109, 110, 124, \\
		& 125, 126, 127, 128, 129, 130, 144, 145, 146, 147, 148, 149, 164, 165, 166, 167, 168, 169, \\
		& 170, 172, 173, 174, 188, 189, 190, 191, 192, 193, 194, 454, 455, 456, 458, 459, 460, 474, \\
		& 475, 476, 477, 478, 479, 480, 481, 483, 484, 498, 501, 503, 504, 518, 519, 520, 522, 523, \\
		& 524, 538, 539, 540, 542, 543, 544, 545, 547, 548, 597, 615, 616, 617, 635, 636, 637 \\
		\hline
		5 & 43, 44, 45, 52, 53, 55, 56, 57, 58, 59, 60, 61, 63, 64, 65, 66, 67, 68, 69, 71, 72, 73, \\
		& 74, 75, 76, 77, 78, 80, 81, 83, 84, 85, 86, 91, 92, 93, 94, 216, 217, 218, 219, 223, 225, \\
		& 226, 227, 231, 233, 234, 235, 237, 238, 240, 241, 242, 243, 245, 246, 247, 251, 253, 254, \\
		& 255, 257, 383, 390, 391, 392, 394, 395, 396, 400, 401, 402, 403, 405, 406, 407, 409, 410, \\
		& 411, 412, 414, 415, 416, 421, 422, 423, 424, 429, 430, 431, 432, 554, 555, 556, 557, 563, \\ 
		& 564, 565, 567, 568, 570, 571, 572, 573, 575, 576, 577, 579, 580, 581, 582, 583, 584, 585, \\
		& 587, 588, 589, 591, 592, 594, 595, 596, 603, 604, 605 \\
		\hline
		6 & 96, 97, 98, 116, 117, 118, 140, 141, 142, 160, 161, 162, 184, 185, 186, 204, 205, 206, \\
		& 229, 230, 248, 249, 250, 268, 269, 270, 271, 273, 274, 285, 286, 287, 288, 289, 290, 291, \\
		& 293, 294, 295, 297, 298, 299, 305, 306, 307, 310, 311, 313, 314, 315, 317, 318, 319, 329, \\
		& 330, 331, 332, 333, 334, 335, 336, 337, 341, 342, 343, 349, 350, 351, 352, 354, 355, 356, \\
		& 357, 358, 359, 361, 362, 363, 373, 374, 375, 376, 377, 378, 379, 380, 398, 399, 418, 419, \\
		& 420, 442, 443, 444, 462, 463, 464, 486, 487, 488, 506, 507, 508, 530, 531, 532, 550, 551, 552 \\
		\hline
	\end{array}
	\]
	\vspace{-3ex}
	\caption{Weakly sum-free partition of \([\![ 1, 646 ]\!] \) into 6 subsets.}
	\vspace{-2ex}
	\end{table}



\end{document}
