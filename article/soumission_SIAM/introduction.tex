\section{Introduction}

We are interested in partioning the set of integers \(\{1, ..., p\}\) into \(n\) subsets such that there is no 
subset containing three integers \(x\), \(y\) and \(z\) verifying \(x + y = z\). We say these subsets are 
\textit{sum-free}. If we lessen the above constraint by only considering integers \(x,y\) such that \(x \neq y\), we say the subsets are \textit{weakly sum-free}. The 
greatest \(p\) for which there is a partition into \(n\) sum-free subsets is called the \(n^{\text{th}}\) Schur 
number and is denoted \(S(n)\) \cite{Schur1917}. Likewise for weakly sum-free partitions we define \(\WS(n)\) 
the \(n^{\text{th}}\) weak Schur number \cite{Irving1973}. Values of \(S(n)\) and \(\WS(n)\) are known for small \(n\) only.


\subsection{State of the art}

Up to recently, the most efficient generic construction for Schur numbers was given by Abbott and Hanson 
\cite{AbbottHanson} in 1972 with a recursive construction. It gave the best lower bounds for all sufficiently large 
numbers. No equivalent was known for weak Schur numbers and, as a result, the best known partitions for large 
weak Schur numbers did not use the \textit{weakly} sum-free property. 

\par
As for smaller weak Schur numbers, the best lower bounds were obtained by computations. Eliahou 
\cite{EliahouBook}, Bouzy \cite{Bouzy2015AnAP} and Rafilipojaona \cite{Rafilipojaona} improved the lower 
bounds with Monte-Carlo methods. This was the main approach during the past decade.

\par
In 2020, Rowley introduced the notion of templates for Schur and Ramsey numbers \cite{RowleyRamsey} which 
generalizes Abbott and Hanson's construction and produces new lower bounds (and inequalties) for Schur numbers. 
He also provided two inequalities for weak Schur numbers \cite{RowleyWS}. Besides, these two inequalities do use the \textit{weakly} sum-free property.

\par The hereunder tables recap the lower bounds for Schur and weak Schur numbers, respectively. Gray cells 
indicate the exact values.


\begin{table}[H]
  \centering
\begin{NiceTabular}{*{13}{c}}[vlines]
\CodeBefore
	\cellcolor{gray!40}{2-2}
	\cellcolor{gray!40}{2-3}
	\cellcolor{gray!40}{2-4}
	\cellcolor{gray!40}{2-5}
	\cellcolor{gray!40}{2-6}
	\cellcolor{gray!40}{3-2}
	\cellcolor{gray!40}{3-3}
	\cellcolor{gray!40}{3-4}
	\cellcolor{gray!40}{3-5}
	\cellcolor{gray!40}{3-6}
\Body
	\hline
	\(n\) & 1 & 2 & 3 & 4 & 5 & 6 & 7 & 8 & 9 & 10 & 11 & 12 \\
	\hline
	\Block{2-1}{State of the art} & 1 & 4 & 13 & 44 & 160 & 536 & 1\,696 & 5\,286 & 17\,694 & 60\,320 & 201\,696 & 637\,856 \\
	& & & & & \cite{Heule2017} & \cite{Fredricksen} & \cite{rowley2021improved} & \cite{RowleyRamsey} & \cite{RowleyRamsey} & 
		\cite{RowleyRamsey} & \cite{RowleyRamsey} & \cite{RowleyRamsey} \\
	\hline
	\hyperref[Schur]{\textbf{Our results}}  & & & & & & & & & \textbf{17\,803} & \textbf{60\,948} & \textbf{203\,828} & \textbf{644\,628} \\
	\hline
\end{NiceTabular}
\caption{Comparison of lower bounds for Schur numbers}
\label{table:1}
\end{table}


\begin{table}[H]
  \centering
\begin{NiceTabular}{*{13}{c}}[vlines]
\CodeBefore
	\cellcolor{gray!40}{2-2}
	\cellcolor{gray!40}{2-3}
	\cellcolor{gray!40}{2-4}
	\cellcolor{gray!40}{2-5}
	\cellcolor{gray!40}{3-2}
	\cellcolor{gray!40}{3-3}
	\cellcolor{gray!40}{3-4}
	\cellcolor{gray!40}{3-5}
\Body
	\hline
	\(n\) & 1 & 2 & 3 & 4 & 5 & 6 & 7 & 8 & 9 & 10 & 11 & 12 \\
	\hline
	\Block{2-1}{State of the art} & 2 & 8 & 23 & 66 & 196 & 642 & 2\,146 & 6\,976 & 22\,056 & 70\,778 & 241\,282 & 806\,786 \\
	& & & & & \cite{ELIAHOU2012175} &\cite{RowleyWS} & \cite{RowleyWS} & \cite{RowleyWS} & \cite{RowleyWS} & 
		\cite{RowleyWS} & \cite{RowleyWS} & \cite{RowleyWS} \\
	\hline
	\hyperref[WeakSchur]{\textbf{Our results}}  & & & & & & \textbf{646} & & & \textbf{22\,536} & \textbf{71\,256} & \textbf{243\,794} & \textbf{815\,314} \\
	\hline
\end{NiceTabular}
\caption{Comparison of lower bounds for weak Schur numbers}
\label{table:2}
\end{table}


\subsection{Organization of this article}

Our main contribution is a generalization of the concept of template to weak Schur numbers. Our templates 
provide new lower bounds (and inequalities) for weak Schur numbers. As a special case, our construction also includes  
a construction similar to Abbott and Hanson's \cite{AbbottHanson}, but for \textit{weak} Schur numbers this time.

\par
In Section \ref{Schur}, we explain Rowley's template-based construction for 
Schur numbers. Then, we give new templates, thus providing new lowers bounds and inequalities as well as 
showing that the growth rates for both Schur and Ramsey numbers \(R_n(3)\) exceed 3.28. 

\par
In Section \ref{WeakSchur}, we generalize the concept of template to weak Schur numbers 
and provide new lower bounds for weak Schur numbers. Then, we use a different approach and give a new 
lower bound for \(\WS(6)\).

\par
Now, we introduce notations and definitions used throughout this article.

\subsection{Definitions and notations}

We start by defining sum-free and weakly sum-free subsets to introduce regular and weak Schur numbers. 
The set of positive natural numbers is denoted by \(\mathbb{N}^* =\mathbb{N} \backslash \{0\}\).

\begin{definition}
A subset \(A\) of \(\mathbb{N}\) is said to be \textit{sum-free} if
\[ \forall (a,b) \in A^2 \text{, } a+b \notin A.\]
\end{definition}

\begin{definition}
A subset \(B\) of \(\mathbb{N}\) is said to be \textit{weakly sum-free} if
\[ \forall (a,b) \in B^2 \text{, } a \neq b \Longrightarrow a+b \notin B.\]
\end{definition}

Let us notice that a sum-free subset is also weakly sum-free, hence justifying the name of \textit{weakly} sum-free
subsets. Given \(p\) and \(n\) two integers, we are interested in partitioning the set of integers \(\{1, 2, ..., p\}\), 
denoted by \([\![1,p]\!]\), into \(n\) (weakly) sum-free subsets.

\par
Schur proved \cite{Schur1917} that given a number of subsets \(n\), there is a value of \(p\)
such that \([\![1,q]\!]\) cannot be partionned into \(n\) sum-free subsets for \(q \geqslant p\). A similar
property holds for weakly sum-free subsets \cite{Irving1973}. These observations lead to the following definitions
written for \(n \in \mathbb{N}^*\): 

\begin{definition}
There is a largest integer denoted by \(S(n)\) such that \([\![1, S(n)]\!]\) can be
 partitioned into \(n\) sum-free subsets. \(S(n)\) is called the \textit{\(n\)\textsuperscript{th} Schur number}.
\end{definition}

\begin{definition}
There is a largest integer denoted by \(\WS (n)\) such that \([\![1, \WS (n)]\!]\) 
can be partitioned into \(n\) weakly sum-free subsets. \(\WS(n)\) is called the \(n\)\textsuperscript{th} weak Schur 
number.
\end{definition}

Subsets of an \(n\)-partition are denoted \(A_1, ..., A_n\). The smallest element of \(A_i\) is denoted  
\(m_i = \min(A_i)\). By ordering the subsets, we mean assuming that \(m_1 < ... < m_n\). We will make clear in the text 
when we use a subset ordering.

A partition may be seen as a number coloring.

\begin{definition}
The coloring associated to a partition \(A_1, ..., A_n\) of 
\([\![1, p]\!]\) is the function \(f\) such that \(\forall x \in [\![1, p]\!], x \in A_{f(x)}\). Likewise, the partition associated to
a coloring \(f\) of \([\![1, p]\!]\) with \(n\) colors is \(\forall c \in [\![1, n]\!], A_c = f^{-1}(c)\).
\end{definition}
