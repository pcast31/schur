\section{Templates for Schur numbers}
\label{Schur}

We use Rowley's template-based constructions \cite{RowleyRamsey} for Schur 
numbers. To improve lower bounds for Schur and Ramsey numbers, he has introduced special sum-free 
partitions verifying some additional properties which can be extended using a method generalizing Abbott and 
Hanson's construction \cite{AbbottHanson}. He called these partitions "templates", and we keep this nomination. 
Then, we propose new templates and use them to produce new lower bounds for Schur numbers.

\subsection{Definition of \( [\![ 1, S^+ ]\!] \)}

We begin by introducing \textit{S-templates}, standing for \textit{"Schur templates"}. The idea is to consider the first line
of \hyperref[figure:1]{Figure 1} not as a combination of two blocs, like in \cite{AbbottHanson} but as a whole, single construction. 
An S-template is then defined as a new object playing the role of the first line but with less, yet sufficient, 
constraints which allow for an expansion of the partition using a sum-free partition.

\begin{definition}
Let \((p,n) \in (\mathbb{N}^*)^2\). An S-template with \(n\) colors is defined as a partition of 
\([\![1,p]\!]\) into \(n\) sum-free subsets \(A_1, A_2, ..., A_n\) verifying for all subsets but \(A_n\)
\begin{equation}
\label{eq:constemp}
\forall (x,y) \in A_i^2, x+y > p
\Longrightarrow x+y-p \notin A_i.
\end{equation}
The integer \(p\) is called the \textit{width} of the template.
\end{definition}

Here \(n\) can be seen as a "special" color in the sense that it is not constrained by Assertion (\ref{eq:constemp}).
However, note that \(n\) is not necessarily the last color by order of appearance, any color can play this role.


\begin{proposition}
	Let \(n \in [\![2, +\infty[\![\). We define \(S^+(n)\) as the greatest integer \(w\) such that there is an S-template with 
	width \(w\) and \(n\) colors. 
	\(S^+(n)\) is well defined and verifies
	\[
	2S(n-1)+1 \leqslant S^+(n) \leqslant S(n).
	\]
\end{proposition}

\begin{proof}[\textsc{Proof.}]
The upper bound comes from the fact that an S-template with width \(p\) and \(n\) colors is also a sum-free \(n\)-partition 
of \([\![1, p]\!]\). The lower bound comes from Abbott and Hanson's construction \cite{AbbottHanson}. \\
\end{proof}


\subsection{Construction of Schur partitions using S-templates}

We start by reminding the explicit construction of a sum-free partition with the use of an S-template and a sum-free partition. 
We rephrase in terms of Schur numbers this contruction stated by Rowley in the context of Ramsey numbers \cite{RowleyRamsey}.

\begin{theorem}
\label{thm:Stemp}
	Let \((p,k), (q,n) \in (\mathbb{N}^*)^2\). If there are an \(S\)-template with width \(q\) and \(n+1\) colors 
	and a sum-free \(k\)-partition of \([\![1,p]\!]\) then there is a partition of \([\![1,pq+m_{n+1}
	-1]\!]\) into \(n+k\) sum-free subsets. \(m_{n+1}\) denotes the minimum element colored with the special color in the \(S\)-template.
\end{theorem}

The idea lying beneath this theorem is similar to Abbott and Hanson's contruction \cite{AbbottHanson}. First, they
put the integers from \([\![1,S(k)(2S(n)+1)+S(n)]\!]\) in a table with width \(2S(n)+1\) and height \(S(k)+1\) and they 
numbered both the rows and the columns, starting from one upwards. Then, they 
extended a sum-free partition vertically by repeating it. Next, they used another sum-free partition to color the other half according 
to the line number, as in \hyperref[figure:1]{Figure 1}. We give an example for \(p = 4\), 
\(q = 9\), \(n = 2\) and \(k = 2\) showing that \(S(2 + 2) \geqslant S(2) \left(2 S(2) + 1\right) + S(2)\), both with Abbott and 
Hanson's construction (\hyperref[figure:1]{Figure 1}) and with an S-template which is not included in Abbott and Hanson's construction 
(\hyperref[figure:2]{Figure 2}). In both cases, the special color is grey.

\begin{figure}[H]
\begin{center}

\textbf{Abbott and Hanson's construction}
\label{figure:1}
\vspace{1.7ex}

\renewcommand{\arraystretch}{1.5}
\setlength{\tabcolsep}{1.2ex}
\begin{NiceTabular}{|*{9}{c|}}[corners=SE,standard-cline,hlines]
\CodeBefore
	\cellcolor{red}{1-1}
	\cellcolor{green}{1-2}
	\cellcolor{green}{1-3}
	\cellcolor{red}{1-4}
	\cellcolor{cyan}{1-5}
	\cellcolor{cyan}{1-6}
	\cellcolor{cyan}{1-7}
	\cellcolor{cyan}{1-8}
	\cellcolor{cyan}{1-9}
	\cellcolor{red}{2-1}
	\cellcolor{green}{2-2}
	\cellcolor{green}{2-3}
	\cellcolor{red}{2-4}
	\cellcolor{yellow}{2-5}
	\cellcolor{yellow}{2-6}
	\cellcolor{yellow}{2-7}
	\cellcolor{yellow}{2-8}
	\cellcolor{yellow}{2-9}
	\cellcolor{red}{3-1}
	\cellcolor{green}{3-2}
	\cellcolor{green}{3-3}
	\cellcolor{red}{3-4}
	\cellcolor{yellow}{3-5}
	\cellcolor{yellow}{3-6}
	\cellcolor{yellow}{3-7}
	\cellcolor{yellow}{3-8}
	\cellcolor{yellow}{3-9}
	\cellcolor{red}{4-1}
	\cellcolor{green}{4-2}
	\cellcolor{green}{4-3}
	\cellcolor{red}{4-4}
	\cellcolor{cyan}{4-5}
	\cellcolor{cyan}{4-6}
	\cellcolor{cyan}{4-7}
	\cellcolor{cyan}{4-8}
	\cellcolor{cyan}{4-9}
	\cellcolor{red}{5-1}
	\cellcolor{green}{5-2}
	\cellcolor{green}{5-3}
	\cellcolor{red}{5-4}
\Body
	1 & 2 & 3 & 4 & 5 & 6 & 7 & 8 & 9 \\
	10 & 11 & 12 & 13 & 14 & 15 & 16 & 17 & 18 \\
	19 & 20 & 21 & 22 & 23 & 24 & 25 & 26 & 27 \\
	28 & 29 & 30 & 31 & 32 & 33 & 34 & 35 & 36 \\
	37 & 38 & 39 & 40 \\
\end{NiceTabular}

\vspace{1ex}
\setlength{\tabcolsep}{2ex}

\begin{tabular}{c c}
	\textbf{Corresponding S-template} & \textbf{Corresponding sum-free partition} \\
	\setlength{\tabcolsep}{1.5ex}
	\begin{NiceTabular}{|*{9}{c|}}[standard-cline,hlines]
	\CodeBefore 
		\cellcolor{red}{1-1}
		\cellcolor{green}{1-2}
		\cellcolor{green}{1-3}
		\cellcolor{red}{1-4}
		\cellcolor{gray!40}{1-5}
		\cellcolor{gray!40}{1-6}
		\cellcolor{gray!40}{1-7}
		\cellcolor{gray!40}{1-8}
		\cellcolor{gray!40}{1-9}
	\Body
		1 & 2 & 3 & 4 & 5 & 6 & 7 & 8 & 9 \\
	\end{NiceTabular} &
	\setlength{\tabcolsep}{1.5ex}
	\begin{NiceTabular}{|*{4}{c|}}[standard-cline,hlines]
	\CodeBefore
		\cellcolor{cyan}{1-1}
		\cellcolor{yellow}{1-2}
		\cellcolor{yellow}{1-3}
		\cellcolor{cyan}{1-4}
	\Body
		1 & 2 & 3 & 4 \\
	\end{NiceTabular}
\end{tabular}

\setlength{\tabcolsep}{6pt}
\caption{Visualisation of an Abbott and Hanson construction}
\end{center}
\end{figure}
In the general construction with S-templates, the special color may not paint consecutive 
numbers any longer. However, the special color is still replaced by the colors of the sum-free partition according to the 
line number and the other colors are still vertically extended.

\begin{figure}[H]
\begin{center}
\textbf{S-template construction}
\label{figure:2}
\setlength{\tabcolsep}{1.2ex}
\renewcommand{\arraystretch}{1.5}

\vspace{1.7ex}
\begin{NiceTabular}{|*{9}{c|}}[corners=SE,standard-cline,hlines]
\CodeBefore
	\cellcolor{red}{1-1}
	\cellcolor{green}{1-2}
	\cellcolor{green}{1-3}
	\cellcolor{red}{1-4}
	\cellcolor{cyan}{1-5}
	\cellcolor{cyan}{1-6}
	\cellcolor{red}{1-7}
	\cellcolor{cyan}{1-8}
	\cellcolor{cyan}{1-9}
	\cellcolor{red}{2-1}
	\cellcolor{green}{2-2}
	\cellcolor{green}{2-3}
	\cellcolor{red}{2-4}
	\cellcolor{yellow}{2-5}
	\cellcolor{yellow}{2-6}
	\cellcolor{red}{2-7}
	\cellcolor{yellow}{2-8}
	\cellcolor{yellow}{2-9}
	\cellcolor{red}{3-1}
	\cellcolor{green}{3-2}
	\cellcolor{green}{3-3}
	\cellcolor{red}{3-4}
	\cellcolor{yellow}{3-5}
	\cellcolor{yellow}{3-6}
	\cellcolor{red}{3-7}
	\cellcolor{yellow}{3-8}
	\cellcolor{yellow}{3-9}
	\cellcolor{red}{4-1}
	\cellcolor{green}{4-2}
	\cellcolor{green}{4-3}
	\cellcolor{red}{4-4}
	\cellcolor{cyan}{4-5}
	\cellcolor{cyan}{4-6}
	\cellcolor{red}{4-7}
	\cellcolor{cyan}{4-8}
	\cellcolor{cyan}{4-9}
	\cellcolor{red}{5-1}
	\cellcolor{green}{5-2}
	\cellcolor{green}{5-3}
	\cellcolor{red}{5-4}
\Body
	1 & 2 & 3 & 4 & 5 & 6 & 7 & 8 & 9 \\
	10 & 11 & 12 & 13 & 14 & 15 & 16 & 17 & 18 \\
	19 & 20 & 21 & 22 & 23 & 24 & 25 & 26 & 27 \\
	28 & 29 & 30 & 31 & 32 & 33 & 34 & 35 & 36 \\
	37 & 38 & 39 & 40 \\
\end{NiceTabular}

\vspace{1ex}
\setlength{\tabcolsep}{1.5ex}

\begin{tabular}{c c}
	\textbf{Corresponding S-template} & \textbf{Corresponding sum-free partition} \\
	\setlength{\tabcolsep}{1.5ex}
	\begin{NiceTabular}{|*{9}{c|}}[standard-cline,hlines]
	\CodeBefore 
		\cellcolor{red}{1-1}
		\cellcolor{green}{1-2}
		\cellcolor{green}{1-3}
		\cellcolor{red}{1-4}
		\cellcolor{gray!40}{1-5}
		\cellcolor{gray!40}{1-6}
		\cellcolor{red}{1-7}
		\cellcolor{gray!40}{1-8}
		\cellcolor{gray!40}{1-9}
	\Body
		1 & 2 & 3 & 4 & 5 & 6 & 7 & 8 & 9 \\
	\end{NiceTabular} &
	\setlength{\tabcolsep}{1.5ex}
	\begin{NiceTabular}{|*{4}{c|}}[standard-cline,hlines]
	\CodeBefore
		\cellcolor{cyan}{1-1}
		\cellcolor{yellow}{1-2}
		\cellcolor{yellow}{1-3}
		\cellcolor{cyan}{1-4}
	\Body
		1 & 2 & 3 & 4 \\
	\end{NiceTabular}
\end{tabular}

\setlength{\tabcolsep}{6pt}
\caption{Visualisation of a template-based construction}

\end{center}
\end{figure}


Now, we prove Theorem \ref{thm:Stemp}.

\begin{proof}[\textsc{Proof.}]
	Let \(f\) and \(g\) be colorings associated to the S-template with width \(q\) and the sum-free partition of 
	\([\![1,p]\!]\), respectively: \(f : [\![1,q]\!] \longrightarrow [\![1,n+1]\!]\) and 
	\(g : [\![1,p]\!] \longrightarrow [\![1,k]\!]\).
	
	\par
	NB: In the following three predicates, the conditions \(x + y \leqslant q\)  and \(x + y \leqslant p\) are omitted for readability.
	\par
	The sum-free condition is expressed as:
	\[\forall (x,y) \in [\![1,q]\!]^2, f(x) = f(y) \Longrightarrow f(x+y) \neq f(x),\]
	\[\forall (x,y) \in [\![1,p]\!]^2, g(x) = g(y) \Longrightarrow g(x+y) \neq g(x).\]
	
	The additionnal constraint for the S-template is:
	\[
	\forall (x,y) \in [\![1,q]\!]^2, \left\{
	\begin{array}{l}
		f(x) = f(y) \leqslant n \\
		x + y > q
	\end{array}
	\right. \Longrightarrow f(x+y-q) \neq f(x).
	\]
	
	For \(x \in [\![1,pq+m_{n+1}-1]\!]\), we write \(x = (\alpha - 1) q + u\) for certain integers \(\alpha \in \mathbb{N}^*\) 
	and \(u \in [\![1,q]\!]\). Obviously, this decomposition unique. Coefficients \(\alpha\) and \(u\) may be interpreted as line 
	and column coordinates of \(x\) in the template. We define a coloring \(h\) of \([\![1,pq+m_{n+1}-1]\!]\) using \( [\![1,n+k]\!]\) colors:
	\[
	\begin{array}{c c c l}
		h : & [\![1,pq+m_{n+1}-1]\!] & \longrightarrow & [\![1,n+k]\!], \\
		& x & \longmapsto & 
		\left\{ \begin{array}{l l}
			f(u), & \text{if}~f(u) \leqslant n, \\
			n + g(\alpha), & \text{if}~f(u) = n + 1. \\
		\end{array} \right.
	\end{array}
	\]
	
	Function \(h\) is well-defined since, by definition of \(m_{n+1}\), \(\forall x \in [\![p q + 1, p q + m_{n + 1} - 1 ]\!], f(u) 
	\leqslant n\) and therefore \(\forall x \in [\![1,pq+m_{n+1}-1]\!], f(u) = n + 1 \implies \alpha \in [\![1, p]\!]\).
	
	\par
	Now, we prove that \(h\) is a sum-free coloring. Let \(x,y \in [\![1,pq + m_{n+1}-1]\!]\) such that \(h(x) = h(y)\) 
	and \(x+y \leqslant pq+m_{n+1}-1\). We claim that \(h(x+y) \neq h(x)\). We write \(x = (\alpha - 1) q + u\) and 
	\(y = (\beta - 1) q + v\) where \(\alpha, \beta \in \mathbb{Z}\) and \(u, v \in [\![1,q]\!]\). Two cases are to be
	distinguished according to the value of \(h(x)\). \\
	
	\noindent \underline{\textbf{Case 1:} \(h(x) \leqslant n\)}
	\par
	Let us assume that \(h(x+y) \leqslant n\), otherwise \(h(x + y) \neq h(x)\) obviously holds. By definition of function 
	\(h\) and given that \(h(u) = h(v)\), we conclude \(f(u) = f(v)\). Two cases are to be distinguished according to the value of \(x + y\).
	
	\begin{itemize}
	\item If \(u + v > q\), we write \(w = u + v - q \in [\![1, q]\!]\). Consequentely \(x + y = (\alpha + \beta - 1) q + w\). By definition, 
		\(h(x + y) = f(w)\). Given that \(f(u) = f(v) \leqslant n\), the additionnal constraint on \(f\) implies \(f(w) 
		\neq f(u)\), that is \(h(x + y) \neq h(x)\).
	\item If \(u + v \leqslant q\), we write \(w = u + v \in [\![1, q]\!]\). Consequentely \(x+y = (\alpha + \beta- 2) q + w\). By definition, 
		\(h(x + y) = f(w)\). Given that \(f(u) = f(v) \leqslant n\), the sum-free property of \(f\) implies \(f(w) \neq f(u)\), 
		that is \(h(x + y) \neq h(x)\).
	\end{itemize} 
	  
	\noindent \underline{\textbf{Case 2:} \(h(x) \geqslant n + 1\)}
	\par
	Now we have \(h(x) = n + g(\alpha) = 
	n + g(\beta) = h(y)\), hence \(g(\alpha) = g(\beta)\). As above, we distinguish between two cases according 
	to the value of \(x + y\).
	
	\begin{itemize}
	\item \begin{sloppypar}
		If \(u + v > q\), write \(w = u + v - q \in [\![1, q]\!]\). Then \(x + y = (\alpha + \beta - 1) q + w\). Assume that 
		\({h(x+y) \geqslant n + 1}\),  otherwise \(h(x + y) \neq h(x)\) obviously holds. By definition, \({h(x + y) = n + g(\alpha + 
		\beta)}\). Given that \(g(\alpha) = g(\beta)\), the sum-free property of \(g\) implies \(g(\alpha + \beta) 
		\neq g(\alpha)\) that is \(h(x + y) \neq h(x)\).
		\end{sloppypar}
	\item  If \(u + v \leqslant q\), write \(w = u + v \in [\![1, q]\!]\). Then \(x+y = (\alpha + \beta- 2) q + w\). The sum-free 
		property of \(f\) implies \(f(w) \neq f(u)\). Therefore \(f(w) \leqslant n\) and thus \(h(x + y) \leqslant n\). In particular,
		given that \(h(x) \geqslant n + 1\), \(h(x + y) \neq h(x)\).
	\end{itemize}
\end{proof}

Setting \(q = S^+(n+1)\) and \(p = S(k)\) in Theorem \ref{thm:Stemp} yields the following corollary.

\begin{corollary}
\label{cor:ineqS}
	Let \(n, k \in \mathbb{N}^*\). Then
	\[ S(n+k) \geqslant S^+(n+1)S(k) + m_{n+1} - 1.\]
\end{corollary}

The following proposition may improve the additive constant of Corollary \ref{cor:ineqS}.

\begin{proposition}
\label{prop:rafStemp}
Let \((q, n) \in \left(\mathbb{N}^* \right)^2\) and let \(f\) be a coloring associated to an S-template with width \(q\) and \(n+1\) 
colors. Let \(b \in \mathbb{N}\) and assume there is a coloring \(g\) of 
\([\![1, b]\!]\) with \(n+1\) colors such that:
	
\begin{itemize}
\item \(\forall (x, y) \in [\![1, q]\!]^2, \left\{
	\begin{array}{l}
		f(x) = f(y) \\
		(x + y) \mod q \leqslant b
	\end{array}
	\right. \implies g((x + y) \mod q) \neq f(x)\),
\item \(\forall (x, y) \in [\![1, q]\!] \times  [\![1, b]\!],  \left\{
	\begin{array}{l}
		f(x) = g(y) \\
		x + y \leqslant b
	\end{array}
	\right. \implies g(x + y) \neq f(x)\).
\end{itemize}
	
Then, for every \(n \in \mathbb{N}^*\), by using on the last row the coloring \(x \longmapsto g(x - q S(n))\), we have
\[ S(n+k) \geqslant q S(k) + b.\]
\end{proposition}
This proposition corresponds to the fact that the hypotheses made on the coloring of the last row can be weakened because 
the constraints in the definition of an S-template prevent a number from: 
\begin{itemize}
\item being the sum of two of numbers of its own color. 
\item creating a sum in its own color. 
\end{itemize}
The later is not always relevant on the last row given that it is not completely filled.
As a result, we can change the coloring on \([\![ q S(k) + 1,  q S(k) + b]\!]\) (the last row) to extend the previous partitions.
\newline

There is a construction theorem for S-templates as well.

\begin{theorem}
	Let \((p,k), (q,n) \in (\mathbb{N}^*)^2\). If there are an S-template with width \(q\) and \(n+1\) colors,
	and an S-template with width \(p\) and \(k\) colors, then there also is an S-template with width \(pq\) and \((n+k)\) 
	colors.
\end{theorem}

\begin{proof}[\textsc{Proof.}]
The idea is the same as in the Theorem \ref{thm:Stemp}. The only difference is the S-template property inherited 
from the second S-template. \\
\end{proof}
	
The inequality associated with this theorem is given by:
	
\begin{corollary}
	Let \(n, k \in \mathbb{N}^*\). Then
	\[ S^+(n+k) \geqslant S^+(n+1)S^+(k).\]
\end{corollary}


\subsection{New lower bounds for Schur numbers}
\label{subsec:lowS}

We now give the inequalities corresponding to the current  best S-templates.

\begin{definition}
A sum-free partition \(A_1, ..., A_n\) of \([\![1, p]\!]\) is said to be symmetric if for all \( x \in [\![1, p]\!]\), 
\(x\) and \(p + 1 - x\) belong to the same subset (except if \(x = p + 1 - x\)).

An S-template with \(n\) colors is said to be symmetric if the sum-free \(n\)-partition derived 
from this template by applying the extension procedure of Theorem \ref{thm:Stemp} with a sum-free 
partition with length one is symmetric. 
\end{definition}

We produced S-templates using the lingeling SAT solver \cite{Lingeling2017}, hence providing lower bounds on 
\(S^+\) and inequalities of the type \(S(n+k) \geqslant a S(n) + b\). We sought templates providing us with 
the largest value of \((a, b)\) (in the lexicographic order). In order to reduce the search space when the number 
of colors exceeded five we only looked for symmetric S-templates, we assumed that the special color was the last 
color to appear and we constrained the \(m_c\)'s out of being too small. Details concerning the encoding as a SAT 
problem can be found in \cite{Heule2017}.

\par
The following six inequalities are given by the current best S-templates with \(n \leqslant 7\) colors.
\begin{align}
	S(n + 1) \geqslant &~3\,S(n)  + 1 \label{S(n+1)}\\
	S(n + 2) \geqslant &~9\,S(n)  + 4 \label{S(n+2)}\\
	S(n + 3) \geqslant &~33\,S(n) + 6 \label{S(n+3)}
\end{align}

Inequality (\ref{S(n+1)}) comes from  Schur's original article \cite{Schur1917}. Inequality (\ref{S(n+2)}) is due to
Abott and Hanson \cite{AbbottHanson} and inequality (\ref{S(n+3)}) to Rowley \cite{RowleyRamsey}. 
The three following inequalities are our result:

\begin{align}
	S(n + 4) \geqslant &~111\,S(n) + 43 \label{S(n+4)}\\
	S(n + 5) \geqslant &~380\,S(n) + 148 \label{S(n+5)}\\
	S(n + 6) \geqslant &~1\,160\,S(n) + 536 \label{S(n+6)}
\end{align}

\par The \hyperref[S-templates]{templates} corresponding to inequalities (\ref{S(n+3)}), (\ref{S(n+4)}), and 
(\ref{S(n+5)}) are listed in \ref{S-templates}.

\par
Inequalities (\ref{S(n+1)}), (\ref{S(n+2)}), and (\ref{S(n+3)}) cannot be further improved (with this definition of S-template). 
Inequality (\ref{S(n+4)}) cannot be further improved by only searching for symmetric S-templates whose special color is the 
last in the order of apparition (and with a multiplicative factor less than or equal to 118). Inequality (\ref{S(n+5)}) can most 
likely be further improved but the improvement probably will not be substantial. Finally, an S-template corresponding to inequality 
(\ref{S(n+6)}) was found by extending into a S-template the Schur 6-partition used in \cite{rowley2021improved} (owing to the 
size of this template, its interest is limited and since it can easily be derived from above mentioned partition, it is not given in 
\ref{S-templates}). Although we could not find a better S-template with seven colors, inequality (\ref{S(n+6)}) can definitely be 
improved by a wide margin.

\par
The previous inequalities give new lower bounds for \(S(n)\) for
\( n \geqslant 9 \). We compute the lower
bounds for \( n \in [\![8,15]\!] \) using the four different inequalities. The best lower bounds are highlighted.


\begin{table}[H]

\label{LowerBoundsS}
\[
\begin{array}{c}
	\begin{NiceArray}{cwc{8ex}wc{10ex}wc{10ex}wc{11ex}}[hvlines]
	\CodeBefore
		\cellcolor{yellow}{2-2}
		\cellcolor{yellow}{3-3}
		\cellcolor{yellow}{4-4}
		\cellcolor{yellow}{4-5}
	\Body
		n & 8 & 9 & 10 & 11 \\
		33 \, S(n-3) + 6 & 5\,286 & 17\,694 & 55\,974 & 174\,444 \\
		111 \, S(n-4) + 43 & 4927 & 17\,803 & 59\,539 & 188\,299 \\
		380 \, S(n-5) + 148 & 5\,088 & 16\,868 & 60\,948 & 203\,828 \\
	\end{NiceArray}
	\\ \\
	\begin{NiceArray}{cwc{8ex}wc{10ex}wc{10ex}wc{11ex}}[hvlines]
	\CodeBefore
		\cellcolor{yellow}{2-3}
		\cellcolor{yellow}{4-2}
		\cellcolor{yellow}{4-4}
		\cellcolor{yellow}{4-5}
	\Body
		n & 12 & 13 & 14 & 15 \\
		33 \, S(n-3) + 6 & 587\,505 & 2\,011\,290 & 6\,726\,330 & 21\,272\,730 \\
		111 \, S(n-4) + 43 & 586\,789 & 1\,976\,176 & 6\,765\,271 & 22\,624\,951 \\
		380 \, S(n-5) + 148 & 644\,628 & 2\,008\,828 & 6\,765\,288 & 23\,160\,388 \\
	\end{NiceArray}
\end{array}
\]
\caption{New lower bounds for \( n \in [\![8,15]\!] \)}
\end{table}

Except for \(S(8)\), \(S(9)\), and \(S(13)\), the best lower bounds are obtained thanks to
the fifth inequality \( S(n+5) \geqslant 380 \, S(n) + 148\). The table
does not go any further, but the same inequality allows one to improve the
lower bounds for \( n \geqslant 15 \).

\begin{corollary}
\begin{sloppypar}
The growth rate for Schur numbers (and Ramsey numbers \(R_n(3)\))  satisfies \({\gamma \geqslant \sqrt[5]{380} \approx 3.28}\).
\end{sloppypar}
\end{corollary}

\begin{proof}[\textsc{Proof.}]
It is a mere consequence of the inequality \( S(n+5) \geqslant 380 \, S(n) + 148\). As for Ramsey
numbers, the following inequality holds \(S(n) \leqslant R_n(3) - 2\) (see \cite{AbbottHanson}) hence the result. \\
\end{proof}


\subsection{Conclusion on S-templates}

First, we formalized Rowley's template-based constructions \cite{RowleyRamsey} in the context of Schur numbers 
by introducing S-templates as well as a new sequence, \(S^+\). We found new S-templates allowing us to obtain 
new lower bounds for schur numbers. One may notice that we mostly gave only lower bounds for \(S^+\). It should be possible to 
find better S-templates by making different assumptions or using a different method (Monte-Carlo methods, for instance).

\par
In the next section, we provide similar results for weak Schur numbers. We introduce WS-templates and a corresponding sequence, 
\(\WS^+\). Then, we derive similar relations and a construction method allowing us to find new lower bounds for weak Schur numbers.