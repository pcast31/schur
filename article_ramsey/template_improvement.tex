\documentclass{article}
\title{Improving the templates for multicolor Ramsey numbers and Schur numbers}
\author{Romain Ageron}

\usepackage{amsmath}
\usepackage{amssymb}
\usepackage{amsthm}
\usepackage{dsfont}
\usepackage{mathtools}
\usepackage{MyMnSymbol}

\usepackage{enumerate}

\usepackage[hyphens]{url}
\usepackage{hyperref}

\hypersetup{hidelinks}

\usepackage[utf8]{inputenc}
\usepackage{siunitx}
\usepackage{float}

\usepackage[table]{xcolor}
\usepackage{multirow}
\usepackage{nicematrix}
\usepackage[a4paper, total={6in, 8in}]{geometry}

\newtheorem{definition}{Definition}[section]
\newtheorem{theorem}[definition]{Theorem}
\newtheorem{corollary}[definition]{Corollary}
\newtheorem{lemma}[definition]{Lemma}
\newtheorem{proposition}[definition]{Proposition}

\DeclareMathOperator{\cons}{Cons}
\DeclareMathOperator{\coldef}{Default}
\DeclareMathOperator{\tree}{Tree}
\DeclareMathOperator{\treeset}{TreeSet}

\newcommand{\defeq}{\vcentcolon=}
\newcommand{\TS}{\mathit{TS}}


\begin{document}

\maketitle

\begin{abstract}
Lower bounds for several multicolor Ramsey numbers as well as Schur numbers have recently been improved using a 
pattern based construction with a particular type of coloring nammed "sf-template". This article describes a 
generalization of the sf-templates: the \(b\)-templates.
\end{abstract}

\section{Introduction}

This article addresses the edge-coloring of complete graphs in an arbitrary number of colors while limiting the clique 
number in each color, as well as sum-free colorings of positive integers. A pattern based construction which uses a 
particular type of coloring nammed sf-template was described in \cite{RowleyRamsey}. This construction was used to 
improve lower bounds on several Ramsey numbers as well as Schur numbers. Lower bounds were then improved by describing 
larger templates in \cite{schurboyz} for Schur numbers and in \cite{rowleyramseysat} for multicolor Ramsey numbers.

Ramsey numbers and Schur numbers are defined in Section \ref{sec:def}. Section \ref{sec:temp} defines the 
\(b\)-templates and describes the corrresponding construction. Section \ref{sec:results} states and proves the main 
results on \(b\)-templates and on the construction. Section \ref{sec:bounds} records the new inequalities and some new 
lower bounds. The improved templates can be found in Appendix \ref{appendix}.

\section{Definitions}
\label{sec:def}

\begin{definition}[Coloring and partition]
Let \(S\) and \(T\) be two sets. A coloring of \(S\) with colors in \(T\) is an application 
\(\gamma: S \rightarrow T\). The partition associated to the coloring \(\gamma\) is the partition 
\((\gamma^{-1}(t))_{t \in T ~\text{and}~ \gamma^{-1}(\{t\}) \neq \varnothing}\). Conversely, the coloring associated to 
a partition \((A_i)_{i \in I}\) for some \(I \subset T\) is the application \(\gamma: S \rightarrow T\) such that 
\(\forall s \in S, \forall t \in T, \gamma(s) = t \iff s \in A_t\).
\end{definition}

The set of positive natural numbers is denoted by \(\mathbb{N}^* = \mathbb{N} \backslash \{0\}\). Given two integers 
\(a, b \in \mathbb{Z}\), the set of integers \(\{a, a + 1, ..., b - 1, b\}\) is denoted by \([\![a, b]\!]\). For 
\(n \in \mathbb{N}\),  the complete graph of order \(n\) is denoted by \(K_n\).

\begin{definition}[Ramsey coloring]
Let  \(n \in \mathbb{N}\), the order of the graph, and let \(r \in \mathbb{N}\), the number of colors. Let 
\((k_c)_{1 \leqslant c \leqslant r} \in {\mathbb{N}^*}^r\). A Ramsey coloring is an edge coloring of \(K_n\) with 
\(r\) colors such that for any color \(c \in [\![1, r]\!]\), the coloring does not contain any monochramatic complete 
sub-graph of order \(k_c\) with color \(c\). The set of these colorings is denoted by 
\(\mathcal{R}(k_1, ..., k_r ; n)\).
\end{definition}

\begin{theorem}[Ramsey's theorem \cite{Ramsey}]
\label{thm:ram}
Let \(r \in \mathbb{N}\) and let \((k_c)_{1 \leqslant c \leqslant r} \in {\mathbb{N}^*}^r\). Then there is 
\(n \in \mathbb{N}\) such that \(\mathcal{R}(k_1, ..., k_r ; n) = \varnothing\).
\end{theorem}

Theorem \ref{thm:ram} leads to the following definition.

\begin{definition}[Multicolor Ramsey number]
Let \(r \in \mathbb{N}\) and let \((k_c)_{1 \leqslant c \leqslant r} \in {\mathbb{N}^*}^r\). The multicolor Ramsey  
number \(R(k_1, ..., k_r)\) is defined as the smallest integer \(n \in \mathbb{N}\) such 
\(\mathcal{R}(k_1, ..., k_r ; n) = \varnothing\). If all the \(k_c\)'s are equal to some \(k\), this Ramsey number is 
also denoted by \(R_r(k)\).
\end{definition}

The construction described in \cite{RowleyRamsey} as well as the one described in this article use a subset of Ramsey 
colorings, the linear Ramsey colorings.

\begin{definition}[Linear Ramsey coloring]
Let \(n \in \mathbb{N}\), and let \(r \in \mathbb{N}\). Let \((k_c)_{1 \leqslant c \leqslant r} \in {\mathbb{N}^*}^r\). 
We assume that the vertices of \(K_{n + 1}\) are the integers from \([\![0, n]\!]\). A linear Ramsey coloring is any 
Ramsey coloring such that the color of the edge \((u, v)\) only depends on the value of \(|u - v|\). The set of these 
colorings is denoted by \(\mathcal{L}(k_1, ..., k_r ; n)\). The integer coloring associated to a linear Ramsey coloring 
is the coloring of \([\![1, n]\!]\) such that every \(x \in [\![1, n]\!]\) is colored with the color of the edges \((u, v)\) 
such that \(|u - v| = x\).
\end{definition}

Unlike Ramsey colorings, the set \(\mathcal{L}(k_1, ..., k_r ; n)\) corresponds to graph colorings of order \(n + 1\). 
This definition is chosen in this article because linear Ramsey colorings are studied using their associated integer 
coloring. Also, in this article the linear Ramsey numbers are defined as the largest size of a coloring (contrary to 
Ramsey numbers) because it is more convenient in inequalities.

\begin{definition}[Linear Ramsey number]
Let \(r \in \mathbb{N}\) and let \((k_c)_{1 \leqslant c \leqslant r} \in {\mathbb{N}^*}^r\). The linear Ramsey number 
\(L(k_1, ..., k_r)\) is defined as the largest integer \(n \in \mathbb{N}\) such that 
\(\mathcal{L}(k_1, ..., k_r ; n) \neq \varnothing\). If all the \(k_c\)'s are equal to some \(k\), this linear Ramsey 
number is also denoted by \(L_r(k)\).
\end{definition}

Linear Ramsey numbers can be used to construct lower bounds for Ramsey numbers because a linear Ramsey coloring is 
also a Ramsey coloring. Therefore for all \(r \in \mathbb{N}\) and for all 
\((k_c)_{1 \leqslant c \leqslant r} \in {\mathbb{N}^*}^r\), \(R(k_1, ..., k_r) \geqslant L(k_1, ..., k_r) + 2\).

Let \(n \in \mathbb{N}\). Any linear coloring of \(K_{n + 1}\) induces a coloring of \([\![1, n]\!]\). Conversely, any 
coloring of \([\![1, n]\!]\) corresponds to a linear coloring of \(K_{n + 1}\). Theorem \ref{thm:linkintegers} gives a 
link between some sum-free properties of the coloring of \([\![1, n]\!]\) and the clique numbers of the associated 
coloring of \(K_{n+1}\).

\begin{theorem}[Link to sum-free colorings \cite{AbbottHanson}]
\label{thm:linkintegers}
Let \(r \in \mathbb{N}\) and let \((k_c)_{1 \leqslant c \leqslant r} \in {\mathbb{N}^*}^r\). Let \(n \in \mathbb{N}^*\) 
and let \(\gamma\) be a coloring of \([\![1, n]\!]\) with \(r\) colors. Then the two following conditions are equivalent.
\begin{enumerate}[(i)]
\item There is a monochromatic subgraph of order \(k_c\) with color \(c\) in the linear coloring of \(K_{n+1}\) 
	associated  to \(\gamma\).
\item There is a monochromatic subset \(S\) of \([\![1, n]\!]\) with color \(c\) and cardinality \(k_c - 1\) such that 
	for all distinct \(x, y \in S\), \(\gamma(|x - y|) = c\).
\end{enumerate}
\end{theorem}

\begin{definition}[Schur number]
A subset \(A\) of \(\mathbb{N}\) is said to be sum-free if \(\forall (a, b) \in A^2, a + b \notin A\). Let 
\(n \in \mathbb{N}\). Schur proved in \cite{Schur1917} that there is a largest integer denoted by \(S(n)\) such that 
\([\![1, S(n)]\!]\) can be partioned into \(n\) sum-free subsets. \(S(n)\) is called the \(n\)\textsuperscript{th} 
Schur number.
\end{definition}

Theorem \ref{thm:linkintegers} shows a link between Schur numbers and Ramsey numbers:
\[\forall n \in \mathbb{N}^*, S(n) = L_n(3) \leqslant R_n(3) - 2.\]

\section{\(b\)-Templates}
\label{sec:temp}

\begin{definition}[Construction method]
Let \((a, p) \in {\mathbb{N}^*}^2\) and let \(b \in \mathbb{N}\). Let \((r, s) \in {\mathbb{N}^*}^2\). Let \(H\) 
(standing for "horizontal") be a coloring of \([\![1, a + b]\!]\) with \(r + 1\) colors, and let \(V\)  (standing for 
"vertical") be a coloring of \([\![1, p]\!]\) with \(s\) colors. Let \(d \in \mathbb{N}\) and let \(F\) (standing for 
"final") be a coloring of \([\![1, d]\!]\) with \(r + 1\) colors. The empty coloring (i.e. \(d = 0\)) is denoted by 
\(F = \varnothing\). If \(s \geqslant 3\) or \(s = 2\) and \(V(p) \neq 2\), set \(c = \max ([\![2, s]\!] \backslash 
\{V(p)\})\). Otherwise, set \(c = s\). The construction produces a coloring \(\cons_b(H, V, F)\) of 
\([\![1, a p + b + d]\!]\) with \(r + s\) colors.

Informally, one starts by using the initial coloring \(I \defeq H_{| \, [\![1, b]\!]}\), then repeats \(p\) times the 
pattern \(P \defeq H_{| \, [\![b + 1, a + b]\!]}\) while replacing at the \(i\)\textsuperscript{th} iteration the color 
\(r + 1\) by \(r + \mathcal{V}(i)\), and eventually appends the coloring \(\mathcal{F}\) in which color \(r + 1\) is 
replaced by \(r + c\). Formally:
\[\begin{array}{ccl}
	[\![1, a p + b + d]\!]  & \longrightarrow &  [\![1, r + s]\!] \\
 	x & \longmapsto & 
		\left\{\begin{array}{l l}
			H(x) & ~\text{if}~ x \leqslant b \\
			H(\pi(x)) & ~\text{if}~  b + 1 \leqslant x \leqslant a p + b ~\text{and}~ H(\pi(x)) \neq r + 1 \\
			r + V \left(\left\lceil \dfrac{x - b}{a} \right\rceil\right) & ~\text{if}~ b + 1 \leqslant x \leqslant a p + b 
				~\text{and}~ H(\pi(x)) = r + 1 \\
			F(x - a p - b) & ~\text{if}~ x \geqslant a p + b + 1 ~\text{and}~ F(x - a p - b) \neq r + 1 \\
			r + c & ~\text{if}~ x \geqslant a p + b + 1 ~\text{and}~ F(x - a p - b) = r + 1 \\
		\end{array}\right.
\end{array}\]
\end{definition}

\begin{figure}[H]
\caption{Example of construction with \(b = 5\)}
\renewcommand{\arraystretch}{1.6}
\small
\begin{center}

\begin{tabular}{c}
	\begin{NiceTabular}{c}[hvlines]
		\CodeBefore
			\cellcolor{cyan}{1-1}
		\Body
			template color
	\end{NiceTabular} \\
	\(H\) coloring \\
	\(
		\begin{NiceArray}{*{17}{c}}[hvlines,last-row]
		\CodeBefore
			\cellcolor{red}{1-1}
			\cellcolor{red}{1-2}
			\cellcolor{cyan}{1-3}
			\cellcolor{red}{1-4}
			\cellcolor{green}{1-5}
			\cellcolor{green}{1-6}
			\cellcolor{red}{1-7}
			\cellcolor{green}{1-8}
			\cellcolor{cyan}{1-9}
			\cellcolor{cyan}{1-10}
			\cellcolor{cyan}{1-11}
			\cellcolor{cyan}{1-12}
			\cellcolor{cyan}{1-13}
			\cellcolor{red}{1-14}
			\cellcolor{green}{1-15}
			\cellcolor{cyan}{1-16}
			\cellcolor{red}{1-17}
		\Body
			\,\,1\,\, & \,\,2\,\, & \,\,3\,\, & \,\,4\,\, & \,\,5\,\, & \,\,6\,\, & \,\,7\,\, & \,\,8\,\, & \,\,9\,\, & 10 & 
				11 & 12 & 13 & 14 & 15 & 16 & 17 \\
			\Block{1-5}{\underbrace{\hphantom{................................}}_{\textstyle I}} & & & & & \Block{1-12}{\underbrace{\hphantom{..................................................................................}}_{\textstyle P}}
		\end{NiceArray}\) \\
	\begin{tabular}{c c}
		\begin{tabular}{c}
			\(V\) coloring \\
			\(
				\begin{NiceArray}{c}[hvlines]
				\CodeBefore
					\cellcolor{cyan}{1-1}
					\cellcolor{yellow}{2-1}
					\cellcolor{yellow}{3-1}
					\cellcolor{cyan}{4-1}
				\Body
					\,\,1\,\, \\
					\,\,2\,\, \\
					\,\,3\,\, \\
					\,\,4\,\, \\
				\end{NiceArray}\)
		\end{tabular} &
		\begin{tabular}{c}
			\(F\) coloring \\
			\(
				\begin{NiceArray}{*{4}{c}}[hvlines]
				\CodeBefore
					\cellcolor{red}{1-1}
					\cellcolor{cyan}{1-2}
					\cellcolor{green}{1-3}
					\cellcolor{green}{1-4}
				\Body
					\,\,1\,\, & \,\,2\,\, & \,\,3\,\, & \,\,4\,\,
				\end{NiceArray}\)
		\end{tabular}
	\end{tabular} \\
	\(\cons_5(H, V, F)\) \\
	\(
		\begin{NiceArray}{*{12}{c}}[corners={NW,SE},hvlines]
		\CodeBefore
		    	\cellcolor{red}{1-8}
			\cellcolor{red}{1-9}
			\cellcolor{cyan}{1-10}
			\cellcolor{red}{1-11}
			\cellcolor{green}{1-12}
			\cellcolor{green}{2-1}
			\cellcolor{red}{2-2}
			\cellcolor{green}{2-3}
			\cellcolor{cyan}{2-4}
			\cellcolor{cyan}{2-5}
			\cellcolor{cyan}{2-6}
			\cellcolor{cyan}{2-7}
			\cellcolor{cyan}{2-8}
			\cellcolor{red}{2-9}
			\cellcolor{green}{2-10}
			\cellcolor{cyan}{2-11}
			\cellcolor{red}{2-12}
			\cellcolor{green}{3-1}
			\cellcolor{red}{3-2}
			\cellcolor{green}{3-3}
			\cellcolor{yellow}{3-4}
			\cellcolor{yellow}{3-5}
			\cellcolor{yellow}{3-6}
			\cellcolor{yellow}{3-7}
			\cellcolor{yellow}{3-8}
			\cellcolor{red}{3-9}
			\cellcolor{green}{3-10}
			\cellcolor{yellow}{3-11}
			\cellcolor{red}{3-12}
			\cellcolor{green}{4-1}
			\cellcolor{red}{4-2}
			\cellcolor{green}{4-3}
			\cellcolor{yellow}{4-4}
			\cellcolor{yellow}{4-5}
			\cellcolor{yellow}{4-6}
			\cellcolor{yellow}{4-7}
			\cellcolor{yellow}{4-8}
			\cellcolor{red}{4-9}
			\cellcolor{green}{4-10}
			\cellcolor{yellow}{4-11}
			\cellcolor{red}{4-12}
			\cellcolor{green}{5-1}
			\cellcolor{red}{5-2}
			\cellcolor{green}{5-3}
			\cellcolor{cyan}{5-4}
			\cellcolor{cyan}{5-5}
			\cellcolor{cyan}{5-6}
			\cellcolor{cyan}{5-7}
			\cellcolor{cyan}{5-8}
			\cellcolor{red}{5-9}
			\cellcolor{green}{5-10}
			\cellcolor{cyan}{5-11}
			\cellcolor{red}{5-12}
			\cellcolor{red}{6-1}
			\cellcolor{yellow}{6-2}
			\cellcolor{green}{6-3}
			\cellcolor{green}{6-4}
		 \Body
		    	   & & & & & & & 1 &  2 & 3 &  4 &  5 \\
		    	 6 &  7 &  8 &  9 & 10 & 11 & 12 & 13 & 14 & 15 & 16 & 17 \\
		    	18 & 19 & 20 & 21 & 22 & 23 & 24 & 25 & 26 & 27 & 28 & 29 \\
		    	30 & 31 & 32 & 33 & 34 & 35 & 36 & 37 & 38 & 39 & 40 & 41 \\
			42 & 43 & 44 & 45 & 46 & 47 & 48 & 49 & 50 & 51 & 51 & 53 \\
			54 & 55 & 56 & 57 \\
		\end{NiceArray}\)
\end{tabular}
\end{center}
\end{figure}

In some cases, it may be useful to use a specific coloring for the final coloring. However, a default coloring 
associated to the horizontal coloring can be used instead.

\begin{definition}[Default coloring]
Let \(n \in \mathbb{N}^*\), \(b \in \mathbb{N}\) and \(r \in \mathbb{N}\). Let \(H\) be a coloring of \([\![1, n]\!]\) 
with \(r +1\) colors. Set \(d = \min \left(H^{-1}(\{r + 1\}) \backslash [\![1, b]\!] \right) - b  - 1\). The default 
coloring \(\coldef_b(H)\)  associated to \(H\) is defined as follows.
\[\begin{array}{c c c c}
	\coldef_b(H) : & [\![1, d]\!] & \longrightarrow & [\![1, r + 1]\!] \\
	 & x & \longmapsto & H(x + b)
\end{array}\]
\end{definition}

The construction described in \cite{RowleyRamsey} for sf-templates corresponds to the case \(b = 0\) and it uses the 
default coloring. Contrary to sf-templates, \(b\)-templates are directly defined as particular colorings for which the 
clique numbers remain unchanged in the construction. The rest of this section defines the \(b\)-templates as well as 
the notion of compatibility for the final coloring.

In order to describe the configurations in which a given tuple can induce a monochromatic clique when using the 
construction, a tree is associated to every tuple.

\begin{definition}
A tree node has two components: the first one is the label and the second one is the set of descendants of this node. A 
branch of a tree \(T\) is a tuple \((x_i)_{1 \leqslant i \leqslant k}\) such that there is a sequence of trees 
\((t_i)_{1 \leqslant i \leqslant k}\) satisfying: \(t_1 = T\) and for all \(i \in [\![1, k]\!]\) \(x_i\) is the label 
of \(t_i\) and for all \(i \in [\![1, k - 1]\!]\) \(t_{i + 1}\) is a descendant of \(t_i\).

Let \(a \in \mathbb{N}^*\) and \(b \in \mathbb{N}\). The \(\tree_{a, b}\) application is recurssively defined as 
follows.
\[ \left\{
\begin{array}{l}
	\forall x \in \mathbb{Z}, \tree_{a, b}(x) = (x, \varnothing), \\
	\forall k \in [\![2, +\infty[\![, \forall x \in \mathbb{Z}^k, \\
	\mspace{30mu}
		\left\{
		\begin{array}{l l}
			\tree_{a, b}(x) = (x_1, \{\tree_{a, b}(x_2, ..., x_k)\}) &  ~\text{if}~ x_2 - x_1 > b \\
			\tree_{a, b}(x) = (x_1, \{\tree_{a, b}(x_2, ..., x_k), \tree_{a, b}(x_2 + a, ..., x_k)\}) &  ~\text{if}~ 0 \leqslant 
				x_2 - x_1 \leqslant b \\
			\tree_{a, b}(x) = \tree_{a, b} \left(x_1, x_2 + a \left \lceil \dfrac{x_1 - x_2}{a} \right \rceil, x_3, ..., x_k 
				\right) & ~\text{if}~ x_2 < x_1
		\end{array}
		\right.
\end{array}
\right. \]
Let \(x\) be a non-empty tuple of integers. The set of all the branches of the tree \(\tree_{a, b}(x)\) is denoted by 
\(\treeset_{a, b}(x)\).
\end{definition}

Let \(A \subset \mathbb{N}^*\) and let \(k \in [\![2, +\infty]\!]\). When using the construction, not every tuple in 
\(A^k\) have a tree which describes cliques that appear in the construction.

\begin{definition}
Let \(A \subset \mathbb{N}^*\), \(b \in \mathbb{N}\) and \(k \in [\![2, +\infty[\![\). The set of \(k\)-tuples of 
\(A^k\) whose values are pariwise distinct is denoted by \(\mathfrak{S}_k(A)\). The set of \(k\)-tuples of 
\(\mathfrak{S}_k(A)\) such that the elements of a tuple that are in \([\![1, b]\!]\) appear in increasing order at 
begining of this tuple is denoted by \(S_{b,k}(A)\). That is 
\(S_{b,k}(A) = \left\{ x \in \mathfrak{S}_k(A) ~:~ \forall i \in  [\![2, k]\!], x_i \leqslant b \implies x_i \geqslant 
x_{i -1} \right\}\). The set of branches of trees of tuples in \(S_{b, k}(A)\) is denoted by \(\TS_{b,k}(A)\): 
\(\TS_{b,k}(A) = \displaystyle \bigcup \limits_{x \in S_{b,k}(A)} \treeset(x)\).
\end{definition}

Most of the integers which appear in the tuples of \(\TS_{b,k}(A)\) (as well as the differences of these integers) are 
not in \([\![1, a + b]\!]\). The role of the projection \(\pi_{a, b}\) is to project these integers onto 
\([\![1, a + b]\!]\) in order to describe the constraints defning the \(b\)-templates.

\begin{definition}
Let \(a \in \mathbb{N}^*\) and \(b \in \mathbb{N}\).The projection \(\pi_{a, b}\) is defined as follows.
\[ \begin{array}{c c c l}
	\pi_{a, b} : & \mathbb{Z} & \longrightarrow & [\![1, a + b]\!] \\
	 & x & \longmapsto & 
		\left\{
		\begin{array}{l l}
			x & ~\text{if}~ 1 \leqslant x \leqslant b \\
			(x \mod a) + a \mathds{1}_{[\![0, b]\!]}(x \mod a) & ~\text{if}~ x > b
		\end{array}
		\right.
\end{array} \]
\end{definition}

The \(b\)-templates are designed to be used as the horizontal coloring in the construction.

\begin{definition}[\(b\)-Templates]
Let \(r \in \mathbb{N}\) and \((k_c)_{1 \leqslant c \leqslant r} \in {[\![3, +\infty[\![}^r\). Let 
\(a \in \mathbb{N}^*\) and \(b \in [\![0, a - 1]\!]\).  A \(b\)-template with width \(a\) and \(r + 1\) colors is 
defined as a partition of \([\![1, a + b]\!]\) into \(r+1\) subsets \(A_1, ..., A_{r+1}\) such that:
\begin{enumerate}[(i)]
\item \(a \in A_{r+1}\),
\item \(\forall (x, y) \in A_{r + 1}^2, x + y \notin [\![a + b + 1, 2 a + b]\!] \implies \pi_{a, b}(x + y) \notin 
	A_{r + 1}\),
\item \(\forall c \in [\![1, c]\!], \forall x \in \TS_{b, k_c - 1}(A_c), \exists (i, j) \in {[\![1, k_c - 1]\!]}^2, i < 
	j ~\text{et}~ \pi_{a, b}(x_j - x_i) \notin A_c\).
\end{enumerate}
The set of these \(b\)-templates is denoted by \(\mathcal{T}_b(k_1, ..., k_r, t; a)\).  Color \(r + 1\) plays a special 
role and is nammed "template color".
\end{definition}

Color \(r + 1\)  is not necessarily the last color by order of appearance, the designation of \(r+1\) as template color 
symbolized by "\(t\)" is a convention that lightens notations and avoids writting 
\(\mathcal{T}_b(k_1, ..., k_{i - 1}, t,  k_{i + 1}, ..., k_{r + 1}; n)\) for instance.

If all the \(k_c\)'s are equal to 3, these \(b\)-templates can be used to produce sum-free partitions for Schur numbers.

\(b\)-Templates are a particular case of linear colorings. In the template color, the corresponding clique number is 
equal to 3. That is \(T_b(k_1, ..., k_r, t; a) \subset L(k_1, ..., k_r, 3; a + b)\).

The notion of compatibility with a \(b\)-template illustrates the fact that the final coloring in the construction is 
less constrained than the horizontal coloring, and that the constraints on this final coloring depends only on the 
horizontal coloring.

\begin{definition}[Compatibility with a \(b\)-template]
For \(A \subset \mathbb{N}\) and \(k \in \mathbb{N}\), the set of increasing sequences of length \(k\) with values in 
\(A\) is denoted by \(\mathfrak{I}_k(A)\).

Let \(a \in \mathbb{N}^*\), \(b \in [\![0, a - 1]\!]\) and \(r \in \mathbb{N}\). Let 
\((k_c)_{1 \leqslant c \leqslant r} \in {[\![3, + \infty [\![}^r\). Let \(T \in \mathcal{T}_b(k_1, ..., k_r, t; a)\) 
and denote by \(A_1, ..., A_{r+1}\) the associated partition. Let \(d \in [\![0, a]\!]\). Let \(F\) be a coloring of 
\([\![1, d]\!]\) with \(r +1\) colors and denote by \(B_1, ..., B_{r + 1}\) the associated partition. The coloring 
\(F\) is said to be strongly compatible with the \(b\)-template \(T\) if it satisfies:
\begin{enumerate}[(i)]
\item \label{cond:nocol} \(a - b \notin B_{r + 1}\)
\item \label{cond:noover} \(\forall (x,y) \in A_{r+1}^2, (x + y > a + b ~\text{and}~ \pi_{a, b}(x + y) - b \leqslant d) 
	\implies \pi_{a, b}(x + y) - b \notin B_{r+1}\)
\item \label{cond:nolin} \(\forall (x,y) \in A_{r+1} \times B_{r+1}, x + y \leqslant d \implies x + y  \notin B_{r+1}\)
\item \label{cond:notree} \(\forall c \in [\![1, r]\!], \forall k \in [\![1, k_c-1]\!], \forall x \in \TS_{b, k}(A_c), 
	\forall y \in \mathfrak{I}_{k_c - k - 1}(B_c), \\
	\begin{array}{l}
		\left\{ \text{or} 
		\begin{array}{l}
			\exists (i, j) \in {[\![1, k]\!]}^2, \pi_{a, b}(x_j - x_i) \notin A_c, \\
			\exists (i, j) \in [\![1, k]\!] \times [\![1, k_c - k - 1]\!], \\
			\mspace{15mu}
				\left \{
				\begin{array}{l l}
					(y_j + b - x_i) \mod a \notin A_c & ~\text{if}~ (y_j + b - x_i) \mod a > b \\
					(y_j + b - x_i) \mod a \notin A_c ~\text{and}~ \pi_{a, b}(y_j + b - x_i) \notin A_c & ~\text{if}~ (y_j + b - x_i) 
						\mod a \leqslant b \\
				\end{array}
				\right. \\
			\exists (i, j) \in {[\![1, k_c - k - 1]\!]}^2, i < j ~\text{and}~ \pi_{a, b}(y_j - y_i) \notin A_c.
		\end{array}
		\right.
	\end{array}\)
\end{enumerate}

In practice, conditions (\ref{cond:noover}) and (\ref{cond:nolin}) can be weakened by replacing \(A_{r + 1}\) by 
\(A_{r + 1} \backslash [\![1, b]\!]\); those modified conditions are respectively denoted by 
(\(\widetilde{\ref{cond:noover}}\)) and (\(\widetilde{\ref{cond:nolin}}\)). The coloring \(F\) is said to be compatible 
with the \(b\)-template \(T\) if it satisfies conditions (\ref{cond:nocol}), (\(\widetilde{\ref{cond:noover}}\)), 
(\(\widetilde{\ref{cond:nolin}}\)) and (\ref{cond:notree}). The coloring \(F\) is said to be weakly compatible with the
 \(b\)-template \(T\) if it satisfies conditions (\ref{cond:nocol}), (\(\widetilde{\ref{cond:noover}}\)) and 
(\ref{cond:notree}). 
\end{definition}

\section{Properties of the \(b\)-templates}
\label{sec:results}

\begin{proposition}
Let \(r \in \mathbb{N}\) and let \((k_c)_{1 \leqslant c \leqslant r}  \in {[\![3, +\infty[\![}^r\). Let 
\(a \in \mathbb{N}^*\) and let \(b \in [\![0, a - 1]\!]\). Let \(T \in \mathcal{T}_b(k_1, ..., k_r, t; a)\). Then 
\(\coldef_b(T)\) is strongly compatible with \(T\).
\end{proposition}

\begin{proof}
TODO
\end{proof}

\begin{lemma}
Let \(a \in \mathbb{N}^*\), \(b \in [\![0, a - 1 ]\!]\) and \(r \in \mathbb{N}^*\). Let 
\((k_c)_{1 \leqslant c \leqslant r} \in {[\![3, + \infty [\![}^r\). Let \(T\) be a coloring of \([\![1,  a + b]\!]\) 
with \(r + 1\) colors. Let \(d \in [\![0, a]\!]\) and let \(F\) be a coloring of \([\![1, d]\!]\) with \(r + 1\) 
colors. Set \(k = \max_{1 \leqslant c \leqslant r} k_c - 1\).

Define the following three colorings.
\[\begin{array}{l c c c l}
	V_\mathit{strong} : & [\![1, k - 2]\!] & \longrightarrow & \{1\} & \\
	 & x & \longmapsto & 1 & \\
	 & & & & \\
	V_\mathit{reg} : & [\![1, k - 2]\!] & \longrightarrow & \{1\} & \\
	 & x & \longmapsto & 1 & \\
	 & & & & \\
	V_\mathit{weak} : & [\![1, k - 2]\!] & \longrightarrow & \{1\} & \\
	 & x & \longmapsto & 1 & \\
\end{array}\]
\end{lemma}

\begin{theorem}[Construction of linear colorings]
\label{thm:b-temp}
Soit \(r \in \mathbb{N}\) et soit \((k_c)_{1 \leqslant c \leqslant r} \in {[\![3, +\infty[\![}^r\). Soit \(a \in \mathbb{N}^*\) et soit \(b \in \mathbb{N}\). Soit \(H\) un coloriage de \([\![1, a + b]\!]\) à \(r + 1\) couleurs. Soit \(d \in [\![0, a - 1]\!]\) et soit \(F\) un coloriage de \([\![1, d]\!]\) à \(r + 1\) couleurs. Les deux conditions suivantes sont équivalentes.

\begin{enumerate}[(i)]
\item \(H \in \mathcal{T}_b(k_1, ..., k_r, t; a)\) et \(F\) est compatible  avec \(H\).
\item \(\forall s \in \mathbb{N}^*, \forall l \in {[\![3, +\infty[\![}^s, \forall V \in \mathcal{L}(l_1, ..., l_s ; p), \cons_b(H, V, F) \in  \mathcal{L}(k_1, ..., k_r, l_1, ..., l_s ; a \times p + b + d)\)
\end{enumerate}
\end{theorem}

\begin{corollary}[Inégalités pour les nombres de Ramsey linéaires]
\label{thm:ineq}
Soit \(r \in \mathbb{N}\) et soit \((k_c)_{1 \leqslant c \leqslant r} \in {[\![3, +\infty[\![}^r\). Soit \(s \in \mathbb{N}\) et soit \((l_c)_{1 \leqslant c \leqslant s} \in {[\![3, +\infty[\![}^s\). Soient \(a \in \mathbb{N}^*\) et \(b \in \mathbb{N}\) tels que \(\mathcal{T}_b(l_1, ..., l_s, t; a) \neq \varnothing\), et on fixe \(T\) un tel \(b\)-template. Soit \(d \in [\![1, a]\!]\) tel qu'il existe un coloriage de \([\![1, d]\!]\) à \(s + 1\) couleurs compatible avec \(T\).
Alors \(L(k_1, ..., k_r, l_1, ..., l_s) - 2 \geqslant a (L(k_1, ..., k_r) - 2) + b + d\).
\end{corollary}

La construction présentée dans \cite{rowleyramseyabott} utilise des coloriages linéaires afin de construire des \(0\)-templates. La proposition suivante présente cette construction sous l'angle des \(b\)-templates.
\begin{proposition}
Soit \(r \in \mathbb{N}\) et soit \((k_c)_{1 \leqslant c \leqslant r} \in {[\![3, +\infty[\![}^r\). Soit \(n \in \mathbb{N}\) tel que \(\mathcal{L}(k_1, ..., k_r; n) \neq \varnothing\) et on se donne \(\gamma\) un tel coloriage. On définit un nouveau coloriage T de la manière suivante.
\[\begin{array}{c c c l}
	T : & [\![1, 2 n + 1]\!] & \longrightarrow & [\![1, r + 1]\!] \\
	 & x & \longmapsto & \left\{
		\begin{array}{l l}
			\gamma(x) & ~\text{si}~ x \leqslant n \\
			r + 1 & ~\text{si}~ x > n \\
		\end{array}
		\right.
\end{array}\]
Alors \(T \in \mathcal{T}_0(k_1, ..., k_r, t; 2 \, n + 1)\).
\end{proposition}

\begin{lemma}[associativité de la construction]
TODO
\[
\cons_{b_{H_1}}(H_1, , )
\]
\end{lemma}

\begin{theorem}[Construction de \(b\)-templates]
TODO
\end{theorem}

\section{New inequalites and lower bounds}
\label{sec:bounds}

As in \cite{schurboyz} and \cite{rowleyramseysat}, the \(b\)-templates described in this article were found by encoding 
the existence of the templates into a boolean formula and then using a SAT solver (here, the parallel version of the 
Lingeling SAT solver \cite{Lingeling2017}).

\subsection{Multicolor Ramsey numbers}

The templates in Appendix \ref{appendix} provide inequalities of the form 
\(L(k_1, ..., k_r, k_{r+1}, ..., k_{r+s}) \geqslant a \, L(k_1, ..., k_r) - 2 + d\). In table \ref{tab:ineq},"added 
\(k_i\)'s" denotes the \( k_{r+1}, ..., k_{r+s}\) in the previous inequality and "\(b\)-template" indicates a 
\(b\)-template which cannot be expressed as a sf-template.

\begin{table}[H]
\begin{center}
\caption{Summarizing table of the new inequalities}
\begin{tabular}{| c | c | c | c | c |}
	\hline
	added \(k_i\)'s & former (a, d) & new (a, d) & template type & final coloring \\
	\hline
	3, 3 & 9, 4 \cite{AbbottHanson} & 10, 2 & \(b\)-template & default \\
	\hline
	3, 4 & 18, 7 \cite{rowleyramseysat} & 19, 2 & \(b\)-template & default \\
	\hline
	3, 5 & 30, 12 \cite{rowleyramseysat} & 31, 3 & \(b\)-template & specific \\
	\hline
	4, 5 & 51, 21 \cite{rowleyramseysat} & 55, 23 & sf-template & default \\
	\hline
\end{tabular}
\end{center}
\label{tab:ineq}
\end{table}

\begin{table}[H]
\begin{center}
\caption{Summarizing table of the new lower bounds for small Ramsey numbers}
\begin{tabular}{| c | c | c |}
	\hline
	Ramsey number & former lower bound & new lower bound \\
	\hline
	\(R_8(3)\) & \(5\,288\) \cite{RowleyRamsey} & \(5\,364\)  \\
	\hline
	\(R_{13}(3)\) & \(2\,011\,292\) \cite{schurboyz} & \(2\,038\,284\) \\
	\hline
	\(R(3 , 4 , 5 , 5)\) & 729 \cite{rowleyramseysat} & 764 \\
	\hline
\end{tabular}
\end{center}
\label{tab:bounds}
\end{table}
The template in \(\mathcal{L}_2(3, 3, t; 10)\) induces an inequality which improves the lower bounds for 
\(R_{5 n + 3}(3)\) for all \(n \in \mathbb{N}^*\).

\subsection{Schur numbers}

The existence of a 2-template with width 10 and 3 colors which cannot be expressed as a sf-template provides an improved 
inequality. The former corresponding inequality is \(S(n + 2) \geqslant 9 \, S(n) + 4\) \cite{AbbottHanson}.

\begin{proposition}
	\[\forall n \in \mathbb{N}, S(n + 2) \geqslant 10 \, S(n) + 2\]
\end{proposition}

This inequality provides the following improved lower bounds: \(S(8) \geqslant 5\,362\) and 
\(S(13) \geqslant 2\,038\,282\) (the former lower bounds were respectively \(5\,286\) \cite{RowleyRamsey} and 
\(2\,011\,290\) \cite{schurboyz}). More generally, this inequality improves the lower bounds for \(S(5 n + 3)\) for all 
\(n \in \mathbb{N}^*\).

\section{Acknowledgements}

This study results from a long-term project performed within the framework of the Project Cluster "Training for 
Research" of CentraleSupélec. I also would like to address warm thanks to Paul Casteras, Thibaut Pellerin and Yann 
Portella with whom I started working on Schur and weak Schur numbers.

\bibliographystyle{IEEEtran}
\bibliography{biblio}

\appendix

\section{\(b\)-Templates}
\label{appendix}

\begin{table}[H]
\caption{\(T \in \mathcal{T}_2(3, 3, t; 10)\)}
\begin{center}
\begin{tabular}{c}
	\begin{tabular}{| c | c |}
		\hline
		1 & 1, 5, 8, 12 \\
		\hline
		2 & 2, 6, 7 \\
		\hline
		3 & 3, 4, 9, 10, 11 \\
		\hline
	\end{tabular} \\ \\
	 Final coloring: \(\varnothing\)
\end{tabular}
\end{center}
\end{table}

\begin{table}[H]
\caption{\(T \in \mathcal{T}_2(3, 4, t; 19)\)}
\begin{center}
\begin{tabular}{c}
	\begin{tabular}{| c | c |}
		\hline
		1 & 1, 6, 9, 13, 17, 21 \\
		\hline
		2 & 2, 7, 8, 10, 14, 15, 16 \\
		\hline
		3 & 3, 4, 5, 11, 12, 18, 19, 20 \\
		\hline
	\end{tabular} \\ \\
	 Final coloring: \(\varnothing\)
\end{tabular}
\end{center}
\end{table}

\begin{table}[H]
\caption{\(T \in \mathcal{T}_2(3, 5, t; 31)\)}
\begin{center}
\begin{tabular}{c}
	\begin{tabular}{| c | c |}
		\hline
		1 & 1, 5, 9, 12, 16, 19, 23, 26, 30 \\
		\hline
		2 & 2, 6, 7, 8, 10, 11, 17, 18, 24, 25, 27, 28, 29, 33 \\
		\hline
		3 & 3, 4, 13, 14, 15, 20, 21, 22, 31, 32 \\
		\hline
	\end{tabular} \\ \\
	 Final coloring: \(F(1) = 1\)
\end{tabular}
\end{center}
\end{table}

\begin{table}[H]
\caption{\(T \in \mathcal{T}_0(5, 4, t; 55)\)}
\begin{center}
\begin{tabular}{c}
	\begin{tabular}{| c | c |}
		\hline
		1 & 1, 5, 6, 7, 9, 10, 12, 14, 15, 17, 18, 19, 23, 25, 31, 48, 54 \\
		\hline
		2 & 2, 3, 4, 8, 11, 13, 16, 20, 21, 22, 26, 28, 51, 52, 53 \\
		\hline
		3 & 24, 27, 29, 30, 32, 33, 34, 35, 36, 37, 38, 39 \\
		 & 40, 41, 42, 43, 44, 45, 46, 47, 49, 50, 55 \\
		\hline
	\end{tabular} \\ \\
	 Final coloring: \(\coldef_0(T)\)
\end{tabular}
\end{center}
\end{table}

\end{document}
