\documentclass[roundedcorners=true, titleposition=left]{beamerthemeruhuisstijlposter}

\usepackage{grffile}
\usepackage[french]{babel}
\usepackage[utf8]{inputenc}

\usepackage{amsmath,amsthm, amssymb, latexsym}
\usepackage{nicematrix}
\boldmath

\usepackage{array,booktabs,tabularx}
\listfiles

\newcommand{\WS}{\mathit{WS}}

\institute[CLS]{CentraleSupélec, Pôle Projet \og{}Formation à la recherche \fg{}}
\title{Les nombres de Schur et de Schur faibles}
\date{\today}
\author{\underline{Romain Ageron}, \underline{Paul Castéras}, \underline{Thibaut Pellerin}, Yann Portella,\\ Arpad Rimmel, Joanna Tomasik (encadrants)}

%%%%%%%%%%%%%%%%%%%%%%%%%%%%%%%%%%%%%%%%%%%%%%%%%%%%%%%%%%%%%%%%%%%%%%%%%%%%%%%%%%%%%%
\begin{document}

\setbeamertemplate{caption}{\raggedright\insertcaption\par}

\begin{frame}
\begin{columns}
\begin{column}{0.49\textwidth}
\begin{beamercolorbox}[center, wd=\textwidth]{postercolumn}
\begin{minipage}[T]{0.95\textwidth}
\parbox[t][\columnheight]{\textwidth}{%
    \begin{block}{Un peu d'histoire}
    	\begin{itemize}
    	    \item 1917 : Introduction par Schur des partitions sans somme lors d'une étude du \textbf{grand théorème de Fermat}
    	    \item 1930 : Introduction de la \textbf{théorie de Ramsey}
    	    \item 1972 : \textbf{Inégalités récursives} obtenues par Abbott et Hanson 
    	    \item 1990-2020 : Approche \textbf{numérique} (MCTS, SAT, ...)
    	    \item 2020 : Introduction des \textbf{templates} par Rowley, nouvelles inégalités
    	    \item Notre contribution : \textbf{généralisation} des templates, \textbf{nouvelles bornes inférieures}
    	\end{itemize}
    \end{block}
  \begin{block}{Définitions et propriétés}
	\textbf{Issai Schur} a posé le problème suivant :

	\begin{itemize}
		\item Pour \(n \geqslant 1\) un entier ( = taille du problème)
		\item \(k \geqslant 1\) un autre entier ( = nombre de \textbf{couleurs})
	
	\vspace{1ex}
	\textbf{Peut-on colorier les entiers de \(1\) à \(n\) de sorte que si deux 
	nombres ont la même couleur, leur somme n'est pas de cette couleur ?}
	
    \vspace{1ex}
    Si oui, un tel coloriage est dit \textbf{sans somme}.
			\vspace{1ex}

		\item On note \(S(k)\) le plus grand entier \(n\) vérifiant cette propriété.
\vspace{2ex}

Les nombres de Schur faibles sont une variante des nombres de Schur. On se demande : Peut-on colorier les entiers de \(1\) à \(n\) de sorte que si deux 
	nombres \textbf{distincts} ont la même couleur, leur somme n'est pas de cette couleur ?

    \vspace{1ex}
	\item On note \(\WS(k)\) le plus grand entier \(n\) vérifiant cette propriété.
	
	\vspace{2ex}
	Le comportement asymptotique de ces suites est mal connu.
	\vspace{1ex}	
	\item \(c \, \sqrt[5]{380}^k \leqslant S(k) \leqslant \WS(k)\)
	\item Lien avec les nombres de Ramsey \(S(k) \leqslant R_k(3) - 2 \leqslant \left\lfloor k! \left(e - \frac{1}{6}\right) \right\rfloor - 1\)
    \end{itemize}
  \end{block}
  
  
  
 	\vspace{-0.5ex}
    \begin{block}{Exemples}
    \centering
    On a \(S(3) = 13\) et \(WS(2) = 8\)
    \setlength{\arraycolsep}{0.8ex}
    \renewcommand{\arraystretch}{1.5}
    \begin{figure}
    
    \caption{\large Partition sans-somme à 3 couleurs}
    \vspace{1ex}
   	\(\begin{NiceArray}{|*{13}{c|}}[standard-cline,hlines]
		\CodeBefore
			\cellcolor{red}{1-1}
			\cellcolor{cyan}{1-2}
			\cellcolor{cyan}{1-3}
			\cellcolor{red}{1-4}
			\cellcolor{green}{1-5}
			\cellcolor{green}{1-6}
			\cellcolor{cyan}{1-7}
			\cellcolor{green}{1-8}
			\cellcolor{green}{1-9}
			\cellcolor{red}{1-10}
			\cellcolor{cyan}{1-11}
			\cellcolor{cyan}{1-12}
			\cellcolor{red}{1-13}
		\Body
			\,\,1\,\,\, & \,\,2\,\,\, & \,\,3\,\,\, & \,\,4\,\,\, & \,\,5\,\,\, & \,\,6\,\,\, & \,\,7\,\,\, & \,\,8\,\,\, & \,\,9\,\,\, & 10 & 11 & 12 & 13\\
	\end{NiceArray}\)
	
	\end{figure}
	\bigskip
	
	\setlength{\arraycolsep}{1.5ex}
    \renewcommand{\arraystretch}{1.5}
    \begin{figure}
    \caption{\large Partition \textit{faiblement} sans-somme à 2 couleurs}
    \vspace{1ex}
	\(\begin{NiceArray}{|*{8}{c|}}[standard-cline,hlines]
		\CodeBefore
			\cellcolor{red}{1-1}
			\cellcolor{red}{1-2}
			\cellcolor{cyan}{1-3}
			\cellcolor{red}{1-4}
			\cellcolor{cyan}{1-5}
			\cellcolor{cyan}{1-6}
			\cellcolor{cyan}{1-7}
			\cellcolor{red}{1-8}
		\Body
			1 & 2 & 3 & 4 & 5 & 6 & 7 & 8 \\
	\end{NiceArray}\)
    
	\end{figure}
    	
    \end{block}
    
    \vfill
    
}
\end{minipage}
\end{beamercolorbox}
\end{column}

\begin{column}{0.49\textwidth}
\begin{beamercolorbox}[center, wd=\textwidth]{postercolumn}
\begin{minipage}[T]{0.95\textwidth}
\parbox[t][\columnheight]{\textwidth}{%
  \begin{block}{Templates}
    \vspace{-1ex}
    \setlength{\arraycolsep}{0.8ex}\renewcommand{\arraystretch}{1.5}
    \begin{figure}
    \caption{\large Partition sans somme construite à l'aide d'un template}
    \vspace{1ex}
    \(\begin{NiceArray}{*{10}{c}}[corners={NW},hvlines]
    \CodeBefore
    	\cellcolor{red}{1-9}
    	\cellcolor{green}{1-10}
    	\cellcolor{cyan}{2-1}
    	\cellcolor{cyan}{2-2}
	\cellcolor{red}{2-3}
	\cellcolor{green}{2-4}
	\cellcolor{green}{2-5}
	\cellcolor{red}{2-6}
    	\cellcolor{cyan}{2-7}
    	\cellcolor{cyan}{2-8}
    	\cellcolor{cyan}{2-9}
	\cellcolor{red}{2-10}
	\cellcolor{yellow}{3-1}
    	\cellcolor{yellow}{3-2}
	\cellcolor{red}{3-3}
	\cellcolor{green}{3-4}
	\cellcolor{green}{3-5}
	\cellcolor{red}{3-6}
    	\cellcolor{yellow}{3-7}
    	\cellcolor{yellow}{3-8}
    	\cellcolor{yellow}{3-9}
	\cellcolor{red}{3-10}
	\cellcolor{yellow}{4-1}
    	\cellcolor{yellow}{4-2}
	\cellcolor{red}{4-3}
	\cellcolor{green}{4-4}
	\cellcolor{green}{4-5}
	\cellcolor{red}{4-6}
    	\cellcolor{yellow}{4-7}
    	\cellcolor{yellow}{4-8}
    	\cellcolor{yellow}{4-9}
	\cellcolor{red}{4-10}
	\cellcolor{cyan}{5-1}
    	\cellcolor{cyan}{5-2}
	\cellcolor{red}{5-3}
	\cellcolor{green}{5-4}
	\cellcolor{green}{5-5}
	\cellcolor{red}{5-6}
    	\cellcolor{cyan}{5-7}
    	\cellcolor{cyan}{5-8}
    	\cellcolor{cyan}{5-9}
	\cellcolor{red}{5-10}
    \Body
    	   &    &    &    &    &    &    &    &  1 &  2 \\
    	 3 &  4 &  5 &  6 &  7 &  8 &  9 & 10 & 11 & 12 \\
    	13 & 14 & 15 & 16 & 17 & 18 & 19 & 20 & 21 & 22 \\
    	23 & 24 & 25 & 26 & 27 & 28 & 29 & 30 & 31 & 32 \\
    	33 & 34 & 35 & 36 & 37 & 38 & 39 & 40 & 41 & 42 \\
    \end{NiceArray}\)
    \end{figure}
\end{block}
  
  
    \begin{block}{Forme des inégalités obtenues}
    	\begin{itemize}
    	
   \item Forme des inégalités pour les nombres de Schur :
   \[S(k+l) \geqslant aS(k) + b.\]
   
   Une approche précédente avait obtenu $a=2S(l)+1$ et $b=S(l)$.
   \vspace{1ex}
   \item Forme des inégalités pour les nombres de Schur faibles :
   \[ \WS(k+l) \geqslant cS(k) + d.\]
   
   Nous avons obtenu : $c \geqslant \WS (l) + \left \lceil \WS(l)/2 \right \rceil +1$.

     	
  \end{itemize}
  \end{block}

  \medskip
  \begin{block}{Reformulation en formule booléenne (CNF)}
  \begin{itemize}
      \item \(x_i^{(c)} = 1\) si le nombre \(i\) est dans la couleur \(c\), \(0\) sinon.
      \item \(\mathit{Col}_i = \bigvee\limits_{c} x_i^{(c)} \) : \(i\) est dans au moins une couleur.
      \item \(\mathit{Disj}_i = \bigwedge\limits_{c_1 \neq c_2} \neg x_i^{(c_1)} \vee \neg x_i^{(c_2)}\) : \(i\) a au plus une couleur.
      \item \(\mathit{Som}_c = \bigwedge\limits_{i + j \leqslant n} \neg x_i^{(c)} \vee \neg x_j^{(c)} \vee \neg x_{i+j}^{(c)}\) : \(c\) est sans somme.
  \end{itemize}
  \end{block}
  \medskip
  \vspace{-1ex}
\begin{block}{Résultats}
\vspace{-1ex}
\begin{center}
    \begin{table}
		\caption{\large Bornes inférieures pour les nombres de Schur}
        
		 \begin{normalsize}
		  
		\begin{tabular}{|*{11}{c|}}
		    \hline
		    \(k\) & 1 & 2 & 3 & 4 & 5 & 6 & 7 & 8 & 9 & 10 \\
		    \hline
		    Etat de l'art & 1* & 4* & 13* & 44* & 160* & 536 & 1\,696 &5\,286 & 17\,694 & 60\,320 \\
		    \hline
		    \textbf{Nos résultats} & & & & & & & & \textbf{5\,362} & \textbf{17\,803} & \textbf{60\,948} \\
		    \hline
		\end{tabular}
						  
		\end{normalsize}
	\end{table}
	
	\vspace{1ex}

    \begin{table}
		\caption{\large Bornes inférieures pour les nombres de Schur faibles}
        \begin{normalsize}
    		\begin{tabular}{|*{11}{c|}}
    		    \hline
    		    \(k\) & 1 & 2 & 3 & 4 & 5 & 6 & 7 & 8 & 9 & 10 \\
    		    \hline
    		    Etat de l'art & 2* & 8* & 23* & 66* & 196 & 642 & 2\,146 & 6\,976 & 22\,056 & 70\,778 \\
    		    \hline
    		    \textbf{Nos résultats} & & & & & &\textbf{646} & & & \textbf{22\,536} & \textbf{71\,256} \\
    		    \hline
    		\end{tabular}
    	\end{normalsize}
    \end{table}
    \vspace{1ex}
	* désigne une valeur exacte
\end{center}
Nous avons aussi obtenu de nouvelles bornes inférieures pour tout \(k \geqslant 11\).
\end{block}
\medskip

}
\end{minipage}
\end{beamercolorbox}
\end{column}
\end{columns}
\end{frame}

\end{document}